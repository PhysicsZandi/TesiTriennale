\begin{Abstract}
\begin{changemargin}{1cm}{1cm}
    Ad una simmetria globale dell'azione è possibile associare una legge di conservazione di una quantità fisica; se la simmetria dell'azione invece è locale, è possibile associare una relazione vincolante tra le equazioni del moto. Questo è il contenuto dei due teoremi di Noether e oggetto di studio di questa tesi. Nella prima parte, ci concentriamo sul primo teorema, sia per sistemi meccanici che per sistemi descritti da campi: introduciamo le nozioni di simmetria e di legge di conservazione, forniamo una dimostrazione del teorema e accompagniamo con numerosi esempi tra cui le dieci quantità conservate della particella libera o il tensore energia-impulso dell'elettrodinamica. Nella seconda parte, ci soffermiamo sul secondo teorema e in particolare sulle teorie che presentano simmetrie di gauge: ne descriviamo la struttura generale, con attenzione per i vincoli che emergono nella formulazione hamiltoniana, dimostriamo l'inverso del teorema, ovvero come ricavare una simmetria da un vincolo, e anche in questo caso forniamo degli esempi, tra cui la particella relativistica e nuovamente l'elettrodinamica.
\end{changemargin}
\end{Abstract}