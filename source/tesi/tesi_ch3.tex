\chapter{Secondo teorema e teorie di gauge}

    Nei primi due capitoli, abbiamo studiato sistemi fisici che presentano simmetrie globali, riconducibili a trasformazioni applicate a tutti i punti dello spaziotempo. Tuttavia in fisica sono presenti sistemi che possiedono simmetrie locali, ovvero trasformazioni che sono funzioni dei punti dello spaziotempo. Esse non associano quantità conservate che soddisfano leggi di conservazione ma identità o relazioni tra le equazioni del moto. Ciò significa che il sistema ha un numero di gradi di libertà sovrastimato, poichè non tutti sono ricavati dalle equazioni del moto.

    Questo è il contenuto del cosidetto secondo teorema di Noether. In questo capitolo non enunceremo né dimostreremo questo teorema, ma studieremo come esso sia strettamente collegato alle teorie di gauge. L'importanza di tali teorie risiede nello sviluppo della fisica moderna: tre delle quattro interazioni fondamentali (forte, debole ed elettromagnetica) sono descritte da teorie di gauge.

\section{Teoria di gauge}
    Prima di studiare in dettaglio la struttura matematica di una teoria di gauge, presentiamo brevemente quali caratteristiche possiede. Consideriamo un sistema fisico descritto da una lagrangiana nell'ambito della teoria classica dei campi. Definiamo una simmetria di gauge come una simmetria locale, ovvero una trasformazione contenente un'arbitraria funzione delle coordinate spaziotemporali che lascia invariata l'azione del sistema. Ne consegue che ci saranno relazioni che legano le equazioni del moto: fissando arbitrariamente la nostra funzione, i gradi di libertà del sistema diminuiscono. Inoltre, passando alla descrizione hamiltoniana, emerge la presenza di vincoli generati dalla simmetria, a differenza delle cariche conservate nel caso di simmetrie globali. Particolare attenzione è posta su quanto possa essere fisica una teoria formulata in questo modo, che a prima vista potrebbe non sembrarlo affatto. Tuttavia possiamo introdurre il concetto di classi di equivalenza di configurazioni, ovvero campi che differiscono matematicamente solamente per una simmetria di gauge descrivono fisicamente la stessa realtà. Nel prossimo paragrafo mostreremo quantitativamente tutte queste caratteristiche attraverso un semplice esempio di un sistema motivato dall'elettrodinamica di Maxwell.
    
\subsection{Un esempio motivato dall'elettrodinamica}

    Mostriamo dapprima che l'elettrodinamica è una teoria di gauge. Sia $A_\mu$ un quadripotenziale che soddisfa le equazioni di Maxwell, allora è possibile dimostrare che, aggiungendo un termine derivante dalla simmetria di gauge, il nuovo quadripotenziale continua a soddisfare le equazioni di Maxwell. Infatti prendendo 
    \begin{equation}\label{gaugeelettro}
        A'_\mu = A_\mu - \partial_\mu \Lambda
    \end{equation}
    e mettendolo nelle equazioni di Maxwell in assenza di sorgenti~\eqref{gauge1}
    \begin{equation*}
        \partial_\nu \partial^\nu A_\mu - \partial_\mu (\partial^\nu A_\nu) = 0
    \end{equation*}
    otteniamo 
    \begin{equation*}
    \begin{aligned}
        \partial_\nu \partial^\nu A'_\mu - \partial_\mu (\partial^\nu A'_\nu) & = \partial_\nu \partial^\nu A_\mu - \partial_\nu \partial^\nu \partial_\mu \Lambda - \partial_\mu (\partial^\nu A_\nu) + \partial_\mu \partial^\nu \partial_\nu \Lambda \\ & = \partial_\nu \partial^\nu A'_\mu - \partial_\mu (\partial^\nu A_\nu) - \partial_\mu \partial^\nu \partial_\nu \Lambda + \partial_\mu \partial^\nu \partial_\nu \Lambda \\ & = \partial_\nu \partial^\nu A_\mu - \partial_\mu (\partial^\nu A_\nu) = 0
    \end{aligned}
    \end{equation*}
    dove abbiamo usato il fatto che il quadridivergenza commuta e nell'ultimo passaggio la~\eqref{gauge1}.

    \hfill

    Sulla falsariga di tale teoria, prendiamo ora in considerazione un'azione dipendente da due campi $A$ e $\psi$, analoghi ai potenziali elettromagnetici, definito nel seguente modo
    \begin{equation*}
        S[A(t),~\psi(t)] = \frac{1}{2} \int dt ~ {(\dot \psi - A)}^2
    \end{equation*}
    dove la lagrangiana del sistema è
    \begin{equation*}
        L = \frac{1}{2} {(\dot \psi - A)}^2
    \end{equation*}
    
    Il primo passo da compiere è trovare la simmetria di gauge: l'azione è invariante sotto la trasformazione
    \begin{equation}\label{arbitraria}
        \psi' = \psi + \epsilon(t) \qquad A' = A + \dot \epsilon(t)
    \end{equation}
    dove $\epsilon(t)$ è un'arbitraria funzione del tempo. Infatti calcolando la variazione sotto tale trasformazione otteniamo 
    \begin{equation*}
    \begin{aligned}
        S[A'(t),~\psi'(t)] & = \frac{1}{2} \int dt ~ {(\dot \psi' - A')}^2 \\ & = \frac{1}{2} \int dt ~ {(\dot \psi + \dot \epsilon - A - \dot \epsilon )}^2 \\ & = \frac{1}{2} \int dt ~ {(\dot \psi - A)}^2 \\ & = S[A(t),\psi(t)]
    \end{aligned}
    \end{equation*}
    
    Il secondo passo da compiere è trovare le relazioni tra le equazioni del moto. Come ampiamente descritto nei precedenti capitoli, data un'azione, le equazioni del moto sono le equazioni di Eulero-Lagrange~\eqref{eullag}.
    Per il campo $A$ otteniamo 
    \begin{equation*}
        0 = \pdv{\mathcal L}{A} - \dv{}{t} \pdv{\mathcal L}{\dot A} = \pdv{}{A} \Big (\frac{1}{2} {(\dot \psi - A)}^2 \Big ) - \dv{}{t} \pdv{}{\dot A} \Big (\frac{1}{2} {(\dot \psi - A)}^2 \Big ) = - (\dot \psi - A)
    \end{equation*}
    che porta alla prima equazione del moto 
    \begin{equation} \label{moto1}
        \dot \psi - A = 0
    \end{equation}
    invece per il campo $\psi$ troviamo
    \begin{equation*}
        0 = \pdv{\mathcal L}{A} - \dv{}{t} \pdv{\mathcal L}{\dot \psi} = 2 \dv{}{t} (\dot \psi - A)
    \end{equation*}
    che conduce alla seconda equazione del moto 
    \begin{equation} \label{moto2}
        \dv{}{t} (\dot \psi - A) = 0
    \end{equation}
    Confrontando le equazioni del moto, notiamo che non sono indipendenti: l'equazione per il campo $A$~\eqref{moto1} contiene già l'equazione per $\psi$~\eqref{moto2} (se una funzione è nulla, anche la derivata necessariamente lo è). Conseguentemente, non abbiamo trovato due equazioni del moto, ma ne è presente soltanto una. 
    
    Mostriamo ora che la trasformazione di simmetria~\eqref{arbitraria}, non modifica le soluzioni delle equazioni del moto. Scriviamo esplicitamente quest'ultime ipotizzando che $\psi(t)$ sia una funzione $f(t)$ e dalle equazioni del moto~\eqref{moto1} ricaviamo che $A(t)$ è la sua derivata
    \begin{equation*}
        \psi(t) = f(t) \qquad A(t) = \dot f(t)
    \end{equation*}
    Applicando la~\eqref{arbitraria}, otteniamo
    \begin{equation*}
        \psi'(t) = \psi(t) + \epsilon(t) = f(t) + \epsilon(t) \qquad A'(t) = A(t) + \dot \epsilon(t) = \dot f(t) + \dot \epsilon(t)
    \end{equation*}
    e mostriamo che soddisfano anch'esse le equazioni del moto
    \begin{equation*}
        \dot \psi'- A' = \dot \psi(t) + \dot \epsilon(t) - A - \dot \epsilon(t) = \dot \psi - A = 0
    \end{equation*}
    dove nell'ultimo passaggio abbiamo usato la~\eqref{moto1}.
    
    Infine il terzo passo da compiere è passare dal formalismo lagrangiano alla descrizione hamiltoniana e trovare i vincoli generati dalla simmetria di gauge. Definiamo prima il momento coniugato al campo $\psi$
    \begin{equation*}
        p_\psi = \pdv{L}{\dot \psi} = \dot \psi - A
    \end{equation*}
    e quello coniugato al campo $A$
    \begin{equation*}
        p_A = \pdv{L}{\dot A} = 0
    \end{equation*}
    Osserviamo che il momento coniugato ad $A$ è nullo, quindi non comparirà nell'hamiltoniana, mentre invertiamo la prima relazione per isolare la derivata di $\psi$
    \begin{equation} \label{derpsi}
        \dot \psi = p_\psi + A
    \end{equation}
    Successivamente applichiamo la trasformazione di Legendre~\eqref{hamiltoniana} per ottenere l'hamiltoniana del sistema in funzione dei campi $\psi$ e $A$ e del momento coniugato $p_\psi$ 
    \begin{equation*}
    \begin{aligned}
        H(\psi,~A,~p_\psi) & = p_\psi \dot \psi - L \\ & = p_\psi \dot \psi - \frac{1}{2} {(\dot \psi - A)}^2 \\ & = p_\psi (p_\psi + A) - \frac{1}{2} {(p_\psi + A - A)}^2 \\ &  = \frac{1}{2} p^2_\psi  + A p_\psi
    \end{aligned}
    \end{equation*}
    dove abbiamo utilizzato la~\eqref{derpsi}. Nel formalismo hamiltoniano, le equazioni del moto sono le equazioni di Hamilton~\eqref{ham1} e~\eqref{ham2}. Per il campo $A$ abbiamo
    \begin{equation} \label{Ah}
        \dot p_A = - \pdv{H}{A} = - p_\psi = 0
    \end{equation}
    poichè $p_A = 0$, mentre per il campo $\psi$
    \begin{equation} \label{psih}
        \dot \psi = \pdv{H}{p_\psi} = p_\psi + A \qquad \dot p_\psi = - \pdv{H}{\psi} = 0
    \end{equation}
    Notiamo nella prima equazione l'assenza di derivate temporali, ciò significa che non è un'equazione di evoluzione temporale ma un vincolo
    \begin{equation*}
        p_\psi = 0
    \end{equation*}
    dove $A$ è il moltiplicatore di Lagrange corrispondente. Inoltre osserviamo che anche in questo caso le equazioni del moto non sono indipendenti, infatti l'ultima~\eqref{psih} è la derivata temporale della prima~\eqref{Ah}. Analogamente al formalismo lagrangiano, abbiamo nuovamente equazioni del moto non indipendenti.

    \hfill

    Concludiamo il paragrafo con una nota importante. Dato che le equazioni del moto per $\psi$ sono contenute in quelle per $A$,~\eqref{moto2} in~\eqref{moto1} oppure~\eqref{psih} in~\eqref{Ah}, non è possibile scegliere arbitrariamente $A = 0$ ma posso soltanto imporre $\psi = 0$. In effetti, ponendo $\psi = 0$ 
    \begin{equation*}
        S[\psi = 0,~A] = \int dt ~ A^2
    \end{equation*}
    porta all'equazione accettabile di un vincolo
    \begin{equation*}
        A = 0
    \end{equation*}
    mentre scegliendo $A = 0$ 
    \begin{equation*}
        S[\psi,~A=0]=0 = \int dt ~\dot \psi^2
    \end{equation*}
    conduce ad una inaccettabile equazione del moto 
    \begin{equation*}
        \ddot \psi = 0
    \end{equation*}
    poichè un'evoluzione temporale non può emergere dalla simmetria di gauge, soltanto un vincolo può.

\subsection{Struttura generale delle teorie di gauge} 

    Presentiamo in questo paragrafo la struttura matematica di una teoria di gauge nella sua formulazione hamiltoniana. Il vantaggio di studiare tale teoria in questa formulazione risiede nella possibilità di analizzare in generale caratteristiche valide per molti esempi. In presenza di vincoli, il metodo più generale per passare dalla descrizione lagrangiana a quella hamiltoniana è il metodo sviluppato da P. A. M. Dirac $(1902- 1984)$. Tuttavia salteremo quella procedura e partiremo direttamente dalla conoscenza dell'hamiltoniana e dell'azione. 
    
    Consideriamo dunque un'azione hamiltoniana nella forma generica
    \begin{equation} \label{azionevincolo}
        S[q^i,~p_i,~\lambda^a] = \int dt ~ (p_i \dot q^i - H_0(q^i, ~p_i) + \lambda^a \phi_a (q^i, ~p_i))
    \end{equation}
    dove le variabili indipendenti sono $p_i$, $q^i$ e $\lambda^a$, rispettivamente i momenti coniugati, le coordinate e i moltiplicatori di Lagrange. L'hamiltoniana totale è definita come 
    \begin{equation*}
        H = H_0 - \lambda^a \phi_a
    \end{equation*} 
    ed è composta da una parte che non contiene vincoli $H_0$ e da una parte che contribuisce ad aggiungerli, dove le equazioni dei vincoli sono in funzione delle $\phi_a$. Le corrispondenti equazioni del moto, variando le tre variabili, sono
    \begin{equation} \label{hamvinc1}
        \dot q^i = \pdv{H}{p_i} = \pdv{H_0}{p_i} - \lambda^a \pdv{\phi_a}{p_i}
    \end{equation}
    \begin{equation} \label{hamvinc2}
        \dot p_i = - \pdv{H}{q^i} = - \pdv{H_0}{q^i} + \lambda^a \pdv{\phi_a}{q^i}
    \end{equation} 
    \begin{equation} \label{hamvinc3}
        \phi_a(q^i, ~p_i) = 0
    \end{equation}
    Le prime due equazioni determinano la dinamica del sistema, ovvero come evolvono $q$ e $p$ date le condizioni iniziali $q_0$ e $p_0$. Tuttavia le condizioni iniziali non possono essere del tutto arbitrarie ma devono soddisfare la terza equazione. Inoltre come scelgo le $\lambda$? Con l'utilizzo delle parentesi di Poisson~\eqref{poisson} e delle equazioni di Hamilton~\eqref{hamvinc1} e~\eqref{hamvinc2}, calcoliamo la derivata temporale dei vincoli 
    \begin{equation*}
    \begin{aligned}
        \dv{}{t} \phi_a(q^i, ~p_i) & = \pdv{\phi_a}{q^i} \dot q^i + \pdv{\phi_a}{p_i} \dot p_i = \pdv{\phi_a}{q^i} \pdv{H_0}{p_i} - \lambda^b \pdv{\phi_a}{q^i} \pdv{\phi_b}{p_i} - \pdv{\phi_a}{p_i} \pdv{H_0}{q^i} + \lambda^b \pdv{\phi_a}{p_i} \pdv{\phi_b}{p_i} \\ & = \pdv{\phi_a}{q^i} \pdv{H_0}{p_i} - \pdv{H_0}{q^i} \pdv{\phi_a}{p_i} - \lambda^b \Big ( \pdv{\phi_a}{q^i} \pdv{\phi_b}{p_i} -  \pdv{\phi_b}{p_i} \pdv{\phi_a}{p_i} \Big) \\ & = \poi{\phi_a}{H_0} - \lambda^b \poi{\phi_a}{\phi_b} \\ & = \poi{\phi_a}{H_0} - \lambda^b C_{ab}
    \end{aligned}
    \end{equation*}
    dove abbiamo chiamato la matrice $C_{ab} = \poi{\phi_a}{\phi_b}$. Dato che la condizione vincolante è $\phi_a = 0$, non ci restringiamo a porre la derivata del vincolo nulla ma introduciamo il concetto di debolmente nulla, ovvero nulla soltanto quando $\phi_a = 0$
    \begin{equation} \label{vincoli}
        \poi{\phi_a}{H_0} - \lambda^b C_{ab} \approx 0
    \end{equation}

    Distinguiamo ora due possibili casi, in base alle proprietà della matrice $C_{ab}$. 
    Nel caso in cui sia invertibile, i moltiplicatori di Lagrange $\lambda^a$ sono fissati dall'equazione~\eqref{vincoli}
    \begin{equation*}
        \lambda^b (t) = C^{-1}_{ab} \poi{\phi_a}{H_0}
    \end{equation*}
    dove il loro ruolo è quello di mantenere il vincolo nullo $\phi = 0$ nel tempo. Vincoli di questo tipo sono chiamati di seconda classe e offrono una dinamica del sistema semplice: date le condizioni iniziali in modo che rispettino i vincoli, le $\lambda$ assicurano che il vincolo sia mantenuto negli istanti successivi. Siamo in presenza di teorie non di gauge.
    Caso più interessante è quando la matrice è debolmente nulla $C_{ab} \approx 0$. I moltiplicatori di Lagrange non impongono alcuna condizione, poichè non compaiono nella~\eqref{vincoli}, e le equazioni del moto rimangono non determinate univocamente. Vincoli di questo tipo sono chiamati di prima classe, ovvero quando soddisfano le condizioni
    \begin{equation} \label{secondaclasse}
        \poi{\phi_a}{H_0} = C_a^{\phantom a b} \phi_b \approx 0 \qquad \poi{\phi_a}{\phi_b} = C_{ab}^{\phantom{ab} c} \phi_c \approx 0
    \end{equation}
    In questo caso siamo in presenza di teorie di gauge.
    
    \hfill 

    Particolare attenzione bisogna porre ai gradi di libertà del sistema. Teorie non di gauge con l'azione del tipo 
    \begin{equation*}
        S = \int dt ~ (p_i q^i - H_0)
    \end{equation*}
    con $i=1,~2, \ldots, N$, possiedono 2N costanti di integrazione per risolvere le equazioni di Hamilton, dunque i gradi di libertà sono $\frac{1}{2} 2N = N$. Teorie di gauge invece, con l'azione del tipo
    \begin{equation*}
        S = \int dt ~(p_i q^i - H_0 - \lambda^a \phi_a)
    \end{equation*}
    con $a=1,~2, \ldots, V$, possiedono anch'essi 2N costanti di integrazione ma ci sono $V$ vincoli dovuti a simmetrie di gauge che implicano la non necessità di conoscerle tutte: il numero di gradi di libertà è $\frac{1}{2} (2N - 2V) = N - V$.

\subsection{Il teorema di Noether inverso}
    La presenza di vincoli di prima classe e dunque di teorie di gauge, porta come conseguenza il fatto che l'azione è invariante per trasformazioni del tipo
    \begin{equation} \label{gaugee1}
        \delta q^i = \poi{q^i}{\phi_a} \epsilon^a (t)
    \end{equation}
    \begin{equation}\label{gaugee2}
        \delta p_i = \poi{p_i}{\phi_a} \epsilon^a (t)
    \end{equation}
    \begin{equation}\label{gaugee3}
        \delta \lambda^c = \dot \epsilon^c (t) + \epsilon^a(t) C_a^{\phantom a c} - \lambda^a \epsilon^b(t) C_{ab}^{\phantom{ab} c}
    \end{equation}
    dove $\epsilon^a(t)$ è un'arbitraria funzione del tempo. Queste sono proprio trasformazioni di simmetria di gauge. Siamo quindi in presenza dell'inverso del secondo teorema di Noether.
    \begin{theorem}[L'inverso del secondo teorema di Noether]
        Un'azione del tipo~\eqref{azionevincolo} tale che i vincoli e l'hamiltoniana soddisfano~\eqref{secondaclasse}, presenta delle trasformazioni di simmetria di gauge~\eqref{gaugee1}~\eqref{gaugee2}~\eqref{gaugee3}.
    \end{theorem}
    Dimostriamo ora questo enunciato. 
    \begin{proof}
    La variazione delle coordinate è data dalla~\eqref{gaugee1}
    \begin{equation} \label{varq}
        \delta q^i = \poi{q^i}{\phi_a} \epsilon^a = \Big(\poiexp{q^i}{\phi_a}\Big) \epsilon^a = \pdv{\phi_a}{p_i} \epsilon^a 
    \end{equation}
    mentre quella dei momenti è data dalla~\eqref{gaugee2}
    \begin{equation} \label{varp}
        \delta p_i = \poi{p_i}{\phi_a} \epsilon^a = \Big(\poiexp{p_i}{\phi_a} \Big) \epsilon^a = - \pdv{\phi_a}{q^i} \epsilon^a 
    \end{equation}
    dove abbiamo utilizzato le relazioni canoniche~\eqref{canqq}.

    Calcoliamo ora la variazione dell'azione
    \begin{equation*}
    \begin{aligned}
        \delta S & = \delta \int dt ~ (\dot q^i p_i - H_0(q^i, p_i) + \lambda^a \phi_a(p, q)) \\ & = \int dt ~ (\delta \dot q^i p_i + \dot q^i \delta p_i - \delta H_0(q^i, p_i) + \delta \lambda^a \phi_a(p, q) + \lambda^a \delta \phi_a(p, q)) \\ & = \int dt ~ (- \delta q^i \dot p_i + \dot q^i \delta p_i - \delta H_0(q^i, p_i) + \delta \lambda^a \phi_a(p, q) + \lambda^a \delta \phi_a(p, q) + \dv{}{t} (q^i p_i)) \\ & = \int dt ~ \Big (- \epsilon^a \pdv{\phi_a}{p_i} \dot p_i - \epsilon^a \dot q^i \pdv{\phi_a}{q^i} + \epsilon^a \Big (\pdv{H_0}{q^i} \pdv{\phi_a}{p_i} - \pdv{H_0}{p_i} \pdv{\phi_a}{q^i} \Big ) + \delta \lambda^a \phi_a \\ & \qquad \qquad + \lambda^a \epsilon^b \Big (\pdv{\phi_a}{q^i} \pdv{\phi_b}{p_i} - \pdv{\phi_a}{p_i} \pdv{\phi_b}{q^i} \Big) + \dv{}{t} (q^i p_i) \Big) \\ & = \int dt ~ \Big (- \epsilon^a \Big (\pdv{\phi_a}{p_i} \dot p_i - \dot q^i \pdv{\phi_a}{q^i} \Big ) + \epsilon^a \poi{H_0}{\phi_a} + \delta \lambda^a \phi_a + \lambda^a \epsilon^b \poi{\phi_a}{\phi_b} + \dv{}{t} (q^i p_i) \Big ) \\ & = \int dt ~ \Big (- \epsilon^a \dv{}{t} \phi_a + \epsilon^a \poi{H_0}{\phi_a} + \delta \lambda^a \phi_a + \lambda^a \epsilon^b \poi{\phi_a}{\phi_b} + \dv{}{t} (q^i p_i) \Big ) 
    \end{aligned}
    \end{equation*}
    dove abbiamo integrato per parti nel terzo passaggio, sostituito le~\eqref{varq} e~\eqref{varp} nel terzo, applicato la definizione di parentesi di Poisson~\eqref{poisson} e nell'ultimo passaggio la definizione di derivata del vincolo. Sviluppiamo ulteriormente i calcoli, 
    \begin{equation*}
    \begin{aligned}
        \delta S & = \int dt ~ \Big (- \dot \epsilon^a \phi_a + \epsilon^a \poi{H_0}{\phi_a} + \delta \lambda^a \phi_a + \lambda^a \epsilon^b \poi{\phi_a}{\phi_b} + \dv{}{t} (q^i p_i + \epsilon^a \phi_a) \Big ) \\ & = \int dt ~ \Big (- \dot \epsilon^a \phi_a - \epsilon^a  C_a^{\phantom a b} \phi_b + \delta \lambda^a \phi_a + \lambda^a \epsilon^b C_{ab}^{\phantom{ab} c} \phi_c + \dv{}{t} (q^i p_i + \epsilon^a \phi_a) \Big ) \\ & = \int dt ~ \Big ((- \dot \epsilon^c - \epsilon^a  C_a^{\phantom a c} + \delta \lambda^c + \lambda^a \epsilon^b C_{ab}^{\phantom{ab} c} )\phi_c + \dv{}{t} (q^i p_i + \epsilon^a \phi_a) \Big ) 
    \end{aligned}
    \end{equation*}
    dove abbiamo nuovamente per parti nel primo passaggio, nel secondo abbiamo usato l'ipotesi che i vincoli sono di seconda classe~\eqref{secondaclasse} e ridefinito gli indici. Usando infine la~\eqref{gaugee3}, otteniamo
    \begin{equation*}
       \delta S = \int dt ~ \dv{}{t} (q^i p_i + \epsilon^a \phi_a)
    \end{equation*}
    che essendo un termine al bordo, non influenza le equazioni del moto, chiudendo la dimostrazione, mostrando che è in effetti una trasformazione di simmetria di gauge.
    \end{proof}

    Un'altra caratteristica di una simmetria di gauge è che la carica di Noether associata ad essa è nulla.

    \hfill

    Per comprendere al meglio le nozioni annunciate nei paragrafi precedenti, proponiamo due esempi esplicatori: la particella descritta nell'ambito della relatività speciale e l'elettrodinamica in assenza di sorgenti. 

\section{La particella relativistica}

    Consideriamo un sistema fisico composto da una particella relativistica, descritta dalla linea di universo nello spazio di Minkovski $x^\mu(\tau)$. Il parametro $\tau$ non deve necessariamente essere il tempo proprio. In unità naturali, la costante della luce è unitaria. L'azione associata a tale sistema è
    \begin{equation} \label{azionerel}
        S = - m \int ds = - m \int d\tau ~ \sqrt{- \dv{x^\mu}{\tau}\dv{x_\mu}{\tau}} = - m \int d\tau ~ \sqrt{- \dot x^2}
    \end{equation}
    
    Cambiando il parametro $\tau$ non ha alcun risvolto fisico, poiché la linea di universo non cambia. Tale trasformazione è chiamata riparametrizzazione
    \begin{equation*}
        \tau' = \tau'(t) \qquad x'^\mu (\tau') = x^\mu(\tau)
    \end{equation*} 
    e la sua versione infinitesima, prendendo $\tau' = \tau + \epsilon(\tau)$, diventa
    \begin{equation*}
        \delta x^\mu(\tau) = - \epsilon(\tau) \dot x^\mu
    \end{equation*}
    Un simile risultato lo abbiamo già ottenuto nella~\eqref{simmtempo}.
    
    Calcoliamo ora la variazione dell'azione 
    \begin{equation*}
        \delta S = - m ~ \delta \int d\tau ~ \sqrt{- \dot x^2} = \int d\tau ~ \frac{- \dot x^\mu \delta \dot x_\mu}{\sqrt{- \dot x^2}} = - m \int d\tau ~\dv{}{\tau} \Big ( \epsilon(\tau) \sqrt{- \dot x^2} \Big)
    \end{equation*}
    e dunque il termine al bordo è 
    \begin{equation*}
        K = - m \epsilon(\tau) \sqrt{- \dot x^2}
    \end{equation*}

    Calcoliamo ora attraverso il teorema di Noether, la carica associata~\eqref{carica}
    \begin{equation*}
    \begin{aligned}
        Q & = K - \pdv{L}{\dot x^\mu} \delta_s x^\mu \\ & =  - m \epsilon(\tau) \sqrt{- \dot x^2} - \pdv{}{\dot x^\mu} (- m \sqrt{- \dot x^2} ) (- \epsilon(\tau) \dot x^\mu) \\ & = - m \epsilon(\tau) \sqrt{- \dot x^2} + m \epsilon(\tau) \frac{- \dot x^2}{\sqrt{- \dot x^2}} \\ & = - m \epsilon(\tau) \sqrt{- \dot x^2} + m \epsilon(\tau) \sqrt{- \dot x^2} = 0
    \end{aligned}
    \end{equation*}
    Dunque risulta che la carica associata è nulla, dato che è una simmetria di gauge. 

    Passiamo al formalismo hamiltoniano, scriviamo dapprima il momento canonico 
\begin{equation*}
    p_\nu = \pdv{L}{\dot x^\mu} = \frac{m \dot x_\nu}{\sqrt{- \dot x^\mu \dot x_\mu}}
\end{equation*}
    Tuttavia notiamo che non siamo in grado di risolvere $\dot x^\mu$ in termini di $p_\mu$. Ciò è dovuto al fatto che non ci sono come sembrerebbe 4 equazioni indipendenti, ma soltanto 3 a causa del vincolo 
\begin{equation*}
    \phi = p_\nu p^\nu + m^2 = 0
\end{equation*}

\subsection{L'azione di Polyakov}
    A questo punto entrerebbe in gioco il metodo di Dirac per risolvere questa problematica. Noi ci soffermeremo invece sulla definizione di un'altra azione del tutto equivalente, dovuta a Polyakov. Introduciamo una variabile ausiliaria chiamata einbein $e(\tau)$ che avrà il ruolo di moltiplicatore di Lagrange. La nuova azione è
\begin{equation} \label{poly}
    S[x^\mu(\tau), ~ e(\tau)] = \frac{1}{2} \int d\tau ~ \Big( \frac{1}{e} \dot x^\mu \dot x_\mu - e m^2 \Big)
\end{equation}
    Siccome la lagrangiana non contiene alcuna derivata di $e$, l'equazione del moto corrispondente ad $e(\tau)$ 
\begin{equation*}
    \pdv{L}{\dot e} = 0
\end{equation*}
    si riduce a
\begin{equation*}
    e(x^\mu) = \frac{1}{m} \sqrt{-\dot x^2}
\end{equation*}
    che sostituendole alla~\eqref{poly} ritroviamo~\eqref{azionerel}. Abbiamo appena dimostrato che le due azioni sono equivalenti. 

    Anche l'azione di Polyakov è invariante per riparametrizzazione, e in aggiunta alla variazione di $x$ troviamo come varia $e$
\begin{equation*}
    e'(\tau') = e(\tau) \dv{\tau}{\tau'}
\end{equation*}
    e la controparte infinitesima $e'(\tau + \epsilon) = e(1 - \dot \epsilon)$ conduce alla variazione
\begin{equation*}
    \delta e(\tau) = - \dv{}{\tau} (\epsilon(\tau) e)
\end{equation*}
    Ricapitolando, le variazione dei campi $x^\mu$ e $e$ in seguito ad una riparametrizzazione di $\tau$ porta a 
\begin{equation} \label{varcamp}
    \delta x^\mu(\tau) = - \epsilon(\tau) \dot x^\mu \qquad \delta e(\tau) = - \dv{}{\tau} (\epsilon(\tau) e)
\end{equation}

    Calcoliamo ora la variazione dell'azione con l'aiuto della~\eqref{varcamp}
\begin{equation*}
\begin{aligned}
    \delta S & = \frac{1}{2} ~ \delta \int d\tau ~ \Big( \frac{1}{e} \dot x^\mu \dot x_\mu - e m^2 \Big) \\ & = \frac{1}{2} \int d\tau ~ \Big ( \frac{2 e \dot x^\mu \delta x_\mu - \dot x^2 \delta e}{e^2} - m^2 \delta e \Big) \\ & = -\frac{1}{2} \int d\tau ~ \dv{}{\tau} \Big(\epsilon \Big (\frac{\dot x^2}{e} - m^2 e \Big )\Big)
\end{aligned}
\end{equation*}
    e troviamo dunque il termine al bordo  
\begin{equation*}
    K = - \frac{1}{2} \epsilon(\tau) \Big (\frac{\dot x^2}{e} - m^2\Big ) = - \epsilon(\tau) L
\end{equation*}
    Anche in questo caso la carica associata alla simmetria di gauge è nulla.

    \hfill

    Passiamo alla descrizione hamiltoniana associata, utilizzando la trasformata di Legendre 
\begin{equation*}
    H(p_\mu, ~x^\mu, ~e) = p_\mu x^\mu - L = \frac{1}{2} e (p_\mu p^\mu + m^2)
\end{equation*}
    e dunque l'azione hamiltoniana diventa 
\begin{equation*}
    S[p_\mu, ~x^\mu, ~e] = \int d\tau ~ \Big(p_\mu \dot x^\mu - \frac{1}{2} e (p_\mu p^\mu + m^2) \Big)
\end{equation*}
    Osserviamo che ha la stessa identica forma di~\eqref{azionevincolo}, con $H_0 = 0$ dato che per riparametrizzazioni l'hamiltoniana non è collegata ad alcun tempo. Dunque il vincolo è proprio $\phi = \frac{1}{2} (p^2 + m^2)$ ed $e$ sono i moltiplicatori di Lagrange. 

    Siccome è presente un solo vincolo, sarà di prima classe e $C_{ab} = 0$. Quindi la trasformazione di gauge è
\begin{equation*}
    \delta x^\mu = [x^\mu, ~\epsilon(\tau) \frac{1}{2} (p^2 + m^2)] = \epsilon(\tau) p^\mu
\end{equation*}
\begin{equation*}
    \delta p_\mu = [p_\mu, ~\epsilon(\tau) \frac{1}{2} (p^2 + m^2)] = 0
\end{equation*}
\begin{equation*}
    \delta e = \dot \epsilon(\tau)
\end{equation*}
    e quindi la variazione dell'azione è davvero un termine al bordo
\begin{equation*}
    \delta S = \int d\tau ~ \Big( p_\mu \delta \dot x^\mu - \frac{1}{2} \delta e (p^2 + m^2)\Big) = \int d\tau ~ \dv{}{\tau} \Big (\frac{1}{2} \epsilon(\tau) (p^2 - m^2) \Big)
\end{equation*}

\section{Elettrodinamica} 

    Consideriamo l'azione associata all'elettrodinamica di Maxwell
    \begin{equation*}
        S = - \frac{1}{4} \int d^4 x ~ F^{\mu\nu} F_{\mu\nu}
    \end{equation*}
    e la separiamo in componenti spaziali e temporali\footnote{Ricordiamo che gli indici derivanti dall'alfabeto greco corrispondono a $\mu = 0, ~1, ~2, ~3$, mentre quelli derivanti dall'alfabeto latino corrispondono a $i = 1, ~2, ~3$}
    \begin{equation*}
    \begin{aligned}
        S & = \int d^4 x ~ \Big ( -\frac{1}{4} F^{0i}F_{0i} - \frac{1}{4} F^{i0}F_{i0} - \frac{1}{4} F^{ij} F{ij} \Big) \\ & = \int d^4 x ~ \Big ( -\frac{1}{2} F^{0i}F_{0i} - \frac{1}{4} F^{ij} F{ij} \Big) \\ & = \int d^4 x ~ \Big( \frac{1}{2} (\dot A_i - \partial_i A_0) (\dot A^i - \partial^i A_0) - \frac{1}{4} F^{ij} F{ij} \Big ) \\ & = \int d^4 x ~ \Big (\frac{1}{2} \dot A_i \dot A^i - \dot A_i \partial^i A_0 + \frac{1}{2} \partial_i A_0 \partial^i A_0 - \frac{1}{4} F^{ij} F{ij} \Big )
    \end{aligned}
    \end{equation*}
    dove abbiamo usato la proprietà di antisimmetria del tensore elettromagnetico $F^{\mu\nu}$ e la sua definizione in termini del quadripotenziale $A_\mu$.

    Definiamo dunque il momento associato alle componenti spaziali $A_i$, essendo le uniche che presentano derivate temporali
    \begin{equation} \label{momele}
        \pi_i = \pdv{\mathcal L}{\dot A_i} = \dot A_i - \partial_i A_0
    \end{equation}
    ed usando la definizione di quadripotenziale $A^\mu = (\phi, ~ \mathbf A)$ e l'espressione del campo elettrico in termini dei potenziali scalare e vettore, mostriamo che lo possiamo intepretare fisicamente come il campo elettrico
    \begin{equation*}
        \mathbf \pi = - \pdv{\mathbf A}{t} - \nabla \phi = \mathbf E
    \end{equation*}
    
    Attraverso una trasformazione di Legendre~\eqref{hamiltoniana} troviamo l'hamiltoniana  
    \begin{equation*}
    \begin{aligned}
        H(p, ~A) & = \pi_i \dot A^i - L = \pi_i \dot A^i - \frac{1}{2} \dot A_i \dot A^i + \dot A_i \partial^i A_0 - \frac{1}{2} \partial_i A_0 \partial^i A_0 + \frac{1}{4} F^{ij} F{ij}  \\ & = \pi_i (\pi^i + \partial^i A_0) - \frac{1}{2} (\pi^i + \partial^i A_0) (\pi_i + \partial_i A_0) + (\pi_i + \partial_i A_0)\partial^i A_0 \\ & \qquad - \frac{1}{2} \partial_i A_0 \partial^i A_0 + \frac{1}{4} F^{ij} F{ij} \\ & = \pi_i \pi^i + \pi_i \partial^i A_0 - \frac{1}{2}
        \pi^i \pi_i - \pi_i \partial^i A_0 - \frac{1}{2} \partial^i A_0 \partial_i A_0 \\ & \qquad + \pi_i \partial^i A_0 + \partial_i A_0 \partial^i A_0 - \frac{1}{2} \partial_i A_0 \partial^i A_0 + \frac{1}{4} F^{ij} F{ij} \\ & = \frac{1}{2} \pi^i \pi_i + \frac{1}{4} F^{ij} F{ij} + \pi_i \partial^i A_0 \\ & = \frac{1}{2} \pi_i \pi^i + \frac{1}{4} F_{ij} F^{ij} - A_0 \partial_i \pi^i + \partial^i (\pi_i A_0)
    \end{aligned}
    \end{equation*}
    dove abbiamo sostituito l'inverso della~\eqref{momele} e integrato per parti. L'ultimo termine, essendo un termine al bordo, può essere trascurato. L'azione nella descrizione hamiltoniana per l'elettrodinamica diventa
\begin{equation*}
    S[A_i, ~\pi_i, ~A_0] = \int d^4 x ~ \Big ( \pi_i \dot A^i - \Big ( \frac{1}{2} \pi_i \pi^i + \frac{1}{4} F_{ij} F^{ij} \Big) + A_0 \partial_i \pi^i \Big)
\end{equation*}
    che si presenta nella forma cercata~\eqref{azionevincolo}, in cui definiamo $H_0$ come l'energia corrispondente all'hamiltoniana dinamica 
\begin{equation*}
    H_0 = \frac{1}{2} \pi_i \pi^i + \frac{1}{4} F_{ij} F^{ij} = \frac{1}{2} (E^2 + B^2)
\end{equation*}
    e $A_0$ come il moltiplicatore di Lagrange, il cui vincolo è esattamente una delle equazioni di Maxwell, ovvero l'equazione di Gauss
\begin{equation*}
    \phi = \partial_i \pi^i = \nabla \cdot \mathbf E = 0
\end{equation*}

    Calcolando le corrispondenti equazioni del moto~\eqref{hamvinc1},~\eqref{hamvinc2} e~\eqref{hamvinc3} 
\begin{equation*}
    \dot \pi_i = - \pdv{H}{A^i} \qquad \dot A_i = \pdv{H}{\pi^i} \qquad \phi = 0
\end{equation*}
    ci troviamo nello stesso problema di dover garantire che il vincolo sia preservato durante l'evoluzione, cioè che $\dot \phi = \poi{\phi}{H} = 0$. Una ulteriore difficoltà sorge essendo nell'ambito della teoria dei campi, dove le parentesi di Poisson sono definite per due punti differenti dello spazio $x$ e $x'$
\begin{equation*}
    \poi{\phi(x)}{H(x')} = \Big[\partial_i \pi^i (x), ~ \frac{1}{2} \Big( \pi_i \pi^i (x') + F_{ij} F^{ij}(x') - \frac{1}{2} A_0 \partial_i \pi^i (x') \Big) \Big]
\end{equation*}
    Utilizzando le parentesi di Poisson canoniche derivanti dalla struttura simplettica, dove essendo in presenza di sistemi continui l'usuale delta di Kronecker è sostituita con la delta di Dirac,
\begin{equation*}
    \poi{\pi_i(x)}{\pi_j(x')} = 0 \qquad [A_i(x), ~A_j(x')] = 0 \qquad [A_i(x), ~\pi_j(x')] = \delta_{ij} \delta^3(x - x')
\end{equation*}
    otteniamo 
\begin{equation*}
    \dv{}{t} \phi = [\phi(x), ~H(x')] = 2 \partial_k \partial'_i [\pi_k(x), ~A_j(x')] F^{ij(x')} = -2(\partial_k \partial'_i \delta^3(x - x')) F^{ij} (x') = 0
\end{equation*}
    a causa della cancellazione dei termini dovuta alla simmetria delle derivate parziali e all'antisimmetria del tensore elettromagnetico. Conseguentemente quindi il vincolo si conserva sulle equazioni del moto e genera una simmetria di gauge.
    
    A questo punto mostriamo il gauge, attraverso il seguente espediente: consideriamo come vincolo un funzionale $\Phi[\Lambda(x)]$ dipendente dalla funzione di gauge $\Lambda(x)$ definito come
\begin{equation*}
    \Phi[\Lambda(x)] = \int d^3 x ~ \Lambda(x) \partial_i \pi^i(x)
\end{equation*}
    dove non abbiamo integrato nel tempo per mantenere la dipendenza temporale al fine di poter calcolare parentesi di Poisson allo stesso istante. La trasformazione di gauge sono definite come 
\begin{equation*}
\begin{aligned}
    \delta A_i (x) & = [A_i(t, x), ~\Phi[\Lambda]] \\ & = \int d^3 x' ~ \Lambda(x', ~t) \partial'_j [A_i(t, ~x), ~ \pi^j(t, ~x')] \\ & = \int d^3 x' ~ \Lambda(t, ~x') \delta_{ji} \delta^3 (x-x') \\ & = - \partial_i \Lambda(t, ~x)
\end{aligned}
\end{equation*}
    che è proprio l'invarianza di gauge presente in elettrodinamica~\eqref{gaugeelettro}. D'altra parte la trasformazione di gauge del momento coniugato deve essere nulla 
\begin{equation*}
    \delta \pi_i(t, ~x) = [\pi_i, ~\Phi[\Lambda]] = \int d^3 x' ~ \Lambda(t, ~x')[\pi_i, ~\partial_j \pi^j] = 0
\end{equation*}
    Infine calcoliamo la variazione dell'azione 
\begin{equation*}
    \delta S = \delta \int d^4 x ~ ( \pi_i \dot A^i - H_0 + A_0 \partial_i \pi^i) = \int d^4 x ~ ( \delta \pi_i \dot A^i + \pi_i \delta \dot A^i - \delta H_0 + \delta A_0 \phi + A_0 \delta \phi )
\end{equation*}
    Essendo invarianti di gauge, poniamo $\delta \pi_i = 0$, $\delta H_0 = 0$ e $\delta \phi = 0$ e sviluppiamo ulteriormente i calcoli
\begin{equation*}
\begin{aligned}
    \delta S & = \int d^4 x ~ (\pi_i \delta \dot A^i + \delta A_0 \phi) \\ & = \int d^4 (-\pi_i\partial^i \partial_0 \Lambda + \delta A_0 \partial_i \pi^i ) \\ & = \int d^4 x (\partial_i (-\pi^i \partial_o \Lambda) + \partial_i \pi^i \partial_0 \Lambda + \delta A_0 \partial_i \pi^i ) \\ & = \int d^4 x (\partial_\mu K^\mu + \partial_i \pi^i (\delta A_0 + \partial_9 \Lambda))
\end{aligned}
\end{equation*}
    dove abbiamo posto il termine al bordo $K^\mu = (0, ~- \pi^i \partial_0 \Lambda)$. L'annullarsi della variazione dell'azione, ci porta a concludere che $\delta A_0 = - \partial_0 \Lambda$, per avere soltanto un termine al bordo che ai fini delle equazioni del moto è trascurabile. In questo modo, la trasformazione di gauge del campo $A_\mu$ generata dal vincolo dell'equazione di Gauss è proprio quella aspettata: infatti abbiamo dimostrato che
\begin{equation*}
    \phi = \partial_i \pi^i = \nabla \cdot E = 0
\end{equation*}
    genera 
\begin{equation*}
    \delta A_\mu = - \partial_\mu \Lambda(x)
\end{equation*}