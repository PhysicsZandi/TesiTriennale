\chapter{Secondo teorema e teorie di gauge}

    Nei primi due capitoli, abbiamo studiato sistemi fisici che presentano simmetrie riconducibili a trasformazioni applicate a tutti i punti dello spaziotempo, dette simmetrie globali. Tuttavia in fisica moderna sono presenti sistemi che possiedono simmetrie locali, ovvero trasformazioni che sono funzioni dei punti dello spaziotempo. Esse non associano quantità conservate che soddisfano leggi di conservazione alle simmetrie ma un sistema di equazioni differenziali. Ciò significa che il sistema ha un numero di gradi di libertà sovrastimato, perchè ci sono relazioni che legando le variabili attraverso equazioni, ne abbassano il numero potendone ricavarne una in funzione dell'altra. In letteratura, questo teorema viene chiamato secondo teorema di Noether perchè presente nel già citato suo articolo del 1916. In questo capitolo, non enunceremo né dimostreremo questo teorema, ma studieremo una conseguenza: le teorie di gauge. L'importanza di tali teorie risiede nello sviluppo della fisica moderna: tre delle quattro interazioni fondamentali (forte, debole ed elettromagnetica) sono descritte da teorie di gauge. I riferimenti bibliografici sono \cite{barone} \cite{banados}.

\section{Teoria di gauge}
    Prima di studiare in dettaglio la struttura matematica di una teoria di gauge, presentiamo brevemente quali caratteristiche possiede. Consideriamo un sistema fisico descritto da una lagrangiana nell'ambito della teoria dei campi. Definiamo una simmetria di gauge come una simmetria locale, ovvero una trasformazione contenente un'arbitraria funzione delle coordinate che lascia invariata l'azione del sistema. Ne consegue che ci saranno relazioni che legano le equazioni del moto: fissando arbitrariamente la nostra funzione, i gradi di libertà del sistema sono diminuiti. Inoltre, passando alla descrizione hamiltoniana, emerge la presenza di vincoli generati dalla simmetria a differenza delle cariche conservate nel caso di simmetrie globali. Particolare attenzione è posta su quanto possa essere fisica una teoria formulata in questo modo, che a prima vista potrebbe sembrarlo per nulla. Tuttavia possiamo introdurre il concetto di classi di equivalenza di configurazioni, ovvero campi che differiscono matematicamente solamente per una simmetria di gauge, descrivono fisicamente la stessa realtà. 
    
\subsection{Un esempio motivato dall'elettrodinamica}

    Il primo esempio che si trova è quello dell'elettrodinamica: dato un quadripotenziale $A_\mu$ che soddisfa le equazioni di Maxwell, è possibile dimostrare che anche con l'aggiunta di una simmetria di gauge le continua a soddisfare. Infatti prendendo $A'_\mu = A_\mu + \partial_\mu \Lambda$ e mettendolo nelle equazioni di Maxwell nel vuoto
    \begin{equation*}
        \Box A_\mu - \partial_\mu (\partial^\nu A_\nu) = 0
    \end{equation*}
    otteniamo 
    \begin{equation*}
        \Box A'_\mu - \partial_\mu (\partial^\nu A'_\nu) = \Box A_\mu - \partial_\mu (\partial^\nu A_\nu) + \partial^\nu \partial_\mu \partial^\mu \Lambda - \partial^\nu \partial_\mu \partial^\mu \Lambda = \Box A_\mu - \partial_\mu (\partial^\nu A_\nu) = 0
    \end{equation*}

    Prendiamo dunque in considerazione un esempio motivato dalla teoria appena descritta, considerando un funzionale di azione dipendente da due campi $A$ e $\psi$ definito nel seguente modo
    \begin{equation*}
        S[A(t),\psi(t)] = \frac{1}{2} \int dt ~ (\dot \psi - A)^2
    \end{equation*}
    Mostriamo ora quantitativamente tutte le proprietà di una teoria di gauge. 
    
    Innanzitutto vediamo che è presente una simmetria di gauge: l'azione è invariante sotto la trasformazione
    \begin{equation}\label{arbitraria}
        \psi' = \psi + \epsilon(t) \qquad A' = A + \dot \epsilon(t)
    \end{equation}
    dove $\epsilon(t)$ è un'arbitraria funzione del tempo. Infatti calcolando la variazione sotto tale trasformazione otteniamo 
    \begin{equation*}
    \begin{aligned}
        S[A'(t),\psi'(t)] & = \frac{1}{2} \int dt ~ (\dot \psi' - A')^2 = \frac{1}{2} \int dt ~ (\dot \psi + \dot \epsilon - A - \dot \epsilon )^2 \\ & = \frac{1}{2} \int dt ~ (\dot \psi - A)^2 = S[A(t),\psi(t)]
    \end{aligned}
    \end{equation*}
    
    Calcolando ora le equazioni del moto per i due campi, utilizzando le equazioni di Eulero-Lagrange \eqref{eullagcampi}. Per il campo $A$ otteniamo 
    \begin{equation*}
        0 = \pdv{\mathcal L}{A} - \dv{}{t} \pdv{\mathcal L}{\dot A} = - 2 (\dot \psi - A)
    \end{equation*}
    che porta alla prima equazione del moto 
    \begin{equation} \label{moto1}
        \dot \psi - A = 0
    \end{equation}
    invece per il campo $\psi$ troviamo
    \begin{equation*}
        0 = \pdv{\mathcal L}{A} - \dv{}{t} \pdv{\mathcal L}{\dot \psi} = 2 \dv{}{t} (\dot \psi - A)
    \end{equation*}
    che conduce alla seconda equazione del moto 
    \begin{equation} \label{moto2}
        \dv{}{t} (\dot \psi - A) = 0
    \end{equation}
    Confrontando le equazioni del moto, notiamo che non sono indipendenti: l'equazione per il campo $A$ \eqref{moto1} contiene già l'equazione per $\psi$ \eqref{moto2} (se una funzione è nulla, anche la derivata). Non ci sono quindi due equazioni, ne è presente soltanto una. Scriviamo esplicitamente la soluzione delle equazioni del moto 
\begin{equation*}
    \psi(t) = f(t) \qquad A(t) = \dot f(t)
\end{equation*}
    e mostriamo che applicando la simmetria di gauge, attraverso la funzione $\epsilon(t)$ introdotta in \eqref{arbitraria}
\begin{equation*}
    \psi'(t) = \psi(t) + \epsilon(t) = f(t) + \epsilon(t) \qquad A'(t) = A(t) + \dot \epsilon(t) = \dot f(t) + \dot \epsilon(t)
\end{equation*}
    le soluzioni soddisfano ancora le equazioni del moto 
\begin{equation*}
    \dot \psi' - A' = 0 = \dot \psi(t) + \dot \epsilon(t) - A - \dot \epsilon(t) = \dot \psi - A = 0
\end{equation*}
    Come conseguenza, abbiamo arbitrarie condizioni condizioni iniziali sono date: è sempre possibile modificare la soluzione delle equazioni del moto attraverso la simmetria di gauge. 

    Infine, passiamo dal formalismo lagrangiano alla descrizione hamiltoniano. Per trovare l'hamiltoniana del sistema, definiamo prima il momento coniugato al campo $\psi$
\begin{equation*}
    p_\psi = \pdv{L}{\dot \psi} = \dot \psi - A
\end{equation*}
    e quello coniugato al campo $A$
\begin{equation*}
    p_A = \pdv{L}{\dot A} = 0
\end{equation*}
    e successivamente applichiamo una trasformazione di Legendre \eqref{hamiltoniana}
\begin{equation*}
\begin{aligned}
    H(p_\psi,~\psi,~A) & = p_\psi \dot \psi - L = p_\psi \dot \psi - \frac{1}{2} (\dot \psi - A)^2 \\ & = p_\psi (p_\psi + A) - \frac{1}{2} (p_\psi + A - A)^2 = \frac{1}{2} p^2_\psi  + A p_\psi
\end{aligned}
\end{equation*}
    Applichiamo ora le equazioni di Hamilton \eqref{ham1} e \eqref{ham2} per trovare le equazioni del moto. Per il campo $A$ abbiamo
\begin{equation} \label{Ah}
    \dot p_A = - \pdv{H}{A} = - p_\psi = 0 
\end{equation}
    mentre per il campo $\psi$
\begin{equation} \label{psih}
    \dot \psi = \pdv{H}{p_\psi} = p_\psi + A \qquad \dot p_\psi = - \pdv{H}{\psi} = 0
\end{equation}
    Notiamo nella prima equazione l'assenza di derivate temporali, non è un'equazione di evoluzione temporale ma un vincolo $p_\psi = 0$ dove $A$ è il moltiplicatore di Lagrange corrispondente. Inoltre osserviamo che anche in questo caso le equazioni del moto non sono indipendenti, infatti l'ultima è la derivata temporale della prima. Abbiamo nuovamente meno equazioni che variabili.

    Concludiamo il paragrafo con una nota importante. Dato che le equazioni del moto per $\psi$ sono contenute in quelle per $A$, \eqref{moto2} in \eqref{moto1} oppure \eqref{psih} in \eqref{Ah}, non posso scegliere arbitrariamente $A = 0$ ma posso soltanto imporre $\psi = 0$. Lo si può dimostrare, vedendo che se si pone $\psi = 0$ 
\begin{equation*}
    S[\psi = 0,~A] = \int dt A^2
\end{equation*}
    che porta all'equazione di un vincolo
\begin{equation*}
    A = 0
\end{equation*}
    mentre la scelta di $A = 0$ 
\begin{equation*}
    S[\psi,~A]=0 = \int dt \dot \psi^2
\end{equation*}
    conduce ad un equazione del moto 
\begin{equation*}
    \ddot \psi = 0
\end{equation*}
    non accettabile, essendo una evoluzione temporale e non un vincolo che deve emergere dalla simmetria di gauge. 

\subsection{Struttura generale delle teorie di gauge} 

    Il vantaggio di studiare la formulazione hamiltoniana risiede nella possibilità di studiare in generale caratteristiche valide per molti esempi: dalla particella relativistica all'elettrodinamica. Il metodo più generale per passare dalla descrizione lagranginana a quella hamiltoniana è il metodo di Dirac. Tuttavia non sceglieremo quel percorso, ma partiremo direttamente dalla conoscenza dell'hamiltoniana e dal suo funzionale di azione che ha la forma generica
\begin{equation} \label{azionevincolo}
    S[p_i,~q^i,~\lambda^a] = \int dt (p_i \dot q^i - H_0(p_i, ~q^i) + \lambda^a \phi_a (p, ~q))
\end{equation}
    dove le variabili indipendenti dei campi sono $p_i, q^i e \lambda^a$, rispettivamente i momenti coniugati, le coordinate e i moltiplicatori di Lagrange. L'hamiltoniana totale è definita come $H = H_0 - \lambda^a \phi_a$ ed è composta da una parte che non contiene vincoli $H_0$ e da una parte che contribuisce ad aggiungerli, dove le equazioni dei vincoli sono date dalle $\phi_a$. Le corrispondenti equazioni del moto, variando le tre variabili, sono
\begin{equation}
    \dot q^i = \pdv{H_0}{p_i} - \lambda^a \pdv{\phi_a}{p_i}
\end{equation}
\begin{equation}
    \dot p^i = - \pdv{H_0}{q^i} + \lambda^a \pdv{\phi_a}{q^i}
\end{equation}
\begin{equation}
    \phi_a(p, ~q) = 0
\end{equation}
    Le prime due equazioni determinano la dinamica del sistema, come evolvono $p$ e $q$ date le condizioni iniziali $p_0$, $q_0$. Tuttivia le condizioni iniziali non possono essere del tutto arbitrarie ma devono soddisfare la terza equazione. Inoltre un altro dubbio sorge: come scelgo le $\lambda$? Con l'ultilizzo delle parentesi di Poisson, calcoliamo la derivata temporale dei vincoli 
\begin{equation}
    \dv{}{t} \phi_a(p, ~q) = [\phi_a, ~H_0] - [\phi_a, ~\phi_b] \lambda^b
\end{equation}
    e la poniamo debolmente nulla, ovvero nulla soltanto quando $\phi_a = 0$
\begin{equation*} \label{vincoli}
    [\phi_a, ~H_0] - [\phi_a, ~\phi_b] \lambda^b \approx 0
\end{equation*}
    Distinguiamo due possibili casi, chiamando la matrice $C_{ab} = [\phi_a, ~\phi_b]$
\begin{enumerate}
    \item Teorie non di gauge: se la matrice $C_{ab}$ è invertibile, i moltiplicatori di Lagrange $\lambda^a$ sono fissati dall'equazione \eqref{vincoli}
\begin{equation}
    \lambda^a (t) = C^{ab} [\phi_b, ~H_0]
\end{equation}
    In questo caso il loro ruolo è di mantenere il vincolo nullo $\phi = 0$ nel tempo. Questi vincoli sono chiamati di seconda classe e la dinamica del sistema è semplice: date le condizioni iniziali in modo che rispettino i vincoli, le $\lambda$ assicurano che il vicnolo sia mantenuto negli istanti successivi.
    \item Teoria di gauge: se la matrice è debolmente nulla $C_{ab} \approx 0$, allora i moltiplicatori di Lagrange non impongono alcuna condizione e le equazioni del moto rimangono non determinate. Definiamo vincoli di prima classe se soddisfano 
\begin{equation} \label{secondaclasse}
    [\phi_a, ~ H_0] = C_a^{\phantom a b} \phi_b \approx 0 \qquad   [\phi_a, ~ \phi_b] = C_{ab}^{\phantom{ab} c} \phi_c \approx 0
\end{equation}
    Una conseguenza dei vincoli di prima classe, è che l'azione è invariante per trasformazioni di gauge 
\begin{equation}
    \delta q^i = [q^i, ~\phi_a] \epsilon^a (t)
\end{equation}
\begin{equation}
    \delta p_i = [p_i, ~\phi_a] \epsilon^a (t)
\end{equation}
\begin{equation}
    \delta \lambda^c = \dot \epsilon^c (t) + \epsilon^a(t) C_a^{\phantom a c} - \lambda^a \epsilon^b(t) C_{ab}^{\phantom{ab} c}
\end{equation}
    dove $\epsilon^a(t)$ è un'arbitraria funzione del tempo. 
\end{enumerate}

    Abbiamo quindi dimostrato l'inverso del teorema di Noether: un'azione del tipo \eqref{azionevincolo} tale che i vincoli e l'hamiltoniana soddisfano \eqref{secondaclasse}, presentano delle simmetrie di gauge Dimostriamo ora questo enunciato. 
\begin{proof}
    La variazione delle variabili canoniche è data da 
\begin{equation}
    \delta q_i = \epsilon^a \pdv{\phi_a}{p_i} \qquad \delta p_i = - \epsilon^a \pdv{\phi_a}{q^i}
\end{equation}
    dunque calcoliamo la variazione dell'azione
\begin{equation}
\begin{gathered}
    \delta S = \delta \int dt ~ (q^i p_i - H_0(q^i, p_i) + \lambda^a \phi_a(p, q)) = \\
    = dt ~ (-\epsilon^a \pdv{\phi_a}{q^i} \dot q^i - \dot p^i \epsilon^a \pdv{\phi_a}{p_i} - \pdv{H_0}{q^i} \epsilon \pdv{\phi_a}{p_i} + \pdv{H_0}{p^i} \epsilon^a \pdv{\phi_a}{q_i} \\ - \delta \lambda^a \phi_a - \lambda^a (\pdv{\phi_a}{q^j} \epsilon^b \pdv{\phi_b}{p_j} - \pdv{\phi^a}{p_j}\epsilon^b \pdv{\phi_b}{q^j}) + B) \\ 
    = \int dt ~(-\epsilon^a \frac{d}{dt} \phi_a - \epsilon^a [H_0, \phi_a] - \delta \lambda^a \phi_a - \lambda^a \epsilon^b [\phi_a, \phi_b] + B)
\end{gathered}
\end{equation}
    dove $B$ è un termine al contorno. Usando \eqref{secondaclasse} e ridefinendo $B$, otteniamo
\begin{equation}
\begin{gathered}
    \delta S = \int dt ~ (\dot \epsilon^a \phi_a + \epsilon^a C_a^{\phantom{a}b} \phi_b - \delta \lambda^a \phi_a - \lambda^a \epsilon^b C_{ab}^{\phantom{ab}c} \phi_c) + B \\ 
    = \int dt ~ (\dot \epsilon^c + \epsilon^a C_a^{\phantom{c}} - \delta \lambda^c - \lambda^a \epsilon^b C_{ab}^{\phantom{ab}c})\phi_c + B = B
\end{gathered}
\end{equation}
\end{proof}

    Calcolando la carica di Noether associata ad una simmetria di gauge, scopre che la carica è nulla. Inoltre bisogna fare attenzione ai gradi di libertà. Teorie non di gauge con il funzionale di azione del tipo 
\begin{equation*}
    S = \int dt (p_i q^i - H_0)
\end{equation*}
    con $i=1,~2, ldots, N$, possiedono 2N costanti di integrazione per risolvere le equazioni di Hamilton, dunque i gradi di libertà sono $\frac{1}{2} 2N = N$. Teorie di gauge invece, con funzionale di azione
\begin{equation*}
    S = \int dt (p_i q^i - H_0 - \lambda^a \phi_a
\end{equation*}
    con $a=1,~2, ldots, g$, possiedono anch'essi 2n costanti di integrazione ma ci sono $g$ vincoli dovuti a simmetrie di gauge che implicano la non necessità di conoscerle tutte: il numero di gradi di libertà è $\frac{1}{2} (2n - 2g) = N - g$.

    Nei prossimi paragrafi, mostriamo due esempi di teorie di gauge: la particella relativistica e l'elettrodinamica libera. Applicheremo la stessa procedura, ovvero scriveremo la lagrangiana, troveremo l'hamiltoniana attraverso una trasformazione di Legendre e vedremo come i vincoli generano le simmetrie di gauge. 

\section{La particella relativistica}

    Consideriamo una particella relativistica, descritta dalla curva $x^\mu(\tau)$ con $\tau$ un parametro qualsiasi, non necessariamente il tempo proprio. La sua azione è
\begin{equation} \label{azionerel}
    S = -m \int ds = - m \int d\tau \sqrt{- \dv{x^\mu}{\tau}\dv{x^\nu}{\tau} g_{\mu\nu}}
\end{equation}
    Questa azione è invariante per riparametrizzazione, una trasformazione non fisica 
\begin{equation*}
    \tau' = \tau'(t) \qquad x'^\mu (\tau') = x^\mu(\tau)
\end{equation*} 
    Rendendo tale trasformazione infinitesima, prendendo $\tau' = \tau + \epsilon(\tau)$, otteniamo sulla falsa riga di \eqref{simmtempo}
\begin{equation*}
    \delta x^\mu(\tau) = - \epsilon(\tau) \dot x^\mu
\end{equation*}
    Calcoliamo la variazione dell'azione 
\begin{equation*}
    \delta S = - m \int d\tau \frac{- \dot x^\mu \delta \dot x_\mu}{\sqrt{- \dot x^2}} = - m \int d\tau \dv{}{\tau} \Big ( \epsilon(\tau) \sqrt{- \dot x^2} \Big)
\end{equation*}
    e dunque il termine al bordo $K$ sarà 
\begin{equation*}
    K = - m \epsilon(\tau) \sqrt{- \dot x^2}
\end{equation*}

    Calcoliamo ora attraverso il teorema di Noether, la carica associata \eqref{carica}
\begin{equation*}
    Q = K - \frac{\partial L}{\partial \dot x} \delta_s x = ?
\end{equation*}
    Dunque risulta che la carica associata è nulla, dato che è una simmetria di gauge. 

    Cercando di trovare l'hamiltoniana, scrivendo il momento canonico 
\begin{equation*}
    p_\nu = \frac{m \dot x_\nu}{\sqrt{- \dot x^\mu \dot x_\mu}}
\end{equation*}
    non ci troviamo in grado di risolvere $\dot x^\mu$ in termini di $p_\mu$. Ciò è dovuto al fatto che non ci sono come sembrerebbe 4 equazioni indipendenti, ma soltanto 3 a causa del vincolo dovuto alla conservazione dell'energia 
\begin{equation*}
    \phi = p_\nu p^\nu + m^2 = 0
\end{equation*}
    Non useremo il metodo di Dirac ma definiamo un'altra azione equivalente, dovuta a Polyakov, che ci permetterà di risolvere il problema. L'azione di cui stiamo parlando è introdotta con l'aggiunta di una variabile ausiliaria $e(\tau)$, chiamata einbein, che come vedremo in seguito corrisponde ad un moltiplicatore di Lagrange
\begin{equation} \label{poly}
    S[x^\mu(\tau), ~ e(\tau)] = \frac{1}{2} \int d\tau \Big( \frac{1}{e} \dot x^\mu \dot x_\mu - e m^2 \Big)
\end{equation}
    Siccome la lagrangiana non contiene alcuna derivata di $e$, le equazioni del moto corrispondenti ad $e(\tau)$  
\begin{equation*}
    \pdv{L}{e} = 0
\end{equation*}
    si riducono a
\begin{equation*}
    e(x^\mu) = \frac{}{m} \sqrt{-\dot x^2}
\end{equation*}
    che sostituendole alla \eqref{poly} ritroviamo \eqref{azionerel}. Perciò le due azioni sono equivalenti. 

    Anche l'azione di Polyakov è invariante per riparametrizzazione, e in aggiunta alla variazione di $x$ troviamo come varia $e$
\begin{equation*}
    e'(\tau') = e(\tau) \dv{\tau}{\tau'}
\end{equation*}
    e la controparte infinitesima $e'(\tau + \epsilon) = e(1 - \dot \epsilon)$ conduce alla variazione
\begin{equation*}
    \delta e(\tau) = - \dv{}{\tau} (\epsilon(\tau) e)
\end{equation*}
    Ricapitolando, le variazione dei campi in seguito ad una riparametrizzazione di $\tau$ porta a 
\begin{equation}
    \delta x^\mu(\tau) = - \epsilon(\tau) \dot x^\mu \qquad \delta e(\tau) = - \dv{}{\tau} (\epsilon(\tau) e)
\end{equation}
    Calcoliamo la variazione dell'azione
\begin{equation*}
    \delta S  = \frac{1}{2} \int d\tau \Big ( \frac{2 e \dot x^\mu \delta x_\mu - \dot x^2 \delta e}{e^2} - m^2 \delta e \Big) = -\frac{1}{2} \int d\tau \dv{}{\tau} \Big(\epsilon \Big (\frac{\dot x^2}{e} - m^2 e \Big )\Big)
\end{equation*}
    e dunque il termine al bordo è 
\begin{equation*}
    K = - \frac{1}{2} \epsilon(\tau) \Big (\frac{\dot x^2}{e} - m^2\Big ) = - \epsilon(\tau) L
\end{equation*}

    L'hamiltoniana associata all'azione di Polyakov è 
\begin{equation*}
    H(p_\mu, ~x^\mu, ~e) ) \frac{1}{2} e (p_\mu p^\mu + m^2)
\end{equation*}
    e dunque l'azione hamiltoniana diventa 
\begin{equation*}
    S[p_\mu, ~x^\mu, ~e] = \int d\tau \Big(p_\mu \dot x^\mu - \frac{1}{2} e (p_\mu p^\mu + m^2) \Big)
\end{equation*}
    Osserviamo che ha la stessa identica forma di \eqref{azionevincolo}, con $H_0 = 0$ dato che per riparametrizzazioni l'hamiltoniana non è collegata ad alcun tempo. Dunque il vincolo è proprio $\phi = \frac{1}{2} (p^2 = m^2)$ ed $e$ sono i moltiplicatori di Lagrange. 

    Siccome è presente un solo vincolo, sarà di prima classe e $C_{ab} = 0$. Quindi la trasformazione di gauge è
\begin{equation}
    \delta x^\mu = [x^\mu, ~\epsilon(\tau) \frac{1}{2} (p^2 + m^2)] = \epsilon(\tau) p^\mu
\end{equation}
\begin{equation}
    \delta p_\mu = [p_\mu, ~\epsilon(\tau) \frac{1}{2} (p^2 + m^2)] = 0
\end{equation}
\begin{equation}
    \delta e = \dot \epsilon(\tau)
\end{equation}
    e quindi la variazione dell'azione è davvero un termine al bordo
\begin{equation*}
    \delta S = \int d\tau \Big( p_\mu \delta \dot x^\mu - \frac{1}{2} \delta e (p^2 + m^2)\Big) = \int d\tau \dv{}{\tau} \Big (\frac{1}{2} \epsilon(\tau) (p^2 - m^2) \Big)
\end{equation*}

\section{Elettrodinamica libera} 

    Consideriamo l'azione associata alla teoria di Maxwell d la separiamo in componenti spaziali e temporali
\begin{equation*}
\begin{aligned}
    S & = -\frac{1}{4} \int d^4 x F^{\mu\nu} F_{\mu\nu} = \int d^4 x \Big ( -\frac{1}{2} F^{0i}F_{0i} - \frac{1}{4} F^{ij} F{ij} \Big) \\ & = \int d^4 x \Big( \frac{1}{2} (\dot A_i - \partial_i A_0) (\dot A^i - \partial^i A_0) - \frac{1}{4} F^{ij} F{ij} \Big ) \\ & = \int d^4 x (\frac{1}{2} \dot A_i \dot A^i - \dot A_i \partial^i A_0 + \frac{1}{2} \partial_i A_0 \partial^i A_0 - \frac{1}{4} F^{ij} F{ij})
\end{aligned}
\end{equation*}

    Definiamo dunque il momento associato alle uniche componenti del campo che presentano derivate temporali $A_i$
\begin{equation*}
    \pi_i = \pdv{\mathcal L}{\dot A_i} = \dot A_i - \partial_i A_0
\end{equation*}
    che possiamo intepretare fisicamente come il campo elettrico $p^i = E^i$. Attraverso una trasformazione di Legendre troviamo l'hamiltoniana  
\begin{equation*}
\begin{aligned}
    H(p, ~A) & = \pi_i \dot A^i - L \\ & = \pi_i (\pi^i + \partial^i A_0) - \Big(\frac{1}{2} (\pi_i + \partial_i A_0)(\pi^i + \partial^i A_0) - (\pi_i + \partial_i A_0) \partial^i A_0 \\ & ~ + \frac{1}{2} \partial_i A_0 \partial^i A_0 - \frac{1}{4} F^{ij} F{ij}\Big) \\ & = \frac{1}{2} \pi_i \pi^i + \frac{1}{4} F_{ij} F^{ij} - A_0 \partial_i \pi^i
\end{aligned}
\end{equation*}
    dove abbiamo cancellato i termini al bordo nell'ultimo passaggio. Quindi l'azione hamiltoniana per l'elettromagnetismo è
\begin{equation}
    S[A_i, \pi_i, ~A_0] = \int d^4 x \Big ( \pi_i \dot A^i - \Big ( \frac{1}{2} \pi_i \pi^i + \frac{1}{4} F_{ij} F^{ij} \Big) + A_0 \partial_i \pi^i \Big)
\end{equation}
    che si presenta nella forma cercata \eqref{azionevincolo}, dove $H_0$ è l'energia corrispondente all'hamiltoniana dinamica 
\begin{equation*}
    H_0 = \frac{1}{2} \pi_i \pi^i + \frac{1}{4} F_{ij} F^{ij} = \frac{1}{2} (E^2 + B^2)
\end{equation*}
    e $A_0$ è il moltiplicatore di Lagrange, il cui vincolo è l'equazione di Gauss
\begin{equation*}
    \phi = \partial_i \pi^i = \nabla \cdot E = 0
\end{equation*}

    Calcolando le equazioni del moto 
\begin{equation*}
    \dot \pi_i = - \pdv{H}{A^i} \qquad \dot A_i = \pdv{H}{\pi^i} \qquad \phi = 0
\end{equation*}
    ci troviamo nello stesso problema di dover garantire che il vincolo sia preservato durante l'evoluzione, cioè che $\dot \phi = [\phi, ~H] = 0$. Essendo in teoria dei campi, dobbiamo però calcolare le parentesi di Poisson in due coordinate differenti $x$ e $x'$
\begin{equation*}
    [\phi(x), ~ H(x')] = \Big[\partial_i \pi^i, ~ \frac{1}{2} \Big( \pi_i \pi^i + F_{ij} F^{ij}(x') - \frac{1}{2} A_0 \partial_i \pi^i(x') \Big) \Big]
\end{equation*}
    Utilizzando le parentesi di Poisson canoniche derivanti dalla stuttura simplettica della teoria dei campi
\begin{equation*}
    [\pi_i(x), ~\pi_j(x')] = 0 \qquad [A_i(x), ~A_j(x')] = 0 \qquad [A_i(x), ~\pi_j(x')] = \delta_{ij} \delta^3(x - x')
\end{equation*}
    otteniamo 
\begin{equation*}
    \dv{}{t} \phi = [\phi(x), ~H(x')] = 2 \partial_k \partial'_i [\pi_k(x), ~A_j(x')] F^{ij(x')} = -2(\partial_k \partial'_i \delta^3(x - x')) F^{ij} (x') = 0
\end{equation*}
    a causa della cancellazione dei termini dovuta alla simmetria-antisimmetria. Conludiamo quindi che il vincolo si conserva sulle equazioni del moto e genera una simmetria di gauge. Per mostrare il gauge, consideriamo come vincolo un funzionale $\Phi[\Lambda(x)]$ dipendente dalla funzione di gauge $\Lambda(x)$ nel seguente modo 
\begin{equation*}
    \Phi[\Lambda(x)] = \int d^3 x \Lambda(x) \partial_i \pi^i(x)
\end{equation*}
    dove non abbiamo integrato nel tempo per mantenere la dipendenza temporale al fine di poter calcolare parentesi di Poisson allo stesso istante. La trasformazione di gauge sono definite come 
\begin{equation*}
\begin{aligned}
    \delta A_i (x) & = [A_i(t, x), ~\Phi[\Lambda]] = \int d^3 x' \Lambda(x', ~t) \partial'_j [A_i(t, ~x), ~ \pi^j(t, ~x')] \\ & = \int d^3 x' \Lambda(t, ~x') \delta_{ji} \delta^3 (x-x') = - \partial_i \Lambda(t, ~x)
\end{aligned}
\end{equation*}
    che è proprio l'invarianza di gauge presente in elettrodinamica. D'altra parte la trasfrmazione di gauge del momento coniugato deve essere nulla 
\begin{equation*}
    \delta \pi_i(t, ~x) = [\pi_i, ~\Phi[\Lambda]] = \int d^3 x' \Lambda(t, ~x')[\pi_i, ~\partial_j \pi^j] = 0
\end{equation*}
    Infine calcoliamo la variazione dell'azione 
\begin{equation*}
\begin{aligned}
    \delta S & = \int d^4 x (\pi_i \delta \dot A^i + \delta A_0 \phi) = \int d^4 (-\pi_i\partial^i \partial_0 \Lambda + \delta A_0 \partial_i \pi^i ) \\ & = \int d^4 x (\partial_i (-\pi^i \partial_o \Lambda) + \partial_i \pi^i \partial_0 \Lambda + \delta A_0 \partial_i \pi^i ) \\ & = \int d^4 x (\partial_\mu K^\mu + \partial_i \pi^i (\delta A_0 + \partial_9 \Lambda))
\end{aligned}
\end{equation*}
    dove abbiamo posto $K^\mu = (0, ~- \pi^i \partial_0 \Lambda)$ e, essendo invarianti di gauge, abbiamo usato $\delta \pi_i = 0$, $\delta H_0 = 0$ e $\delta \phi = 0$. Dunque concludiamo che dobbiamo avere $\delta A_0 = - \partial_0 \Lambda$ per avere un termine al bordo. In questo modo, la trasformazione di gauge del campo $A_\mu$ generata dal vincolo dell'equazione di Gauss è proprio quella aspettata:
\begin{equation*}
    \phi = \partial_i \pi^i = \nabla \cdot E = 0
\end{equation*}
    genera 
\begin{equation*}
    \delta A_\mu = - \partial_\mu \Lambda(x)
\end{equation*}
