\chapter*{Introduzione}
\addcontentsline{toc}{chapter}{Introduzione}

    L'oggetto di studio della presente tesi sono le conseguenze in fisica di due teoremi che la matematica tedesca Emmy Noether (1882-1935) ha pubblicato nel paper del 1918 \cite{noether}. Noether lavorò a Gottinga, matematicamente molto prolifica grazie alla presenza di Hilbert e Klein. La formulazione originale prevede nell'ambito del calcolo delle variazioni una corrispondenza sia tra gruppi continui finiti di simmetria e leggi di conservazionec che tra gruppi continui infiniti di simmetria e identità. In letteratura vengono chiamati rispettivamente il primo e il secondo teorema di Noether, anche se molto spesso si riferisce con l'espressione teorema di Noether solamente al primo. Il primo permette quindi di trovare quantità conservate già conosciute come l'energia, la quantità di moto e il momento angolare a partire da simmetrie traslazionali o rotazionali. Il secondo invece apre la strada alle moderne teorie di gauge, ovvero teorie che presentano funzioni arbitrarie come l'elettromagnetismo. Anche se il motivo originale del paper era lo studio della conservazione di energia e quantità di moto nello spaziotempo della relatività generale, viene poi successivamente applicato da Eric Bessel-Hagen (1898-1946) per studiare le leggi di conservazione della meccanica classica e l'invarianza conforme dell'elettrodinamica. In realtà nella formulazione originale di Noether, non erano presenti invarianze con termini al bordo mentre furono introdotte nel paper di Bessel-Hagen per giustificare l'invarianza del moto del centro di massa in seguito ad una simmetria dovuta ad un boost di Galileo. Inoltre nel paper originale è presente un terzo teorema.

    In questa tesi non enunceremo e dimostreremo le formulazioni orginali del teorema, ma ci limiteremo a fornire una versione più fisica più utili ai fini di applicazioni. La tesi è strutturata nel seguente modo. 
    
    Nel primo capitolo tratteremo sistemi fisici appartenenti alla meccanica classica, ovvero non quantistica né relativistica. Dopo una iniziale descrizione dei concetti di equazioni del moto, gradi di libertà e spazio delle configurazioni, introdurremo il formalismo lagrangiano con la funzione lagrangiana e le equazioni di Eulero-Lagrange. Successivamente studieremo le nozioni di simmetria e che cosa vuol dire che una quantità si conserva, per poi mostrare come si legano questi due concetti, attraverso il primo teorema di Noether per sistemi meccanici. Concluderemo il capitolo con il formalismo hamiltoniano, descrivendo il passaggio dallo spazio delle configurazioni allo spazio delle fasi, la funzione hamultoniana e le equazioni di Hamilton. Attraverso questa descrizione, dimostreremo due proposizioni collegate al primo teorema: il teorema inverso e l'algebra di Lie delle cariche, fornendo un esempio finale.

    Nel secondo capitolo, seguiremo le stesse linee guida del precedente, passando perà alla teoria dei campi classica, in questo caso relativistici ma non quantizzati. Studieremo qundi il formalismo lagrangiano, la densità di lagrangiana e le equazioni di Eulero-Lagrange per campi. Successivamente analizzaremo il legame tra simmetrie e quantità conservate, che in questo caso non si conservano più nel tempo ma soddisfano un'equazione di continuità, attraverso il primo teorema per campi, mostrando come la corrente conservata delle simmetrie spaziotemporali porta alla definizione di tensore energia-impulso. Dopo aver brevemente presentato un generico campo scalare, introdurremo la teoria elettromagnetica di Maxwell con il formalismo covariante e ne studieremo la simmetria spaziotemporale e la più generale simmetria conforme. Infine studieremo come applicando il primo teorema alla lagrangiana di Schroedinger, è possibile ritrovare la conservazione della probabilità, necessaria per l'interpretazione probabilistica.

    Nel terzo e ultimo capitolo, non enunceremo il secondo teorema di Noether, ma mostreremo che cosa significhi una teoria di gauge attraverso prima un esempio motivato da una azione simile a quella elettromagnetica e poi attraverso un elenco più formale di quale sia la struttura matematica di tale teoria. Infine ci focalizzeremo sulla particella relativistica e sulla già studiata elettrodinamica, per mostrare come tutte le caratteristiche siano presenti nella descrizione hamiltoniana: dalla simmetria di gauge ai vincoli. 