\chapter{Introduzione}

    \epigraph{As far as I can see, all a priori statements in physics have their origin in symmetry.}{Hermann Weyl}

    L'importanza dell'articolo pubblicato dalla matematica tedesca Emmy Noether $(1882-1935)$ nel 1918 dal titolo ``Invariante Variationsprobleme'' (problemi sulle variazioni invarianti) è incredibile~\cite{noether}~\cite{noether2}\footnote{Nella traduzione inglese dovuta a M. A. Tavel}. Le moderne teorie fisiche che descrivono le interazioni fondamentali sono formulate a partire da una teoria di gauge, che trova le sue basi proprio sul lavoro di Noether. La formulazione originale prevede nell'ambito del calcolo delle variazioni una corrispondenza sia tra gruppi continui finiti di simmetria e leggi di conservazione che tra gruppi continui infiniti di simmetria e identità. In letteratura vengono chiamati rispettivamente il primo e il secondo teorema di Noether, anche se molto spesso si riferisce con l'espressione teorema di Noether solamente al primo. 
    
    Noether lavorò a Gottinga, matematicamente molto prolifica grazie alla presenza di Klein $(1894-1977)$ e Hilbert $(1862-1943)$. Fu proprio da un'idea di quest'ultimo sullo studio della conservazione di energia e quantità di moto nello spaziotempo della relatività generale, che diede la possibilità a Noether di dimostrare i teoremi, che vennero successivamente applicati da Eric Bessel-Hagen $(1898-1946)$ per studiare le leggi di conservazione della meccanica classica e l'invarianza conforme dell'elettrodinamica. In realtà nella formulazione originale di Noether, non erano presenti invarianze con termini al bordo mentre furono introdotte da Bessel-Hagen per giustificare l'invarianza del moto del centro di massa in seguito ad una simmetria dovuta ad un boost di Galileo.

    In questa tesi non enunceremo o dimostreremo le formulazioni originali del teorema, ma ci limiteremo a fornire una versione più fisica al fine di mostrare applicazioni immediate. Di seguito introduciamo brevemente in che modo è strutturata questa tesi.
    
    Nel secondo capitolo tratteremo sistemi fisici appartenenti alla meccanica classica, ovvero non quantistica né relativistica. Dopo un'iniziale descrizione dei concetti di equazioni del moto, gradi di libertà e spazio delle configurazioni, introdurremo il formalismo lagrangiano con la funzione lagrangiana e le equazioni di Eulero-Lagrange. Successivamente studieremo le nozioni di simmetria e che cosa vuol dire che una quantità si conserva, per poi mostrare come si legano questi due concetti, attraverso il primo teorema di Noether per sistemi meccanici. Tratteremo quattro esempi che ci permettono di comprendere il teorema: la particella conforme, la particella libera, una particella in un campo di background e due particelle in un campo centrale. Concluderemo il capitolo con il formalismo hamiltoniano, descrivendo il passaggio dallo spazio delle configurazioni allo spazio delle fasi, la funzione hamiltoniana e le equazioni di Hamilton. Attraverso questa descrizione, dimostreremo due proposizioni collegate al primo teorema: il teorema inverso e l'algebra di Lie delle cariche, fornendo un esempio finale nuovamente sula particella conforme.

    Nel terzo capitolo, seguiremo le stesse linee guida del precedente, passando però alla teoria classica dei campi, in questo caso relativistici ma non quantizzati. Studieremo qundi il formalismo lagrangiano, la densità di lagrangiana e le equazioni di Eulero-Lagrange per campi. Successivamente analizzaremo il legame tra simmetrie e quantità conservate, che in questo caso non si conservano più nel tempo ma soddisfano un'equazione di continuità, attraverso il primo teorema per campi. Mostreremo come la corrente conservata delle simmetrie spaziotemporali porta alla definizione di tensore energia-impulso. Dopo aver brevemente presentato un generico campo scalare, introdurremo la teoria elettromagnetica di Maxwell e ne studieremo sia la simmetria spaziotemporale che la più generale simmetria conforme. Infine studieremo come applicando il primo teorema alla lagrangiana di Schroedinger, è possibile ritrovare la conservazione della probabilità, necessaria per l'interpretazione probabilistica.

    Nel quarto e ultimo capitolo, non enunceremo il secondo teorema di Noether, ma mostreremo che cosa significhi una teoria di gauge attraverso prima un esempio motivato da una azione simile a quella elettromagnetica e poi attraverso un elenco più formale di quale sia la struttura matematica di tale teoria. Dimostreremo il secondo teorema inverso, che ad una azione hamiltoniana in cui è presente un vincolo, descritto da moltiplicatori di Lagrange, è possibile associare una simmetria di gauge che generi tale vincolo. Infine ci focalizzeremo sulla particella relativistica e sulla già studiata elettrodinamica, per mostrare come tutte le caratteristiche siano presenti nella descrizione hamiltoniana: dalla simmetria di gauge ai vincoli. 

    \subsubsection{Riferimenti bibliografici}
    I riferimenti bibliografici per il secondo capitolo sono~\cite{landaumecc},~\cite{goldstein},~\cite{banados},~\cite{hill} e~\cite{bastianelli}.
    
    I riferimenti bibliografici per il terzo capitolo sono~\cite{banados},~\cite{landaucampi},~\cite{barone} e~\cite{weinberg}.

    I riferimenti bibliografici per il quarto capitolo sono~\cite{banados},~\cite{barone} e~\cite{weinberg}.
