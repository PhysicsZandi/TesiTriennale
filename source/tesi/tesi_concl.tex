\chapter*{Conclusioni}
\addcontentsline{toc}{chapter}{Conclusioni}

    Concludiamo la presente tesi, mostrando che i teoremi di Noether non si fermano solamente ad applicazioni riguardanti la meccanica classica o la teoria relativistica dei campi, ma che può essere uno strumento molto potente anche nell'ambito della fisica moderna: dall'elettrodinamica quantistica fino alla fisica delle particelle. Mostriamo qui una tabella che mostra come le principali leggi di conservazione siano associate a simmetrie continue di gauge\footnote{Alla parità e ad altre simmetrie discrete non si applicano i teoremi di Noether}

    \begin{center}
    \begin{tabular}{ c | c } 
      Legge di conservazione & Simmetria \\ 
      \hline
      Conservazione dell'energia & Traslazione temporale \\ 
      \hline
      Conservazione della quantità di moto & Traslazione spaziale \\ 
      \hline
      Conservazione del momento angolare & Rotazione spaziale \\ 
      \hline
      Conservazione del centro di massa & Boost di Lorentz \\ 
      \hline
      Conservazione della carica elettrica & Invarianza di gauge $U(1)$ \\ 
      \hline
      Conservazione della carica di colore & Invarianza di gauge $SU(3)$ \\ 
      \hline
      Conservazione dell'isospin debole & Invarianza di gauge $SU(2)$ \\ 
    \end{tabular}
    \end{center}