\documentclass[a4paper, 12pt, twoside]{book}

\usepackage{lipsum}

\usepackage{newlfont} 
\usepackage{color} 

%   Reduce the margin of the summary:
\def\changemargin#1#2{\list{}{\rightmargin#2\leftmargin#1}\item[]}
\let\endchangemargin=\endlist 

%   Generate the environment for the abstract:
\newcommand\summaryname{Abstract}
\newenvironment{Abstract}%
    {\small\begin{center}%
    \bfseries{\summaryname} \end{center}}

\usepackage[a4paper, top = 4cm, bottom = 4cm, left = 2.5cm, right = 2.5cm]{geometry}
\usepackage{etoc}

\usepackage{array}

\usepackage{emptypage}

\usepackage[utf8]{inputenc}

\usepackage{indentfirst}

\usepackage{epigraph}
\usepackage{amssymb}

\usepackage{fancyhdr}


\usepackage{hyperref}
%\hypersetup{colorlinks, linkcolor={black}, citecolor={black}, urlcolor={black}}

\usepackage[backend=biber, style=numeric, sorting=none]{biblatex}
\addbibresource{../bibliography.bib}
\usepackage{csquotes}

\usepackage{amsfonts}
\usepackage{amsmath}
\usepackage{amsthm}
\usepackage{cancel}
\usepackage{mathtools}
\usepackage{bbm}

\newtheorem{theorem}{Teorema}[section]
\newtheorem{principle}{Principio}[section]
\newtheorem{definition}{Definizione}[section]
\newtheorem{lemma}{Lemma}[section]
\theoremstyle{definition}
\newtheorem{example}{Esempio}[section]
\renewcommand\qedsymbol{q.e.d.}

\theoremstyle{remark}
\newtheorem{case}{Case}

\newcommand{\dv}[2]{\frac{d#1}{d#2}}
\newcommand{\dvd}[2]{\frac{d^2#1}{d#2^2}}
\newcommand{\dvf}[2]{\frac{\delta #1}{\delta #2}}
\newcommand{\pdv}[2]{\frac{\partial#1}{\partial#2}}
\newcommand{\pdvd}[3]{\frac{\partial^2 #1}{\partial#2 \partial#3}}
\newcommand{\integ}[3]{\int_{#1}^{#2}d#3~}
\newcommand{\poi}[2]{[#1, ~#2]}
\newcommand{\poiexp}[2]{\pdv{#1}{q^i} \pdv{#2}{p_i} - \pdv{#2}{q^i} \pdv{#1}{p_i}}

\usepackage{tikz}
\tikzset{every picture/.style={line width=0.75pt}} %set default line width to 0.75pt


\usepackage[italian]{babel}

\begin{document}

\textwidth=450pt\oddsidemargin=0pt

\begin{titlepage}

\begin{center}
{{\Large{\textsc{Alma Mater Studiorum $\cdot$ Universit\`a di Bologna}}}} 
\rule[0.1cm]{15.8cm}{0.1mm}
\rule[0.5cm]{15.8cm}{0.6mm}
\\\vspace{3mm}

{\small{\bf Scuola di Scienze \\ 
Dipartimento di Fisica e Astronomia\\
Corso di Laurea in Fisica}}

\end{center}

\vspace{23mm}

\begin{center}
{\LARGE{\bf I teoremi di Noether: simmetrie, leggi di conservazione e teorie di gauge}}\\
\end{center}

\vspace{50mm} \par \noindent

\begin{minipage}[t]{0.47\textwidth}
{\large{\bf Relatore:\vspace{2mm}\\
Prof. Fiorenzo Bastianelli}\\\\
\bf
\vspace{2mm}\\
\\\\}
\end{minipage}
\hfill
\begin{minipage}[t]{0.47\textwidth}\raggedleft 
{\large{\bf Presentata da:
\vspace{2mm}\\
Matteo Zandi}}
\end{minipage}

\vspace{40mm}

\begin{center}

Anno Accademico 2022/2023
\end{center}

\end{titlepage}


\begin{Abstract}
\begin{changemargin}{1cm}{1cm}
    Ad una simmetria globale dell'azione è possibile associare una legge di conservazione di una quantità fisica. Per sistemi meccanici, ciò comporta ad una quantità che rimane invariante nel tempo, mentre per sistemi continui, ciò comporta ad una equazione di continuità. Esempi dei primi sono la conservazione di energia, quantità di moto e momento angolare, mentre per gli ultimi sono il campo elettromagnetico o quello di Schroedinger. Inoltre, in seguito a simmetrie spaziotemporali, è possibile definire il tensore energia-impulso. Infine, se è presente una simmetria locale è possibile trovare delle equazioni che legano le equazioni del moto. Nel formalismo hamiltoniana questo comporta la presenza di vincoli generati dalle simmetrie. Esempi sono la particella relativistica o l'elettrodinamica.  
\end{changemargin}
\end{Abstract}

\pagestyle{empty}

\frontmatter

\tableofcontents
\addcontentsline{toc}{section}{Indice}

\mainmatter
\pagestyle{headings}

\chapter{Introduzione}

    La lungimiranza dell'articolo pubblicato dalla matematica tedesca Emmy Noether $(1882-1935)$ nel 1918 dal titolo ``Invariante Variationsprobleme'' (problemi sulle variazioni invarianti) è incredibile~\cite{noether}~\cite{noether2}\footnote{Nella traduzione inglese dovuta a M. A. Tavel}. Le moderne teorie fisiche che descrivono le interazioni fondamentali sono formulate a partire da una teoria di gauge, che trova le sue basi proprio nel lavoro di Noether. La formulazione originale prevede che nell'ambito del calcolo delle variazioni ci sia una corrispondenza sia tra gruppi continui finiti di simmetria e leggi di conservazione che tra gruppi continui infiniti di simmetria e identità. In letteratura vengono chiamati rispettivamente il primo e il secondo teorema di Noether, anche se molto spesso si riferisce con l'espressione teorema di Noether solamente al primo. 
    
    Noether lavorò a Gottinga, matematicamente molto prolifica grazie alla presenza di Klein $(1894-1977)$ e Hilbert $(1862-1943)$. Fu proprio da un'idea di quest'ultimo sullo studio della conservazione di energia e quantità di moto nello spaziotempo della relatività generale, che diede la possibilità a Noether di dimostrare i teoremi, che vennero successivamente applicati da Eric Bessel-Hagen $(1898-1946)$ per studiare le leggi di conservazione della meccanica classica e l'invarianza conforme dell'elettrodinamica. In realtà nella formulazione originale di Noether, non erano presenti invarianze con termini al bordo mentre furono introdotte da Bessel-Hagen per giustificare l'invarianza del moto del centro di massa in seguito ad una simmetria dovuta ad un boost di Galileo.

    In questa tesi non enunceremo o dimostreremo le formulazioni originali del teorema, ma ci limiteremo a fornire una versione più fisica al fine di mostrare applicazioni immediate. Di seguito introduciamo brevemente in che modo è strutturata questa tesi.
    
    Nel secondo capitolo tratteremo sistemi fisici appartenenti alla meccanica classica, ovvero non quantistica né relativistica. Dopo un'iniziale descrizione dei concetti di equazioni del moto, gradi di libertà e spazio delle configurazioni, introdurremo il formalismo lagrangiano con la funzione lagrangiana e le equazioni di Eulero-Lagrange. Successivamente studieremo le nozioni di simmetria e che cosa vuol dire che una quantità si conservi, per poi mostrare come si legano questi due concetti, attraverso il primo teorema di Noether per sistemi meccanici. Tratteremo quattro esempi che ci permettono di comprendere il teorema: la particella conforme, la particella libera, una particella in un campo di background e due particelle in un campo centrale. Concluderemo il capitolo con il formalismo hamiltoniano, descrivendo il passaggio dallo spazio delle configurazioni allo spazio delle fasi, la funzione hamiltoniana e le equazioni di Hamilton. Attraverso questa descrizione, dimostreremo due proposizioni collegate al primo teorema: il teorema inverso e l'algebra di Lie delle cariche, fornendo un esempio finale nuovamente sulla particella conforme.

    Nel terzo capitolo, seguiremo le stesse linee guida del precedente, passando però alla teoria classica dei campi, in questo caso relativistici ma non quantizzati. Studieremo quindi il formalismo lagrangiano, la densità di lagrangiana e le equazioni di Eulero-Lagrange per campi. Successivamente analizzaremo il legame tra simmetrie e quantità conservate, che in questo caso non si conservano più nel tempo ma soddisfano un'equazione di continuità, attraverso il primo teorema di Noether per campi. Mostreremo poi come la corrente conservata delle simmetrie spaziotemporali porta alla definizione di tensore energia-impulso. Dopo aver brevemente presentato un generico campo scalare, introdurremo la teoria elettromagnetica di Maxwell e ne studieremo sia la simmetria spaziotemporale che la più generale simmetria conforme. Infine studieremo come applicando il primo teorema alla lagrangiana di Schroedinger, è possibile ritrovare la conservazione della probabilità, necessaria per l'interpretazione probabilistica.

    Nel quarto e ultimo capitolo, non enunceremo il secondo teorema di Noether, ma mostreremo che cosa significhi una teoria di gauge attraverso prima un esempio motivato da una azione simile a quella elettromagnetica e poi attraverso un elenco più formale di quale sia la struttura matematica di tale teoria. Dimostreremo il secondo teorema inverso, che ad un'azione hamiltoniana in cui è presente un vincolo, descritto da moltiplicatori di Lagrange, è possibile associare una simmetria di gauge che generi tale vincolo. Infine ci focalizzeremo sulla particella relativistica e sulla già studiata elettrodinamica, per mostrare come tutte le caratteristiche siano presenti nella loro descrizione hamiltoniana: dalla simmetria di gauge ai vincoli. 

    \subsubsection{Riferimenti bibliografici}
    I riferimenti bibliografici per il secondo capitolo sono~\cite{landaumecc},~\cite{goldstein},~\cite{banados},~\cite{hill} e~\cite{bastianelli}.
    
    I riferimenti bibliografici per il terzo capitolo sono~\cite{banados},~\cite{landaucampi},~\cite{barone} e~\cite{weinberg}.

    I riferimenti bibliografici per il quarto capitolo sono~\cite{banados},~\cite{barone} e~\cite{weinberg}.
\chapter{Il primo teorema in meccanica classica}
    
    Il primo sistema fisico che prendiamo in considerazione è un sistema di particelle o punti materiali, ovvero un insieme discreto costituito da oggetti le cui dimensioni posso essere trascurate quando ad esserne descritto è il loro moto. Al fine di studiare la relazione tra simmetrie e leggi di conservazione, è necessario approfondire la struttura matematica per descrivere tali sistemi. Introduciamo dapprima quali sono le quantità fisiche che ci permettono di analizzare il moto: posizione, velocità, accelerazione. Nell'ambito della fisica newtoniana classica, lo spazio matematico che utilizziamo per studiare il sistema fisico è lo spazio tridimensionale euclideo $\mathbb R^3$ e assumiamo che il tempo sia assoluto.
    
    Nel sistema cartesiano di assi coordinati ortogonali, la posizione di una particella viene generalmente indicata con un vettore $\mathbf r$, le cui componenti sono 
    \begin{equation*}
        r^i = (x, ~y, ~z)
    \end{equation*}
    con $i = 1, ~2, ~3$. La derivata prima rispetto al tempo del vettore posizione viene chiamata velocità $\mathbf v = \mathbf {\dot r}$, le cui componenti sono 
    \begin{equation*}
        \dot r^i = \dv{r^i}{t} = (\dot x, ~\dot y, ~\dot z)
    \end{equation*}
    mentre la derivata seconda rispetto al tempo del vettore posizione, o equivalentemente la derivata prima della velocità, viene chiamata accelerazione $\mathbf a = \mathbf{\dot v} = \mathbf{\ddot r}$, le cui componenti sono 
    \begin{equation*}
        \ddot r^i = \dvd{r^i}{t} = (\ddot x, ~\ddot y, ~\ddot z)
    \end{equation*}
    
    Se per una singola particella abbiamo definito 3 coordinate, per un sistema di N particelle libere sono necessarie 3N coordinate, 3N velocità, 3N accelerazioni per descriverlo. Tuttavia il sistema cartesiano non è l'unico disponibile, infatti è possibile utilizzare un generico sistema di assi coordinati purchè siano di numero pari al numero di gradi di libertà $d$, che corrisponde al numero di quantità indipendenti necessarie per definire univocamente la posizione di un sistema fisico. Una particella libera ha $d = 3$, mentre un sistema di N particelle ha $d = 3N$. Chiameremo queste quantità coordinate generalizzate $q^i$ e, analogamente, chiameremo velocità generalizzate $\dot q^i$ e accelerazioni generalizzate $\ddot q^i$ le rispettive derivate prime e seconde. La configurazione istantanea del sistema, cioè ad un tempo fissato, può quindi essere quindi rappresentata da un punto in uno spazio $\mathcal C$, chiamato spazio delle configurazioni, di dimensione pari al numero di gradi di libertà ed i cui assi coordinati sono le coordinate generalizzate. Come precedentemente indicato, nel caso di una particella, lo spazio delle configurazioni è lo spazio tridimensionale euclideo $\mathbb R^3$, mentre per un sistema di N particelle libere sarà il prodotto cartesiano $\mathbb R^{3N}$.
    
    Il moto del sistema è matematicamente descritto dalla posizione del sistema in funzione del tempo, chiamata traiettoria, ciò che geometricamente viene rappresentato da una curva $q^i(t)$ nello spazio delle configurazioni $\mathcal C$. Dunque conoscendo quest'ultima, siamo a conoscenza della completa evoluzione temporale del sistema. Come possiamo ricavarla? La traiettoria è soluzione di un sistema di equazioni che legano accelerazioni, velocità e posizioni, chiamate equazioni del moto
    \begin{equation} \label{eqmoto}
        f_j(q^i, ~ \dot q^i, ~ \ddot q^i) = 0
    \end{equation}
    dove $j = 1, ~2, ~\ldots ~d$. Sono un sistema formato da $d$ equazioni differenziali, una per ogni grado di libertà e la presenza delle accelerazioni ci suggerisce che sono al secondo ordine, caratteristica che porta una importante conseguenza: per studiare come il sistema evolve nel tempo, la conoscenza delle sole coordinate in un dato istante non è sufficiente. Infatti, in modo tale che le accelerazioni siano univocamente determinate, è necessario fornire anche le velocità. Ciò deriva da un risultato dell'analisi matematica conosciuto: conoscendo le condizioni iniziali, ovvero le $d$ posizioni $q^i(0) = q^i_0$ e le $d$ velocità $\dot q^i(0) = v^i_0$, la soluzione del problema di Cauchy delle equazioni del moto è unica. 
    
    Come possiamo trovare esplicitamente le \eqref{eqmoto}? In meccanica classica, si distinguono principalmente tre differenti formalismi dovuti a tre importanti figure nell'ambito della fisica: Isaac Newton $(1642-1726)$, Joseph-Louis Lagrange $(1736-1813)$ e William Rowan Hamilton $(1805-1865)$. 

    La descrizione newtoniana consiste brevemente nel trovare le forze $F^i(t, ~q^i, ~ \dot q^i)$ che agiscono sul sistema e poi risolvere la celeberrima equazione conosciuta come secondo principio della dinamica
    \begin{equation*}
        F^i(t, ~q^i, ~ \dot q^i) = m \ddot q^i
    \end{equation*}
    o qualora la massa $m$ non sia costante 
    \begin{equation*}
        F^i(t, ~q^i, ~ \dot q^i) = \dv{p^i}{t}
    \end{equation*}
    dove $p^i = m \dot q^i$ è la componente $i$-esima della quantità di moto. In ogni caso, in questa tesi non tratteremo questo formalismo, bensì nel prossimo paragrafo ci concentreremo principalmente sulla descrizione lagrangiana, mentre alla fine del capitolo illustreremo anche quella hamiltoniana. I riferimenti bibliografici sono \cite{landaumecc} \cite{goldstein} \cite{banados}.
    
\section{Formalismo lagrangiano}

    Il formalismo lagrangiano consiste nell'associare ad un generico sistema meccanico una funzione $L$ delle coordinate $q^i$, velocità $\dot q^i$ ed eventualmente del tempo $t$, chiamata lagrangiana del sistema
    \begin{equation} \label{lagrangiana}
        L = L(t, ~q^i, ~\dot q^i) 
    \end{equation}  
    Non c'è un criterio generale per trovare tale funzione, ma c'è una classe di sistemi meccanici in cui la lagrangiana può essere scritta nella forma $L = T - U$, dove $T$ e $U$ sono rispettivamente l'energia cinetica e l'energia potenziale del sistema. Tale sistema viene chiamato sistema conservativo e richiede che l'energia potenziale $U = U(q^i)$ sia funzione soltanto delle coordinate.
    
    Utilizzando la Lagrangiana $\eqref{lagrangiana}$, introduciamo il funzionale di azione $S = S[q^i(t)]$ che associa un numero reale ad ogni curva $q^i(t)$ dello spazio delle configurazioni, definita all'interno di un intervallo temporale fisso $[t_1, ~t_2]$
    \begin{equation} \label{azione}
        S[q^i(t)] = \integ{t_1}{t_2}{t} L(t, ~q^i, ~\dot q^i)
    \end{equation}

\subsection{Equazioni di Eulero-Lagrange per sistemi meccanici}

    Enunciamo il principio che ci permette di trovare le equazioni del moto a partire dalla Lagrangiana del sistema: il principio di Hamilton, chiamato anche incorrettamente principio di minima azione.

    \begin{principle}[di Hamilton]
        Tra tutte le curve nello spazio delle configurazioni $\mathcal C$ che collegano due estremi fissi $q_1$ e $q_2$ di un intervallo temporale anch'esso fisso $[t_1, ~t_2]$ tale che 
    \begin{equation*}
        q^i(t_1) = q^i_1 \qquad q^i(t_2) = q^i_2
    \end{equation*}
        il sistema meccanico associato alla lagrangiana \eqref{lagrangiana} percorre la curva $q^i(t)$ che rende stazionario il suo funzionale di azione \eqref{azione}
    \begin{equation} \label{azionestazionaria}
        \delta S [q^i(t)] = 0
    \end{equation}
    \end{principle}

    Cerchiamo dunque di tradurre questo principio in equazioni in funzione della lagrangiana del sistema e delle sue derivate che siano equivalenti alle equazioni del moto \eqref{eqmoto}. Siccome l'azione è un funzionale, ovvero una funzione che non dipende da un numero discreto di variabili, ma il cui dominio è in uno spazio di funzioni infinito dimensionale, non possiamo utilizzare direttamente il calcolo differenziale ma è necessario attingere alla branca della matematica, chiamata calcolo delle variazioni, che studia come trovare la funzione nel dominio del funzionale tale che quest'ultimo sia stazionario, nel nostro caso quella che soddisfa il principio di Hamilton. Supponiamo che la curva $q^i(t)$ sia quella cercata, ovvero quella che rende stazionario il funzionale d'azione \eqref{azionestazionaria}, e prendiamo un'altra curva $\delta q^i(t)$, che chiameremo variazione, definita sempre nell'intervallo $[t_1, ~t_2]$, arbitraria tranne che per gli estremi
    \begin{equation} \label{estreminulli}
        \delta q^i(t_1) = \delta q^i(t_2) = 0
    \end{equation}
    Introduciamo ora una famiglia di curve $q^i(t, ~\epsilon)$ dipendenti da un parametro $\epsilon$ definita nel seguente modo 
    \begin{equation} \label{famigliacurve}
        q^i(t, \epsilon) = q^i(t, ~0) + \epsilon \delta q^i(t)
    \end{equation}
    ovvero una famiglia di curve che connetta la curva cercata $q^i(t)$ con la variazione $\delta q^i(t)$. Segue dalla definizione \eqref{famigliacurve} che se il parametro si annulla $\epsilon=0$, la famiglia di curve si riduce alla curva cercata
    \begin{equation*}
        q^i(t, ~ 0) = q^i(t)
    \end{equation*}
    In questo modo, sostituendo nel funzionale di azione \eqref{azione} la famiglia di curve \eqref{famigliacurve} possiamo parametrizzarlo con il parametro reale $\epsilon$
    \begin{equation*}
        S(\epsilon) = \integ{t_1}{t_2}{t} L(t, ~q^i(t, ~\epsilon), ~\dot q^i(t, \epsilon))
    \end{equation*}
    riconducendoci ad una funzione dipendente da una variabile reale, così da poter utilizzare le conosciute tecniche del calcolo differenziale. Definiamo la variazione del funzionale d'azione in $q^i(t)$ come la derivata rispetto al parametro $\epsilon$ calcolata in $\epsilon = 0$
    \begin{equation} \label{variazioneazione}
        \delta S[q^i(t), ~ \delta q^i(t)] = \dv{}{\epsilon} S(\epsilon) \Big \vert_{\epsilon = 0} = \dv{}{\epsilon} S[q^i(t) + \epsilon \delta q^i(t)] \Big \vert_{\epsilon = 0}
    \end{equation}
    e calcolando esplicitamente la derivata, differenziando sotto il segno di integrale, troviamo che 
    \begin{equation*}
        \dv{S}{\epsilon} = \integ{t_1}{t_2}{t} \Big (\pdv{L}{q^i} \pdv{q^i}{\epsilon} + \pdv{L}{\dot q^i} \pdv{\dot q^i}{\epsilon} \Big)
    \end{equation*}\label{prova1}
    Osserviamo che non è presente la derivata rispetto ad t perché quest'ultimo non dipende da $\epsilon$, dunque le loro derivate commutano. L'integrazione per parti del secondo termine nell'integrale ci conduce a
    \begin{equation}\label{prova2}
        \dv{S}{\epsilon} = \integ{t_1}{t_2}{t} \Big (\pdv{L}{q^i} \pdv{q^i}{\epsilon} - \dv{}{t} \Big (\pdv{L}{\dot q^i} \Big) \pdv{q^i}{\epsilon} \Big) + \pdv{L}{\dot q^i} \pdv{q^i}{\epsilon} \Big \vert_{t_1}^{t_2}
    \end{equation}
    Riprendendo la \eqref{famigliacurve}, notiamo che la derivata parziale della curva $q^i$ rispetto al parametro $\epsilon$ è la variazione della curva $\delta q^i$
    \begin{equation}\label{prova3}
        \pdv{q^i}{\epsilon} = \pdv{}{\epsilon} (q^i(t, ~ 0) + \epsilon \delta q^i(t)) = \delta q^i(t)
    \end{equation}
    Sostituendo \eqref{prova3} in \eqref{prova2}, otteniamo 
    \begin{equation*}
    \begin{aligned}
        \dv{S}{\epsilon} & = \integ{t_1}{t_2}{t} \Big (\pdv{L}{q^i} \delta q^i(t) - \dv{}{t} \Big (\pdv{L}{\dot q^i} \Big) \delta q^i(t) \Big) + \pdv{L}{\dot q^i} \delta q^i(t) \Big \vert_{t_1}^{t_2} \\  & = \integ{t_1}{t_2}{t} \Big (\pdv{L}{q^i} \delta q^i(t) - \dv{}{t} \Big (\pdv{L}{\dot q^i} \Big) \delta q^i(t) \Big) + \pdv{L}{\dot q^i} \delta q^i(t_2) - \pdv{L}{\dot q^i} \delta q^i(t_1) \\ & = \integ{t_1}{t_2}{t} \delta q^i(t) \Big (\pdv{L}{q^i}  - \dv{}{t} \Big (\pdv{L}{\dot q^i} \Big) \Big) + \pdv{L}{\dot q^i} \delta q^i(t_2) - \pdv{L}{\dot q^i} \delta q^i(t_1)
    \end{aligned}
    \end{equation*}
    dove nell'ultimo passaggio abbiamo raccolto un fattore comune $\delta q^i(t)$. Ricordando che agli estremi la variazione si annulla \eqref{estreminulli}, gli ultimi due termini si cancellano, otteniamo la relazione finale
    \begin{equation} \label{prova4}
        \dv{S}{\epsilon} = \integ{t_1}{t_2}{t} \delta q^i(t) \Big (\pdv{L}{q^i}  - \dv{}{t} \pdv{L}{\dot q^i} \Big)
    \end{equation}
    Adoperiamo il fatto che condizione necessaria affinchè la curva cercata $q^i(t)$ renda il funzionale d'azione \eqref{azione} stazionario è che la variazione di quest'ultimo si annulli, qualunque sia la variazione della curva $\delta q^i(t)$, poniamo la variazione nulla
    \begin{equation} \label{prova5}
        \delta S[q^i(t), ~ \delta q^i(t)] = 0 \qquad \forall ~ \delta q^i(t)
    \end{equation}
    Mettendo insieme \eqref{prova4} e \eqref{prova5}, attraverso la \eqref{variazioneazione}, otteniamo
    \begin{equation*}
        \delta S = \integ{t_1}{t_2}{t} \delta q^i(t) \Big (\pdv{L}{q^i}  - \dv{}{t} \pdv{L}{\dot q^i} \Big) = 0
    \end{equation*}
    Utilizzando il lemma fondamentale del calcolo delle variazioni, possiamo asserire che l'integranda si annulla per qualsiasi variazione della curva $\delta q^i(t)$ e dunque giungere al sistema di equazioni
    \begin{equation} \label{eullag}
        \dvf{S}{q^i} = \pdv{L}{q^i}  - \dv{}{t} \pdv{L}{\dot q^i} = 0
    \end{equation}
    oppure anche scritte esplicitando la derivata temporale nel secondo termine
    \begin{equation*} 
        \pdv{L}{q^i}  - \pdvd{L}{\dot q^i}{t} - \pdvd{L}{\dot q^i}{q^j} \dot q^j - \pdvd{L}{\dot q^i}{\dot q^j} \ddot q^j = 0
    \end{equation*}
    Queste sono le equazioni del moto che stavamo cercando, equazioni differenziali al secondo ordine della lagrangiana, chiamate equazioni di Eulero-Lagrange. Una volta trovata la lagrangiana $L$ associata al sistema fisico che stiamo studiando, abbiamo trovato che le \eqref{eullag} si riducono alle \eqref{eqmoto}.

    \hfill 

    Per completezza, concludiamo il paragrafo enunciando e dimostraziondo il lemma fondamentale del calcolo delle variazioni.
    \begin{lemma}
        Sia $f(x)$ una funzione continua tale che 
    \begin{equation*}
        \integ{x_1}{x_2}{x} g(x) f(x) = 0 
    \end{equation*}
        per ogni funzione $g(x)$, continua con derivata prima e seconda continua tale, che si annulli agli estremi
    \begin{equation*}
        g(x_1) = g(x_2) = 0
    \end{equation*}
        allora $f(x)$ = 0
    \end{lemma}
    Nel nostro caso avevamo al posto di $f$ le equazioni di Eulero-Lagrange e al posto di $g$ la variazione $\delta q^i(t)$. È qui presente una dimostrazione non formale ma molto intuitiva.
    \begin{proof}
        Per assurdo se $f$ è positiva in un intorno di un punto $[x'_1, ~ x'_2]$ e nulla in ogni altro punto dell'intervallo $[x_1, ~ x_2]$, data l'arbitrarietà di $g$, per qualsiasi funzione che sia positiva otteniamo
    \begin{equation*}
        \integ{x_1}{x_2}{x} g(x) f(x) = \integ{x'_1}{x'_2}{x} g(x) f(x) > 0
    \end{equation*} 
        che contraddice l'ipotesi che questo integrale sia nullo e mostrando l'assurdo.
    \end{proof}

\subsection{Simmetrie del funzionale di azione}
    
    Una volta introdotti i concetti di funzionale di azione e di equazioni del moto, iniziamo ora a studiare il concetto di simmetria del funzionale di azione \cite[Capitolo 2]{banados}. Innanzitutto le simmetrie sono trasformazioni di coordinate, ovvero un cambio di coordinate spaziali e temporale per descrivere il sistema. Distinguiamo due modi differenti di interpretare una trasformazione di coordinate: quello passivo e quello attivo. Una trasformazione passiva è un cambio del sistema di riferimento, cioè l'oggetto viene descritto da due differenti osservatori; mentre in una trasformazione attiva si sposta effettivamente il sistema preso in considerazione mantenendo invariato il sistema di riferimento, cioè si sposta l'oggetto mantenendo l'osservatore invariato. Inoltre le simmetrie sono una classe speciale di trasformazioni di coordinate con una importante proprietà: il funzionale di azione rimane invariato, ovvero la nuova azione $S'$ conduce alle stesse equazioni del moto della vecchia $S$. Portiamo particolare attenzione al significato di invariato: non necessariamente significa che la nuova e la vecchia azione siano identiche $S' = S$ ma che siano uguali a meno di un termine al bordo $K$. Ciò è conseguenza del fatto che la lagrangiana \eqref{lagrangiana} che porta alle equazioni del moto \eqref{eullag} non è unica ma presenta un'ambiguità: è definita a meno di una derivata totale rispetto ad un'arbitraria funzione $K(q^i, ~ t)$. Infatti aggiungendolo alla lagrangiana di un funzionale d'azione $S'$
    \begin{equation*}
        S' = \integ{t_1}{t_2}{t} \Big ( L(q^i, ~ \dot q^i, ~t) + \dv{}{t} K(q^i, ~t) \Big)
    \end{equation*}
    notiamo che, usando \eqref{estreminulli}, la sua variazione è annulla,
    \begin{equation*}
        \delta \integ{t_1}{t_2}{t} \dv{}{t} K(q^i, ~t) = \pdv{K}{q^i} \delta q^i \Big \vert_{t_1}^{t_2} = \pdv{K}{q^i} \delta q^i (t_2) - \pdv{K}{q^i} \delta q^i (t_1) = 0
    \end{equation*} 
    Dunque $\delta S = \delta S'$ e le equazioni del moto non cambiano.

    Formalizziamo il tutto, introducendo il primo concetto di simmetria.

\begin{definition}[Simmetria dell'azione]
    Definiamo una simmetria dell'azione una trasformazione qualsiasi, non necessariamente infinitesima, delle coordinate spaziali 
\begin{equation}\label{simmazione}
    \delta_s q^i(t) = q'(t) - q(t)
\end{equation} 
    tale che l'azione sia invariante a meno di un termine al bordo $K$
\begin{equation} \label{invazione}
    \delta[q^i(t), ~\delta_s q^i(t)] = S[q^i(t) + ~\delta_s q^i(t)] - S[q^i(t)] = \integ{t_1}{t_2}{t} \dv{}{t} K(q^i, ~t)
\end{equation}
\end{definition}    
    Le simmetrie $\delta_s q^i$ sono quindi direzioni dello spazio generate dalle coordinate $q^i$ che rendono invariante l'azione $S$. Vedremo come il termine al bordo non interferisce con l'esistenza delle quantità conservate ma sarà necessario per trovarle. Sottolineiamo il fatto che le variazioni di simmetria $\delta_s q^i(t)$ devono soddisfare il vincolo dato dall'equazione \eqref{invazione}, mentre le $q^i(t)$ sono del tutto arbitrarie. 

    Finora non abbiamo considerato trasformazioni riguardanti il tempo perché definiamo simmetrie solo cambi di coordinate spaziali mentre le traslazioni temporali saranno tradotte in deformazioni delle coordinate. Consideriamo il caso più semplice unidimensionale dove due curve $q^i(t)$ e $q'^i(t)$ sono collegate da un traslazione temporale $t' = t + \epsilon$ nel seguente modo 
\begin{equation*}
    q'(t') = q'(t + \epsilon) = q(t)
\end{equation*} 
    Trattando il parametro $\epsilon$ come una quantità infinitesima, ovvero con $\epsilon << 1$, espandiamo al primo ordine il termine $q'(t + \epsilon)$ e otteniamo
\begin{equation*}
    q'(t + \epsilon) \simeq q'(t) + \epsilon \dot q(t) = q(t)
\end{equation*}     
    Quindi possiamo scrivere la variazione di simmetria \eqref{simmazione} come 
\begin{equation}\label{simmtempo}
    \delta_s q(t) = - \epsilon \dot q(t)
\end{equation}
    che dunque coinvolge soltanto un istante $t$ e non anche $t'$ come prima. Conseguentemente, essendo $\delta_s q(t)$ una differenza valutata allo stesso istante $t$, commuta con la derivata rispetto al tempo
\begin{equation}\label{commutatore}
    \delta_s \dot q(t) = \delta_s \dv{}{t} q(t) = \dv{}{t} \delta_s q(t) 
\end{equation}
    Questa considerazione può essere evidenziata ricordando che nel funzionale di azione \eqref{azione} il tempo è la variabile di integrazion, che per definizione è muta ovvero effettuato una sostituzione l'integrale non ne subisce gli effetti. Questa trasformazione non ha nulla a che vedere con le variazioni di simmetria dell'azione. 

    \hfill

    Le variazioni di simmetria del funzionale di azione \eqref{simmazione} non sono le uniche tipologie di variazioni. Infatti è possiible introdurre le cosidette variazioni on-shell, dove il termine on-shell significa che vengono applicate le equazioni del moto. 
\begin{definition}[Variazione on-shell]
    Definiamo una variazione on-shell una trasformazione infinitesima delle coordinate spaziali 
\begin{equation}\label{simmonshell}
    \delta q = q' - q
\end{equation} 
    tale che l'azione sia invariante a meno di una derivata temporale, dopo aver applicato le equazioni di Eulero-Lagrange \eqref{eullag}. Infatti, variando il funzionale di azione abbiamo
\begin{equation}
    \delta S[q^i,~\delta q^i] = S[q^i + \delta q^i] - S[q^i] = \integ{t_1}{t_2}{t} \Big( \pdv{L}{q^i} \delta q^i + \pdv{L}{\dot q^i} \delta \dot q^i \Big)
\end{equation}
    Utilizzando l'identità, ricordando che le variazioni commutano con la derivata temporale \eqref{commutatore}
\begin{equation*}
    \pdv{L}{\dot q^i} \delta \dot q^i = \pdv{L}{\dot q^i} \delta \dv{}{t} q^i = \dv{}{t} \Big (\pdv{L}{\dot q^i} \delta q^i \Big) - \dv{}{t} \Big ( \pdv{L}{\dot q^i} \Big) \delta q^i 
\end{equation*}
    per integrare per parti, otteniamo
\begin{equation*}
    \delta S[q^i,~\delta q^i] = \integ{t_1}{t_2}{t} \Big( \pdv{L}{q^i}  - \dv{}{t} \pdv{L}{\dot q^i} \Big ) \delta q^i + \integ{t_1}{t_2}{t} \dv{}{t} \Big( \pdv{L}{\dot q^i} \delta q^i \Big )
\end{equation*}
    A questo punto otteniamo esplicitamente la variazione on-shell, notando che il primo termine comprende le equazioni di Eulero-Lagrange \eqref{eullag}. Applicandole, passando on-shell, e otteniamo 
\begin{equation} \label{invonshell}
    \delta S[\overline q^i,~\delta q^i] = S[\overline q^i + \delta q^i] - S[\overline q^i] = \integ{t_1}{t_2}{t} \dv{}{t} \Big( \pdv{L}{\dot q^i} \delta q^i \Big )
\end{equation}
    dove la notazione $\overline q^i$ indica che le $q^i$ rispettano le equazioni del moto.
\end{definition}  

    Concettualmente questa variazione è oppsta rispetto a quella dell'azione, perchè quella sull'azione pone vincoli su $\delta_s q^i(t)$ e lasciano libere le $q^i(t)$ mentre specularmente le variazioni on-shell pongono vincoli sulle $\overline q^i$ e lasciano libere le $\delta q^i$. Queste proprietà saranno il punto di partenza per trovare le quantità conservate del nostro sistema, che studieremo nel prossimo paragrafo. 

\subsection{Leggi di conservazione}
    Soffermiamoci tuttavia prima sulla nozione di quantità conservata. Definiamo una quantità conservata come una funzione $Q(t, ~q^i)$ tale che il suo valore rimane constante nel tempo lungo ogni traiettoria nello spazio delle configurazioni del sistema
\begin{equation*}
    \dv{}{t} Q(t, ~q^i) = \pdv{Q}{t} + \pdv{Q}{q^i} \dot q^i = 0
\end{equation*}
    Dallo studio delle equazioni di Eulero-Lagrange, notiamo che possiamo definirne già una: il momento coniugato di una coordinata ciclica. Una coordinata ciclica $q^j$ è una coordinata che non è presente esplicitamente nella lagrangiana
\begin{equation} \label{cyclic}
    \pdv{L}{q^j} = 0
\end{equation}
    mentre definiamo il momento coniugato $p_j$ ad una coordinata $q^j$ come la derivata parziale della lagrangiana rispetto alla velocità corrispondente a questa coordinata
\begin{equation} \label{coniugato}
    p_j = \pdv{L}{\dot q^j}
\end{equation}
    Prendendo in considerazione le \eqref{eullag}
\begin{equation*}
    \pdv{L}{q^j}  - \dv{}{t} \pdv{L}{\dot q^j} = 0
\end{equation*}
    notiamo che se non è presente alcuna dipendenza esplicita della lagrangiana da una coordinata ciclica, il primo termine sarà nullo- Inserendo \eqref{coniugato} e semplificando con \eqref{cyclic}, otteniamo
\begin{equation*}
    \dv{}{t} p_j = 0
\end{equation*}
    Concludiamo quindi che nel caso di una coordinata ciclica, il suo momento coniugato si conserva. 

    Una importante osservazione sulle quantità conservate è che possono essere usate come condizioni iniziali del sistema e dunque per trovare le soluzioni del moto. Considerando per semplicità un sistema unidimensionale, abbiamo bisogno di due condizioni iniziali: invece di usare due posizioni oppure una posizione ed una velocità come solitamente, possiamo sfruttare la conoscenza di due quantità conservate per scrivere le soluzioni del moto in funzione di essere e dunque raggiungere il nistro obiettivo senza neanche aver risolto le equazioni del moto, che potrebbero essere troppo complicate. Più avanti, tratteremo un esempio che mostra esplicitamente questa procedura, ma prima ci soffermiamo sul risultato alla base di questa tesi: come trovare altre quantità conservate?

\section{Il primo teorema di Noether in meccanica classica}
    Al cuore di questa tesi è presente quello che viene chiamato il primo teorema di Noether, in onore del paper pubblicato nel 1918 dal matematico Emmy Noether che per prima l'ha dimostrato \cite{noether}. Qualitativamente, asserisce che per ogni simmetria è possibile associare una quantità conservata. Inoltre, non solo ci dice quante sono le quantità conservate, ma anche come si calcolano, fornendo un vero e proprio algoritmo. Tuttavia non presenteremo il teorema originale, ma una versione più fisica che ci permette di trovare immediate applicazioni. 

    Ricapitolando ciò che abbiamo concluso nel precedente paragrafo, una variazione di simmetria $\delta_s q^i(t)$ porta ad una variazione dell'azione data da \eqref{invazione} 
\begin{equation*}
    \delta[q^i(t), ~\delta_s q^i(t)] = S[q^i(t) + ~\delta_s q^i(t)] - S[q^i(t)] = \integ{t_1}{t_2}{t} \dv{}{t} K(q^i, ~t)
\end{equation*}
    mentre una variazione on-shell porta ad una variazione dell'azione data da \eqref{invonshell} 
\begin{equation*}
    \delta S[\overline q^i,~\delta q^i] = S[\overline q^i + \delta q^i] - S[\overline q^i] = \integ{t_1}{t_2}{t} \dv{}{t} \Big( \pdv{L}{\dot q^i} \delta q^i \Big )
\end{equation*}
    Entrambi presentano termini al bordo ma per ragioni differenti: la prima perché $\delta_s q^i(t)$ soddisfa una determinata equazione mentre nella seconda è $\overline q^i$ che ne soddisfa un'altra.

\subsection{Enunciato e dimostrazione}

    A questo punto abbiamo tutti gli strumenti per enunciare e dimostrare il primo teorema di Noether per sistemi fisici appartenenti al dominio della meccanica classica.

\begin{theorem}[Primo teorema di Noether in meccanica classica]
    Sia $L(q^i,~\dot q^i)$ la lagrangiana di un sistema fisico con $i=1,~2,\ldots d$, dove $d$ indica il numero digradi di libertà. Il funzionale di azione $S$ associato al sistema nell'intervallo temporale $[t_1,~t_2]$ è
\begin{equation*}
    S[q^i(t)] = \integ{t_1}{t_2}{t} L(L(q^i,~\dot q^i))
\end{equation*}
    Sia $\delta_s q^i(t)$ una trasformazione delle coordinate che individua una variazione di simmetria del sistema \eqref{simmazione}, ovvero che lascia invariate le equazioni di Eulero-Lagrange \eqref{eullag}. Allora esiste una quantità $Q$, definita come
\begin{equation}\label{carica}
    Q = K - \frac{\partial L}{\partial \dot q^i} \delta_s q^i
\end{equation}
    tale che sia conservata lungo le equazioni del moto, ovvero soddisfi la legge di conservazione
\begin{equation} \label{th}
    \frac{d}{dt} Q = 0
\end{equation}
\end{theorem}

\begin{proof}
    Inserendo $q^i(t) = \overline q^i(t)$ nella \eqref{invazione}, ovvero vincolando che le $q^i(t)$ soddisfino le equazioni del moto, otteniamo 
\begin{equation}\label{prova6}
    \delta[\overline q^i(t), ~\delta_s q^i(t)] = \integ{t_1}{t_2}{t} \dv{}{t} K(q^i, ~t)
\end{equation}
    Dall'altra parte, inserendo $\delta q^i(t) = \delta_s q^i(t)$ in \eqref{invonshell}, ovvero vincolando le le variazioni on-shell siano anche di simmetria, abbiamo
\begin{equation}\label{prova7}
    \delta S[\overline q^i,~\delta_s q^i] = \integ{t_1}{t_2}{t} \dv{}{t} \Big( \pdv{L}{\dot q^i} \delta q^i \Big )
\end{equation}
    In questi due passaggi abbiamo sostanzialmente vincolato che $delta_s q^i(t)$ sia una variazione di simmetria e dunque necessariamente soddisfi l'equazione \eqref{simmazione}, mentre che $\overline q^i(t)$ soddisfi le equazioni del moto \eqref{eullag}. Ricordiamo che prima soltanto uno dei due era soggetto a rispettare un'equazione, mentre l'altra era arbitraria, e inoltre che in questa maniera la variazione debba essere infinitesima. Notando che i membri sinistri delle \eqref{prova6} e \eqref{prova7} sono identici, abbiamo
\begin{equation}
    \delta[\overline q^i(t), ~\delta_s q^i(t)] = \integ{t_1}{t_2}{t} \dv{}{t} K(q^i, ~t) = \integ{t_1}{t_2}{t} \dv{}{t} \Big( \pdv{L}{\dot q^i} \delta q^i \Big )
\end{equation}
    A questo punto sottraiamo il secondo con il terzo membro, ottenendo la sottrazione nulla
\begin{equation}
    \integ{t_1}{t_2}{t} \dv{}{t} \Big (K - \pdv{L}{\dot q^i} \delta q^i \Big )
\end{equation}
    Infine avremo che l'integranda si annulla e, riconoscendi la definizione di carica \eqref{carica}, otteniamo la tesi
\begin{equation}
    \dv{}{t} \Big (K - \pdv{L}{\dot q^i} \delta q^i \Big ) = \dv{}{t} Q = 0
\end{equation}
\end{proof}
    Chiameremo d'ora in poi la quantità $Q$ carica conservata o carica di Noether associata alla simmetria. Chiudiamo il paragrafo precisando che sarebbe possibile ottenere tutte le quantità conservate direttamente dalle equazioni del moto ma che in questa tesi abbiamo scelto di ricavarle attraverso il primo teorema di Noether. 

\subsection{Esempi}

    Le applicazioni sono molto spesso fondamentali per comprendere meglio e a fondo la teoria. In questo paragrafo proponiamo quattro differenti esempi in cui applicheremo il teorema appena dimostrato. Il primo esempio sarà la particella conforme, ovvero una particella in cui il potenziale in cui è immerso dipende dall'inverso del  quadrato della distanza, e vedremo come sarà possibile risolvere il moto senza conoscerne le equazioni, soltanto utilizzando le carice di Noether. Nel secondo esempio studieremo la particella libera classica e ricaveremo le sue quantità conservate: energia, quantità di moto e momento angolare. Queste leggi di conservazione sono quelle che notoriamente si trovano in tutti i manuali di fisica di base, possono quindi essere un banco di prova per il nostro teorema e vederle da una propsettiva differente: ognuna di esse è conseguenza di una simmetria dello spazio-tempo. Il terzo esempio mostrerà come una particella in un campo di background uniforme non avrà una carica conservata associata ad una simmetria a causa proprio del campo ma sarà ugualmente possibile ottenere l'equazione differenziale che descrive la carica non conservata con l'aiuto dei risultati ottenuti. Infine il quarto esempio mostrerà come quando siamo in presenza di due particelle ad essere conservate non sono le singole cariche ma soltanti la carica totale: è la quantità di moto o il momento angolare totale a conservarsi, non quella delle singole particelle. 

\begin{example}[Particella conforme] 
    Consideriamo una particella di massa $m$ che può muoversi soltanto in una dimensione $x$, soggetta ad un potenziale $U$ dipendente dall'inverso del quadrato della posizione. La lagrangiana del sistema \eqref{lagrangiana}, scritta nella forma $T - U$, è
\begin{equation} \label{lag1}
    L = \frac{m}{2} \dot x^2 - \frac{\alpha}{x^2} 
\end{equation}
    e di conseguenza, usando la definizione \eqref{azione}, l'azione $S$ del sistema è
\begin{equation}
    S[x] = \integ{t_1}{t_2}{t} \Big( \frac{m}{2} \dot x^2 - \frac{\alpha}{x^2} \Big)
\end{equation}

    La prima simmetria che osserviamo è la traslazione temporale: l'azione non dipende esplicitamente dal tempo, quindi la trasformazione
\begin{equation*}
    t' = t + \epsilon
\end{equation*}
    dove $\epsilon$ è una costante, è una simmetria del funzionale di azione. Recuperando la \eqref{simmtempo}, ovvero come viene deformata la $x$ a seguito di una traslazione temporale,
\begin{equation*}
    \delta_s x = x(t + \epsilon) - x(t) = - \epsilon \dot x(t)
\end{equation*}
    calcoliamo il termine al bordo $K$ 
\begin{equation*}
\begin{aligned}
    \delta S[x] & = \integ{t_1}{t_2}{t} \Big( \frac{m}{2} \delta_s (\dot x^2) - \alpha \delta_s \Big (\frac{1}{x^2} \Big) \Big) = \integ{t_1}{t_2}{t} \Big( - \epsilon m \dot x \ddot x - \epsilon \alpha \frac{2 \dot x}{x^3} \Big) \\ & = \integ{t_1}{t_2}{t} \dv{}{t} \Big( -\epsilon \Big( \frac{m}{2} \dot x^2 - \frac{\alpha}{x^2} \Big) \Big)
\end{aligned}
\end{equation*}
    Dunque abbiamo trovato dalla \eqref{invazione}, che il termina al bordo $K$ è
\begin{equation*}
    K = - \epsilon \Big (\frac{m}{2} \dot x^2 - \frac{\alpha}{x^2} \Big)
\end{equation*}
    Ora applichiamo il primo teorema di Noether e troviamo la carica \eqref{carica} 
\begin{equation*}
    Q = K - \frac{\partial L}{\partial \dot x} \delta_s x = - \epsilon \Big (\frac{m}{2} \dot x^2 - \frac{\alpha}{x^2} \Big) - m \dot x (-\epsilon \dot x) = - \epsilon (\frac{m}{2} \dot x^2 + \frac{\alpha}{x^2} \Big)
\end{equation*}
    che a meno di un segno meno e di un fattore $\epsilon$ costante, è l'energia del sistema $E$
\begin{equation}\label{energia1}
    E = \frac{m}{2} \dot x^2 + \frac{\alpha}{x^2}
\end{equation}
    Troviamo quindi la prima relazione algebrica tra $x(t)$ e $\dot x(t)$. 

    La seconda simmetria che notiamo è quella di scala, chiamata anche simmetria di Weyl
\begin{equation*}
    t' = \lambda t \qquad x'(t') = \sqrt{\lambda} x(t)
\end{equation*} 
    dove $\lambda$ è una costante. Osserviamo dapprima che la velocità si trasforma nel seguente modo
\begin{equation*}
    \dot x' = \dv{x'}{t'} = \dv{\sqrt{\lambda}}{\lambda t} = \frac{1}{\sqrt{\lambda}} \dv{dx}{dt} = \frac{1}{\sqrt{\lambda}} \dot x
\end{equation*} 
    e quindi l'azione rimane invariante
\begin{equation*}
    \integ{t_1}{t_2}{t'} \Big( \frac{m}{2} \dot x'^2 - \frac{\alpha}{x'^2} \Big) = \integ{t_1}{t_2}{t} \lambda \Big( \frac{m}{2} \frac{\dot x^2}{\lambda} - \frac{\alpha}{\lambda x^2} \Big) = \integ{t_1}{t_2}{t} \Big( \frac{m}{2} \dot x^2 - \frac{\alpha}{x^2} \Big)
\end{equation*}
    Questa trasformazione mostra chiaramente che la simmetria non deve essere infinitesima, tuttavia per applicare il teorema di Noether, è necessario renderla tale. Scriviamo $\lambda$ come $\lambda = 1 + \epsilon$, dove $\epsilon$ è una quantità infinitesima. Espandendo al primo ordine in $\epsilon$, otteniamo la variazione di simmetria
\begin{equation*}
    x'(t') = x'((1+\epsilon)t) = x(t) + \frac{\epsilon}{2} x(t)
\end{equation*}
    D'altra parte abbiamo che 
\begin{equation*}
    x'(t') = x'((1+\epsilon)t) = x'(t) + \dot x(t) \epsilon t
\end{equation*}
    Mettendo insieme, otteniamo dunque
\begin{equation*}
    \delta_s x(t) = x'(t) - x(t) = x'(t') - \dot x(t) \epsilon t - x(t) = x(t) + \frac{\epsilon}{2} x(t) - \dot x(t) \epsilon t - x(t) = \frac{\epsilon}{2} x(t) - \dot x(t) \epsilon t 
\end{equation*}
    Alla luce di questo risultato, calcoliamo il termine al bordo $K$ 
\begin{equation}
\begin{aligned}
    \delta S[x] & = \integ{t_1}{t_2}{t} \Big( \frac{m}{2} \delta_s (\dot x^2) - \alpha \delta_s \Big (\frac{1}{x^2} \Big) \Big) \\ & = \integ{t_1}{t_2}{t} \Big( m \dot x \Big( \frac{\epsilon}{2} \dot x(t) - \ddot x(t) \epsilon t - \dot x(t) \epsilon \Big) + \frac{2 \alpha}{x^3} \Big( \frac{\epsilon}{2} x(t) - \dot x(t) \epsilon t \Big) \Big) \\ & = \integ{t_1}{t_2}{t} \epsilon \Big( \frac{m}{2} \dot x^2 - m t \dot x \ddot x - m \dot x^2 + \frac{\alpha x - 2 \alpha t \dot x}{x^3}\Big) \\ & = \integ{t_1}{t_2}{t} \epsilon \Big( - \frac{m}{2} \dot x^2 - m t \dot x \ddot x + \frac{\alpha x - 2 \alpha t \dot x}{x^3}\Big) \\ & = \integ{t_1}{t_2}{t} \dv{}{t} \Big ( - \frac{\epsilon t m \dot x^2}{2} + \frac{\alpha \epsilon t}{x^2} \Big)
\end{aligned}
\end{equation}
    e quindi abbiamo trovato, dalla \eqref{invazione}, che il termina al bordo $K$ è 
\begin{equation*}
    K = - \epsilon t \Big (\frac{m}{2} \dot x^2 - \frac{\alpha}{x^2} \Big)
\end{equation*}
    Ora applichiamo il primo teorema di Noether e otteniamo la carica \eqref{carica} 
\begin{equation}
\begin{aligned}
    Q & = K - \frac{\partial L}{\partial \dot x} \delta_s x = - \epsilon t \Big (\frac{m}{2} \dot x^2 - \frac{\alpha}{x^2} \Big) - m \dot x \Big (\frac{\epsilon}{2} x(t) - \dot x(t) \epsilon t \Big ) \\ & = - \epsilon \Big ( \frac{m x \dot x }{2} + t \frac{m \dot x^2}{2} - t \frac{\alpha}{x^2} \Big) = - \epsilon \Big ( \frac{m x \dot x}{2} - t \Big ( \frac{m}{2} \dot x^2 + \frac{\alpha}{x^2} \Big ) \Big )
\end{aligned}
\end{equation}
    che a meno di un segno meno e di un fattore $\epsilon$ costante, è la carica di Noether associata alla simmetria di Weyl
\begin{equation} \label{carica1}
    Q = \frac{m x \dot x}{2} - t \Big( \frac{m}{2} \dot x^2 + \frac{\alpha}{x^2} \Big)
\end{equation}
    Notiamo che almeno una carica deve avere una dipendenza esplicita dal tempo, altrimenti non ci sarebbe dinamica del sistema. Nel nostro caso è $Q$.
    
    Una volta trovate le 2 cariche di Noether, che definiscono due equazioni algebriche che legano $x(t)$ e $\dot x(t)$, è possibile risolvere il moto del sistema senza scrivere l'equazione del moto esplicita. Innanzitutto notiamo che gli ultimi due termini della \eqref{carica1} sono uguali all'energia \eqref{energia1} moltiplicata per t, e quindi $Q$ può esser riscritta come
\begin{equation*}
    Q = \frac{m x \dot x}{2} - Et
\end{equation*}
    Abbiamo ottenuto un'equazione differenziale, che separando le variabili, è possibile risolvere e trovare la soluzione dell'equazione del moto
\begin{equation} \label{soleqmoto}
    x = \pm \sqrt{\frac{4Qt + 2Et^2}{m}}
\end{equation}

    Per controllare il nostro lavoro, ricaviamo l'equazione del moto e verifichiamo che in effetti la nostra soluzione la soddisfa. Applicando le equazioni di Eulero-Lagrange \eqref{eullag} alla nostra Lagrangiana \eqref{lag1}, abbiamo
\begin{equation*}
    0 = \pdv{L}{x}  - \dv{}{t} \pdv{L}{\dot x} = \pdv{}{x} \Big(\frac{m}{2} \dot x^2 - \frac{\alpha}{x^2} \Big) - \dv{}{t} \pdv{}{\dot x} \Big(\frac{m}{2} \dot x^2 - \frac{\alpha}{x^2} \Big) = \frac{2 \alpha}{x^3} - \dv{}{t} (m \dot x) = \frac{2 \alpha}{x^3} - m \ddot x 
\end{equation*}
    e quindi l'equazione del moto è
\begin{equation}\label{eqmoto1}
    m \ddot x = \frac{2 \alpha}{x^3}
\end{equation}
    È dunque possibile verificare che \eqref{soleqmoto} soddisfi \eqref{eqmoto1}.
\end{example}

\begin{example}[Le leggi di conservazione della particella libera]
    Consideriamo una particella di massa $m$ libera di spostarsi nello spazio tridimensionale, ovvero in cui non è presente nessun potenziale $U$. In questo caso, la lagrangiana del sistema \eqref{lagrangiana} coinciderà con l'energia cinetica 
\begin{equation} \label{lag2}
    L = \frac{m}{2} \mathbf{\dot x} = \frac{m}{2} (\dot x^2 + \dot y^2 + \dot z^2)
\end{equation}
    e di conseguenza, usando la definizione \eqref{azione}, l'azione $S$ del sistema sarà
\begin{equation}
    S[x] = \integ{t_1}{t_2}{t} \frac{m}{2} \mathbf{\dot x} = \integ{t_1}{t_2}{t} \frac{m}{2} (\dot x^2 + \dot y^2 + \dot z^2)
\end{equation}
    Dalla fisica di base sappiamo che questa particella obbedirà alle leggi di conservazione della fisica classica. Attraverso il primo teorema di Noether è possibile mostrare come queste quantità derivino da simmetrie spaziotemporali della particella
\begin{enumerate}
    \item Energia $E = \frac{m}{2} \dot x^2$ che deriva dalla simmetria rispetto alle traslazioni temporali (omogeneità del tempo),
    \item Quantità di moto $\mathbf p = \mathbf{\dot x}$ che deriva dalla simmetria rispetto alle traslazioni spaziali (omogeneità dello spazio),
    \item Momento angolare $\mathbf L = \mathbf x \times m \mathbf{\dot x}$ che deriva dalla simmetria rispetto alle rotazioni spaziali (isotropia dello spazio).
\end{enumerate}

\subsubsection{1. Energia}
    La prima simmetria che andremo a indagare sarà una traslazione temporale
\begin{equation}
    t' = t + \epsilon
\end{equation}
    dove $\epsilon$ è una costante. L'azione è invariante perchè la lagrangiana non dipende esplicitamente dal tempo $t$. Recuperando la variazione di simmetria \eqref{simmtempo}, calcoliamo la variazione dell'azione
\begin{equation}
\begin{aligned}
    \delta S[x] & = \integ{t_1}{t_2}{t} \frac{m}{2} \delta_s (\dot x^2) = \integ{t_1}{t_2}{t} ( - \epsilon m \dot x \ddot x) = \integ{t_1}{t_2}{t} \dv{}{t} \Big( -\epsilon \frac{m}{2} \dot x^2 \Big)
\end{aligned}
\end{equation}
    e quindi abbiamo trovato, dalla \eqref{invazione}, che il termina al bordo $K$ è 
\begin{equation*}
    K = - \epsilon \frac{m}{2} \dot x^2 
\end{equation*}
    Ora applichiamo il primo teorema di Noether e otteniamo la carica \eqref{carica} 
\begin{equation}
    Q = K - \frac{\partial L}{\partial \dot x} \delta_s x = - \epsilon \frac{m}{2} \dot x^2 - m \dot x (-\epsilon \dot x) = \epsilon \frac{m}{2} \dot x^2
\end{equation}
    che a meno di un fattore $\epsilon$ costante, è l'energia del sistema $E$
\begin{equation*}
    E = \frac{m}{2} \dot x^2
\end{equation*}
    Dunque abbiamo dimostrato che una simmetria di traslazione temporale comporta la conservazione dell'energia $E$.

\subsubsection{2. Quantità di moto}
    La seconda simmetria che andremo a studiare sarà una traslazione spaziale
\begin{equation}
    \mathbf x' = \mathbf x + \mathbf a
\end{equation}
    dove $\mathbf a = (a_x,~a_y,~a_z)$ è un vettore dalle componenti costanti. L'azione è invariante perchè la lagrangiana non dipende esplicitamente dalla posizione $\mathbf{x}$. La variazione di simmetria è 
\begin{equation*}
    \delta_s \mathbf x = \mathbf a
\end{equation*}
    Calcoliamo la variazione dell'azione
\begin{equation}
    \delta S[x] = \integ{t_1}{t_2}{t} \frac{m}{2} \delta_s \mathbf{\dot x^2} = 0
\end{equation}
    e quindi abbiamo trovato, dalla \eqref{invazione}, che il termina al bordo $K$ è nullo
\begin{equation*}
    K = 0
\end{equation*}
    Ora applichiamo il primo teorema di Noether e otteniamo la carica \eqref{carica} 
\begin{equation}
    Q = K - \frac{\partial L}{\partial \mathbf{\dot x}} \cdot \delta_s \mathbf x = - m \mathbf{\dot x} \cdot \mathbf a
\end{equation}
    che è la quantità di moto del sistema $\mathbf p$ lungo la direzione della traslazioen $\mathbf a$ a meno di un fattore costante $\|\mathbf a \|$
\begin{equation*}
    \mathbf p = m \mathbf{\dot x}
\end{equation*}
    Dunque abbiamo dimostrato che una simmetria di traslazione spaziale comporta la conservazione della quantitò di moto $\mathbf p$.

\subsubsection{3. Momento angolare}
    La terza simmetria che andremo a indagare sarà una rotazione spaziale, che per semplicità prendiamo rispetto ad un asse fissato con angolo $\theta$
\begin{equation}
    \mathbf x' = R(\theta) \mathbf x + \mathbf a
\end{equation}
    dove $R(\theta)$ è la matrice di rotazione. L'azione è invariante perchè una rotazione lascia invariata la norma del vettore velocità $\mathbf{\dot x^2}$. A questo punto rendiamo infinitesima la rotazione, introducendo il vettore $\mathbf \omega = (\omega_x,~\omega_y,~\omega_z)$
\begin{equation}
    \mathbf x' = \mathbf x + \mathbf \omega \times \mathbf x
\end{equation}
    La variazione di simmetria 
\begin{equation*}
    \delta_s \mathbf x = \mathbf \omega \times \mathbf x
\end{equation*}
    Calcoliamo la variazione dell'azione
\begin{equation}
    \delta S[x] = \integ{t_1}{t_2}{t} \frac{m}{2} \delta_s \mathbf{\dot x^2} = \integ{t_1}{t_2}{t} \frac{m}{2} \mathbf{\dot x} \cdot (\omega \times \mathbf{\dot x} ) = 0
\end{equation}
    e quindi abbiamo trovato, dalla \eqref{invazione}, che il termina al bordo $K$ è nullo
\begin{equation*}
    K = 0
\end{equation*}
    Ora applichiamo il primo teorema di Noether e otteniamo la carica \eqref{carica} 
\begin{equation}
    Q = K - \frac{\partial L}{\partial \mathbf{\dot x}} \cdot \delta_s \mathbf x = - m \mathbf{\dot x} \cdot \omega \times \mathbf x 
\end{equation}
    che è il momento angolare del sistema $\mathbf L$ lungo la direzione dell'asse di rotazione $\mathbf \omega$ a meno di un fattore costante $\|\mathbf \omega\|$ e di un segno meno
\begin{equation*}
    \mathbf L = m \mathbf{\dot x} \times \mathbf x 
\end{equation*}
    Dunque abbiamo dimostrato che una simmetria di rotazione spaziale comporta la conservazione del momento angolare $\mathbf L$.
\end{example}

\begin{example}[Particella in campo di background uniforme] 
    Consideriamo una particella di massa $m$ immersa in un campo uniforme il cui potenziale dipendente dalla coordinata z. 
\begin{equation}\label{enpot3}
    U(x) = B z
\end{equation}
    dove $B$ è una costante. Scrivendo la Lagrangiana nella forma $T - U$, sostituendo \eqref{enpot3}, otteniamo 
\begin{equation} \label{lag3}
    L = \frac{m}{2} \mathbf{\dot x} - B z
\end{equation}
    e di conseguenza, usando la definizione \eqref{azione}, l'azione $S$ del sistema sarà
\begin{equation}
    S[x] = \integ{t_1}{t_2}{t} \Big( \frac{m}{2} \mathbf{\dot x} - B z \Big)
\end{equation}
    Riscriviamo il potenziale in modo differente, introducendo il vettore $\mathbf B = (0,~0,~B)$ e riscrivendo l'azione come
\begin{equation}
    S[x] = \integ{t_1}{t_2}{t} \Big( \frac{m}{2} \mathbf{\dot x} - \mathbf B \cdot \mathbf x \Big)
\end{equation}
    Notiamo che questo applicando una rotazione, sia il primo termine che il secondo termine sono invarianti, mostrando quindi una simmetria rotazionale. Infatti essendo entrambi prodotti scalari, il primo tra $\mathbf x$ e $\mathbf x$ mentre il secondo tra $\mathbf B$ e $\mathbf x$, sono scalari e dunque invarianti per trasformazione di coordinate. Tuttavia ciò non implica che la carica di Noether associata a questa simmetria, ovvero il momento angolare, si conservi. Ciò è dovuto alla presenza di un campo in background. Consideriamo una rotazione infinitesima come variazione di simmetria sia delle coordinate 
\begin{equation}
    \delta_s \mathbf x = \omega \times \mathbf x
\end{equation}
    che del campo
\begin{equation}
    \delta_s \mathbf B = \omega \times \mathbf B
\end{equation}
    il cui vettore $\omega$ è lo stesso introdotto precedentemente. Derivando la variazione delle coordinate, abbiamo
\begin{equation}
    \delta_s \mathbf{\dot x} = \omega \times \mathbf{\dot x}
\end{equation}
    
    Alla luce di questo risultato, calcoliamo il termine al bordo $K$ 
\begin{equation}
\begin{aligned}
    \delta S[x] & = \integ{t_1}{t_2}{t} \Big( \frac{m}{2} \delta_s \mathbf{\dot x^2} - \delta_s (\mathbf B \cdot \mathbf x ) \Big ) \\ & = \integ{t_1}{t_2}{t} \Big( \frac{m}{2} \mathbf{\dot x} \cdot (\omega \times \mathbf{\dot x} ) - \mathbf B \cdot \omega \times \mathbf x - \omega \times \mathbf B \cdot \mathbf x \Big ) = 0
\end{aligned}
\end{equation}
    che risulta essere nullo essendo tutti prodotti scalari tra vettori perpendicolari
\begin{equation*}
    K = 0
\end{equation*}

    Calcoliamo l'equazione del moto applicando le equazioni di Eulero-Lagrange \eqref{eullag} alla nostra Lagrangiana \eqref{lag3}, abbiamo
\begin{equation*}
    0 = \pdv{L}{\mathbf x}  - \dv{}{t} \pdv{L}{\mathbf{\dot x}} = \pdv{}{\mathbf x} \Big(\frac{m}{2} \mathbf{\dot x} - \mathbf B \cdot \mathbf x \Big) - \dv{}{t} \pdv{}{\mathbf{\dot x}} \Big(\frac{m}{2} \mathbf{\dot x} - \mathbf B \cdot \mathbf x \Big) = \mathbf B - m \mathbf{\ddot x}
\end{equation*}
    e quindi l'equazione del moto è
\begin{equation}\label{eqmoto3}
    m \mathbf{\ddot x} = \mathbf B
\end{equation}

    Effettuiamo ora la variazione on-shell, utilizzando le equazioni del moto
\begin{equation*}
    \delta S = \integ{t_1}{t_2}{t} \Big( \frac{m}{2} \delta \mathbf{\dot x^2} - \delta (\mathbf B \cdot \mathbf x ) \Big ) = \integ{t_1}{t_2}{t} \Big( m \mathbf{\ddot x } \cdot \delta \mathbf{\dot x} - \delta \mathbf B \cdot \mathbf x - \mathbf B \cdot \delta \mathbf x \Big )
\end{equation*} 
    Usando l'identità
\begin{equation*}
    m \mathbf{\ddot x} \cdot \delta \mathbf x = \dv{}{t} (m \mathbf{\dot x} \cdot \delta \mathbf x) - m \mathbf{\dot x} \cdot \delta \mathbf{\dot x}
\end{equation*}
    riscriviamo la variazione dell'azione
\begin{equation*}
    \delta S = \integ{t_1}{t_2}{t} \Big( \dv{}{t} (m \mathbf{\dot x} \cdot \delta \mathbf x) - m \mathbf{\dot x} \cdot \delta \mathbf{\dot x} - \delta \mathbf B \cdot \mathbf x - \mathbf B \cdot \delta \mathbf x \Big )
\end{equation*} 
    Sostituiamo ora le variazioni
\begin{equation*}
    \delta S = \integ{t_1}{t_2}{t} \Big( \dv{}{t} (m \mathbf{\dot x} \cdot \omega \times \mathbf x) - m \mathbf{\dot x} \cdot \omega \times \mathbf{\dot x} - \omega \times \mathbf B \cdot \mathbf x - \mathbf B \cdot \omega \times \mathbf x \Big ) = 
\end{equation*} 
    ed eliminando i termini nulli, otteniamo
\begin{equation*}
    \delta S = \integ{t_1}{t_2}{t} \Big( \dv{}{t} (m \mathbf{\dot x} \cdot \omega \times \mathbf x) - \mathbf B \cdot \omega \times \mathbf x \Big )
\end{equation*} 
    Riconoscendo il momento angolare $\mathbf L$, abbiamo
\begin{equation*}
    \delta S = \integ{t_1}{t_2}{t} \omega \cdot \Big( \dv{\mathbf L}{t} - \mathbf x \times \mathbf B \Big)
\end{equation*}
    Dunque, ponendo la simmetria uguale a zero, abbiamo ottenuto l'equazione differenziale che governa come varia il momento angolare, che quindi non si conserva. 
\begin{equation*}
    \dv{\mathbf L}{t} = \mathbf x \times \mathbf B 
\end{equation*}
    Infine, affermiamo dunque che questa non è una simmetria dal punto di vista di Noether perchè la sua variazione non è una derivata totale ma presenta in termine ulteriore dipendente dal campo $B$.
\end{example}

\begin{example}[Due particelle in campo centrale] 
    Consideriamo due particelle rispettivamente di massa $m_1$ e $m_2$ tali che la prima particella $q$ sia immersa nel campo molto più intenso della seconda particella $Q$ ma che quest'ultima sia ferma. La Lagrangiana sarà dunque 
\begin{equation}
    L = \frac{m_1}{2} \mathbf{\dot x_1^2} - \frac{qQ}{|\mathbf x_1 - \mathbf x_2|}
\end{equation}
    e di conseguenza, usando la definizione \eqref{azione}, l'azione $S$ del sistema sarà
\begin{equation}
    S[x] = \integ{t_1}{t_2}{t} \Big( \frac{m_1}{2} \mathbf{\dot x_1^2} - \frac{qQ}{|\mathbf x_1 - \mathbf x_2|} \Big)
\end{equation}
    A questo punto, notiamo che l'azione non è invariante per la trasformazione 
\begin{equation}
    \mathbf x'_1 = \mathbf x_1 + \mathbf a
\end{equation}
    dove $\mathbf a$ è una costante. Essendo questa una traslazione spaziale, dunque possiamo concludere che la quantità di moto $\mathbf p_1$ della prima particella non si conserva. Tuttavia, aggiungendo anche l'energia cinetica della seconda particella e quindi la sua possibilità di muoversi, otteniamo una Lagrangiana  
\begin{equation}
    L = \frac{m_1}{2} \mathbf{\dot x_1^2} + \frac{m_2}{2} \mathbf{\dot x_2^2} - \frac{qQ}{|\mathbf x_1 - \mathbf x_2|}
\end{equation}
    e di conseguenza l'azione $S$ del sistema diventerà
\begin{equation}
    S[x] = \integ{t_1}{t_2}{t} \Big( \frac{m_1}{2} \mathbf{\dot x_1^2} + \frac{m_2}{2} \mathbf{\dot x_2^2} - \frac{qQ}{|\mathbf x_1 - \mathbf x_2|} \Big)
\end{equation}
    Ora, notiamo che la traslazione spaziale 
\begin{equation}
    \mathbf x'_1 = \mathbf x_1 + \mathbf a \qquad \mathbf x'_2 = \mathbf x_2 + \mathbf a
\end{equation}
    è una simmetria per cui l'azione è invariante. Come consequenza, abbiamo la conservazione della quantità di moto totale del sistema
\begin{equation*}
    \mathbf P = \mathbf p_1 + \mathbf p_2 = m_1 \mathbf{\dot x_1} + m_2 \mathbf{\dot x_2}
\end{equation*}
    e non delle singole quantità di moto. Ciò si può vedere notando che le energie cinetiche sono invarianti per singole traslazioni $\mathbf x'_1 = \mathbf x_1 + \mathbf a_1$ e $\mathbf x'_2 = \mathbf x_2 + \mathbf a_2$ ma il potenziale è invariante soltanto sotto la condizione $\mathbf a_1 = \mathbf a_2 = \mathbf a$. Analoghi ragionamenti possono essere ripetuti per il momento angolare.
\end{example}

\section{Formalismo hamiltoniano}

    Il formalismo lagrangiano non è l'unico che possiamo utilizzare per scrivere e risolvere un sistema meccanico. Le equazioni di Eulero-Lagrange sono un sistema di equazioni differenziali al secondo ordine, la cui soluzione descrive una traiettoia nello spazio delle configurazioni generato da N variabili indipendenti. Il formalismo hamiltoniano invece fornisce una equivalente descrizione, utilizzando come equazioni del moto un sistema di 2N equazioni differenziali del primo ordine, la cui soluzione è una traiettoria in un nuovo spazio, lo spazio delle fasi, formato dalle coordinate e dai momenti associati che abbiamo introdotto \eqref{coniugato}, e quindi a 2N variabili indipendenti. Vediamo come è possible passare dalla formulazione lagrangiana a quella hamiltoniana, attraverso un cambio di coordinate che permette di parametrizzare lo spazio delle fasi. 
    
\subsection{Equazioni di Hamilton}

    Consideriamo un sistema meccanico descritto da una Lagrangiana \eqref{lagrangiana} nello spazio delle configurazioni generato dalle coordinate generalizzate $q^i$. Supponiamo inoltre di trattare con Lagrangiane non singolari. Ricordando la definizione di momento coniugato 
\begin{equation*}
    p_i = \pdv{L(q^i, ~\dot q^i)}{\dot q^i} 
\end{equation*}
    ed effettuiamo un cambio di variabili, da $(q, ~\dot q, ~t)$ a $(q, ~p, ~t)$, ovvero con $\dot q = \dot q (q, ~ p)$. Dalle equazioni di Eulero-Lagrange \eqref{eullag}, osserviamo che 
\begin{equation} \label{proof13}
    \pdv{L}{q^i} = \dv{}{t} \pdv{L}{\dot q^i} = \dv{}{t} p_i = \dot p_i
\end{equation}
    Calcoliamo la derivata parziale della Lagrangiana nello spazio delle fasi rispetto alle coordinate
\begin{equation*}
    \pdv{L(q^i, ~p_j)}{q^i} = \pdv{L(q^i, ~\dot q^i)}{q^i} + \pdv{L(q^i, ~ \dot q^i)}{\dot q^j} \pdv{\dot q^i}{q^i} = \pdv{L(q^i, ~\dot q^i)}{q^i} + p_i \pdv{\dot q^i}{q^i} 
\end{equation*}
    dove nell'ultima uguaglianza abbiamo usato la definizione di momento coniugato \eqref{coniugato}. Isolando la derivata della Lagrangiana nello spazio delle configurazioni, otteniamo 
\begin{equation} \label{proof14}
    \pdv{L(q^i, ~\dot q^i)}{q^i} = - \pdv{}{q^i} (p_j \dot q^j - L(q, ~p))
\end{equation}
    A questo punto, utilizziamo una trasformazione di Legendre che ci permette di passare dalla Lagrangiana definita nello spazio delle configurazioni alla Hamiltoniana nello spazio delle fasi, una funzione delle coordinate $q^i$, dei momenti $p_j$ ed eventualmente del tempo $t$
\begin{equation} \label{hamiltoniana}
    H(q, ~p) = p_j \dot q^j - L
\end{equation} 
    La prima equazione del moto può essere ricavata mettendo insieme \eqref{proof14}, sostituendo la definizione di Hamiltoniana \eqref{hamiltoniana}, con \eqref{proof13}
\begin{equation*}
    \pdv{L(q^i, ~\dot q^i)}{q^i} = p_i = - \pdv{}{q^i} (p_j \dot q^j - L(q, ~p)) = \pdv{H}{q^i}
\end{equation*}
    La seconda equazione invece si deduce derivando l'Hamiltoniana rispetto al momento coniugato, utilizzando ancora la definizione di momento coniugato \eqref{coniugato}
\begin{equation*}
\begin{gathered}
    \pdv{H}{p_i} = \pdv{}{p_i} (p_j \dot q^j - L) = \pdv{p_j}{p_i} \dot q^j + p_j \pdv{\dot q^j}{p_i} - \pdv{L}{p_i} = \delta^i_{\phantom i j} \dot q^j + p_j \pdv{\dot q^j}{p_i} - \pdv{L}{\dot q^j} \pdv{\dot q^j}{p_i} \\ = \delta^i_{\phantom i j} \dot q^j + \cancel{p_j \pdv{\dot q^j}{p_i}} - \cancel{p_j \pdv{\dot q^j}{p_i}} = \dot q^i
\end{gathered}
\end{equation*}
    Ricapitolando, le equazioni di Hamilton, ovvero le equazioni che governano il moto e che ci dicono come evolvono le 2N variabili indipendenti $q^i$ e $p_j$ sono
\begin{equation} \label{ham1}
    \dot q^i = \pdv{H}{p_i}
\end{equation}
\begin{equation} \label{ham2}
    \dot p_j = - \pdv{H}{q^j}
\end{equation}
    Completiamo le equazioni introducendo la dipendenza sia della Lagrangiana che dell'Hamiltoniana dal tempo e vedendo come dipendono le derivate temporali tra loro. Nel farlo utilizziamo un altro modo di calcolare le equazioni del moto, attraverso l'uso dei differenziali. Ricordiamo che la Lagrangiana è definita come $L = L(q^i, ~\dot q^i, ~t)$ e quindi il suo differenziale sarà
\begin{equation*}
    dL = \pdv{L}{q^i} dq^i + \pdv{L}{\dot q^i} d\dot q^i + \pdv{L}{t} dt 
\end{equation*}
    Sostituendo \eqref{coniugato} e \eqref{proof13}, otteniamo 
\begin{equation*}
    dL = \dot p_i dq^i + p_i d\dot q^i + \pdv{L}{t} dt 
\end{equation*}
    Ora calcoliamo il differenziale della Hamiltoniana \eqref{hamiltoniana} utilizzando la 
\begin{equation*}
    dH = \dot q^i dp_i - d \dot q^i p_i - dL = \dot q^i dp_i + \cancel{d \dot q^i p_i} - \dot p_i dq^i - \cancel{p_i d\dot q^i} - \pdv{L}{t} dt = \dot q^i dp_i - \dot p_i dq^i - \pdv{L}{t} dt
\end{equation*}
    e notiamo che può anche essere riscritto attraverso le derivate parziali 
\begin{equation*}
    dH = \pdv{H}{q^i} dq^i + \pdv{H}{p_i} dp_i + \pdv{H}{t} dt
\end{equation*}
    Mettendo insieme otteniamo le 2N+1 equazioni differenziali, le 2N equazioni di Hamilton 
\begin{equation*}
    \dot q^i = \pdv{H}{p_i} \qquad \dot p_j = - \pdv{H}{q^j}
\end{equation*}
    con l'aggiunta della ricercata dipendenza dal tempo
\begin{equation*}
    \pdv{L}{t} = - \pdv{H}{t}
\end{equation*}

\hfill 

    Solitamente, si usa esprimere il formalismo hamiltoniano introducendo le parentesi di Poisson. Date due funzioni $f(q, ~p)$ e $g(q, ~p)$ nello spazio delle fasi, definiamo 
\begin{equation}
    \poi{f}{g} = \pdv{f}{q^i} \pdv{g}{p_i} - \pdv{g}{q^i} \pdv{f}{p_i}
\end{equation}
    Dalla definizione è possibile dimostrare le seguenti proprietà, dove $h(q, ~p)$ è un'altra funzione nello spazio delle fasi e $c, ~c_1, ~c_2$ sono costanti
\begin{enumerate}
    \item Antisimmetria
\begin{equation}
    \poi{f}{g} = \poi{g}{f}
\end{equation}
    \item Bilinearità
\begin{equation}
    \poi{c_1f_1+c_2f_2}{g} = c_1 \poi{f_1}{g} + c_2 \poi{f_2}{g}
\end{equation}
\begin{equation}
    \poi{f}{c_1g_1+c_2g_2} = c_1 \poi{f}{g_1} + c_2 \poi{f}{g_2}
\end{equation}
    \item Regola di Leibniz
\begin{equation}
    \poi{fg}{h} = f \poi{g}{h} + g \poi{f}{h}
\end{equation}
\begin{equation}
    \poi{f}{gh} = g \poi{f}{h} + h \poi{f}{g}
\end{equation}
    \item Identità di Jacobi
\begin{equation}
    \poi{f}{\poi{g}{h}} + \poi{g}{\poi{h}{f}} + \poi{h}{\poi{f}{g}} = 0
\end{equation}
\end{enumerate}
    Se le funzioni definite nello spazio delle fasi sono le coordinate o i momenti, le parentesi di Poisson vengono chiamate parentesi canoniche ed esplicitamente sono 
\begin{equation} \label{canq}
    \poi{q^i}{q_j} = 0
\end{equation}
\begin{equation} \label{canp}
    \poi{p^i}{p_j} = 0
\end{equation}
\begin{equation} \label{canqq}
    \poi{q^i}{p_j} = \delta^i_{\phantom i j}
\end{equation}
    Attraverso le parentesi di Poisson possiano riscrivere le equazioni di Hamilton. Infatti se sviluppiamo esplicitamente la \eqref{ham1}, otteniamo 
\begin{equation*}
    \dot q^i = \pdv{H}{p_i} = \delta^i_{\phantom i k} \pdv{H}{p_k} - 0 = \pdv{q^i}{q^k} \pdv{H}{p_k} - \pdv{H}{q^k} \pdv{q^i}{p_k} = \poi{q^i}{H}
\end{equation*}
    mentre analogamente se sviluppiano esplicitamente la \eqref{ham2}, abbiamo
\begin{equation*}
    \dot p_i = - \pdv{H}{q^i} = \delta^i_{\phantom i k} \pdv{H}{q^k} - 0 = \pdv{p_i}{p_k} \pdv{H}{q^k} - \pdv{H}{p_k} \pdv{p_i}{q^k} = - \Big( \pdv{p_i}{q^k} \pdv{H}{p_k} - \pdv{H}{q^k} \pdv{p_i}{p_k} \Big) = - \poi{p_i}{H}
\end{equation*}
    Ricapitolando, le equazioni di Hamilton diventano 
\begin{equation}
    \dot q^i = \poi{q^i}{H}
\end{equation}
\begin{equation}
    \dot p_j = - \poi{p_j}{H}
\end{equation}  

    Concludiamo il paragrafo, mostrando che è possibile ricavare le equazioni di Hamilton associando un funzionale di azione. Infatti, partendo dalla trasformazione di Legendre
\begin{equation*}
    H(q^i, ~p_j, ~t) = p_i \dot q^i - L
\end{equation*}
    isoliamo la Lagrangiana e definiamo un funzionale di azione nello spazio delle fasi nel seguente modo 
\begin{equation*}
    S[q, ~p] = \integ{t_1}{t_2}{t} (p_i \dot q^i - H(q^i, ~p_j, ~t))
\end{equation*}
    Calcoliamo la sua variazione con le stesse tecniche sviluppate precedentemente
\begin{equation*}
    \delta S = \integ{t_1}{t_2}{t} \Big ( \delta p_i \dot q^i + p_i \delta \dot q^i - \pdv{H}{q^i} \delta q^i - \pdv{H}{p_i} \delta p_i \Big )
\end{equation*}
    Usando l'identità 
\begin{equation*}
    p_i \delta \dv{}{t} q^i = p_i \dv{}{t} \delta q^i = \dv{}{t} (p_i \delta q^i) - \dv{}{t} (p^i) \delta q^i = \dv{}{t} (p_i \delta q^i) - \dot p^i \delta q^i
\end{equation*}
    integriamo per parti, ponendo nulli i termini agli estremi, ottenendo
\begin{equation*}
\begin{aligned}
    \delta S & = \integ{t_1}{t_2}{t} \Big ( \delta p_i \dot q^i - \dot p^i \delta q^i - \pdv{H}{q^i} \delta q^i - \pdv{H}{p_i} \delta p_i \Big) \\ &= \integ{t_1}{t_2}{t} \Big ( \delta p_i \dot q^i - \dot p^i \delta q^i - \pdv{H}{q^i} \delta q^i - \pdv{H}{p_i} \delta p_i \Big) 
\end{aligned}
\end{equation*}
    Raccogliendo e ponendo la variazione nulla
\begin{equation*}
    \delta S = \integ{t_1}{t_2}{t} \Big ( \delta p_i \Big( \dot q^i - \pdv{H}{p_i} \Big ) - \delta q^i \Big(\dot p_i + \pdv{H}{q^i} \Big ) \Big) = 0
\end{equation*} 
    osserviamo che per il lemma fondamentale del calcolo delle variazione, otteniamo le equazioni cercate. 

    Passando dallo spazio delle configurazioni allo spazio delle fasi, è possibile formulare il teorema di Noether anche in meccanica hamiltoniana. Consideriamo una trasformazione di Legendre che porta un'azione $S[q^i, p_j]$ definita come 
\begin{equation}
    I[q^i, p_j] = \int dt ~ (p_j \dot q^i - H(q, p))
\end{equation}
    Definendo le parentesi di Poisson di due funzioni $F(q, p)$ e $G(q, p)$
\begin{equation}
    [F, G] = \frac{\partial F }{\partial q^i} \frac{\partial G }{\partial p_i} - \frac{\partial F }{\partial p_i} \frac{\partial G }{\partial q^i} 
\end{equation}
    le equazioni del moto diventano 
\begin{equation}
    \dot q^i = \frac{\partial H} {\partial p_i} = [q^i, H]
\end{equation}
\begin{equation}
    \dot p_i = - \frac{\partial H} {\partial q^i} = [p_i, H]
\end{equation}
    Data una funzione $Q(q, p, t)$, la sua derivata rispetto al tempo può essere scritta come
\begin{equation}
    \frac{dQ(q, p, t)}{dt} = [Q, H] + \frac{\partial Q}{\partial t}
\end{equation}
    Scegliendo la carica conservata $Q$ tale che soddisfi \eqref{th} otteniamo
\begin{equation}
    \dv{}{t} Q = [Q, H] + \frac{\partial Q}{\partial t} = 0
\end{equation}

\subsection{Il teorema inverso e algebra di Lie delle cariche}
\begin{theorem}
    Sia $Q$ una carica conservata, allora la seguente trasformazione 
\begin{equation}
    \delta_s q^i = [q^i, \epsilon Q] = \epsilon \frac{\partial Q}{\partial p_i} \qquad \delta_s p_i = [q_i, \epsilon Q] = - \epsilon \frac{\partial Q}{\partial q^i}
\end{equation}
    è una simmetria dell'azione.
\end{theorem}

\begin{proof}
    Variando l'azione
\begin{equation}
\begin{gathered}
    \delta S = \int dt ~  (\delta_s p \dot q + p \frac{d}{dt} \delta_s q - \frac{\partial H}{\partial p} \delta_s p - \frac{\partial H}{\partial q} \delta_s q) \\ = \int dt ~ (- \epsilon \frac{\partial Q}{\partial q} \dot q + \frac{d}{dt} (p \delta_s q) - \epsilon \dot p \frac{\partial Q}{\partial p} + \epsilon \frac{\partial H}{\partial p} \frac{\partial Q}{\partial q} - \epsilon \frac{\partial H}{\partial q}\frac{\partial Q}{\partial p}) \\ = \int dt ~ (\epsilon (- \frac{dQ}{dt} + \frac{\partial Q}{\partial t} + [Q, H] ) + \frac{d}{dt} (p \delta_s q)) = \int dt ~ \frac{d}{dt}(-\epsilon Q + p \delta_s q)
\end{gathered}
\end{equation}
    che essendo una derivata totale, completa il teorema
\end{proof}

\begin{theorem}
    L'insieme di tutte le cariche conservate $Q_a$ con $a = 1, 2, \ldots N$ soddisfano un algebra di Lie chiusa
\begin{equation}
[Q_a, Q_b] = f_{ab}^{\phantom{ab}c} Q_c
\end{equation}
    dove $f_{ab}^{\phantom{ab}c}$ sono le costanti di struttura.
\end{theorem}

\begin{proof}
    Le parentesi di Poisson di due cariche conservate $Q_1$ e $Q_2$ si conservano
\begin{equation}
\frac{d}{dt} [Q_1, Q_2] = [[Q_1, Q_2], H] + \frac{\partial}{\partial t} [Q_1, Q_2] = 0
\end{equation} 
Quindi a prescindere da quale sia il risultato delle parentesi, sicuramente si conserva e quindi definisce un'algebra di Lie
\begin{equation}
[Q_a, Q_b] = f_{ab}^{\phantom{ab}c} Q_c
\end{equation}
\end{proof}

\subsection{Esempi}
    Riproponiamo lo stesso esempio della particella conforme, utilizzata nel formalismo lagrangiano, ma questa volta studiata nella descrizione lagrangiana. L'azione hamiltoniana è 
\begin{equation}
    S = \int dt ~ \Big (p \dot q \Big( \frac{p^2}{2m} + \frac{\alpha}{q^2} \Big) \Big)
\end{equation}
    e presenta tre differenti cariche conservate 
\begin{equation}
    H = \frac{p^2}{2m} + \frac{\alpha}{q^2}
\end{equation}
\begin{equation}
    Q = -tH + \frac{pq}{2}
\end{equation}
\begin{equation}
    K = t^2H + 2Q^2 - \frac{m}{2} q^2
\end{equation}  
    Tuttavia notiamo che esiste una relazione fra le cariche 
\begin{equation}
    2KH + 2Q^2 + m\alpha = 0
\end{equation}

    Infine ci soffermiamo sull'algebra delle cariche, infatti le parentesi di Poisson fra le cariche sono anch'esse stesse delle cariche
\begin{equation}
    [Q, ~H] = H
\end{equation}
\begin{equation}
    [Q, ~K] = -K
\end{equation}
\begin{equation}
    [K, ~H] = 2Q
\end{equation}

\chapter{Il primo teorema in teoria classica dei campi}

    Il secondo sistema fisico che prendiamo in considerazione è un sistema continuo che presenta un'infinità di gradi di libertà, a differenza dei sistemi studiati nel precedente capitolo. Il nostro oggetto di indagine sarà quindi un'entità estesa nello spazio che rappresentiamo con una funzione delle coordinate e del tempo, appartenente alla teoria classica\footnote{In questo caso classica ha l'accezione di relativistica e non quantistica} dei campi. I campi scalari sono semplici funzioni che forniscono un numero per ogni punto, come ad esempio la temperatura, la pressione o la densità. Campi vettoriali (o tensoriali) invece associano vettori (o tensori) ad ogni punto, come ad esempio il campo elettrico o quello gravitazionale.

\section{Cenni di relatività speciale}
    In questo capitolo, non ci limiteremo all'ambito della fisica newtoniana, ma useremo la formulazione relativistica einsteniana. I principi su cui si basa sono il principio di relatività, ovvero che le leggi della fisica sono le stesse in tutti i sistemi di riferimento inerziali, e la costanza della velocità della luce, ovvero che le onde elettromagnetiche viaggiano ad una velocità costante che nessun corpo massivo può superare. Lo spazio matematico che utilizzeremo è lo spaziotempo di Minkowski $\mathcal M$, ovvero uno spazio euclideo con l'aggiunta della coordinata temporale, in cui però è stata definita una metrica differente: non sarà più valido il teorema di Pitagora 
    \begin{equation*}
        ds^2 = dx^2 + dy^2 + dz^2
    \end{equation*}
    ma le distanze verranno misurate nel seguente modo
    \begin{equation*}
        ds^2 = g_{\mu\nu} dx^\mu dx^\nu = c^2 dt^2 - dx^2 - dy^2 - dz^2
    \end{equation*}
    dove $g_{\mu\nu} = diag(1,~-1,~-1,~-1)$ è la metrica dello spazio di Minkovski e $x^\mu$ il quadrivettore, definito come 
    \begin{equation*}
        x^\mu = (ct,~x,~y,~z)
    \end{equation*}
    
    Analogamente al caso discreto, la dinamica del campo è descritta da equazioni differenziali che ne determinano l'evoluzione. Trattiamo solamente il caso di un campo scalare, ovvero definito attraverso una singola funzione $\phi(x^\mu)$, mentre per campi vettoriali (o tensoriali) si generalizza immediatamente. Definiamo le rispettive derivate dei campi rispetto alle coordinate come
    \begin{equation*}
        \phi_{, \mu} = \partial_\mu \phi = \dv{\phi}{x^\mu} = (\frac{1}{c} \pdv{\phi}{t}, ~ \nabla \phi)
    \end{equation*}
    Conseguentemente, le equazioni del moto sono equazioni dipendenti dal campo, dalle sue derivate e dal quadrivettore posizione
    \begin{equation} \label{motocampi}
        f_j(x^\mu, ~\phi(x^\mu), ~\phi_{, \mu}(x^\mu)) = 0
    \end{equation}

\section{Formalismo lagrangiano}    

    Il formalismo lagrangiano si basa sulla densità di lagrangiana $\mathcal L$, funzione del campo $\phi$, delle sue derivate $\phi,_\mu$ ed eventualmente delle coordinate $x^\mu$
    \begin{equation*} 
        \mathcal L = \mathcal L (\phi,~\phi_{, \mu},~x^\mu)
    \end{equation*}
    che integrata sullo spazio tridimensionale permette di ottenere la lagrangiana associata al sistema
    \begin{equation*}
        L = \int d^3 x ~ \mathcal L
    \end{equation*}
    In questo modo possiamo definire l'azione in funzione della densità di lagrangiana
    \begin{equation} \label{azionecampi}
        S[\phi] = \int d^4 x ~ \mathcal L (\phi,~\phi_{, \mu},~x^\mu)
    \end{equation}
    È importante osservare che l'azione è Lorentz-invariante, e quindi non dipende dal sistema di riferimento inerziale scelto.

\subsection{Equazioni di Eulero-Lagrange}

    Seguendo le stesse linee guida che abbiamo utilizzato per sistemi discreti, è intuitivo pensare che il principio di azione stazionaria ci permetterà di trovare le equazioni del moto per la configurazione del campo che rende stazionaria la sua azione~\eqref{azionecampi}
    \begin{equation} \label{azionestazionariacampi}
        \delta S [\phi(x^\mu)] = 0
    \end{equation}

    Calcoliamo la variazione dell'azione
    \begin{equation*}
    \begin{aligned}
        \delta S & = \delta \int d^4 x ~ \mathcal L \\ & = \int d^4 x ~ \Big (\pdv{\mathcal L}{\phi} \delta \phi + \pdv{\mathcal L}{\phi,_\mu} \delta \phi,_\mu \Big) \\ & = \int d^4 x ~ \Big (\pdv{\mathcal L}{\phi} \delta \phi - \partial_\mu \Big (\pdv{\mathcal L}{\phi,_\mu} \Big) \delta \phi \Big) + \int d^4 x ~ \partial_\mu \Big ( \pdv{\mathcal L}{\phi,_\mu} \delta \phi \Big ) \\ & = \int d^4 x ~ \delta \phi \Big (\pdv{\mathcal L}{\phi} - \partial_\mu \Big (\pdv{\mathcal L}{\phi,_\mu} \Big) \Big)
    \end{aligned}
    \end{equation*}
    dove abbiamo integrato per parti e nell'ultimo passaggio posto uguale a zero le quattro componenti dell'ultimo termine, a causa delle condizioni agli estremi analoghe al caso discreto, e raccolto un termine comune $\delta \phi$. Imponendo la~\eqref{azionestazionariacampi}, otteniamo
    \begin{equation*}
        \delta S = \int d^4 x ~ \delta \phi \Big (\pdv{\mathcal L}{\phi} - \partial_\mu \Big (\pdv{\mathcal L}{\phi,_\mu} \Big) \Big)  = 0
    \end{equation*}
    Per la generalizzazione del lemma fondamentale del calcolo delle variazioni, e dunque per l'arbitrarietà della variazione $\delta \phi$, l'integranda deve annullarsi e troviamo
    \begin{equation} \label{eullagcampi}
        \pdv{\mathcal L}{\phi} - \partial_\mu \pdv{\mathcal L}{\phi,_\mu} = 0
    \end{equation}
    Questa è l'equazione del moto che stavamo cercando, una sola equazione differenziale alle derivate parziali. Dunque inserendo esplicitamente la densità di lagrangiana, le~\eqref{eullagcampi} diventano le~\eqref{motocampi}. Nel caso di sistemi discreti a $d$ gradi di libertà, avevamo ottenuto $d$ equazioni differenziali mentre per sistemi continui a infiniti gradi di libertà, soltanto una equazione è apparsa. Tuttavia sono presenti anche le derivate parziali spaziali a differenza del caso precedente. 

\section{Simmetrie}
    
    Indaghiamo ora le simmetrie, generalizzando lo stesso concetto espresso per il capitolo precedente. Le simmetrie sono sempre classi speciali di trasformazioni del campo che lasciano l'azione invariata a meno di un termine al bordo. Quindi definiamo una variazione di simmetria dell'azione come una funzione infinitesima $\delta_s \phi$ vincolata a risolvere l'equazione, per ogni arbitraria $\phi$
    \begin{equation} \label{invazionecampi}
        \delta S[\phi, ~\delta_s \phi] = S[\phi + \delta_s \phi] - S[\phi] = \int d^4 x ~ \partial_\mu K^\mu \qquad \forall \phi
    \end{equation}  
    dove $K^\mu$ è un termine al bordo. Anche in questo caso non chiediamo che sia strettamente invariante ma che la variazione sia nulla a meno di una quadridivergenza. Ciò deriva dalla possibilità di aggiungere una quadridivergenza alla densità della lagrangiana senza che le equazioni del moto cambino. Definiamo perciò una variazione on-shell, ovvero tale che $\phi$ debba soddisfare l'equazione del moto~\eqref{eullagcampi}, come un'arbitraria variazione infinitesima $\delta \phi$ tale che la variazione dell'azione sia una quadridivergenza
    \begin{equation*}
    \begin{aligned}
        \delta S[\phi, ~\delta \phi] & = \int d^4 x ~ \Big(\pdv{\mathcal L}{\phi} \delta \phi + \pdv{\mathcal L}{\phi,_\mu} \delta \phi,_\mu \Big) \\ & = \int d^4 x ~ \delta \phi \Big (\pdv{\mathcal L}{\phi} - \partial_\mu \pdv{\mathcal L}{\phi,_\mu} \Big ) + \int d^4 x ~ \partial_\mu \Big ( \pdv{\mathcal L}{\phi,_\mu}  \delta \phi \Big)
    \end{aligned}
    \end{equation*}
    Applicando le equazioni di Eulero-Lagrange~\eqref{eullagcampi}, il primo integrale si annulla e otteniamo 
    \begin{equation} \label{invonshellcampi}
        \delta S[\overline \phi, ~\delta_s \phi] = \int d^4 x ~ \partial_\mu \Big ( \pdv{\mathcal L}{\phi,_\mu}  \delta \phi \Big)
    \end{equation}

\subsection{Trasformazioni spaziotemporali come deformazioni dei campi}

    Nel precedente paragrafo, abbiamo imposto che le simmetrie agiscano sui campi e non sulle coordinate, allo stesso modo in cui le simmetrie nel caso discreto interessavano le coordinate e non il tempo. Ciò può sempre essere mostrato notando che le coordinate $x$ sono variabili di integrazione e quindi mute. Vediamo dunque come trovare la deformazione dei campi in seguito a trasformazioni spaziotemporali. Consideriamo prima il caso di traslazioni spaziotemporali
    \begin{equation*}
        x'^\mu = x^\mu + \epsilon^\mu
    \end{equation*} 
        dove $\epsilon^\mu$ è un quadrivettore dalle componenti costanti. Dato un campo $\phi(\mu)$, costruiamo il campo traslato $\phi'(x^\mu)$, espandendo in serie e tenendo solo i termini lineari al primo ordine in $\epsilon$
    \begin{equation*}
        \phi'(x) = \phi(x - \epsilon) \simeq \phi(x) - \epsilon^\mu \partial_\mu \phi(x)
    \end{equation*}
        Abbiamo dunque ricavato la deformazione del campo in seguito alla traslazione spaziotemporale 
    \begin{equation*}
        \delta \phi(x) = \phi'(x) - \phi(x) = -\epsilon^\mu \partial_\mu \phi (x)
    \end{equation*}

    Finora abbiamo studiato solamente campi scalari, generalizziamo quindi per quelli vettoriali o tensoriali. Solitamente, questi ultimi vengono definiti in base a come si comportano sotto l'effetto di leggi di trasformazione. Vediamo dunque esempi di come campi scalari, vettoriali, tensoriali di rango $(0,~1)$ si trasformino in seguito all'applicazione di un cambio di coordinate:
    \begin{enumerate}
        \item Per campi scalari
    \begin{equation*}
        \phi'(x') = \phi(x)
    \end{equation*}
        \item Per campi vettoriali
    \begin{equation*}
        V'^\mu(x') = \pdv{x'^\mu}{x^\nu} V^\nu(x)
    \end{equation*}
        \item Per campi tensoriali di rango $(0,~1)$
    \begin{equation*}
        A'_\mu(x') = \pdv{x^\nu}{x'^\mu} A_\nu(x)
    \end{equation*}
    \end{enumerate} 
        Consideriamo un generale cambio delle coordinate infinitesimo
    \begin{equation*}
        x'^\mu = x^\mu + \xi^\mu(x)
    \end{equation*}
        e calcoliamo la versione infinitesima dello jacobiano precedentemente utilizzato 
    \begin{equation*}
        \pdv{x'^\mu}{x^\nu} = \delta^{\mu}_{\phantom \mu \nu} + \partial_\nu \xi^\mu(x)
    \end{equation*}
        L'inverso invece è 
    \begin{equation*}
        \pdv{x^\mu}{x'^\nu} = \delta^{\mu}_{\phantom \mu \nu} - \partial_\nu \xi^\mu(x)
    \end{equation*}
        A questo punto applichiamo ai casi precedenti e troviamo le variazioni dei campi:
    \begin{enumerate}
        \item Per campi scalari, espandiamo al primo ordine
    \begin{equation*}
        \phi'(x) = \phi(x - \xi) = \phi(x) - \xi^\mu(x) \partial_\mu \phi(x)
    \end{equation*}  
        e otteniamo la variazione
    \begin{equation}\label{variazionescalare}
        \delta \phi(x) = - \xi^\mu(x) \partial_\mu \phi(x)
    \end{equation} 
        \item Per campi vettoriali, espandiamo al primo ordine sia il lato destro che sinistro
    \begin{equation*}
        V'^\mu(x) + \xi^\nu(x) \partial_\nu V^\mu (x) = V^\mu(x) + \partial_\nu \xi^\mu(x) V^\nu(x)
    \end{equation*}
        e otteniamo la variazione
    \begin{equation*}
        \delta V^\mu(x) = V'^\mu(x) - V^\mu(x) = \partial_\nu \xi^\mu(x) V^\nu(x) - \xi^\nu (x)\partial_\nu V^\mu(x)
    \end{equation*}
        \item Per campi tensoriali di rango $(0,~1)$, espandiamo entrambi i membri 
    \begin{equation*}
        A'_\mu(x) + \xi^\nu(x) \partial_\nu A_\mu (x) = A_\mu(x) - A_\nu(x) \partial_\mu \xi^\nu(x)
    \end{equation*}
        e otteniamo la variazione
    \begin{equation}\label{variazione10}
        \delta A_\mu(x) = A'_\mu(x) - A_\mu(x) = - \xi^\nu(x) \partial_\nu A_\mu (x) - A_\nu(x) \partial_\mu \xi^\nu(x)
    \end{equation}
    \end{enumerate}

\section{Enunciato e dimostrazione}
    Enunciamo ora la trasposizione del primo teorema di Noether nell'ambito della teoria classica dei campi. Il teorema è del tutto analogo alla versione discreta, con l'unica differenza che la legge di conservazione è espressa tramite un'equazione di continuità.

    \begin{theorem}[Il primo teorema di Noether in teoria dei campi]
        Sia $\mathcal L(\phi)$ la densità di lagrangiana di un sistema fisico con relativa azione
    \begin{equation*}
        S[\phi] = \int d^4 x ~ \mathcal L(\phi,~\phi,_\mu)
    \end{equation*}
        Sia $\delta_s \phi(x)$ una trasformazione del campo che individua una variazione di simmetria del sistema~\eqref{invazionecampi}, ovvero che lascia invariate le equazioni di Eulero-Lagrange~\eqref{eullagcampi}. Allora esiste una quantità $J^\mu$, definita come
    \begin{equation}\label{caricacampi}
        J^\mu = \pdv{\mathcal L}{\phi,_\mu} \delta \phi - K^\mu
    \end{equation}
        tale che soddisfi l'equazione di continuità
    \begin{equation} \label{thcampi}
        \partial_\mu J^\mu = 0
    \end{equation}
    \end{theorem}

    \begin{proof}
        Inserendo $\phi = \overline \phi$ nella~\eqref{invazionecampi}, ovvero vincolando che $\phi$ soddisfi le equazioni del moto, otteniamo 
    \begin{equation}\label{prova6campi}
        \delta S[\overline \phi, ~\delta_s \phi] = \int d^4 x ~ \partial_\mu K^\mu
    \end{equation}
        Dall'altra parte, inserendo $\delta \phi = \delta_s \phi$ in~\eqref{invonshellcampi}, ovvero vincolando le variazioni on-shell siano anche di simmetria, abbiamo
    \begin{equation}\label{prova7campi}
        \delta S[\overline \phi,~\delta_s \phi] = \int d^4 x ~ \partial_\mu \Big ( \pdv{\mathcal L}{\phi,_\mu}  \delta \phi \Big)
    \end{equation}
        Notando che i membri sinistri delle~\eqref{prova6campi} e~\eqref{prova7campi} sono identici, abbiamo
    \begin{equation*}
        \delta S[\overline \phi, ~\delta_s \phi] = \int d^4 x~ \partial_\mu K^\mu = \int d^4 x ~ \partial_\mu \Big ( \pdv{\mathcal L}{\phi,_\mu}  \delta \phi \Big)
    \end{equation*}
        A questo punto sottraiamo il secondo con il terzo membro
    \begin{equation*}
        \int d^4 x ~ \partial_\mu \Big ( K^\mu - \pdv{\mathcal L}{\phi,_\mu} \delta \phi \Big) = 0
    \end{equation*}
        Infine avremo che l'integranda si annulla e, riconoscendo la definizione di corrente~\eqref{caricacampi}, otteniamo la tesi
    \begin{equation*}
        \partial_\mu \Big ( K^\mu - \pdv{\mathcal L}{\phi,_\mu}  \delta \phi \Big) = - \partial_\mu J^\mu = 0
    \end{equation*}
    \end{proof}

    Chiameremo d'ora in poi la quantità $J^\mu$ corrente di Noether associata alla simmetria. 

\subsection{Equazione di continuità}
    Soffermiamoci ora sulla nozione di legge di conservazione. L'equazione di continuità esprime la conservazione di una quantità fisica asserendo che la variazione di questa quantità sarà uguale al suo flusso uscente da una superficia chiusa. Per vedere ciò, scriviamo esplicitamente la~\eqref{thcampi}
    \begin{equation*}
        0 = \partial_\mu J^\mu = \partial_0 J^0 + \partial_i J^i = \partial_0 J^0 + \boldsymbol \nabla \cdot \mathbf J
    \end{equation*}
    Troviamo che l'equazione di continuità può essere riscritta come 
    \begin{equation*}
        \pdv{J^0}{t} = - \boldsymbol \nabla \cdot \mathbf J
    \end{equation*}
    Integriamo entrambi i membri su un volume $V$ abbastanza grande da permettere che $J$ vada a zero più velocemente della crescita dell'area di superficie
    \begin{equation*}
        \int_V d^3 x ~ \pdv{J^0}{t} = - \int_V d^3 x ~ \boldsymbol  \nabla \cdot \mathbf J
    \end{equation*} 
    Utilizziamo il teorema della divergenza e riscriviamo il secondo termine 
    \begin{equation*}
        \int_V d^3 x ~\pdv{J^0}{t} = - \int_{\partial V} d \mathbf A \cdot \mathbf J
    \end{equation*} 
    Dalle ipotesi fatte su $\mathbf J$ e su $V$, il secondo membro si annulla e troviamo la carica che si conserva 
    \begin{equation*}
        Q = \int_V d^3 x ~ J^0(x)
    \end{equation*} 
    ovvero che soddisfi la legge di conservazione
    \begin{equation*}
        \dv{}{t} Q = 0 
    \end{equation*}

\section{Tensore energia-impulso}

    Speciale interesse poniamo sulle traslazioni spaziotemporali, poichè ci conducono alla definizione del tensore energia-impulso, la cui fondamentale applicazione è data dalle equazioni di campo di Einstein in relatività generale. 
    Definiamo il tensore energia-impulso $T^\mu_{\phantom \mu \nu}$ come il tensore formato dalle correnti di Noether associate alle 4 traslazioni spaziotemporali
    \begin{equation*}
        x'^\mu = x^\mu + \epsilon^\mu
    \end{equation*}
    che possono essere decomposte nelle varie componenti: per $\mu=0$ abbiamo una traslazione temporale $t' = t + \epsilon^0$ mentre per $\mu=i=1,~2,~3$ abbiamo una traslazione spaziale $x'^i = x^i + \epsilon^i$. Insieme ci sono 4 correnti di Noether che si conservano, che possiamo scrivere senza separarle come 
    \begin{equation} \label{tensenimp}
        J^\mu = T^\mu_{\phantom \mu \nu} \epsilon^\nu
    \end{equation}
    una per ogni indice $\mu$. La conservazione di $J^\mu$, ovvero l'equazione $\partial_\mu J^\mu = 0 $ implica anche che si conservi $\partial_\mu T^\mu_{\phantom \mu \nu} = 0$, essendo $\epsilon^\nu$ una costante.

    Il significato fisico delle componenti del tensore energia-impulso è il seguente: 
    \begin{enumerate}
        \item $T^{00}$ è la densità di energia,
        \item $T^{0j}$ è il flusso di energia lungo la j-esima direzione,
        \item $T^{i0}$ è la densità di quantità di moto lungo la i-esima direzione,
        \item $T^{ij}$ è il tensore degli sforzi.
    \end{enumerate}

\section{Campo scalare}

    In questo e nel prossimo paragrafo, proponiamo due esempi di campi: un generico campo scalare e un campo tensoriale, ovvero quello elettromagnetico.

    \hfill

    Consideriamo una densità di lagrangiana per un campo scalare $\phi$, che avrà la forma $\mathcal L(\phi,~\phi_{, \mu})$. Dato che la lagrangiana non dipende esplicitamente dalle coordinate $x^\mu$, sarà invariante per la variazione~\eqref{variazionescalare}, dove poniamo $\xi^\mu = \epsilon^\mu$
    \begin{equation*}
        \delta \phi(x) = - \epsilon^\mu \partial_\mu \phi(x)
    \end{equation*}
    Calcoliamo la variazione dell'azione
    \begin{equation*}
    \begin{aligned}
        \delta S & = \delta \int d^4 x ~ \mathcal L \\ &  = \int d^4 x ~ \Big ( \pdv{\mathcal L}{\phi} \delta \phi + \pdv{\mathcal L}{\phi,_\rho} \delta \phi,_\rho \Big) \\ & = - \int d^4 x ~ \epsilon^\sigma \Big ( \pdv{\mathcal L}{\phi} \phi,_\sigma + \pdv{\mathcal L}{\phi,_\rho} \phi,_{\rho\sigma} \Big) \\ & = \int d^4 x ~ \partial_\sigma (- \epsilon^\sigma \mathcal L)
    \end{aligned}
    \end{equation*}
    Quindi, secondo la~\eqref{invazionecampi}, abbiamo trovato che il termine al bordo è
    \begin{equation*}
        K^\mu = - \epsilon^\mu \mathcal L
    \end{equation*}
    Ora applichiamo il primo teorema di Noether e troviamo la corrente~\eqref{caricacampi} generata dalle traslazione spaziotemporali
    \begin{equation*}
        J^\mu = \pdv{\mathcal L}{\phi,_\mu} \delta \phi - K^\mu = \pdv{\mathcal L}{\phi,_\mu} (- \epsilon^\sigma \phi,_\sigma) + \epsilon^\mu \mathcal L = - \epsilon^\sigma \Big ( \pdv{\mathcal L}{\phi,_\mu} \phi,_\sigma - \delta^\mu_{\phantom \mu \sigma} \mathcal L \Big)
    \end{equation*}
    Quindi il tensore energia-impulso, definito dalla~\eqref{tensenimp}, è
    \begin{equation*}
        T^\mu_{\phantom \mu \sigma} = \pdv{\mathcal L}{\phi,_\mu} \phi,_\sigma - \delta^\mu_{\phantom \mu \sigma} \mathcal L
    \end{equation*}

    \hfill

    Proponiamo un esempio concreto, in cui prendiamo la lagrangiana più semplice.

    \begin{example}
    Sia la densità di lagrangiana per un campo scalare nella forma
    \begin{equation*}
        \mathcal L = \frac{1}{2} \partial_\mu \phi \partial^\mu \phi
    \end{equation*}
    Il suo tensore energia-impulso è
    \begin{equation*}
        T_{\mu\nu} = \partial_\mu \phi \partial_\nu \phi - \frac{1}{2} g_{\mu\nu} \partial_\sigma \phi \partial^\sigma \phi
    \end{equation*}
    \end{example}

\section{Campo elettromagnetico}
    I fenomeni elettromagnetici sono interamente caratterizzati dalle equazioni di Maxwell, che possono essere derivate dal principio di azione stazionaria a partire dall'azione di Maxwell\footnote{In questo caso useremo il sistema di unità di misura di Heaviside, dove $4\pi$ viene sostituito con $4$ nell'azione}
\begin{equation} \label{azionemaxwell}
    S[A_\mu] = - \frac{1}{4} \int d^4 x ~ F_{\mu\nu} F^{\mu\nu}
\end{equation}
    dove $A_\mu$ è il tensore $(0, 1)$ definito attraverso il potenziale vettore e scalare, e $F^{\mu\nu} = \partial_\mu A_\nu - \partial_\nu A_\mu$ è il tensore elettromagnetico. Calcoliamo le equazioni del moto attraverso le equazioni di Eulero-Lagrange~\eqref{eullagcampi}
\begin{equation*}
    0 = \pdv{\mathcal L}{A_\nu} - \partial_\mu \pdv{\mathcal L}{A_{\nu,\mu}} = \partial_\mu (\partial^\nu A^\mu - \partial^\mu A^\nu) = \partial_\mu F^{\mu\nu}
\end{equation*}
    che sono proprio le equazioni di Maxwell in assenza di sorgenti\footnote{In realtà ci sarebbe un'altra equazione $\partial_\mu F^{*\mu\nu} = 0$, dove $F^{*\mu\nu}$ è il tensore duale a $F^{\mu\nu}$}
\begin{equation}\label{eqmax}
    \partial_\mu F^{\mu\nu} = 0
\end{equation}
    oppure scritte in funzione del quadripotenziale
\begin{equation}\label{gauge1}
    \partial_\nu \partial^\nu A_\mu - \partial_\mu (\partial^\nu A_\nu) = 0
\end{equation}

    Indaghiamo ora le simmetrie che l'elettrodinamica presenta.

\subsection{Simmetria spaziotemporale}
    La prima simmetria che studiamo è una traslazione spaziotemporale, che porta come abbiamo precedentemente visto al tensore energia-impulso. Sappiamo che l'elettrodinamica è invariante per trasformazioni di gauge 
\begin{equation*}
    \delta A_\mu = \partial_\mu \lambda(x)
\end{equation*}
    dove $\lambda(x)$ è una generica funzione, chiamata funzione di gauge, dipendente dalle coordinate. Come conseguenza, chiediamo che anche le cariche conservate siano invarianti di gauge come la teoria. Osserviamo tuttavia che prendendo una variazione di $A_\mu$, che abbiamo calcolato in~\eqref{variazione10}, 
\begin{equation*}
    \delta A_\mu = - \epsilon^\nu \partial_\nu A_\mu
\end{equation*}
    e che lascia effettivamente invariata l'azione, a meno di un termine al bordo
\begin{equation*}
\begin{aligned}
    \delta F_{\mu\nu} F^{\mu\nu} & = 2 F^{\mu\nu} \delta F_{\mu\nu} \\ & = 2 F^{\mu\nu} \delta (\partial_\mu A_\nu - \partial_\nu A_\mu) \\ & = 2 F^{\mu\nu} ( \partial_\mu \delta A_\nu - \partial_\nu \delta A_\mu) \\ & = 2 F^{\mu\nu} ( \partial_\mu (- \epsilon^\sigma \partial_\sigma A_\nu) - \partial_\nu (- \epsilon^\sigma \partial_\sigma A_\mu)) \\ & = - 2 \epsilon^\sigma F^{\mu\nu} ( \partial_\mu \partial_\sigma A_\nu - \partial_\nu \partial_\sigma A_\mu) \\ & = - 2 \epsilon^\sigma F^{\mu\nu} \partial_\sigma F_{\mu\nu} \\ & = \partial_\sigma (-\epsilon^\sigma F_{\mu\nu} F^{\mu\nu})
\end{aligned}
\end{equation*}
    la variazione non è gauge invariante e si può già prevedere che non lo sarà neanche la corrente. Per ovviare a ciò, introduciamo una nuova variazione che questa volta rispetta i prerequisiti 
\begin{equation*}
    \delta A_\mu = - \epsilon^\nu \partial_\nu A_\mu + \partial_\mu (\epsilon^\nu A_\nu) = F_{\mu\nu} \epsilon^\nu
\end{equation*}
    che è un invariante di gauge essendo il tensore elettromagnetico tale. Calcoliamo la variazione dell'azione
\begin{equation*}
\begin{aligned}
    \delta F_{\mu\nu} F^{\mu\nu} & = 2 F^{\mu\nu} \delta F_{\mu\nu} \\ &= 2 F^{\mu\nu} \delta (\partial_\mu A_\nu - \partial_\nu A_\mu) \\ & = 2 F^{\mu\nu} (\partial_\mu \delta A_\nu - \partial_\nu \delta A_\mu) \\ & = - \epsilon^\sigma 2 F^{\mu\nu} (\partial_\mu F_{\nu\sigma}  - \partial_\nu F_{\mu\sigma}) \\ & = 2 F^{\mu\nu} \epsilon^\sigma (\partial_\mu F_{\nu\sigma} + \partial_\nu F_{\sigma\mu}) \\ & = - 2 F^{\mu\nu} \epsilon^\sigma \partial_\sigma F_{\mu\nu} \\ & = \partial_\sigma (-\epsilon^\sigma F^{\mu\nu} F_{\mu\nu})
\end{aligned}
\end{equation*}
    che essendo un termine al bordo, mostra che è una variazione di simmetria, con 
\begin{equation*}
    K^\sigma = -\epsilon^\sigma \mathcal L
\end{equation*}
    Ora applichiamo il primo teorema di Noether e troviamo la corrente~\eqref{caricacampi} 
\begin{equation*}
\begin{aligned}
    J^\mu & = \pdv{L}{A_{\rho,\mu}} \delta A_\rho - K^\mu \\ & = - F^{\mu\rho} \epsilon^\sigma F_{\rho\sigma} + \epsilon^\mu L \\ & = \epsilon^\sigma (- F^{\mu\rho} F_{\rho \sigma} + \delta^\mu_{\phantom \mu \sigma} L ) \\ & = - \epsilon^\sigma \Big ( F^{\mu\rho} F_{\rho\sigma} + \frac{1}{4} \delta^\mu_{\phantom \mu \sigma} F^{\alpha \beta} F_{\alpha \beta} \Big)
\end{aligned}
\end{equation*}
    ed usando la relazione che lega il tensore energia-impulso, otteniamo 
\begin{equation*}
    T^\mu_{\phantom \mu \sigma} = - F^{\mu\rho} F_{\sigma\rho} + \frac{1}{4} \delta^\mu_{\phantom \mu \sigma} F^{\alpha \beta} F_{\alpha \beta}
\end{equation*}
    Notiamo che $T$ è invariante di gauge dato che $F$ lo è. A questo punto mostriamo che è veramente una carica che si conserva
\begin{equation*}
    \partial_\mu T^\mu_{\phantom \mu \sigma} = - \partial_\mu F^{\mu\rho} F_{\sigma\rho} + \frac{1}{4} \partial_\mu \delta^\mu_{\phantom \mu \sigma} F^{\alpha \beta} F_{\alpha \beta} = 0
\end{equation*} 
    dove abbiamo usato le equazioni di Maxwell~\eqref{eqmax}.

\subsection{Simmetria conforme}
    In realtà, la teoria elettromagnetica è invariante rispetto ad un gruppo molto più grande di trasformazioni, chiamato gruppo conforme. Consideriamo una trasformazione generica
\begin{equation*}
    x'^\mu = x^\mu + \xi^\mu(x)
\end{equation*}
    dove $\xi^\mu$ è un generico quadrivettore. La variazione che subisce $A_\mu$ in seguito a questa trasformazione spaziotemporale è data dalla~\eqref{variazione10}
\begin{equation*}
    \delta A_\mu = - \xi^\nu \partial_\nu A_\mu - \partial_\mu \xi^\nu A_\nu
\end{equation*}
    Tuttavia, come nel paragrafo precedente, chiediamo che la variazione sia invariante di gauge e quindi aggiungiamo un termine alle variazione
\begin{equation*}
    \delta A_\mu = - \xi^\nu \partial_\nu A_\mu - \partial_\mu \xi^\nu A_\nu + \partial_\mu(\xi^\nu A_\nu)
\end{equation*}
    Potrebbe sembrare che l'aggiunta di un termine dovuto ad una trasformazione di gauge, porti ad una corrente di Noether differente e dunque ad una sbagliata legge di conservazione. Ciò non avviene perchè associate alle trasformazioni di gauge, non sono presenti cariche come nel caso di simmetrie incontrate finora, ma bensì vincoli. Nel prossimo capitolo, tratteremo più in dettaglio questa differenza. 

    L'azione~\eqref{azionemaxwell}, non è invariante per arbitrarie scelte di $\xi^\mu$, ma soltanto per trasformazioni appartenenti al gruppo conforme, cioè tali che soddisfi l'equazione
\begin{equation}\label{conforme}
    \xi_{\mu,\nu} + \xi_{\nu,\mu} = \frac{1}{2} g_{\mu\nu} \xi^\alpha_{\phantom \alpha ,\alpha}
\end{equation} 
    Dunque qualsiasi soluzione di questa equazione ci fornisce una simmetria a cui possiamo associare una corrente di Noether. È possibile separare le soluzioni in quattro differenti categorie 
\begin{enumerate}
    \item Traslazioni, già incontrate nel paragrafo precedente, in cui
\begin{equation*}
    \xi^\mu = \epsilon^\mu
\end{equation*}
    dove $\epsilon^\mu$ è un quadrivettore dalle componenti costanti.
    \item Trasformazioni di Lorentz, tali che 
\begin{equation*}
    \xi^\mu = \Lambda^\mu_{\phantom \mu \nu} x^\nu \qquad \Lambda_{\mu\nu} = - \Lambda_{\nu\mu}
\end{equation*}
    dove $\Lambda^\mu_{\phantom \mu \nu}$ è la matrice di Lorentz antisimmetrica.
    \item Dilatazioni, tali che
\begin{equation*}
    \xi^\mu = \lambda x^\mu
\end{equation*}
    dove $\lambda$ è un fattore di scala.
    \item Trasformazioni conformi speciali, tali che
\begin{equation*}
    \xi^\mu = 2 x^\nu b_\nu x^\mu - b^\mu x^\nu x_\nu
\end{equation*}
\end{enumerate}

    Mostriamo ora che l'azione è invariante per trasformazioni conformi 
\begin{equation*}
\begin{aligned}
     \delta \frac{1}{4} F^{\mu\nu} F_{\mu\nu} & = \frac{1}{2} F^{\mu\nu} \delta F_{\mu\nu} \\ & = \frac{1}{2} F^{\mu\nu} 2 \partial_\mu (\xi^\rho F_{\nu \rho}) \\ & = F^{\mu\nu} F_{\nu\rho} \partial_\mu \xi^\rho + F^{\mu\nu} \xi^\rho \partial_\mu F_{\nu\rho} \\ & = F^{\mu\nu} F_\nu^{\phantom \nu \rho} \partial_\mu \xi^\rho + \frac{1}{2} F^{\mu\nu} \xi^\rho (\partial_\mu F_{\nu\rho} + \partial_\nu F_{\rho\mu}) \\ & = - F^{\mu\nu} F^\rho_{\phantom \rho \nu} \partial_\mu \xi^\rho - \frac{1}{2} F^{\mu\nu} \xi^\rho \partial_\rho F_{\mu\nu} \\ & = - F^{\mu\nu} F^\rho_{\phantom \rho \nu} \partial_\mu \xi^\rho - \frac{1}{4} \partial_\rho (F^{\mu\nu} F_{\mu\nu}) \xi^\rho \\ & = - \frac{1}{2} F^{\mu\nu} F^\rho_{\phantom \rho \nu} (\partial_\mu \xi_\rho + \partial_\rho \xi_\mu) + \frac{1}{4} F^{\mu\nu} F_{\mu\nu} \partial_\alpha \xi^\alpha - \frac{1}{4} \partial_\rho (\xi^\rho F^2) \\ & = - \frac{1}{2} F^{\mu\nu} F^\rho_{\phantom \rho \nu} \Big ( \partial_\mu \xi_\rho + \partial_\rho \xi_\mu - \frac{1}{2} g_{\mu\rho} \partial_\alpha \xi^\alpha \Big) - \partial_\rho(\xi^\rho \mathcal L) \\ & = - \partial_\rho(\xi^\rho \mathcal L)
\end{aligned}
\end{equation*}
    dove abbiamo usato l'identità di Bianchi e il fatto che $\xi$ soddisfi l'equazione~\eqref{conforme}. Il termine al bordo è dunque 
\begin{equation*}
    K = - \xi^\rho \mathcal L
\end{equation*}

    Ora applichiamo il primo teorema di Noether e troviamo la corrente~\eqref{caricacampi} 
\begin{equation*}
    J^\mu = \pdv{\mathcal L}{A_{\rho,\mu}} \delta A_\rho - K^\mu = F^{\mu\alpha} \xi^\beta F_{\alpha\beta} +  \xi^\mu \frac{1}{4} F^{\alpha\beta} F_{\alpha\beta} = \xi^\beta \Big ( F^{\mu \alpha} F_{\alpha \beta} + \frac{1}{4} \delta^\mu_{\phantom \mu \sigma} F^{\alpha \beta} F_{\alpha \beta} \Big)
\end{equation*}
    tale che soddisfi l'equazione di continuità $\partial_\mu J^\mu$. 

\section{Campo di Schroedinger}

    Concludiamo il capitolo, trattando brevemente un altro esempio di campo: la funzione d'onda $\psi$ della meccanica quantistica. L'equazione che governa tale oggetto è l'equazione di Schoedinger
    \begin{equation} \label{schro}
        i \hbar \dot \psi = - \frac{\hbar^2}{2m} \nabla^2 \psi + V \psi = H \psi
    \end{equation}
    Anch'essa può essere derivata attraverso il principio di azione stazionaria, utilizzando l'azione 
    \begin{equation*}
        S = \int dt \int d^3 r ~ \Big( i \hbar \psi^* \dot \psi - \frac{\hbar^2}{2m} \nabla \psi^* \cdot \nabla \psi - V(\mathbf r) \psi^* \psi \Big)
    \end{equation*}
    Per mostrare ciò, consideriamo $\psi$ e $\psi^*$ come due campi differenti e applichiamo le equazioni di Eulero-Lagrange~\eqref{eullagcampi}
    \begin{equation*}
        0 = \pdv{\mathcal L}{\psi^*} - \dv{}{t} \pdv{\mathcal L}{\dot \psi^*} = i \hbar \dot \psi + \frac{\hbar^2}{2m} \nabla^2 \psi - V \psi 
    \end{equation*}
    che è proprio la~\eqref{schro}.

    Analogamente, derivando rispetto a $\psi$ invece che a $\psi^*$, le equazioni di Eulero-lagrange ci portano alla stessa equazione ma complessa coniugata. 

    \hfill

    L'azione è invariante rispetto alla trasformazione di fase globale 
\begin{equation*}
    \psi' = \psi e^{i\alpha}
\end{equation*}
    dove $\alpha$ è una costante. La variazione di simmetria è 
\begin{equation*}
    \delta \psi = i \alpha \psi
\end{equation*}
    Il termine al bordo $K$ è nullo, dato che sono presenti termini $e^{i\alpha}$ e $e^{-i\alpha}$ che si elidono a vicenda. Effettuando la variazione on-shell, otteniamo 
\begin{equation*}
    \delta S = \int dt \int d^3 r ~ i \hbar \dv{}{t} (\psi^* \delta \psi) = - \alpha \hbar \int dt ~ \dv{}{t} \int d^3 r ~ \psi^* \psi
\end{equation*}
    che ci porta alla conservazione della quantità
\begin{equation*}
    Q = \int d^3 r ~ \psi^* \psi
\end{equation*}
    Secondo l'intepretazione di Copenaghen, questa quantità è proprio la probabilità di trovare una particella in una data regione dello spazio, quantità che necessariamente si deve conservare. Normalmente si pone $Q = 1$ per avere la condizione di normalizzazione. 


\chapter{Secondo teorema e teorie di gauge}

    Nei primi due capitoli, abbiamo studiato sistemi fisici che presentano simmetrie riconducibili a trasformazioni applicate a tutti i punti dello spaziotempo, dette simmetrie globali. Tuttavia in fisica moderna sono presenti sistemi che possiedono simmetrie locali, ovvero trasformazioni che sono funzioni dei punti dello spaziotempo. Esse non associano quantità conservate che soddisfano leggi di conservazione alle simmetrie ma un sistema di equazioni differenziali. Ciò significa che il sistema ha un numero di gradi di libertà sovrastimato, perchè ci sono relazioni che legando le variabili attraverso equazioni, ne abbassano il numero potendone ricavarne una in funzione dell'altra. In letteratura, questo teorema viene chiamato secondo teorema di Noether perchè presente nel già citato suo articolo del 1916. In questo capitolo, non enunceremo né dimostreremo questo teorema, ma studieremo una conseguenza: le teorie di gauge. L'importanza di tali teorie risiede nello sviluppo della fisica moderna: tre delle quattro interazioni fondamentali (forte, debole ed elettromagnetica) sono descritte da teorie di gauge. I riferimenti bibliografici sono \cite{barone} \cite{banados}.

\section{Teoria di gauge}
    Prima di studiare in dettaglio la struttura matematica di una teoria di gauge, presentiamo brevemente quali caratteristiche possiede. Consideriamo un sistema fisico descritto da una lagrangiana nell'ambito della teoria dei campi. Definiamo una simmetria di gauge come una simmetria locale, ovvero una trasformazione contenente un'arbitraria funzione delle coordinate che lascia invariata l'azione del sistema. Ne consegue che ci saranno relazioni che legano le equazioni del moto: fissando arbitrariamente la nostra funzione, i gradi di libertà del sistema sono diminuiti. Inoltre, passando alla descrizione hamiltoniana, emerge la presenza di vincoli generati dalla simmetria a differenza delle cariche conservate nel caso di simmetrie globali. Particolare attenzione è posta su quanto possa essere fisica una teoria formulata in questo modo, che a prima vista potrebbe sembrarlo per nulla. Tuttavia possiamo introdurre il concetto di classi di equivalenza di configurazioni, ovvero campi che differiscono matematicamente solamente per una simmetria di gauge, descrivono fisicamente la stessa realtà. 
    
\subsection{Un esempio motivato dall'elettrodinamica}

    Il primo esempio che si trova è quello dell'elettrodinamica: dato un quadripotenziale $A_\mu$ che soddisfa le equazioni di Maxwell, è possibile dimostrare che anche con l'aggiunta di una simmetria di gauge le continua a soddisfare. Infatti prendendo $A'_\mu = A_\mu + \partial_\mu \Lambda$ e mettendolo nelle equazioni di Maxwell nel vuoto
    \begin{equation*}
        \Box A_\mu - \partial_\mu (\partial^\nu A_\nu) = 0
    \end{equation*}
    otteniamo 
    \begin{equation*}
        \Box A'_\mu - \partial_\mu (\partial^\nu A'_\nu) = \Box A_\mu - \partial_\mu (\partial^\nu A_\nu) + \partial^\nu \partial_\mu \partial^\mu \Lambda - \partial^\nu \partial_\mu \partial^\mu \Lambda = \Box A_\mu - \partial_\mu (\partial^\nu A_\nu) = 0
    \end{equation*}

    Prendiamo dunque in considerazione un esempio motivato dalla teoria appena descritta, considerando un funzionale di azione dipendente da due campi $A$ e $\psi$ definito nel seguente modo
    \begin{equation*}
        S[A(t),\psi(t)] = \frac{1}{2} \int dt ~ (\dot \psi - A)^2
    \end{equation*}
    Mostriamo ora quantitativamente tutte le proprietà di una teoria di gauge. 
    
    Innanzitutto vediamo che è presente una simmetria di gauge: l'azione è invariante sotto la trasformazione
    \begin{equation}\label{arbitraria}
        \psi' = \psi + \epsilon(t) \qquad A' = A + \dot \epsilon(t)
    \end{equation}
    dove $\epsilon(t)$ è un'arbitraria funzione del tempo. Infatti calcolando la variazione sotto tale trasformazione otteniamo 
    \begin{equation*}
    \begin{aligned}
        S[A'(t),\psi'(t)] & = \frac{1}{2} \int dt ~ (\dot \psi' - A')^2 = \frac{1}{2} \int dt ~ (\dot \psi + \dot \epsilon - A - \dot \epsilon )^2 \\ & = \frac{1}{2} \int dt ~ (\dot \psi - A)^2 = S[A(t),\psi(t)]
    \end{aligned}
    \end{equation*}
    
    Calcolando ora le equazioni del moto per i due campi, utilizzando le equazioni di Eulero-Lagrange \eqref{eullagcampi}. Per il campo $A$ otteniamo 
    \begin{equation*}
        0 = \pdv{\mathcal L}{A} - \dv{}{t} \pdv{\mathcal L}{\dot A} = - 2 (\dot \psi - A)
    \end{equation*}
    che porta alla prima equazione del moto 
    \begin{equation} \label{moto1}
        \dot \psi - A = 0
    \end{equation}
    invece per il campo $\psi$ troviamo
    \begin{equation*}
        0 = \pdv{\mathcal L}{A} - \dv{}{t} \pdv{\mathcal L}{\dot \psi} = 2 \dv{}{t} (\dot \psi - A)
    \end{equation*}
    che conduce alla seconda equazione del moto 
    \begin{equation} \label{moto2}
        \dv{}{t} (\dot \psi - A) = 0
    \end{equation}
    Confrontando le equazioni del moto, notiamo che non sono indipendenti: l'equazione per il campo $A$ \eqref{moto1} contiene già l'equazione per $\psi$ \eqref{moto2} (se una funzione è nulla, anche la derivata). Non ci sono quindi due equazioni, ne è presente soltanto una. Scriviamo esplicitamente la soluzione delle equazioni del moto 
\begin{equation*}
    \psi(t) = f(t) \qquad A(t) = \dot f(t)
\end{equation*}
    e mostriamo che applicando la simmetria di gauge, attraverso la funzione $\epsilon(t)$ introdotta in \eqref{arbitraria}
\begin{equation*}
    \psi'(t) = \psi(t) + \epsilon(t) = f(t) + \epsilon(t) \qquad A'(t) = A(t) + \dot \epsilon(t) = \dot f(t) + \dot \epsilon(t)
\end{equation*}
    le soluzioni soddisfano ancora le equazioni del moto 
\begin{equation*}
    \dot \psi' - A' = 0 = \dot \psi(t) + \dot \epsilon(t) - A - \dot \epsilon(t) = \dot \psi - A = 0
\end{equation*}
    Come conseguenza, abbiamo arbitrarie condizioni condizioni iniziali sono date: è sempre possibile modificare la soluzione delle equazioni del moto attraverso la simmetria di gauge. 

    Infine, passiamo dal formalismo lagrangiano alla descrizione hamiltoniano. Per trovare l'hamiltoniana del sistema, definiamo prima il momento coniugato al campo $\psi$
\begin{equation*}
    p_\psi = \pdv{L}{\dot \psi} = \dot \psi - A
\end{equation*}
    e quello coniugato al campo $A$
\begin{equation*}
    p_A = \pdv{L}{\dot A} = 0
\end{equation*}
    e successivamente applichiamo una trasformazione di Legendre \eqref{hamiltoniana}
\begin{equation*}
\begin{aligned}
    H(p_\psi,~\psi,~A) & = p_\psi \dot \psi - L = p_\psi \dot \psi - \frac{1}{2} (\dot \psi - A)^2 \\ & = p_\psi (p_\psi + A) - \frac{1}{2} (p_\psi + A - A)^2 = \frac{1}{2} p^2_\psi  + A p_\psi
\end{aligned}
\end{equation*}
    Applichiamo ora le equazioni di Hamilton \eqref{ham1} e \eqref{ham2} per trovare le equazioni del moto. Per il campo $A$ abbiamo
\begin{equation} \label{Ah}
    \dot p_A = - \pdv{H}{A} = - p_\psi = 0 
\end{equation}
    mentre per il campo $\psi$
\begin{equation} \label{psih}
    \dot \psi = \pdv{H}{p_\psi} = p_\psi + A \qquad \dot p_\psi = - \pdv{H}{\psi} = 0
\end{equation}
    Notiamo nella prima equazione l'assenza di derivate temporali, non è un'equazione di evoluzione temporale ma un vincolo $p_\psi = 0$ dove $A$ è il moltiplicatore di Lagrange corrispondente. Inoltre osserviamo che anche in questo caso le equazioni del moto non sono indipendenti, infatti l'ultima è la derivata temporale della prima. Abbiamo nuovamente meno equazioni che variabili.

    Concludiamo il paragrafo con una nota importante. Dato che le equazioni del moto per $\psi$ sono contenute in quelle per $A$, \eqref{moto2} in \eqref{moto1} oppure \eqref{psih} in \eqref{Ah}, non posso scegliere arbitrariamente $A = 0$ ma posso soltanto imporre $\psi = 0$. Lo si può dimostrare, vedendo che se si pone $\psi = 0$ 
\begin{equation*}
    S[\psi = 0,~A] = \int dt A^2
\end{equation*}
    che porta all'equazione di un vincolo
\begin{equation*}
    A = 0
\end{equation*}
    mentre la scelta di $A = 0$ 
\begin{equation*}
    S[\psi,~A]=0 = \int dt \dot \psi^2
\end{equation*}
    conduce ad un equazione del moto 
\begin{equation*}
    \ddot \psi = 0
\end{equation*}
    non accettabile, essendo una evoluzione temporale e non un vincolo che deve emergere dalla simmetria di gauge. 

\subsection{Struttura generale delle teorie di gauge} 

    Il vantaggio di studiare la formulazione hamiltoniana risiede nella possibilità di studiare in generale caratteristiche valide per molti esempi: dalla particella relativistica all'elettrodinamica. Il metodo più generale per passare dalla descrizione lagranginana a quella hamiltoniana è il metodo di Dirac. Tuttavia non sceglieremo quel percorso, ma partiremo direttamente dalla conoscenza dell'hamiltoniana e dal suo funzionale di azione che ha la forma generica
\begin{equation} \label{azionevincolo}
    S[p_i,~q^i,~\lambda^a] = \int dt (p_i \dot q^i - H_0(p_i, ~q^i) + \lambda^a \phi_a (p, ~q))
\end{equation}
    dove le variabili indipendenti dei campi sono $p_i, q^i e \lambda^a$, rispettivamente i momenti coniugati, le coordinate e i moltiplicatori di Lagrange. L'hamiltoniana totale è definita come $H = H_0 - \lambda^a \phi_a$ ed è composta da una parte che non contiene vincoli $H_0$ e da una parte che contribuisce ad aggiungerli, dove le equazioni dei vincoli sono date dalle $\phi_a$. Le corrispondenti equazioni del moto, variando le tre variabili, sono
\begin{equation}
    \dot q^i = \pdv{H_0}{p_i} - \lambda^a \pdv{\phi_a}{p_i}
\end{equation}
\begin{equation}
    \dot p^i = - \pdv{H_0}{q^i} + \lambda^a \pdv{\phi_a}{q^i}
\end{equation}
\begin{equation}
    \phi_a(p, ~q) = 0
\end{equation}
    Le prime due equazioni determinano la dinamica del sistema, come evolvono $p$ e $q$ date le condizioni iniziali $p_0$, $q_0$. Tuttivia le condizioni iniziali non possono essere del tutto arbitrarie ma devono soddisfare la terza equazione. Inoltre un altro dubbio sorge: come scelgo le $\lambda$? Con l'ultilizzo delle parentesi di Poisson, calcoliamo la derivata temporale dei vincoli 
\begin{equation}
    \dv{}{t} \phi_a(p, ~q) = [\phi_a, ~H_0] - [\phi_a, ~\phi_b] \lambda^b
\end{equation}
    e la poniamo debolmente nulla, ovvero nulla soltanto quando $\phi_a = 0$
\begin{equation*} \label{vincoli}
    [\phi_a, ~H_0] - [\phi_a, ~\phi_b] \lambda^b \approx 0
\end{equation*}
    Distinguiamo due possibili casi, chiamando la matrice $C_{ab} = [\phi_a, ~\phi_b]$
\begin{enumerate}
    \item Teorie non di gauge: se la matrice $C_{ab}$ è invertibile, i moltiplicatori di Lagrange $\lambda^a$ sono fissati dall'equazione \eqref{vincoli}
\begin{equation}
    \lambda^a (t) = C^{ab} [\phi_b, ~H_0]
\end{equation}
    In questo caso il loro ruolo è di mantenere il vincolo nullo $\phi = 0$ nel tempo. Questi vincoli sono chiamati di seconda classe e la dinamica del sistema è semplice: date le condizioni iniziali in modo che rispettino i vincoli, le $\lambda$ assicurano che il vicnolo sia mantenuto negli istanti successivi.
    \item Teoria di gauge: se la matrice è debolmente nulla $C_{ab} \approx 0$, allora i moltiplicatori di Lagrange non impongono alcuna condizione e le equazioni del moto rimangono non determinate. Definiamo vincoli di prima classe se soddisfano 
\begin{equation} \label{secondaclasse}
    [\phi_a, ~ H_0] = C_a^{\phantom a b} \phi_b \approx 0 \qquad   [\phi_a, ~ \phi_b] = C_{ab}^{\phantom{ab} c} \phi_c \approx 0
\end{equation}
    Una conseguenza dei vincoli di prima classe, è che l'azione è invariante per trasformazioni di gauge 
\begin{equation}
    \delta q^i = [q^i, ~\phi_a] \epsilon^a (t)
\end{equation}
\begin{equation}
    \delta p_i = [p_i, ~\phi_a] \epsilon^a (t)
\end{equation}
\begin{equation}
    \delta \lambda^c = \dot \epsilon^c (t) + \epsilon^a(t) C_a^{\phantom a c} - \lambda^a \epsilon^b(t) C_{ab}^{\phantom{ab} c}
\end{equation}
    dove $\epsilon^a(t)$ è un'arbitraria funzione del tempo. 
\end{enumerate}

    Abbiamo quindi dimostrato l'inverso del teorema di Noether: un'azione del tipo \eqref{azionevincolo} tale che i vincoli e l'hamiltoniana soddisfano \eqref{secondaclasse}, presentano delle simmetrie di gauge Dimostriamo ora questo enunciato. 
\begin{proof}
    La variazione delle variabili canoniche è data da 
\begin{equation}
    \delta q_i = \epsilon^a \pdv{\phi_a}{p_i} \qquad \delta p_i = - \epsilon^a \pdv{\phi_a}{q^i}
\end{equation}
    dunque calcoliamo la variazione dell'azione
\begin{equation}
\begin{gathered}
    \delta S = \delta \int dt ~ (q^i p_i - H_0(q^i, p_i) + \lambda^a \phi_a(p, q)) = \\
    = dt ~ (-\epsilon^a \pdv{\phi_a}{q^i} \dot q^i - \dot p^i \epsilon^a \pdv{\phi_a}{p_i} - \pdv{H_0}{q^i} \epsilon \pdv{\phi_a}{p_i} + \pdv{H_0}{p^i} \epsilon^a \pdv{\phi_a}{q_i} \\ - \delta \lambda^a \phi_a - \lambda^a (\pdv{\phi_a}{q^j} \epsilon^b \pdv{\phi_b}{p_j} - \pdv{\phi^a}{p_j}\epsilon^b \pdv{\phi_b}{q^j}) + B) \\ 
    = \int dt ~(-\epsilon^a \frac{d}{dt} \phi_a - \epsilon^a [H_0, \phi_a] - \delta \lambda^a \phi_a - \lambda^a \epsilon^b [\phi_a, \phi_b] + B)
\end{gathered}
\end{equation}
    dove $B$ è un termine al contorno. Usando \eqref{secondaclasse} e ridefinendo $B$, otteniamo
\begin{equation}
\begin{gathered}
    \delta S = \int dt ~ (\dot \epsilon^a \phi_a + \epsilon^a C_a^{\phantom{a}b} \phi_b - \delta \lambda^a \phi_a - \lambda^a \epsilon^b C_{ab}^{\phantom{ab}c} \phi_c) + B \\ 
    = \int dt ~ (\dot \epsilon^c + \epsilon^a C_a^{\phantom{c}} - \delta \lambda^c - \lambda^a \epsilon^b C_{ab}^{\phantom{ab}c})\phi_c + B = B
\end{gathered}
\end{equation}
\end{proof}

    Calcolando la carica di Noether associata ad una simmetria di gauge, scopre che la carica è nulla. Inoltre bisogna fare attenzione ai gradi di libertà. Teorie non di gauge con il funzionale di azione del tipo 
\begin{equation*}
    S = \int dt (p_i q^i - H_0)
\end{equation*}
    con $i=1,~2, ldots, N$, possiedono 2N costanti di integrazione per risolvere le equazioni di Hamilton, dunque i gradi di libertà sono $\frac{1}{2} 2N = N$. Teorie di gauge invece, con funzionale di azione
\begin{equation*}
    S = \int dt (p_i q^i - H_0 - \lambda^a \phi_a
\end{equation*}
    con $a=1,~2, ldots, g$, possiedono anch'essi 2n costanti di integrazione ma ci sono $g$ vincoli dovuti a simmetrie di gauge che implicano la non necessità di conoscerle tutte: il numero di gradi di libertà è $\frac{1}{2} (2n - 2g) = N - g$.

    Nei prossimi paragrafi, mostriamo due esempi di teorie di gauge: la particella relativistica e l'elettrodinamica libera. Applicheremo la stessa procedura, ovvero scriveremo la lagrangiana, troveremo l'hamiltoniana attraverso una trasformazione di Legendre e vedremo come i vincoli generano le simmetrie di gauge. 

\section{La particella relativistica}

    Consideriamo una particella relativistica, descritta dalla curva $x^\mu(\tau)$ con $\tau$ un parametro qualsiasi, non necessariamente il tempo proprio. La sua azione è
\begin{equation} \label{azionerel}
    S = -m \int ds = - m \int d\tau \sqrt{- \dv{x^\mu}{\tau}\dv{x^\nu}{\tau} g_{\mu\nu}}
\end{equation}
    Questa azione è invariante per riparametrizzazione, una trasformazione non fisica 
\begin{equation*}
    \tau' = \tau'(t) \qquad x'^\mu (\tau') = x^\mu(\tau)
\end{equation*} 
    Rendendo tale trasformazione infinitesima, prendendo $\tau' = \tau + \epsilon(\tau)$, otteniamo sulla falsa riga di \eqref{simmtempo}
\begin{equation*}
    \delta x^\mu(\tau) = - \epsilon(\tau) \dot x^\mu
\end{equation*}
    Calcoliamo la variazione dell'azione 
\begin{equation*}
    \delta S = - m \int d\tau \frac{- \dot x^\mu \delta \dot x_\mu}{\sqrt{- \dot x^2}} = - m \int d\tau \dv{}{\tau} \Big ( \epsilon(\tau) \sqrt{- \dot x^2} \Big)
\end{equation*}
    e dunque il termine al bordo $K$ sarà 
\begin{equation*}
    K = - m \epsilon(\tau) \sqrt{- \dot x^2}
\end{equation*}

    Calcoliamo ora attraverso il teorema di Noether, la carica associata \eqref{carica}
\begin{equation*}
    Q = K - \frac{\partial L}{\partial \dot x} \delta_s x = ?
\end{equation*}
    Dunque risulta che la carica associata è nulla, dato che è una simmetria di gauge. 

    Cercando di trovare l'hamiltoniana, scrivendo il momento canonico 
\begin{equation*}
    p_\nu = \frac{m \dot x_\nu}{\sqrt{- \dot x^\mu \dot x_\mu}}
\end{equation*}
    non ci troviamo in grado di risolvere $\dot x^\mu$ in termini di $p_\mu$. Ciò è dovuto al fatto che non ci sono come sembrerebbe 4 equazioni indipendenti, ma soltanto 3 a causa del vincolo dovuto alla conservazione dell'energia 
\begin{equation*}
    \phi = p_\nu p^\nu + m^2 = 0
\end{equation*}
    Non useremo il metodo di Dirac ma definiamo un'altra azione equivalente, dovuta a Polyakov, che ci permetterà di risolvere il problema. L'azione di cui stiamo parlando è introdotta con l'aggiunta di una variabile ausiliaria $e(\tau)$, chiamata einbein, che come vedremo in seguito corrisponde ad un moltiplicatore di Lagrange
\begin{equation} \label{poly}
    S[x^\mu(\tau), ~ e(\tau)] = \frac{1}{2} \int d\tau \Big( \frac{1}{e} \dot x^\mu \dot x_\mu - e m^2 \Big)
\end{equation}
    Siccome la lagrangiana non contiene alcuna derivata di $e$, le equazioni del moto corrispondenti ad $e(\tau)$  
\begin{equation*}
    \pdv{L}{e} = 0
\end{equation*}
    si riducono a
\begin{equation*}
    e(x^\mu) = \frac{}{m} \sqrt{-\dot x^2}
\end{equation*}
    che sostituendole alla \eqref{poly} ritroviamo \eqref{azionerel}. Perciò le due azioni sono equivalenti. 

    Anche l'azione di Polyakov è invariante per riparametrizzazione, e in aggiunta alla variazione di $x$ troviamo come varia $e$
\begin{equation*}
    e'(\tau') = e(\tau) \dv{\tau}{\tau'}
\end{equation*}
    e la controparte infinitesima $e'(\tau + \epsilon) = e(1 - \dot \epsilon)$ conduce alla variazione
\begin{equation*}
    \delta e(\tau) = - \dv{}{\tau} (\epsilon(\tau) e)
\end{equation*}
    Ricapitolando, le variazione dei campi in seguito ad una riparametrizzazione di $\tau$ porta a 
\begin{equation}
    \delta x^\mu(\tau) = - \epsilon(\tau) \dot x^\mu \qquad \delta e(\tau) = - \dv{}{\tau} (\epsilon(\tau) e)
\end{equation}
    Calcoliamo la variazione dell'azione
\begin{equation*}
    \delta S  = \frac{1}{2} \int d\tau \Big ( \frac{2 e \dot x^\mu \delta x_\mu - \dot x^2 \delta e}{e^2} - m^2 \delta e \Big) = -\frac{1}{2} \int d\tau \dv{}{\tau} \Big(\epsilon \Big (\frac{\dot x^2}{e} - m^2 e \Big )\Big)
\end{equation*}
    e dunque il termine al bordo è 
\begin{equation*}
    K = - \frac{1}{2} \epsilon(\tau) \Big (\frac{\dot x^2}{e} - m^2\Big ) = - \epsilon(\tau) L
\end{equation*}

    L'hamiltoniana associata all'azione di Polyakov è 
\begin{equation*}
    H(p_\mu, ~x^\mu, ~e) ) \frac{1}{2} e (p_\mu p^\mu + m^2)
\end{equation*}
    e dunque l'azione hamiltoniana diventa 
\begin{equation*}
    S[p_\mu, ~x^\mu, ~e] = \int d\tau \Big(p_\mu \dot x^\mu - \frac{1}{2} e (p_\mu p^\mu + m^2) \Big)
\end{equation*}
    Osserviamo che ha la stessa identica forma di \eqref{azionevincolo}, con $H_0 = 0$ dato che per riparametrizzazioni l'hamiltoniana non è collegata ad alcun tempo. Dunque il vincolo è proprio $\phi = \frac{1}{2} (p^2 = m^2)$ ed $e$ sono i moltiplicatori di Lagrange. 

    Siccome è presente un solo vincolo, sarà di prima classe e $C_{ab} = 0$. Quindi la trasformazione di gauge è
\begin{equation}
    \delta x^\mu = [x^\mu, ~\epsilon(\tau) \frac{1}{2} (p^2 + m^2)] = \epsilon(\tau) p^\mu
\end{equation}
\begin{equation}
    \delta p_\mu = [p_\mu, ~\epsilon(\tau) \frac{1}{2} (p^2 + m^2)] = 0
\end{equation}
\begin{equation}
    \delta e = \dot \epsilon(\tau)
\end{equation}
    e quindi la variazione dell'azione è davvero un termine al bordo
\begin{equation*}
    \delta S = \int d\tau \Big( p_\mu \delta \dot x^\mu - \frac{1}{2} \delta e (p^2 + m^2)\Big) = \int d\tau \dv{}{\tau} \Big (\frac{1}{2} \epsilon(\tau) (p^2 - m^2) \Big)
\end{equation*}

\section{Elettrodinamica libera} 

    Consideriamo l'azione associata alla teoria di Maxwell d la separiamo in componenti spaziali e temporali
\begin{equation*}
\begin{aligned}
    S & = -\frac{1}{4} \int d^4 x F^{\mu\nu} F_{\mu\nu} = \int d^4 x \Big ( -\frac{1}{2} F^{0i}F_{0i} - \frac{1}{4} F^{ij} F{ij} \Big) \\ & = \int d^4 x \Big( \frac{1}{2} (\dot A_i - \partial_i A_0) (\dot A^i - \partial^i A_0) - \frac{1}{4} F^{ij} F{ij} \Big ) \\ & = \int d^4 x (\frac{1}{2} \dot A_i \dot A^i - \dot A_i \partial^i A_0 + \frac{1}{2} \partial_i A_0 \partial^i A_0 - \frac{1}{4} F^{ij} F{ij})
\end{aligned}
\end{equation*}

    Definiamo dunque il momento associato alle uniche componenti del campo che presentano derivate temporali $A_i$
\begin{equation*}
    \pi_i = \pdv{\mathcal L}{\dot A_i} = \dot A_i - \partial_i A_0
\end{equation*}
    che possiamo intepretare fisicamente come il campo elettrico $p^i = E^i$. Attraverso una trasformazione di Legendre troviamo l'hamiltoniana  
\begin{equation*}
\begin{aligned}
    H(p, ~A) & = \pi_i \dot A^i - L \\ & = \pi_i (\pi^i + \partial^i A_0) - \Big(\frac{1}{2} (\pi_i + \partial_i A_0)(\pi^i + \partial^i A_0) - (\pi_i + \partial_i A_0) \partial^i A_0 \\ & ~ + \frac{1}{2} \partial_i A_0 \partial^i A_0 - \frac{1}{4} F^{ij} F{ij}\Big) \\ & = \frac{1}{2} \pi_i \pi^i + \frac{1}{4} F_{ij} F^{ij} - A_0 \partial_i \pi^i
\end{aligned}
\end{equation*}
    dove abbiamo cancellato i termini al bordo nell'ultimo passaggio. Quindi l'azione hamiltoniana per l'elettromagnetismo è
\begin{equation}
    S[A_i, \pi_i, ~A_0] = \int d^4 x \Big ( \pi_i \dot A^i - \Big ( \frac{1}{2} \pi_i \pi^i + \frac{1}{4} F_{ij} F^{ij} \Big) + A_0 \partial_i \pi^i \Big)
\end{equation}
    che si presenta nella forma cercata \eqref{azionevincolo}, dove $H_0$ è l'energia corrispondente all'hamiltoniana dinamica 
\begin{equation*}
    H_0 = \frac{1}{2} \pi_i \pi^i + \frac{1}{4} F_{ij} F^{ij} = \frac{1}{2} (E^2 + B^2)
\end{equation*}
    e $A_0$ è il moltiplicatore di Lagrange, il cui vincolo è l'equazione di Gauss
\begin{equation*}
    \phi = \partial_i \pi^i = \nabla \cdot E = 0
\end{equation*}

    Calcolando le equazioni del moto 
\begin{equation*}
    \dot \pi_i = - \pdv{H}{A^i} \qquad \dot A_i = \pdv{H}{\pi^i} \qquad \phi = 0
\end{equation*}
    ci troviamo nello stesso problema di dover garantire che il vincolo sia preservato durante l'evoluzione, cioè che $\dot \phi = [\phi, ~H] = 0$. Essendo in teoria dei campi, dobbiamo però calcolare le parentesi di Poisson in due coordinate differenti $x$ e $x'$
\begin{equation*}
    [\phi(x), ~ H(x')] = \Big[\partial_i \pi^i, ~ \frac{1}{2} \Big( \pi_i \pi^i + F_{ij} F^{ij}(x') - \frac{1}{2} A_0 \partial_i \pi^i(x') \Big) \Big]
\end{equation*}
    Utilizzando le parentesi di Poisson canoniche derivanti dalla stuttura simplettica della teoria dei campi
\begin{equation*}
    [\pi_i(x), ~\pi_j(x')] = 0 \qquad [A_i(x), ~A_j(x')] = 0 \qquad [A_i(x), ~\pi_j(x')] = \delta_{ij} \delta^3(x - x')
\end{equation*}
    otteniamo 
\begin{equation*}
    \dv{}{t} \phi = [\phi(x), ~H(x')] = 2 \partial_k \partial'_i [\pi_k(x), ~A_j(x')] F^{ij(x')} = -2(\partial_k \partial'_i \delta^3(x - x')) F^{ij} (x') = 0
\end{equation*}
    a causa della cancellazione dei termini dovuta alla simmetria-antisimmetria. Conludiamo quindi che il vincolo si conserva sulle equazioni del moto e genera una simmetria di gauge. Per mostrare il gauge, consideriamo come vincolo un funzionale $\Phi[\Lambda(x)]$ dipendente dalla funzione di gauge $\Lambda(x)$ nel seguente modo 
\begin{equation*}
    \Phi[\Lambda(x)] = \int d^3 x \Lambda(x) \partial_i \pi^i(x)
\end{equation*}
    dove non abbiamo integrato nel tempo per mantenere la dipendenza temporale al fine di poter calcolare parentesi di Poisson allo stesso istante. La trasformazione di gauge sono definite come 
\begin{equation*}
\begin{aligned}
    \delta A_i (x) & = [A_i(t, x), ~\Phi[\Lambda]] = \int d^3 x' \Lambda(x', ~t) \partial'_j [A_i(t, ~x), ~ \pi^j(t, ~x')] \\ & = \int d^3 x' \Lambda(t, ~x') \delta_{ji} \delta^3 (x-x') = - \partial_i \Lambda(t, ~x)
\end{aligned}
\end{equation*}
    che è proprio l'invarianza di gauge presente in elettrodinamica. D'altra parte la trasfrmazione di gauge del momento coniugato deve essere nulla 
\begin{equation*}
    \delta \pi_i(t, ~x) = [\pi_i, ~\Phi[\Lambda]] = \int d^3 x' \Lambda(t, ~x')[\pi_i, ~\partial_j \pi^j] = 0
\end{equation*}
    Infine calcoliamo la variazione dell'azione 
\begin{equation*}
\begin{aligned}
    \delta S & = \int d^4 x (\pi_i \delta \dot A^i + \delta A_0 \phi) = \int d^4 (-\pi_i\partial^i \partial_0 \Lambda + \delta A_0 \partial_i \pi^i ) \\ & = \int d^4 x (\partial_i (-\pi^i \partial_o \Lambda) + \partial_i \pi^i \partial_0 \Lambda + \delta A_0 \partial_i \pi^i ) \\ & = \int d^4 x (\partial_\mu K^\mu + \partial_i \pi^i (\delta A_0 + \partial_9 \Lambda))
\end{aligned}
\end{equation*}
    dove abbiamo posto $K^\mu = (0, ~- \pi^i \partial_0 \Lambda)$ e, essendo invarianti di gauge, abbiamo usato $\delta \pi_i = 0$, $\delta H_0 = 0$ e $\delta \phi = 0$. Dunque concludiamo che dobbiamo avere $\delta A_0 = - \partial_0 \Lambda$ per avere un termine al bordo. In questo modo, la trasformazione di gauge del campo $A_\mu$ generata dal vincolo dell'equazione di Gauss è proprio quella aspettata:
\begin{equation*}
    \phi = \partial_i \pi^i = \nabla \cdot E = 0
\end{equation*}
    genera 
\begin{equation*}
    \delta A_\mu = - \partial_\mu \Lambda(x)
\end{equation*}

\chapter*{Conclusioni}
\addcontentsline{toc}{chapter}{Conclusioni}

    Concludiamo la presente tesi, mostrando che i teoremi di Noether non si fermano solamente ad applicazioni riguardanti la meccanica classica o la teoria relativistica dei campi, ma che può essere uno strumento molto potente anche nell'ambito della fisica moderna: dall'elettrodinamica quantistica fino alla fisica delle particelle. Mostriamo qui una tabella che mostra come le principali leggi di conservazione siano associate a simmetrie continue di gauge\footnote{Alla parità e ad altre simmetrie discrete non si applicano i teoremi di Noether}

    \begin{center}
    \begin{tabular}{ c | c } 
      Legge di conservazione & Simmetria \\ 
      \hline
      Conservazione dell'energia & Traslazione temporale \\ 
      \hline
      Conservazione della quantità di moto & Traslazione spaziale \\ 
      \hline
      Conservazione del momento angolare & Rotazione spaziale \\ 
      \hline
      Conservazione del centro di massa & Boost di Lorentz \\ 
      \hline
      Conservazione della carica elettrica & Invarianza di gauge $U(1)$ \\ 
      \hline
      Conservazione della carica di colore & Invarianza di gauge $SU(3)$ \\ 
      \hline
      Conservazione dell'isospin debole & Invarianza di gauge $SU(2)$ \\ 
    \end{tabular}
    \end{center}

\clearpage
\phantomsection
\addcontentsline{toc}{chapter}{Bibliografia}
\printbibliography

\end{document}