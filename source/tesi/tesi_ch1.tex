\chapter{Il primo teorema in meccanica classica}
    
    Il primo sistema fisico che prendiamo in considerazione è un sistema di particelle o punti materiali, ovvero un insieme discreto costituito da oggetti le cui dimensioni posso essere trascurate quando ad esserne descritto è il loro moto. Al fine di studiare la relazione tra simmetrie e leggi di conservazione, è necessario approfondire la struttura matematica per descrivere tali sistemi. 

\section{Le equazioni del moto}
    
    Introduciamo dapprima quali sono le quantità fisiche che ci permettono di analizzare il moto: posizione, velocità, accelerazione. Nell'ambito della fisica classica\footnote{In questo caso classica ha l'accezione di non relativistica e non quantistica}, in cui svilupperemo questo capitolo, lo spazio matematico che utilizziamo per studiare il sistema fisico è lo spazio tridimensionale euclideo $\mathbb R^3$ e inoltre assumiamo che il tempo sia assoluto.
    
    Nel sistema cartesiano di assi coordinati ortogonali, la posizione di una particella viene generalmente indicata con un vettore $\mathbf r$, le cui componenti sono 
    \begin{equation*}
        r^i = (x, ~y, ~z)
    \end{equation*}
    con $i = 1, ~2, ~3$. La derivata prima rispetto al tempo del vettore posizione viene chiamata velocità $\mathbf v = \mathbf {\dot r}$, le cui componenti sono 
    \begin{equation*}
        \dot r^i = \dv{r^i}{t} = (\dot x, ~\dot y, ~\dot z)
    \end{equation*}
    mentre la derivata seconda rispetto al tempo del vettore posizione, o equivalentemente la derivata prima della velocità, viene chiamata accelerazione $\mathbf a = \mathbf{\dot v} = \mathbf{\ddot r}$, le cui componenti sono 
    \begin{equation*}
        \ddot r^i = \dvd{r^i}{t} = (\ddot x, ~\ddot y, ~\ddot z)
    \end{equation*}
    
    Se per una singola particella definiamo $3$ coordinate, per un sistema di N particelle libere sono necessarie $3N$ coordinate, $3N$ velocità, $3N$ accelerazioni per descriverlo. 
    
    Il sistema cartesiano non è l'unico disponibile. Infatti è possibile utilizzare un generico sistema di assi coordinati purchè siano di numero pari al numero di gradi di libertà $d$, che corrisponde al numero di quantità indipendenti necessarie per definire univocamente la posizione di un sistema fisico. Una particella libera ha $d = 3$, mentre un sistema di N particelle ha $d = 3N$. Chiameremo queste quantità coordinate generalizzate $q^i$ e, analogamente, chiameremo velocità generalizzate $\dot q^i$ e accelerazioni generalizzate $\ddot q^i$ le rispettive derivate prime e seconde. La configurazione istantanea del sistema, cioè ad un tempo fissato, può quindi essere rappresentata da un punto in uno spazio $\mathcal C$, chiamato spazio delle configurazioni, di dimensione pari al numero di gradi di libertà ed i cui assi coordinati sono le coordinate generalizzate. Come precedentemente indicato, nel caso di una particella, lo spazio delle configurazioni è lo spazio tridimensionale euclideo $\mathbb R^3$, mentre per un sistema di N particelle libere sarà il prodotto cartesiano $\mathbb R^{3N}$.
    
    Il moto del sistema è matematicamente descritto dalla posizione del sistema in funzione del tempo, chiamata traiettoria, ciò che geometricamente viene rappresentato da una curva $q^i(t)$ nello spazio delle configurazioni. Conoscendo quest'ultima, siamo a conoscenza della completa evoluzione temporale del sistema. Come possiamo ricavarla? La traiettoria è soluzione di equazioni che legano accelerazioni, velocità e posizioni, chiamate equazioni del moto
    \begin{equation} \label{eqmoto}
        f_j(q^i, ~ \dot q^i, ~ \ddot q^i) = 0
    \end{equation}
    dove $j = 1, ~2, ~\ldots ~d$. Esse sono un sistema formato da $d$ equazioni differenziali, una per ogni grado di libertà, e la presenza delle accelerazioni ci suggerisce che sono al secondo ordine, caratteristica che porta una importante conseguenza: per studiare come il sistema evolve nel tempo, la conoscenza delle sole coordinate in un dato istante non è sufficiente. Difatti, in modo tale che le accelerazioni siano univocamente determinate, è necessario fornire anche le velocità. Ciò deriva da un risultato dell'analisi matematica: conoscendo le condizioni iniziali, ovvero le $d$ posizioni $q^i(0) = q^i_0$ e le $d$ velocità $\dot q^i(0) = v^i_0$, la soluzione del problema di Cauchy delle equazioni del moto è unica. 
    
    Come possiamo trovare esplicitamente le~\eqref{eqmoto}? In meccanica classica, si distinguono principalmente tre differenti formalismi dovuti a tre importanti figure della storia della fisica: Isaac Newton $(1642-1726)$, Joseph-Louis Lagrange $(1736-1813)$ e William Rowan Hamilton $(1805-1865)$. 

    La descrizione newtoniana consiste nel trovare le forze $F^j(t, ~q^i, ~ \dot q^i)$ che agiscono sul sistema e poi risolvere la celeberrima equazione conosciuta come secondo principio della dinamica
    \begin{equation*}
        F^j(t, ~q^i, ~ \dot q^i) = m \ddot q^j
    \end{equation*}
    o qualora la massa $m$ non sia costante 
    \begin{equation*}
        F^j(t, ~q^i, ~ \dot q^i) = \dv{p^j}{t}
    \end{equation*}
    dove $p^j = m \dot q^j$ è la componente $j$-esima della quantità di moto. In ogni caso, in questa tesi non tratteremo questo formalismo, bensì nel prossimo paragrafo ci concentreremo principalmente sulla descrizione lagrangiana, mentre alla fine del capitolo illustreremo anche quella hamiltoniana.
    
\section{Meccanica lagrangiana}

    Il formalismo lagrangiano consiste nell'associare ad un generico sistema meccanico una funzione $L$ delle coordinate $q^i$, velocità $\dot q^i$ ed eventualmente del tempo $t$, chiamata lagrangiana del sistema
    \begin{equation} \label{lagrangiana}
        L = L(t, ~q^i, ~\dot q^i) 
    \end{equation}  
    Non c'è un criterio generale per trovare tale funzione, ma c'è una classe di sistemi meccanici in cui la lagrangiana può essere scritta nella forma $L = T - U$, dove $T$ e $U$ sono rispettivamente l'energia cinetica e l'energia potenziale del sistema. Tale sistema viene chiamato sistema conservativo e richiede che l'energia potenziale $U = U(q^i)$ sia funzione soltanto delle coordinate.
    
    Attraverso la lagrangiana~\eqref{lagrangiana}, introduciamo il funzionale di azione $S$ che associa un numero reale ad ogni curva $q^i(t)$ dello spazio delle configurazioni, definita all'interno di un intervallo temporale fisso $[t_1, ~t_2]$
    \begin{equation} \label{azione}
        S[q^i(t)] = \integ{t_1}{t_2}{t} L(t, ~q^i, ~\dot q^i)
    \end{equation}

\subsection{Il principio di azione stazionaria e le equazioni di Eulero-Lagrange}

    Enunciamo ora il principio che ci permette di trovare le equazioni del moto a partire dalla lagrangiana del sistema: il principio di azione stazionaria, chiamato anche impropriamente principio di minima azione.

    \begin{principle}[Di azione stazionaria]
        Tra tutte le curve nello spazio delle configurazioni $\mathcal C$ che collegano due estremi fissati $q_1$ e $q_2$ di un intervallo temporale anch'esso fissato $[t_1, ~t_2]$ tale che 
    \begin{equation*}
        q^i(t_1) = q^i_1 \qquad q^i(t_2) = q^i_2
    \end{equation*}
        il sistema meccanico associato alla lagrangiana~\eqref{lagrangiana} percorre la curva $q^i(t)$ che rende stazionaria la sua azione~\eqref{azione}, in formule
    \begin{equation} \label{azionestazionaria}
        \delta S [q^i(t)] = 0
    \end{equation}
    \end{principle}

    Cerchiamo dunque di tradurre questo principio in equazioni che dipenderanno dalla lagrangiana del sistema e delle sue derivate e che siano equivalenti alle equazioni del moto~\eqref{eqmoto}. Siccome l'azione è un funzionale, ovvero una funzione che non dipende da un numero discreto di variabili, ma il cui dominio è in uno spazio di funzioni infinito dimensionale, non possiamo utilizzare direttamente il calcolo differenziale ma è necessario attingere alla branca della matematica, chiamata calcolo delle variazioni, che studia come trovare la funzione nel dominio del funzionale tale che quest'ultimo sia stazionario, ovvero nel nostro caso quella che soddisfa il principio di azione stazionaria. 
    
    Supponiamo che la curva $q^i(t)$ sia quella cercata, ovvero quella che rende stazionaria l'azione~\eqref{azionestazionaria}, e prendiamo un'altra curva $\delta q^i(t)$, che chiameremo variazione, definita sempre nell'intervallo $[t_1, ~t_2]$, con la proprietà di essere arbitraria tranne che per gli estremi
    \begin{equation} \label{estreminulli}
        \delta q^i(t_1) = \delta q^i(t_2) = 0
    \end{equation}
    Introduciamo una famiglia di curve $q^i(t, ~\epsilon)$ dipendenti da un parametro $\epsilon$, definita nel seguente modo 
    \begin{equation} \label{famigliacurve}
        q^i(t, \epsilon) = q^i(t, ~0) + \epsilon \delta q^i(t)
    \end{equation}
    ovvero una famiglia di curve che connetta la curva cercata $q^i(t)$ con la variazione $\delta q^i(t)$. Segue dalla definizione~\eqref{famigliacurve} che se il parametro si annulla $\epsilon=0$, la famiglia di curve si riduce alla curva cercata
    \begin{equation*}
        q^i(t, ~ 0) = q^i(t)
    \end{equation*}
    In questo modo, sostituendo nell'azione~\eqref{azione} la famiglia di curve~\eqref{famigliacurve}, possiamo parametrizzarlo con il parametro reale $\epsilon$
    \begin{equation*}
        S(\epsilon) = \integ{t_1}{t_2}{t} L(t, ~q^i(t, ~\epsilon), ~\dot q^i(t, \epsilon))
    \end{equation*}
    riconducendoci ad una funzione dipendente da una variabile reale, così da poter utilizzare le conosciute tecniche del calcolo differenziale. Definiamo la variazione dell'azione in $q^i(t)$ come la derivata rispetto al parametro $\epsilon$ calcolata in $\epsilon = 0$
    \begin{equation} \label{variazioneazione}
        \delta S[q^i(t), ~ \delta q^i(t)] = \dv{}{\epsilon} S(\epsilon) \Big \vert_{\epsilon = 0} = \dv{}{\epsilon} S[q^i(t) + \epsilon \delta q^i(t)] \Big \vert_{\epsilon = 0}
    \end{equation}
    e calcoliamo esplicitamente la derivata, differenziando sotto il segno di integrale
    \begin{equation*}
        \dv{S}{\epsilon} = \integ{t_1}{t_2}{t} \Big (\pdv{L}{q^i} \pdv{q^i}{\epsilon} + \pdv{L}{\dot q^i} \pdv{\dot q^i}{\epsilon} \Big)
    \end{equation*}\label{prova1}
    Osserviamo che non è presente la derivata rispetto ad t perché quest'ultimo non dipende da $\epsilon$, conseguentemente le loro derivate commutano. L'integrazione per parti del secondo termine nell'integrale ci conduce a
    \begin{equation}\label{prova2}
        \dv{S}{\epsilon} = \integ{t_1}{t_2}{t} \Big (\pdv{L}{q^i} \pdv{q^i}{\epsilon} - \dv{}{t} \Big (\pdv{L}{\dot q^i} \Big) \pdv{q^i}{\epsilon} \Big) + \pdv{L}{\dot q^i} \pdv{q^i}{\epsilon} \Big \vert_{t_1}^{t_2}
    \end{equation}
    Riprendendo la~\eqref{famigliacurve}, notiamo che la derivata parziale della curva $q^i$ rispetto al parametro $\epsilon$ è la variazione della curva $\delta q^i$
    \begin{equation}\label{prova3}
        \pdv{q^i}{\epsilon} = \pdv{}{\epsilon} (q^i(t, ~ 0) + \epsilon \delta q^i(t)) = \delta q^i(t)
    \end{equation}
    Sostituendo~\eqref{prova3}~in~\eqref{prova2}, otteniamo 
    \begin{equation*}
    \begin{aligned}
        \dv{S}{\epsilon} & = \integ{t_1}{t_2}{t} \Big (\pdv{L}{q^i} \delta q^i(t) - \dv{}{t} \Big (\pdv{L}{\dot q^i} \Big) \delta q^i(t) \Big) + \pdv{L}{\dot q^i} \delta q^i(t) \Big \vert_{t_1}^{t_2} \\  & = \integ{t_1}{t_2}{t} \Big (\pdv{L}{q^i} \delta q^i(t) - \dv{}{t} \Big (\pdv{L}{\dot q^i} \Big) \delta q^i(t) \Big) + \pdv{L}{\dot q^i} \delta q^i(t_2) - \pdv{L}{\dot q^i} \delta q^i(t_1) \\ & = \integ{t_1}{t_2}{t} \delta q^i(t) \Big (\pdv{L}{q^i}  - \dv{}{t} \Big (\pdv{L}{\dot q^i} \Big) \Big) + \pdv{L}{\dot q^i} \delta q^i(t_2) - \pdv{L}{\dot q^i} \delta q^i(t_1)
    \end{aligned}
    \end{equation*}
    dove nell'ultimo passaggio abbiamo raccolto un fattore comune $\delta q^i(t)$. Ricordando che agli estremi la variazione si annulla~\eqref{estreminulli}, gli ultimi due termini si cancellano e otteniamo la relazione
    \begin{equation} \label{prova4}
        \dv{S}{\epsilon} = \integ{t_1}{t_2}{t} \delta q^i(t) \Big (\pdv{L}{q^i}  - \dv{}{t} \pdv{L}{\dot q^i} \Big)
    \end{equation}
    Adoperiamo il fatto che condizione necessaria affinchè la curva cercata $q^i(t)$ renda il funzionale d'azione stazionario è che la variazione di quest'ultimo si annulli~\eqref{azionestazionaria}, qualunque sia la variazione della curva $\delta q^i(t)$
    \begin{equation} \label{prova5}
        \delta S[q^i(t), ~ \delta q^i(t)] = 0 \qquad \forall ~ \delta q^i(t)
    \end{equation}
    Mettendo insieme~\eqref{prova4} e~\eqref{prova5}, attraverso la~\eqref{variazioneazione}, otteniamo
    \begin{equation*}
        \delta S = \integ{t_1}{t_2}{t} \delta q^i(t) \Big (\pdv{L}{q^i}  - \dv{}{t} \pdv{L}{\dot q^i} \Big) = 0
    \end{equation*}
    Utilizzando il lemma fondamentale del calcolo delle variazioni, possiamo asserire che l'integranda si annulla per qualsiasi variazione della curva $\delta q^i(t)$ e dunque giungere al sistema di equazioni
    \begin{equation} \label{eullag}
        \dvf{S}{q^i} = \pdv{L}{q^i}  - \dv{}{t} \pdv{L}{\dot q^i} = 0
    \end{equation}
    oppure anche scritte esplicitando la derivata temporale nel secondo termine
    \begin{equation*} 
        \pdv{L}{q^i}  - \pdvd{L}{\dot q^i}{t} - \pdvd{L}{\dot q^i}{q^j} \dot q^j - \pdvd{L}{\dot q^i}{\dot q^j} \ddot q^j = 0
    \end{equation*}
    Queste sono le equazioni del moto che stavamo cercando, equazioni differenziali al secondo ordine della lagrangiana, chiamate equazioni di Eulero-Lagrange. Una volta trovata la lagrangiana associata al sistema fisico che stiamo studiando, possiamo ricondurci alle~\eqref{eqmoto} attraverso le~\eqref{eullag}.

    \hfill 

    Per completezza, concludiamo il paragrafo enunciando e dimostrando il lemma fondamentale del calcolo delle variazioni.
    \begin{lemma}[Fondamentale del calcolo delle variazioni]
        Sia $f(x)$ una funzione continua tale che 
    \begin{equation*}
        \integ{x_1}{x_2}{x} g(x) f(x) = 0 
    \end{equation*}
        per ogni funzione $g(x)$, continua, insieme alle derivata prima e seconda, e che si annulli agli estremi
    \begin{equation*}
        g(x_1) = g(x_2) = 0
    \end{equation*}
        allora $f(x)$ = 0
    \end{lemma}
    Nel nostro caso abbiamo al posto di $f$ le equazioni di Eulero-Lagrange e al posto di $g$ la variazione $\delta q^i(t)$. Proponiamo una dimostrazione non formale ma intuitiva.
    \begin{proof}
        Per assurdo se $f$ è positiva in un intorno di un punto $[x'_1, ~ x'_2]$ e nulla in ogni altro punto dell'intervallo $[x_1, ~ x_2]$, data l'arbitrarietà di $g$, per qualsiasi funzione che sia positiva otteniamo
    \begin{equation*}
        \integ{x_1}{x_2}{x} g(x) f(x) = \integ{x'_1}{x'_2}{x} g(x) f(x) > 0
    \end{equation*} 
        che contraddice l'ipotesi che questo integrale sia nullo e mostrando l'assurdo.
    \end{proof}

\section{Simmetrie}
    
    Introduciamo ora il concetto di simmetria dell'azione. The fundamental importance in physics of symmetries are summed up by Hermann Weyl, who said that ``As far as I can see, all a priori statements in physics have their origin in symmetry''. Intuitivamente, un oggetto è simmetrico se sotto l'effetto di una trasformazione rimane uguale a se stesso. Esempi sono la rotazione di un arbitrario angolo di un cerchio o la riflessione di un oggetto a simmetria bilaterale rispetto all'asse di simmetria. Deduciamo quindi che una simmetria dell'azione è una speciale classe di trasformazioni delle coordinate che lasciano l'azione invariata. 
    
    Una trasformazione di coordinate è un cambio di coordinate spaziali per descrivere il problema, ovvero parametrizzare la nostra azione con differenti coordinate generalizzate $q'^i$. Distinguiamo due modi differenti di interpretare una trasformazione di coordinate: quello passivo e quello attivo. Una trasformazione passiva è un cambio del sistema di riferimento, cioè l'oggetto viene descritto da due differenti osservatori; mentre in una trasformazione attiva si sposta effettivamente il sistema preso in considerazione mantenendo invariato il sistema di riferimento, cioè si sposta l'oggetto mantenendo l'osservatore invariato. 
    
    Possiamo definire l'invarianza dell'azione imponendo che quello espresso con le nuove coordinate $S'$ conduce alle stesse equazioni del moto di quello espresso con le vecchie $S$. Portiamo particolare attenzione al significato di invarianza: non necessariamente significa che la nuova e la vecchia azione siano identiche $S' = S$ ma che siano uguali a meno di un termine al bordo $K$. Ciò è conseguenza del fatto che ad avere valore fisico sono le equazioni del moto e non la lagrangiana~\eqref{lagrangiana}, che in effetti presenta un'ambiguità: aggiungendo la derivata totale rispetto ad un'arbitraria funzione $K$, il termine al bordo, porta alle stesse equazioni del moto~\eqref{eullag}. Mostriamo quindi che nell'azione $S'$ con il termine al bordo
    \begin{equation*}
        S' = \integ{t_1}{t_2}{t} \Big ( L(q^i, ~ \dot q^i, ~t) + \dv{}{t} K \Big)
    \end{equation*}
    la variazione del secondo termine è nulla
    \begin{equation*}
        \delta \integ{t_1}{t_2}{t} \dv{}{t} K = \pdv{K}{q^i} \delta q^i \Big \vert_{t_1}^{t_2} = \pdv{K}{q^i} \delta q^i (t_2) - \pdv{K}{q^i} \delta q^i (t_1) = 0
    \end{equation*} 
    dove abbiamo usato la~\eqref{estreminulli}. Dunque $\delta S = \delta S'$ e le equazioni del moto non cambiano.

    Formalizziamo il tutto.

    \begin{definition}[Simmetria dell'azione]
        Definiamo una simmetria dell'azione come una trasformazione qualsiasi, non necessariamente infinitesima, delle coordinate spaziali 
    \begin{equation}\label{simmazione}
        \delta_s q^i(t) = q'(t) - q(t)
    \end{equation} 
        tale che l'azione sia invariante a meno di un termine al bordo $K$
    \begin{equation} \label{invazione}
        \delta S[q^i(t), ~\delta_s q^i(t)] = S[q^i(t) + ~\delta_s q^i(t)] - S[q^i(t)] = \integ{t_1}{t_2}{t} \dv{}{t} K
    \end{equation}
    \end{definition}    
    Le simmetrie $\delta_s q^i$ sono quindi direzioni dello spazio generate dalle coordinate $q^i$ che rendono invariante l'azione $S$. Vedremo come il termine al bordo non interferisce con l'esistenza delle quantità conservate ma sarà necessario per trovarle. Sottolineiamo il fatto che le variazioni di simmetria $\delta_s q^i(t)$ devono soddisfare il vincolo dato dall'equazione~\eqref{invazione}, mentre le $q^i(t)$ sono del tutto arbitrarie.

\subsection{Trasformazioni temporali come deformazioni delle coordinate}

    Finora non abbiamo considerato trasformazioni riguardanti il tempo; le simmetrie sono solo cambi di coordinate spaziali. È possibile tradurre le traslazioni temporali in deformazioni delle coordinate. Consideriamo il caso più semplice unidimensionale dove due curve $q^i(t)$ e $q'^i(t)$ sono collegate da un traslazione temporale $t' = t + \epsilon$ nel seguente modo 
    \begin{equation*}
        q'(t') = q'(t + \epsilon) = q(t)
    \end{equation*} 
    Trattando il parametro $\epsilon$ come una quantità infinitesima, ovvero imponendo la condizione $\epsilon \ll 1$, espandiamo in serie di Taylor al primo ordine il termine $q'(t + \epsilon)$ e otteniamo
    \begin{equation*}
        q'(t + \epsilon) \simeq q'(t) + \epsilon \dot q(t) = q(t)
    \end{equation*}     
    Quindi possiamo scrivere la variazione di simmetria~\eqref{simmazione} 
    \begin{equation}\label{simmtempo}
        \delta_s q(t) = - \epsilon \dot q(t)
    \end{equation}
    che dunque coinvolge soltanto un istante $t$ e non anche $t'$ come prima. Conseguentemente, essendo $\delta_s q(t)$ una differenza valutata allo stesso istante $t$, commuta con la derivata rispetto al tempo
    \begin{equation}\label{commutatore}
        \delta_s \dot q(t) = \delta_s \dv{}{t} q(t) = \dv{}{t} \delta_s q(t) 
    \end{equation}
    Questa considerazione può essere evidenziata ricordando che nell'azione~\eqref{azione} il tempo è una variabile di integrazione, per definizione muta: effettuando una sostituzione, l'integrale non ne subisce gli effetti. Questa trasformazione non ha nulla a che vedere con le variazioni di simmetria dell'azione. 

\subsection{Variazioni on-shell}

    Le trasformazioni di simmetria dell'azione~\eqref{simmazione} non sono le uniche tipologie di variazioni, quando sono infinitesime. Infatti è possibile introdurre le cosidette variazioni on-shell, dove il termine on-shell significa che vengono applicate le equazioni del moto. 
    \begin{definition}[Variazione on-shell]
        Definiamo una variazione on-shell una trasformazione infinitesima delle coordinate spaziali 
    \begin{equation*}
        \delta q = q' - q
    \end{equation*} 
        tale che l'azione sia invariante a meno di una derivata temporale, dopo aver applicato le equazioni di Eulero-Lagrange~\eqref{eullag}. 
    \begin{equation} \label{invonshell}
        \delta S[\overline q^i,~\delta q^i] = S[\overline q^i + \delta q^i] - S[\overline q^i] = \integ{t_1}{t_2}{t} \dv{}{t} \Big( \pdv{L}{\dot q^i} \delta q^i \Big )
    \end{equation}  
        dove la notazione $\overline q^i$ indica che le $q^i$ rispettano le equazioni del moto. 
    \end{definition}     

    Infatti, effettuando una variazione dell'azione, otteniamo 
    \begin{equation*}
    \begin{aligned}
        \delta S[q^i,~\delta q^i] & = S[q^i + \delta q^i] - S[q^i] \\ & = \integ{t_1}{t_2}{t} \Big( \pdv{L}{q^i} \delta q^i + \pdv{L}{\dot q^i} \delta \dot q^i \Big) \\ &= \integ{t_1}{t_2}{t} \Big( \pdv{L}{q^i}  \delta q^i - \dv{}{t} \pdv{L}{\dot q^i}  \delta q^i \Big )+ \integ{t_1}{t_2}{t} \dv{}{t} \Big( \pdv{L}{\dot q^i} \delta q^i \Big ) \\ & = \integ{t_1}{t_2}{t} \Big( \pdv{L}{q^i}  - \dv{}{t} \pdv{L}{\dot q^i} \Big ) \delta q^i + \integ{t_1}{t_2}{t} \dv{}{t} \Big( \pdv{L}{\dot q^i} \delta q^i \Big )
    \end{aligned}
    \end{equation*}
    dove abbiamo integrato per parti, usato il fatto che le variazioni commutano con la derivata temporale~\eqref{commutatore} e raccolto un fattore comune $\delta q^i$ nell'ultimo passaggio.
    Notiamo che il primo termine comprende le equazioni di Eulero-Lagrange~\eqref{eullag}, quindi le applichiamo, passando on-shell, e otteniamo 
    \begin{equation*} 
        \delta S[\overline q^i,~\delta q^i] = S[\overline q^i + \delta q^i] - S[\overline q^i] = \integ{t_1}{t_2}{t} \dv{}{t} \Big( \pdv{L}{\dot q^i} \delta q^i \Big )
    \end{equation*}

    Concettualmente questa variazione è opposta rispetto a quella dell'azione, perchè quella sull'azione pone vincoli su $\delta_s q^i(t)$ e lascia libere le $q^i(t)$ mentre specularmente quella on-shell pone vincoli sulle $\overline q^i$ e lascia libere le $\delta q^i$.

\section{Leggi di conservazione}
    Soffermiamoci ora sulla nozione di quantità conservata. 
    \begin{definition}[Quantità conservata]
        Definiamo una quantità conservata come una funzione $Q$ tale che il suo valore rimane constante nel tempo
    \begin{equation} \label{th}
        \dv{}{t} Q = 0
    \end{equation}
    \end{definition} 

    Dallo studio delle equazioni di Eulero-Lagrange, notiamo che possiamo definirne già una: il momento coniugato di una coordinata ciclica. Una coordinata ciclica $q^j$ è una coordinata che non è presente esplicitamente nella lagrangiana
    \begin{equation} \label{cyclic}
        \pdv{L}{q^j} = 0
    \end{equation}
    mentre definiamo il momento coniugato $p_j$ ad una coordinata $q^j$ come la derivata parziale della lagrangiana rispetto alla velocità corrispondente a questa coordinata
    \begin{equation} \label{coniugato}
        p_j = \pdv{L}{\dot q^j}
    \end{equation}
    Prendendo in considerazione le~\eqref{eullag}
    \begin{equation*}
        \pdv{L}{q^j}  - \dv{}{t} \pdv{L}{\dot q^j} = 0
    \end{equation*}
    notiamo che se non è presente alcuna dipendenza esplicita della lagrangiana da una coordinata ciclica, il primo termine sarà nullo. Inserendo~\eqref{coniugato} e semplificando con~\eqref{cyclic}, otteniamo
    \begin{equation*}
        \dv{}{t} p_j = 0
    \end{equation*}
    Concludiamo quindi che nel caso di una coordinata ciclica, il suo momento coniugato si conserva. 

    Una importante osservazione sulle quantità conservate è che possono essere usate per trovare le soluzioni del moto. Considerando per semplicità un sistema unidimensionale, possiamo sfruttare la conoscenza di due quantità conservate per scrivere le soluzioni del moto in funzione di essere e dunque raggiungere il nostro obiettivo senza neanche aver risolto le equazioni del moto, che potrebbero essere di forma troppo complicata. 

\section{Enunciato e dimostrazione}
    Enunciamo ora e dimostriamo il teorema che sta al cuore del primo capitolo di questa tesi: il primo teorema di Noether, per sistemi fisici appartenenti al dominio della meccanica classica. Qualitativamente, asserisce che ad ogni simmetria globale dell'azione è possibile associare una quantità conservata. In questo paragrafo, non presenteremo tuttavia la versione originale, ma una versione più fisica. 

    Ricapitoliamo ciò che abbiamo concluso nei precedenti paragrafi: una variazione di simmetria $\delta_s q^i(t)$ porta ad una variazione dell'azione data da~\eqref{invazione} 
    \begin{equation*}
        \delta S[q^i(t), ~\delta_s q^i(t)] = S[q^i(t) + ~\delta_s q^i(t)] - S[q^i(t)] = \integ{t_1}{t_2}{t} \dv{}{t} K
    \end{equation*}
    mentre una variazione on-shell porta ad una variazione dell'azione data da~\eqref{invonshell} 
    \begin{equation*}
        \delta S[\overline q^i,~\delta q^i] = S[\overline q^i + \delta q^i] - S[\overline q^i] = \integ{t_1}{t_2}{t} \dv{}{t} \Big( \pdv{L}{\dot q^i} \delta q^i \Big )
    \end{equation*}
    Entrambi presentano termini al bordo ma per ragioni differenti: la prima perché $\delta_s q^i(t)$ soddisfa una determinata equazione mentre nella seconda è $\overline q^i$ che ne soddisfa un'altra.

    \begin{theorem}[Il primo teorema di Noether in meccanica classica]
        Sia $L(q^i,~\dot q^i)$ la lagrangiana di un sistema fisico con $i=1,~2,\ldots d$, dove $d$ indica il numero di gradi di libertà, $S$ l'azione associata al sistema nell'intervallo temporale $[t_1,~t_2]$ definita come
    \begin{equation*}
        S[q^i(t)] = \integ{t_1}{t_2}{t} L(q^i,~\dot q^i)
    \end{equation*}
        Sia $\delta_s q^i(t)$ una simmetria del sistema~\eqref{simmazione}, ovvero una variazione che lascia invariate le equazioni di Eulero-Lagrange~\eqref{eullag}. Allora esiste una quantità $Q$ definita come
    \begin{equation}\label{carica}
        Q = K - \frac{\partial L}{\partial \dot q^i} \delta_s q^i
    \end{equation}
        tale che sia conservata lungo le equazioni del moto e quindi soddisfi la legge di conservazione~\eqref{th}
    \begin{equation*} 
        \frac{d}{dt} Q = 0
    \end{equation*}
    \end{theorem}

    \begin{proof}
        Inserendo $q^i(t) = \overline q^i(t)$ nella~\eqref{invazione}, ovvero vincolando che le $q^i(t)$ soddisfino le equazioni del moto, otteniamo 
    \begin{equation}\label{prova6}
        \delta[\overline q^i(t), ~\delta_s q^i(t)] = \integ{t_1}{t_2}{t} \dv{}{t} K
    \end{equation}
        Dall'altra parte, inserendo $\delta q^i(t) = \delta_s q^i(t)$ in~\eqref{invonshell}, ovvero vincolando le le variazioni on-shell siano anche di simmetria, abbiamo
    \begin{equation}\label{prova7}
        \delta S[\overline q^i,~\delta_s q^i] = \integ{t_1}{t_2}{t} \dv{}{t} \Big( \pdv{L}{\dot q^i} \delta q^i \Big )
    \end{equation}
        In questi due passaggi abbiamo sostanzialmente imposto che $\delta_s q^i(t)$ sia una variazione di simmetria e dunque necessariamente soddisfi l'equazione~\eqref{simmazione}, mentre che $\overline q^i(t)$ soddisfi le equazioni del moto~\eqref{eullag}. Ricordiamo che prima soltanto uno dei due era soggetto a rispettare un'equazione, mentre l'altra era arbitraria, e inoltre che in questa maniera la variazione debba essere necessariamente infinitesima. Notando che i membri sinistri delle equazioni~\eqref{prova6} e~\eqref{prova7} sono identici, troviamo
    \begin{equation*}
        \delta[\overline q^i(t), ~\delta_s q^i(t)] = \integ{t_1}{t_2}{t} \dv{}{t} K = \integ{t_1}{t_2}{t} \dv{}{t} \Big( \pdv{L}{\dot q^i} \delta q^i \Big )
    \end{equation*}
        A questo punto sottraiamo il secondo con il terzo membro
    \begin{equation*}
        \integ{t_1}{t_2}{t} \dv{}{t} \Big (K - \pdv{L}{\dot q^i} \delta q^i \Big ) = 0
    \end{equation*}
        Infine avremo che l'integranda si annulla e, riconoscendi la definizione di $Q$ dalla~\eqref{carica}, otteniamo la tesi che si voleva dimostrare
    \begin{equation*}
        \dv{}{t} \Big (K - \pdv{L}{\dot q^i} \delta q^i \Big ) = \dv{}{t} Q = 0
    \end{equation*}
    \end{proof}
    Chiameremo d'ora in poi la quantità $Q$ carica conservata o carica di Noether associata alla simmetria. Chiudiamo il paragrafo precisando che sarebbe possibile ottenere tutte le quantità conservate direttamente dalle equazioni del moto ma che in questa tesi abbiamo scelto di ricavarle attraverso un procedimento più algoritmico, qual è il primo teorema di Noether. 

\section{Esempi}

    Le applicazioni sono molto spesso fondamentali per comprendere meglio e a fondo la teoria. In questo paragrafo proponiamo quattro differenti esempi in cui applicheremo il teorema appena dimostrato. 

\subsection{La particella conforme}
    Il primo esempio che studiamo è la cosidetta particella conforme, ovvero una particella il cui potenziale dipende dall'inverso del quadrato della distanza. Vediamo come sarà possibile risolvere il moto senza conoscerne le equazioni, soltanto utilizzando le cariche di Noether. 

    \begin{example}
        Consideriamo una particella di massa $m$ che può muoversi soltanto in una dimensione $x$, soggetta ad un potenziale $U$ dipendente dall'inverso del quadrato della posizione. La lagrangiana del sistema~\eqref{lagrangiana}, scritta nella forma $T - U$, è
    \begin{equation} \label{lag1}
        L = \frac{m}{2} \dot x^2 - \frac{\alpha}{x^2} 
    \end{equation}
        e di conseguenza, usando la definizione~\eqref{azione}, l'azione del sistema è
    \begin{equation*}
        S = \integ{t_1}{t_2}{t} \Big( \frac{m}{2} \dot x^2 - \frac{\alpha}{x^2} \Big)
    \end{equation*}

        La prima simmetria che osserviamo è la traslazione temporale: l'azione non dipende esplicitamente dal tempo, quindi la trasformazione
    \begin{equation*}
        t' = t + \epsilon
    \end{equation*}
        dove $\epsilon$ è una costante, è una simmetria dell'azione. Recuperiamo la~\eqref{simmtempo}, ovvero come viene deformata la $x$ a seguito di una traslazione temporale
    \begin{equation*}
        \delta_s x = x(t + \epsilon) - x(t) = - \epsilon \dot x(t)
    \end{equation*}
        e poi calcoliamo la variazione dell'azione
    \begin{equation*}
    \begin{aligned}
        \delta S & = \delta \integ{t_1}{t_2}{t} \Big( \frac{m}{2} \dot x^2 - \frac{\alpha}{x^2} \Big) \\ & = \integ{t_1}{t_2}{t} \Big( \frac{m}{2} \delta_s (\dot x^2) - \alpha \delta_s \Big (\frac{1}{x^2} \Big) \Big) \\ & = \integ{t_1}{t_2}{t} \Big( - \epsilon m \dot x \ddot x - \epsilon \alpha \frac{2 \dot x}{x^3} \Big) \\ & = \integ{t_1}{t_2}{t} \dv{}{t} \Big( -\epsilon \Big( \frac{m}{2} \dot x^2 - \frac{\alpha}{x^2} \Big) \Big)
    \end{aligned}
    \end{equation*}
        Dunque abbiamo trovato dalla~\eqref{invazione}, che il termina al bordo $K$ è
    \begin{equation*}
        K = - \epsilon \Big (\frac{m}{2} \dot x^2 - \frac{\alpha}{x^2} \Big)
    \end{equation*}
        Ora applichiamo il primo teorema di Noether e troviamo la carica~\eqref{carica} 
    \begin{equation*}
        Q = K - \frac{\partial L}{\partial \dot x} \delta_s x = - \epsilon \Big (\frac{m}{2} \dot x^2 - \frac{\alpha}{x^2} \Big) - m \dot x (-\epsilon \dot x) = - \epsilon (\frac{m}{2} \dot x^2 + \frac{\alpha}{x^2} \Big)
    \end{equation*}
        che a meno di un segno meno e di un fattore $\epsilon$ costante, è l'energia del sistema $E$
    \begin{equation}\label{energia1}
        E = \frac{m}{2} \dot x^2 + \frac{\alpha}{x^2}
    \end{equation}
        Abbiamo dunque ottenuto la prima relazione algebrica tra $x(t)$ e $\dot x(t)$. 

        \hfill

        La seconda simmetria che notiamo è quella di scala, chiamata anche simmetria di Weyl
    \begin{equation*}
        t' = \lambda t \qquad x'(t') = \sqrt{\lambda} x(t)
    \end{equation*} 
        dove $\lambda$ è una costante. Osserviamo dapprima che la velocità si trasforma nel seguente modo
    \begin{equation*}
        \dot x' = \dv{x'}{t'} = \dv{\sqrt{\lambda}}{(\lambda t)} = \frac{1}{\sqrt{\lambda}} \dv{x}{t} = \frac{1}{\sqrt{\lambda}} \dot x
    \end{equation*} 
        e quindi l'azione rimane invariante
    \begin{equation*}
        S' = \integ{t_1}{t_2}{t'} \Big( \frac{m}{2} \dot x'^2 - \frac{\alpha}{x'^2} \Big) = \integ{t_1}{t_2}{t} \lambda \Big( \frac{m}{2} \frac{\dot x^2}{\lambda} - \frac{\alpha}{\lambda x^2} \Big) = \integ{t_1}{t_2}{t} \Big( \frac{m}{2} \dot x^2 - \frac{\alpha}{x^2} \Big) = S
    \end{equation*}
        Questa trasformazione mostra chiaramente che la simmetria non deve essere infinitesima, tuttavia per applicare il teorema di Noether, è necessario renderla tale. Scriviamo $\lambda$ come $\lambda = 1 + \epsilon$, dove $\epsilon \ll 1$ è una quantità infinitesima. Espandendo al primo ordine in $\epsilon$, otteniamo la variazione di simmetria
    \begin{equation} \label{h1}
        x'(t') = x'((1+\epsilon)t) = x(t) + \frac{\epsilon}{2} x(t)
    \end{equation}
        D'altra parte abbiamo che 
    \begin{equation} \label{h2}
        x'(t') = x'((1+\epsilon)t) = x'(t) + \dot x(t) \epsilon t
    \end{equation}
        Mettendo insieme~\eqref{h1} e~\eqref{h2}, otteniamo dunque
    \begin{equation*}
        \delta_s x(t) = x'(t) - x(t) = x'(t') - \dot x(t) \epsilon t - x(t) = x(t) + \frac{\epsilon}{2} x(t) - \dot x(t) \epsilon t - x(t) = \frac{\epsilon}{2} x(t) - \dot x(t) \epsilon t 
    \end{equation*}
        Alla luce di questo risultato, calcoliamo la variazione dell'azione
    \begin{equation*}
    \begin{aligned}
        \delta S & = \delta \integ{t_1}{t_2}{t} \Big( \frac{m}{2} \dot x^2 - \frac{\alpha}{x^2} \Big) \\ & = \integ{t_1}{t_2}{t} \Big( \frac{m}{2} \delta_s (\dot x^2) - \alpha \delta_s \Big (\frac{1}{x^2} \Big) \Big) \\ & = \integ{t_1}{t_2}{t} \Big( m \dot x \Big( \frac{\epsilon}{2} \dot x(t) - \ddot x \epsilon t - \dot x \epsilon \Big) + \frac{2 \alpha}{x^3} \Big( \frac{\epsilon}{2} x - \dot x \epsilon t \Big) \Big) \\ & = \integ{t_1}{t_2}{t} \epsilon \Big( \frac{m}{2} \dot x^2 - m t \dot x \ddot x - m \dot x^2 + \frac{\alpha x - 2 \alpha t \dot x}{x^3}\Big) \\ & = \integ{t_1}{t_2}{t} \epsilon \Big( - \frac{m}{2} \dot x^2 - m t \dot x \ddot x + \frac{\alpha x - 2 \alpha t \dot x}{x^3}\Big) \\ & = \integ{t_1}{t_2}{t} \dv{}{t} \Big ( - \frac{\epsilon t m \dot x^2}{2} + \frac{\alpha \epsilon t}{x^2} \Big)
    \end{aligned}
    \end{equation*}
        e troviamo, dalla~\eqref{invazione}, che il termina al bordo $K$ è 
    \begin{equation*}
        K = - \epsilon t \Big (\frac{m}{2} \dot x^2 - \frac{\alpha}{x^2} \Big)
    \end{equation*}
        Ora applichiamo il primo teorema di Noether e otteniamo la carica~\eqref{carica} 
    \begin{equation*}
    \begin{aligned}
        Q & = K - \frac{\partial L}{\partial \dot x} \delta_s x \\ & = - \epsilon t \Big (\frac{m}{2} \dot x^2 - \frac{\alpha}{x^2} \Big) - m \dot x \Big (\frac{\epsilon}{2} x(t) - \dot x(t) \epsilon t \Big ) \\ & = - \epsilon \Big ( \frac{m x \dot x }{2} + t \frac{m \dot x^2}{2} - t \frac{\alpha}{x^2} \Big) \\ & = - \epsilon \Big ( \frac{m x \dot x}{2} - t \Big ( \frac{m}{2} \dot x^2 + \frac{\alpha}{x^2} \Big ) \Big )
    \end{aligned}
    \end{equation*}
        che a meno di un segno meno e di un fattore $\epsilon$ costante, è la carica di Noether associata alla simmetria di Weyl
    \begin{equation} \label{carica1}
        Q = \frac{m x \dot x}{2} - t \Big( \frac{m}{2} \dot x^2 + \frac{\alpha}{x^2} \Big)
    \end{equation}
        Notiamo che almeno una carica deve avere una dipendenza esplicita dal tempo, altrimenti non ci sarebbe dinamica del sistema. Nel nostro caso è $Q$.

        \hfill

        Le definizioni delle due cariche di Noether sono due equazioni algebriche che legano $x(t)$ e $\dot x(t)$. A partire da esse è possibile risolvere il moto del sistema senza scrivere l'equazione del moto esplicitamente. Innanzitutto notiamo che gli ultimi due termini della~\eqref{carica1} sono uguali all'energia~\eqref{energia1} moltiplicata per t, e quindi $Q$ può esser riscritta come
    \begin{equation*}
        Q = \frac{m x \dot x}{2} - Et
    \end{equation*}
        Abbiamo dunque ottenuto un'equazione differenziale che, separando le variabili, è possibile risolvere e trovare la soluzione dell'equazione del moto
    \begin{equation} \label{soleqmoto}
        x = \pm \sqrt{\frac{4Qt + 2Et^2}{m}}
    \end{equation}
        
        Applicando le equazioni di Eulero-Lagrange~\eqref{eullag} alla nostra Lagrangiana~\eqref{lag1}, abbiamo
    \begin{equation*}
        0 = \pdv{L}{x}  - \dv{}{t} \pdv{L}{\dot x} = \pdv{}{x} \Big(\frac{m}{2} \dot x^2 - \frac{\alpha}{x^2} \Big) - \dv{}{t} \pdv{}{\dot x} \Big(\frac{m}{2} \dot x^2 - \frac{\alpha}{x^2} \Big) = \frac{2 \alpha}{x^3} - \dv{}{t} (m \dot x) = \frac{2 \alpha}{x^3} - m \ddot x 
    \end{equation*}
        e quindi l'equazione del moto è
    \begin{equation}\label{eqmoto1}
        m \ddot x = \frac{2 \alpha}{x^3}
    \end{equation}
        È infine possibile verificare che~\eqref{soleqmoto} soddisfi~\eqref{eqmoto1}.
    \end{example}

\subsection{La particella libera}
    Il secondo esempio che studiamo è la particella libera, ovvero senza un potenziale. Ricaviamo le sue quantità conservate: energia, quantità di moto (3), momento angolare (3), moto del centro di massa (3). È possibile vedere ognuna di esse come conseguenza delle simmetrie dello spazio-tempo: rispettivamente omogeneità del tempo, omogeneità e isotropia dello spazio e `galileianità`\footnote{Dato che non esiste un termine specifico, con `galileianità` si intende che rispetti le trasformazioni di Galileo} dello spaziotempo.

    \begin{example}
        Consideriamo una particella di massa $m$ libera di spostarsi nello spazio tridimensionale, ovvero in cui non è presente nessun potenziale $U$. In questo caso, la lagrangiana del sistema~\eqref{lagrangiana} coinciderà con l'energia cinetica 
    \begin{equation*}
        L = \frac{m}{2} \mathbf{\dot r}^2 = \frac{m}{2} (\dot x^2 + \dot y^2 + \dot z^2)
    \end{equation*}
        e di conseguenza, usando la definizione~\eqref{azione}, l'azione del sistema sarà
    \begin{equation*}
        S = \integ{t_1}{t_2}{t} \frac{m}{2} \mathbf{\dot r}^2 = \integ{t_1}{t_2}{t} \frac{m}{2} (\dot x^2 + \dot y^2 + \dot z^2)
    \end{equation*}

    \subsubsection{Energia}
        La prima simmetria che andremo a indagare è una traslazione temporale
    \begin{equation*}
        t' = t + \epsilon
    \end{equation*}
        dove $\epsilon$ è una costante. L'azione è invariante perchè la lagrangiana non dipende esplicitamente dal tempo $t$. Recuperando la variazione di simmetria~\eqref{simmtempo}, calcoliamo la variazione dell'azione
    \begin{equation*}
        \delta S = \delta \integ{t_1}{t_2}{t} \frac{m}{2} \mathbf{\dot r}^2 = \integ{t_1}{t_2}{t} \frac{m}{2} \delta_s ( \mathbf{\dot r}^2) = \integ{t_1}{t_2}{t} ( - \epsilon m  \mathbf{\dot r} \cdot \mathbf{\ddot r}) = \integ{t_1}{t_2}{t} \dv{}{t} \Big( -\epsilon \frac{m}{2}  \mathbf{\dot r}^2 \Big)
    \end{equation*}
        e quindi abbiamo trovato, dalla~\eqref{invazione}, che il termine al bordo $K$ è 
    \begin{equation*}
        K = - \epsilon \frac{m}{2} \mathbf{\dot r}^2
    \end{equation*}
        Ora applichiamo il primo teorema di Noether e otteniamo la carica~\eqref{carica} 
    \begin{equation*}
        Q = K - \frac{\partial L}{\partial  \mathbf{\dot r}} \cdot \delta_s  \mathbf r = - \epsilon \frac{m}{2} \mathbf{\dot r}^2 - m \mathbf{\dot r} \cdot (-\epsilon \mathbf{\dot r}) = \epsilon \frac{m}{2} \mathbf{\dot r}^2
    \end{equation*}
        che a meno di un fattore $\epsilon$ costante, è l'energia del sistema $E$
    \begin{equation*}
        E = \frac{m}{2} \mathbf{\dot r}^2
    \end{equation*}
        Dunque abbiamo dimostrato che una simmetria di traslazione temporale comporta la conservazione dell'energia $E$.

    \subsubsection{Quantità di moto}
        La seconda simmetria che andremo a studiare sarà una traslazione spaziale
    \begin{equation*}
        \mathbf r' = \mathbf r + \mathbf a
    \end{equation*}
        dove $\mathbf a$ è un vettore dalle componenti costanti. L'azione è invariante perchè la lagrangiana non dipende esplicitamente dalla posizione $\mathbf r$. La variazione di simmetria è 
    \begin{equation*}
        \delta_s \mathbf r = \mathbf a
    \end{equation*}
        Calcoliamo la variazione dell'azione
    \begin{equation*}
        \delta S = \delta \integ{t_1}{t_2}{t} \frac{m}{2} \mathbf{\dot r}^2 = \integ{t_1}{t_2}{t} \frac{m}{2} \delta_s (\mathbf{\dot r}^2) = 0
    \end{equation*}
        e troviamo, dalla~\eqref{invazione}, che il termine al bordo $K$ è nullo
    \begin{equation*}
        K = 0
    \end{equation*}
        Ora applichiamo il primo teorema di Noether e otteniamo la carica~\eqref{carica} 
    \begin{equation*}
        Q = K - \frac{\partial L}{\partial \mathbf{\dot r}} \cdot \delta_s \mathbf r = - m \mathbf{\dot r} \cdot \mathbf a
    \end{equation*}
        che è la quantità di moto del sistema $\mathbf p$ lungo la direzione della traslazione $\mathbf a$ a meno di un fattore costante $a = \|\mathbf a \|$
    \begin{equation*}
        \mathbf p = m \mathbf{\dot r}
    \end{equation*}
        Dunque abbiamo dimostrato che una simmetria di traslazione spaziale comporta la conservazione della quantitò di moto $\mathbf p$.

    \subsubsection{Momento angolare}
        La terza simmetria che andremo a indagare sarà una rotazione spaziale, che per semplicità prendiamo rispetto ad un asse fissato con angolo $\theta$
    \begin{equation*}
        \mathbf r' = R(\theta) \mathbf r
    \end{equation*}
        dove $R(\theta)$ è la matrice di rotazione. L'azione è invariante perchè una rotazione lascia invariata la norma del vettore velocità e quindi anche il suo quadrato $\mathbf{\dot r}^2$. A questo punto rendiamo infinitesima la rotazione, introducendo il vettore $\boldsymbol \omega = (\omega_x,~\omega_y,~\omega_z)$
    \begin{equation*}
        \mathbf r' = \mathbf r + \boldsymbol \omega \times \mathbf r
    \end{equation*}
        La variazione di simmetria è
    \begin{equation*}
        \delta_s \mathbf r = \boldsymbol \omega \times \mathbf r
    \end{equation*}
        Calcoliamo la variazione dell'azione
    \begin{equation*}
        \delta S = \delta \integ{t_1}{t_2}{t} \frac{m}{2} \mathbf{\dot r}^2 = \integ{t_1}{t_2}{t} \frac{m}{2} \delta_s (\mathbf{\dot r}^2) = \integ{t_1}{t_2}{t} \frac{m}{2} \mathbf{\dot r} \cdot (\omega \times \mathbf{\dot r} ) = 0
    \end{equation*}
        e troviamo, dalla~\eqref{invazione}, che il termine al bordo $K$ è nullo
    \begin{equation*}
        K = 0
    \end{equation*}
        Ora applichiamo il primo teorema di Noether e otteniamo la carica~\eqref{carica} 
    \begin{equation*}
        Q = K - \frac{\partial L}{\partial \mathbf{\dot r}} \cdot \delta_s \mathbf r = - m \mathbf{\dot r} \cdot \boldsymbol\omega \times \mathbf r 
    \end{equation*}
        che è il momento angolare del sistema $\mathbf L$ lungo la direzione dell'asse di rotazione $\boldsymbol \omega$ a meno di un fattore costante $\omega = \|\boldsymbol \omega\|$ e di un segno meno
    \begin{equation*}
        \mathbf L = m \mathbf{\dot r} \times \mathbf r
    \end{equation*}
        Dunque abbiamo dimostrato che una simmetria di rotazione spaziale comporta la conservazione del momento angolare $\mathbf L$.

    \subsubsection{Moto del centro di massa}
        La quarta e ultima simmetria che studiamo è un boost di Galileo lungo una direzione
    \begin{equation*}
        \mathbf r' = \mathbf r + \mathbf v t
    \end{equation*}
        dove $\mathbf v$ è la velocità di movimento del sistema. La variazione di simmetria è 
    \begin{equation*}
        \delta_s \mathbf r = \mathbf v t
    \end{equation*}
        Calcoliamo la variazione dell'azione
    \begin{equation*}
        \delta S = \delta \integ{t_1}{t_2}{t} \frac{m}{2} \mathbf{\dot r}^2 = \integ{t_1}{t_2}{t} \frac{m}{2} \delta_s (\mathbf{\dot r}^2) = \integ{t_1}{t_2}{t} m \mathbf {\dot r} \cdot \delta_s (\mathbf{\dot r})= \integ{t_1}{t_2}{t} \dv{}{t} \Big( m \mathbf r \cdot \mathbf v \Big) 
    \end{equation*}
        e troviamo, dalla~\eqref{invazione}, che il termine al bordo $K$ è 
    \begin{equation*}
        K = m \mathbf r \cdot \mathbf v 
    \end{equation*}
        Ora applichiamo il primo teorema di Noether e otteniamo la carica~\eqref{carica} 
    \begin{equation*}
        Q = K - \frac{\partial L}{\partial \mathbf{\dot r}} \cdot \delta_s \mathbf r =  m \mathbf r \cdot \mathbf v - m \mathbf{\dot r} \cdot \mathbf v t = \mathbf v \cdot (m \mathbf r - m \mathbf{\dot r} t)
    \end{equation*}
        che è la definizione di moto del centro di massa nella direzione di $\mathbf v$ a meno di un fattore costante $v = \|\mathbf v \|$
    \begin{equation*}
        m \mathbf r - m \mathbf{\dot r} t = const
    \end{equation*}
        Dunque abbiamo dimostrato che una simmetria di boost di Galileo comporta la conservazione del moto del centro di massa.

    \end{example}

\subsection{La particella in un campo di background uniforme}

    Il terzo esempio mostra come una particella in un campo di background uniforme non avrà una carica conservata associata ad una simmetria a causa proprio del campo. Tuttavia sarà ugualmente possibile ottenere l'equazione differenziale che governa l'evoluzione temporale della carica non conservata.

    \begin{example}
        Consideriamo una particella di massa $m$ immersa in un campo uniforme il cui potenziale dipendente dalla coordinata z
    \begin{equation}\label{enpot3}
        U = B z
    \end{equation}
        dove $B$ è una costante. Scrivendo la Lagrangiana nella forma $T - U$, sostituendo~\eqref{enpot3}, otteniamo 
    \begin{equation} \label{lag3}
        L = \frac{m}{2} \mathbf{\dot r}^2 - B z
    \end{equation}
        e di conseguenza, usando la definizione~\eqref{azione}, l'azione $S$ del sistema sarà
    \begin{equation*}
        S = \integ{t_1}{t_2}{t} \Big( \frac{m}{2} \mathbf{\dot r} - B z \Big)
    \end{equation*}
        Riscriviamo il potenziale in modo differente, introducendo il vettore $\mathbf B = (0,~0,~B)$ e riscrivendo l'azione come
    \begin{equation*}
        S = \integ{t_1}{t_2}{t} \Big( \frac{m}{2} \mathbf{\dot r}^2 - \mathbf B \cdot \mathbf r \Big)
    \end{equation*}
        Notiamo che applicando una rotazione, sia il primo termine che il secondo termine sono invarianti, mostrando quindi una simmetria rotazionale. Infatti essendo entrambi prodotti scalari, il primo tra $\mathbf{\dot r}$ e $\mathbf{\dot r}$ mentre il secondo tra $\mathbf B$ e $\mathbf r$, sono invarianti per trasformazione di coordinate. Tuttavia ciò non implica che la carica di Noether associata a questa simmetria, ovvero il momento angolare, si conservi. Ciò è dovuto alla presenza di un campo in background. Per vederlo, consideriamo una rotazione infinitesima come variazione di simmetria sia delle coordinate 
    \begin{equation*}
        \delta_s \mathbf r = \boldsymbol \omega \times \mathbf r
    \end{equation*}
        che del campo
    \begin{equation*}
        \delta_s \mathbf B = \boldsymbol \omega \times \mathbf B
    \end{equation*}
        il cui vettore $\boldsymbol \omega$ è lo stesso introdotto nell'esempio precedente. Derivando la variazione delle coordinate, abbiamo
    \begin{equation*}
        \delta_s \mathbf{\dot r} = \boldsymbol \omega \times \mathbf{\dot r}
    \end{equation*}
        Alla luce di questo risultato, calcoliamo il termine al bordo $K$ 
    \begin{equation*}
    \begin{aligned}
        \delta S & = \delta \integ{t_1}{t_2}{t} \Big( \frac{m}{2} \mathbf{\dot r}^2 - \mathbf B \cdot \mathbf r \Big) \\ & = \integ{t_1}{t_2}{t} \Big( \frac{m}{2} \delta_s (\mathbf{\dot r}^2) - \delta_s (\mathbf B \cdot \mathbf r ) \Big ) \\ & = \integ{t_1}{t_2}{t} (m \mathbf{\dot r} \cdot (\boldsymbol \omega \times \mathbf{\dot r} ) - \mathbf B \cdot \boldsymbol \omega \times \mathbf r - \boldsymbol \omega \times \mathbf B \cdot \mathbf r ) = 0
    \end{aligned}
    \end{equation*}
        che risulta essere nullo essendo tutti prodotti scalari tra vettori perpendicolari
    \begin{equation*}
        K = 0
    \end{equation*}
        Calcoliamo ora l'equazione del moto applicando le equazioni di Eulero-Lagrange~\eqref{eullag} alla nostra lagrangiana~\eqref{lag3}
    \begin{equation*}
        0 = \pdv{L}{\mathbf x} - \dv{}{t} \pdv{L}{\mathbf{\dot x}} = \pdv{}{\mathbf x} \Big(\frac{m}{2} \mathbf{\dot x} - \mathbf B \cdot \mathbf x \Big) - \dv{}{t} \pdv{}{\mathbf{\dot x}} \Big(\frac{m}{2} \mathbf{\dot x} - \mathbf B \cdot \mathbf x \Big) = \mathbf B - m \mathbf{\ddot x}
    \end{equation*}
        e quindi l'equazione del moto è
    \begin{equation}\label{eqmoto3}
        m \mathbf{\ddot r} = \mathbf B
    \end{equation}
        Effettuiamo dunque la variazione on-shell, utilizzando le equazioni del moto~\eqref{eqmoto3}
    \begin{equation*}
    \begin{aligned}
        \delta S & = \delta \integ{t_1}{t_2}{t} \Big( \frac{m}{2} \mathbf{\dot r}^2 - \mathbf B \cdot \mathbf r \Big) \\ & = \integ{t_1}{t_2}{t} \Big( \frac{m}{2} \delta_s (\mathbf{\dot r}^2) - \delta_s (\mathbf B \cdot \mathbf r ) \Big ) \\ & = \integ{t_1}{t_2}{t} \Big( m \mathbf{\ddot r } \cdot \delta \mathbf{\dot r} - \delta \mathbf B \cdot \mathbf r - \mathbf B \cdot \delta \mathbf r \Big )  \\ & = \integ{t_1}{t_2}{t} \Big( \dv{}{t} (m \mathbf{\dot r} \cdot \delta \mathbf r) - m \mathbf{\dot r} \cdot \delta \mathbf{\dot r} - \delta \mathbf B \cdot \mathbf r - \mathbf B \cdot \delta \mathbf r \Big )  \\ & = \integ{t_1}{t_2}{t} \Big( \dv{}{t} (m \mathbf{\dot r} \cdot \boldsymbol \omega \times \mathbf r) - m \mathbf{\dot r} \cdot \boldsymbol \omega \times \mathbf{\dot r} - \boldsymbol \omega \times \mathbf B \cdot \mathbf r - \mathbf B \cdot \boldsymbol \omega \times \mathbf r \Big ) \\ & = \integ{t_1}{t_2}{t} \boldsymbol \omega \cdot \Big( \dv{\mathbf L}{t} - \mathbf r \times \mathbf B \Big)
    \end{aligned}
    \end{equation*} 
        dove abbiamo usato l'identità
    \begin{equation*}
        m \mathbf{\ddot r} \cdot \delta \mathbf r = \dv{}{t} (m \mathbf{\dot r} \cdot \delta \mathbf r) - m \mathbf{\dot r} \cdot \delta \mathbf{\dot r} 
    \end{equation*} 
        sostituito le variazioni, eliminato i termini nulli, e riconosciuto la definizione di momento angolare $\mathbf L$. Dunque, ponendo la simmetria uguale a zero, otteniamo l'equazione differenziale che governa come varia il momento angolare, che quindi non si conserva 
    \begin{equation*}
        \dv{\mathbf L}{t} = \mathbf r \times \mathbf B 
    \end{equation*}
        Infine, affermiamo dunque che questa non è una simmetria dal punto di vista di Noether perchè la sua variazione non è una derivata totale ma presenta un termine ulteriore dipendente dal campo $B$.
    \end{example}

\subsection{Due particelle in un campo centrale}

    Il quarto esempio mostra come quando siamo in presenza di due particelle ad essere conservate non sono le singole cariche ma soltanti la carica totale: è la quantità di moto totale a conservarsi, non quella delle singole particelle\footnote{Analoghi ragionamenti possono essere ripetuti per il momento angolare}. 

\begin{example}
    Consideriamo due particelle rispettivamente di massa $m_1$ e $m_2$ tali che la prima particella $q$ sia immersa nel campo molto più intenso della seconda particella $Q$ e che quest'ultima sia vincolata ad essere ferma. La lagrangiana è dunque 
\begin{equation*}
    L = \frac{m_1}{2} \mathbf{\dot r_1^2} - \frac{qQ}{|\mathbf r_1 - \mathbf r_2|}
\end{equation*}
    e di conseguenza, usando la definizione~\eqref{azione}, l'azione del sistema è
\begin{equation*}
    S = \integ{t_1}{t_2}{t} \Big( \frac{m_1}{2} \mathbf{\dot r_1}^2- \frac{qQ}{|\mathbf r_1 - \mathbf r_2|} \Big)
\end{equation*}
    A questo punto, notiamo che l'azione non è invariante per la trasformazione 
\begin{equation*}
    \mathbf r'_1 = \mathbf r_1 + \mathbf a
\end{equation*}
    dove $\mathbf a$ è una costante. Essendo questa una traslazione spaziale, dunque possiamo concludere che la quantità di moto $\mathbf p_1$ della prima particella non si conserva. Tuttavia, aggiungendo anche l'energia cinetica della seconda particella e quindi la sua possibilità di muoversi, otteniamo la lagrangiana  
\begin{equation*}
    L = \frac{m_1}{2} \mathbf{\dot r_1^2} + \frac{m_2}{2} \mathbf{\dot r_2^2} - \frac{qQ}{|\mathbf r_1 - \mathbf r_2|}
\end{equation*}
    e di conseguenza l'azione diventa
\begin{equation*}
    S = \integ{t_1}{t_2}{t} \Big( \frac{m_1}{2} \mathbf{\dot r_1^2} + \frac{m_2}{2} \mathbf{\dot r_2^2} - \frac{qQ}{|\mathbf r_1 - \mathbf r_2|} \Big)
\end{equation*}
    Ora, notiamo che la traslazione spaziale 
\begin{equation*}
    \mathbf r'_1 = \mathbf r_1 + \mathbf a \qquad \mathbf r'_2 = \mathbf r_2 + \mathbf a
\end{equation*}
    è una simmetria per cui l'azione è invariante. Come consequenza, abbiamo la conservazione della quantità di moto totale del sistema
\begin{equation*}
    \mathbf P = \mathbf p_1 + \mathbf p_2 = m_1 \mathbf{\dot r_1} + m_2 \mathbf{\dot r_2}
\end{equation*}
    e non delle singole quantità di moto. Ciò si può vedere notando che le energie cinetiche sono invarianti per singole traslazioni $\mathbf r'_1 = \mathbf r_1 + \mathbf a_1$ e $\mathbf r'_2 = \mathbf r_2 + \mathbf a_2$ ma il potenziale è invariante soltanto sotto la condizione $\mathbf a_1 = \mathbf a_2 = \mathbf a$.
\end{example}

\section{Meccanica hamiltoniana}

    Il formalismo lagrangiano non è l'unico che possiamo utilizzare per descrivere un sistema fisico. Ricordiamo che le equazioni di Eulero-Lagrange sono un sistema di equazioni differenziali al secondo ordine, la cui soluzione è una traiettoia nello spazio delle configurazioni generato da $d$ variabili indipendenti. Diversamente la meccanica hamiltoniana fornisce un sistema di $2d$ equazioni differenziali del primo ordine, la cui soluzione è una traiettoria in un nuovo spazio, lo spazio delle fasi, formato dalle coordinate e dai momenti associati ad esse, definiti nella~\eqref{coniugato}, e quindi generato da 2N variabili indipendenti. I due formalismi sono equivalenti, la scelta di uno o dell'altro dipende da tanti fattori, tra cui la facilità di soluzione in uno delle due descrizioni rispetto all'altra.
    
\subsection{Le equazioni di Hamilton e le parentesi di Poisson}

    Vediamo ora come è possibile passare da una descrizione all'altra, aiutandoci con il principio di azione stazionaria. Consideriamo un sistema meccanico descritto da una lagrangiana~\eqref{lagrangiana} nello spazio delle configurazioni, generato dalle coordinate generalizzate $q^i$. Definiamo l'hamiltoniana del sistema $H$, una funzione delle coordinate $q^i$, dei momenti $p_j$ ed eventualmente del tempo $t$, attraverso una trasformazione di Legendre 
    \begin{equation} \label{hamiltoniana}
        H(q^i, ~p_j, ~t) = p_i \dot q^i - L
    \end{equation} 
    Invertendo quest'ultima equazione, definiamo l'azione nello spazio delle fasi o azione hamiltoniana nel seguente modo 
    \begin{equation*}
        S[q^i, ~p_i] = \integ{t_1}{t_2}{t} (p_i \dot q^i - H(q^i, ~p_j, ~t))
    \end{equation*}
    Applichiamo il principio di azione stazionaria~\eqref{azionestazionaria}, calcolando la variazione dell'azione
    \begin{equation*}
    \begin{aligned}
        \delta S & = \delta \integ{t_1}{t_2}{t} (p_i \dot q^i - H) \\ & = \integ{t_1}{t_2}{t} \Big ( \delta p_i \dot q^i + p_i \delta \dot q^i - \pdv{H}{q^i} \delta q^i - \pdv{H}{p_i} \delta p_i \Big ) \\ & = \integ{t_1}{t_2}{t} \Big ( \delta p_i \dot q^i - \dot p^i \delta q^i - \pdv{H}{q^i} \delta q^i - \pdv{H}{p_i} \delta p_i \Big) \\ & = \integ{t_1}{t_2}{t} \Big ( \delta p_i \Big( \dot q^i - \pdv{H}{p_i} \Big ) - \delta q^i \Big(\dot p_i + \pdv{H}{q^i} \Big ) \Big) = 0
    \end{aligned}
    \end{equation*}
    dove abbiamo integrato per parti, posto gli estremi nulli~\eqref{estreminulli}, e nell'ultimo passaggio abbiamo raccolto un termine $\delta q^i$ comune. Per il lemma fondamentale del calcolo delle variazioni, dato che le $\delta q^i$ sono arbitrarie, l'integranda si annulla e le equazioni del moto, chiamate equazioni di Hamilton, diventano 
    \begin{equation} \label{ham1}
        \dot q^i = \pdv{H}{p_i}
    \end{equation}
    \begin{equation} \label{ham2}
        \dot p_j = - \pdv{H}{q^j}
    \end{equation}
    
    Completiamo le equazioni introducendo la dipendenza sia della Lagrangiana che dell'Hamiltoniana dal tempo e vedendo che le derivate temporali sono uguali tra loro a meno di un segno negativo. Calcoliamo prima il differenziale della lagrangiana
    \begin{equation*}
        dL = \pdv{L}{q^i} dq^i + \pdv{L}{\dot q^i} d\dot q^i + \pdv{L}{t} dt = \dot p_i dq^i + p_i d\dot q^i + \pdv{L}{t} dt 
    \end{equation*}
    dove abbiamo usato le equazioni~\eqref{coniugato} e~\eqref{eullag}, poi il differenziale dell'hamiltoniana
    \begin{equation*}
    \begin{aligned}
        dH & = \dot q^i dp_i - d \dot q^i p_i - dL \\ & = \dot q^i dp_i + d \dot q^i p_i - \dot p_i dq^i - p_i d\dot q^i - \pdv{L}{t} dt \\ & = \dot q^i dp_i - \dot p_i dq^i - \pdv{L}{t} dt \\ & = \pdv{H}{q^i} dq^i + \pdv{H}{p_i} dp_i + \pdv{H}{t} dt
    \end{aligned}
    \end{equation*}
    dove abbiamo usato le equazioni~\eqref{hamiltoniana},~\eqref{ham1} e~\eqref{ham2}. Dalle ultime due uguaglianze, ricaviamo  
    \begin{equation*}
        \pdv{L}{t} = - \pdv{H}{t}
    \end{equation*}

\hfill 

    Solitamente, si usa esprimere la meccanica hamiltoniana introducendo le parentesi di Poisson. 
    \begin{definition}[Parentesi di Poisson]
        Date due funzioni $f(q, ~p)$ e $g(q, ~p)$ nello spazio delle fasi, definiamo le parentesi di Poisson come
    \begin{equation} \label{poisson}
        \poi{f}{g} = \pdv{f}{q^i} \pdv{g}{p_i} - \pdv{g}{q^i} \pdv{f}{p_i}
    \end{equation}
    \end{definition}
    Dalla definizione è possibile dimostrare le seguenti proprietà, dove $h(q, ~p)$ è un'altra funzione nello spazio delle fasi e $c, ~c_1, ~c_2$ sono costanti
    \begin{enumerate}
        \item Antisimmetria
    \begin{equation*}
        \poi{f}{g} = - \poi{g}{f}
    \end{equation*}
        \item Bilinearità di entrambi gli argomenti
    \begin{equation*}
        \poi{c_1f_1+c_2f_2}{g} = c_1 \poi{f_1}{g} + c_2 \poi{f_2}{g}
    \end{equation*}
    \begin{equation*}
        \poi{f}{c_1g_1+c_2g_2} = c_1 \poi{f}{g_1} + c_2 \poi{f}{g_2}
    \end{equation*}
        \item Regola di Leibniz, simile a quella delle derivate
    \begin{equation*}
        \poi{fg}{h} = f \poi{g}{h} + g \poi{f}{h}
    \end{equation*}
    \begin{equation*}
        \poi{f}{gh} = g \poi{f}{h} + h \poi{f}{g}
    \end{equation*}
        \item Identità di Jacobi
    \begin{equation} \label{jacobi}
        \poi{f}{\poi{g}{h}} + \poi{g}{\poi{h}{f}} + \poi{h}{\poi{f}{g}} = 0
    \end{equation}
    \end{enumerate}
    Importanti relazioni emergono quando si sostituisconi le coordinate o i momenti al posto delle funzioni $f$ e $g$. Queste parentesi di Poisson vengono chiamate parentesi canoniche ed esplicitamente sono 
    \begin{equation*}
        \poi{q^i}{q_j} = 0
    \end{equation*}
    \begin{equation*}
        \poi{p^i}{p_j} = 0
    \end{equation*}
    \begin{equation} \label{canqq}
        \poi{q^i}{p_j} = \delta^i_{\phantom i j}
    \end{equation}
    Infine, attraverso le parentesi di Poisson possiano riscrivere le equazioni di Hamilton. Infatti se sviluppiamo esplicitamente la~\eqref{ham1}, otteniamo 
    \begin{equation*}
        \dot q^i = \pdv{H}{p_i} = \delta^i_{\phantom i k} \pdv{H}{p_k} - 0 = \pdv{q^i}{q^k} \pdv{H}{p_k} - \pdv{H}{q^k} \pdv{q^i}{p_k} = \poi{q^i}{H}
    \end{equation*}
    mentre analogamente se sviluppiano esplicitamente la~\eqref{ham2}, abbiamo
    \begin{equation*}
        \dot p_i = - \pdv{H}{q^i} = \delta^i_{\phantom i k} \pdv{H}{q^k} - 0 = \pdv{p_i}{p_k} \pdv{H}{q^k} - \pdv{H}{p_k} \pdv{p_i}{q^k} = - \Big( \pdv{p_i}{q^k} \pdv{H}{p_k} - \pdv{H}{q^k} \pdv{p_i}{p_k} \Big) = - \poi{p_i}{H}
    \end{equation*}
    Quindi le equazioni di Hamilton diventano 
    \begin{equation*}
        \dot q^i = \poi{q^i}{H}
    \end{equation*}
    \begin{equation*} 
        \dot p_j = - \poi{p_j}{H}
    \end{equation*}  

\subsection{Il teorema inverso e algebra di Lie delle cariche}

    Enunciamo e dimostriamo in questo paragrafo due teoremi concernenti le simmetrie e le leggi di dimostrazione: il teorema di Noether inverso e l'algebra di Lie delle cariche. Attraverso le parentesi di Poisson, è possibile definire la legge di conservazione di una carica di Noether $ Q(q^i, ~p_i, ~t)$ nel seguente modo
    \begin{equation} \label{caricaham}
        \dv{}{t} Q(q^i, ~p_i, ~t) = \pdv{Q}{q^i} \dot q^i + \pdv{Q}{p_i} \dot p_i + \pdv{Q}{t} = \poiexp{Q}{H} + \pdv{Q}{t} = \poi{Q}{H} + \pdv{Q}{t} = 0
    \end{equation}

    Il primo teorema che dimostriamo è il teorema inverso: invece che derivare una carica conservata da una simmetria, compiano l'azione inversa, ovvero definita una carica conservata deriviamo una simmetria dell'azione.
    \begin{theorem}[L'inverso del primo teorema di Noether]
        Sia $Q$ una carica di Noether, allora la seguente trasformazione 
    \begin{equation*}
        \delta_s q^i = [q^i, \epsilon Q] = \epsilon \frac{\partial Q}{\partial p_i} \qquad \delta_s p_i = [q_i, \epsilon Q] = - \epsilon \frac{\partial Q}{\partial q^i}
    \end{equation*}
        è una simmetria dell'azione, dove $\epsilon$ è un parametro infinitesimo.
    \end{theorem}

    \begin{proof}
        Calcoliamo la variazione dell'azione
    \begin{equation*}
    \begin{aligned}
        \delta S & = \delta \int dt ~ (p \dot q^i - H) \\ & = \int dt ~  (\delta_s p \dot q + p \frac{d}{dt} \delta_s q - \frac{\partial H}{\partial p} \delta_s p - \frac{\partial H}{\partial q} \delta_s q) \\ & = \int dt ~ (- \epsilon \frac{\partial Q}{\partial q} \dot q + \frac{d}{dt} (p \delta_s q) - \epsilon \dot p \frac{\partial Q}{\partial p} + \epsilon \frac{\partial H}{\partial p} \frac{\partial Q}{\partial q} - \epsilon \frac{\partial H}{\partial q}\frac{\partial Q}{\partial p}) \\ & = \int dt ~ (\epsilon (- \frac{dQ}{dt} + \frac{\partial Q}{\partial t} + [Q, H] ) + \frac{d}{dt} (p \delta_s q)) \\ & = \int dt ~ \frac{d}{dt}(-\epsilon Q + p \delta_s q)
    \end{aligned}
    \end{equation*}
        che essendo una derivata totale, completa la dimostrazione.
    \end{proof}

    Il secondo teorema che dimostriamo è l'algebra di Lie delle cariche di Noether: le parentesi di Poisson di due cariche sono una carica conservata, qualunque forma abbiamo.

    \begin{theorem}[Algebra di Lie delle cariche di Noether]
        L'insieme di tutte le cariche conservate $Q_a$ con $a = 1, 2, \ldots N$ soddisfano un algebra di Lie chiusa
    \begin{equation*}
        [Q_a, Q_b] = f_{ab}^{\phantom{ab}c} Q_c
    \end{equation*}
        dove $f_{ab}^{\phantom{ab}c}$ sono le costanti di struttura.
    \end{theorem}

    \begin{proof}
        La derivata temporale delle parentesi di Poisson di due cariche conservate $Q_1$ e $Q_2$ è
    \begin{equation*}
        \dv{}{t} [Q_1, Q_2] = [[Q_1, Q_2], H] + \pdv{}{t} [Q_1, Q_2] = - \poi{\poi{Q_2}{H}}{Q_1} - \poi{\poi{H}{Q_1}}{Q_2} + \pdv{}{t} [Q_1, Q_2] = 0
    \end{equation*} 
        dove abbiamo usato l'identità di Jacobi~\eqref{jacobi}, essendo cariche la~\eqref{caricaham} e il fatto che non c'è dipendenza esplicita dal tempo\footnote{È possibile estendere il teorema anche nel caso di dipendenza temporale delle cariche}. Quindi a prescindere da quale sia il risultato delle parentesi (zero, una costante indipendente o proporzionale a $Q_1$ o $Q_2$), sicuramente si conserva e quindi definisce un'algebra di Lie
    \begin{equation*}
        [Q_a, Q_b] = f_{ab}^{\phantom{ab}c} Q_c
    \end{equation*}
        che completa la dimostrazione.
    \end{proof}

\subsection{La particella conforme}
    Riproponiamo l'esempio della particella conforme, utilizzata nel formalismo lagrangiano, ma questa volta studiata nella descrizione hamiltoniana. Ricordiamo che la lagrangiana del sistema è 
\begin{equation*}
    L = \frac{m}{2} \dot x^2 - \frac{\alpha}{x^2}
\end{equation*}
    e, attraverso una trasformazione di Legendre~\eqref{hamiltoniana}, calcoliamo l'hamiltoniana
\begin{equation*}
    H = p \dot x - L = \frac{p^2}{2m} + \frac{\alpha}{x^2}
\end{equation*}
    L'azione hamiltoniana è
\begin{equation*}
    S = \int dt~ \Big (\frac{p^2}{2m} - \frac{\alpha}{x^2} \Big)
\end{equation*}
    Il sistema presenta tre differenti cariche conservate: la prima è l'energia, dato che non c'è dipendenza dal tempo
\begin{equation*}
    H = \frac{p^2}{2m} + \frac{\alpha}{x^2}
\end{equation*}
    la seconda è quella che abbiamo trovato nel formalismo lagrangiano
\begin{equation*}
    Q = -tH + \frac{px}{2}
\end{equation*}
    e infine è possibile calcolarne un'ultima
\begin{equation*}
    K = t^2H + 2 t Q^2 - \frac{m}{2} x^2
\end{equation*}  
    Tuttavia non tutte le cariche sono indipendenti, infatti notiamo che esiste una relazione fra le cariche 
\begin{equation*}
    2KH + 2Q^2 + m\alpha = 0
\end{equation*}

    Infine ci soffermiamo sull'algebra delle cariche: come ci dice il teorema del precedente paragrafo, le parentesi di Poisson fra le cariche sono anch'esse stesse delle cariche
\begin{equation*}
    [Q, ~H] = H
\end{equation*}
\begin{equation*}
    [Q, ~K] = -K
\end{equation*}
\begin{equation*}
    [K, ~H] = 2Q
\end{equation*}

    Dimostriamo di seguito la prima delle precedenti relazioni per mostrare il teorema dell'algebra di Lie delle cariche in azione
    \begin{equation*}
    \begin{aligned}
    \poi{Q}{H} & = \pdv{Q}{x}\pdv{H}{p}-\pdv{H}{x}\pdv{Q}{p} \\ & = \Big (-t \pdv{H}{x}+ \frac{p}{2} \Big) \Big( \frac{p}{m} \Big) -  \Big (- \frac{2 \alpha}{x^3} \Big) \Big (-t \pdv{H}{p}+ \frac{x}{2} \Big) \\ & = \Big ( \frac{2 \alpha t}{x^3} + \frac{p}{2} \Big) \Big( \frac{p}{m} \Big) - \Big (- \frac{2 \alpha}{x^3} \Big) \Big (- \frac{pt}{m} + \frac{x}{2} \Big) \\ & = \frac{2 p \alpha t}{m x^3} + \frac{p^2}{2m} - \frac{2 p \alpha t}{m x^3} + \frac{\alpha}{x^2} \\ & = \frac{p^2}{2m} + \frac{\alpha}{x^2} = H
    \end{aligned}
    \end{equation*}
