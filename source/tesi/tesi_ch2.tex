\chapter{Il primo teorema in teoria relativistica dei campi}

    Il secondo sistema fisico che prendiamo in considerazione è un sistema continuo che presenta un'infinità di gradi di libertà, a differenza dai sistemi studiati nel precedente capitolo. Il nostro oggetto di indagine sarà quindi un'entità estesa nello spazio che rappresenteremo da una funzione delle coordinate e del tempo. I campi scalari sono semplici funzioni che forniscono un numero per ogni punto, come ad esempio la temperatura, la pressione o la densità. Campi vettoriali (o tensoriali) invece associano vettori ad ogni punto, come ad esempio il campo elettrico o quello gravitazionale.

\section{Cenni di relatività speciale}
    In questo capitolo, non ci limiteremo all'ambito della fisica newtoniana, ma useremo la formulazione relativistica einsteniana. I principi su cui si basa sono il principio di relatività, ovvero che le leggi della fisica sono le stesse in tutti i sistemi di riferimento inerziali, e la costanza della velocità della luce, ovvero che le onde elettromagnetiche viaggiano ad una velocità costante che nessun corpo massivo può superare. Lo spazio matematico che utilizzeremo è lo spaziotempo di Minkowski $\mathcal M$, ovvero uno spazio euclideo con l'aggiunta della coordinata temporale, in cui però è stata definita una distanza differente: non sarà più valido il teorema di Pitagora 
    \begin{equation*}
        ds^2 = dx^2 + dy^2 + dz^2
    \end{equation*}
    ma le distanze verranno misurate nel seguente modo
    \begin{equation*}
        ds^2 = g_{\mu\nu} dx^\mu dx^\nu = c^2 dt^2 - dx^2 - dy^2 - dz^2
    \end{equation*}
    dove abbiamo utilizzato la metrica $g_{\mu\nu} = diag(1,~-1,~-1,~-1)$ e il quadrivettore $x^\mu$, definito come 
    \begin{equation*}
        x^\mu = (ct,~x,~y,~z)
    \end{equation*}
    
    Analogamente al caso discreto, la dinamica del campo è descritta da equazioni differenziali che ne determinano l'evoluzione. Trattiamo solamente il caso di un campo scalare, ovvero definito attraverso una singola funzione $\phi(x^\mu)$, mentre per campi vettoriali (o tensoriali) si generalizza immediatamente. Definiamo le rispettive derivate dei campi rispetto alle coordinate come
    \begin{equation*}
        \partial_\mu \phi =  \phi,_\mu = \dv{\phi}{x^\mu} = (\frac{1}{c} \pdv{\phi}{t}, ~ \nabla \phi)
    \end{equation*}
    Conseguentemente, le equazioni del moto sono equazioni dipendenti dal campo, dalle sue derivate e dal quadrivettore posizione
    \begin{equation} \label{motocampi}
        f_j(x^\mu, ~\phi(x^\mu), ~\phi,_\nu(x^\mu)) = 0
    \end{equation}

\section{Formalismo lagrangiano}    

    Il formalismo lagrangiano si basa sulla densità di lagrangiana $\mathcal L$, funzione del campo $\phi$, delle sue derivate $\phi,_\mu$ ed eventualmente delle coordinate $x^\mu$
    \begin{equation} \label{lagrangianacampi}
        \mathcal L = \mathcal L (\phi,~\phi,_\mu,~x^\mu)
    \end{equation}
    che integrata sullo spazio tridimensionale permette di ottenere la lagrangiana associata al sistema
    \begin{equation*}
        L = \int d^3 x ~ \mathcal L
    \end{equation*}
    In questo modo possiamo definire l'azione in funzione della densità di lagrangiana
    \begin{equation} \label{azionecampi}
        S[\phi] = \int d^4 x ~ \mathcal L (\phi,~\phi,_\mu,~x^\mu)
    \end{equation}
    È importante osservare che l'azione è Lorentz-invariante, e quindi non dipende dal sistema di riferimento inerziale scelto.

\subsection{Equazioni di Eulero-Lagrange}

    Seguendo le stesse linee guida che abbiamo utilizzato per sistemi discreti, è intuitivo pensare che il principio di azione stazionaria ci permetterà di trovare le equazioni del moto per la configurazione del campo $\phi$ che rende stazionario il suo funzionale di azione~\eqref{azionecampi}
    \begin{equation} \label{azionestazionariacampi}
        \delta S [\phi(x^\mu)] = 0
    \end{equation}

    Calcoliamo la variazione dell'azione
    \begin{equation*}
    \begin{aligned}
        \delta S & = \delta \int d^4 x ~ \mathcal L = \int d^4 x ~ \Big (\pdv{L}{\phi} \delta \phi + \pdv{L}{\phi,_\mu} \delta \phi,_\mu \Big) \\ & = \int d^4 x \Big (\pdv{L}{\phi} \delta \phi - \dv{}{x^\mu} \Big (\pdv{L}{\phi,_\mu} \Big) \delta \phi \Big) + \int d^4 x \dv{}{x^\mu} \Big ( \pdv{L}{\phi,_\mu} \delta \phi \Big ) = 
    \end{aligned}
    \end{equation*}
    dove abbiamo integrato per parti e nell'ultimo passaggio posto uguale a zero le quattro componenti dell'ultimo termine, a causa delle condizioni agli estremi, analoga al caso discreto, e raccolto un termine comune $\delta \phi$. Imponendo la~\eqref{azionecampi}, otteniamo
    \begin{equation*}
        \delta S = \int d^4 x \delta \phi \Big (\pdv{L}{\phi} - \dv{}{x^\mu} \Big (\pdv{L}{\phi,_\mu} \Big) \Big)  = 0
    \end{equation*}
    Per la generalizzazione del lemma fondamentale del calcolo delle variazioni, e dunque per l'arbitrarietà della variazione $\delta \phi$, l'integranda deve annullarsi
    \begin{equation} \label{eullagcampi}
        \pdv{L}{\phi} - \partial_\mu \pdv{L}{\phi,_\mu} = 0
    \end{equation}
    Questa è l'equazione del moto che stavamo cercando, una sola equazione differenziale alle derivate parziali. Dunque inerendo esplicitamente la densità di lagrangiana, le~\eqref{eullagcampi} diventano le~\eqref{motocampi}. Nel caso di sistemi discreti a $d$ gradi di libertà, avevamo ottenuto $d$ equazioni differenziali mentre per sistemi continui a infiniti gradi di libertà, soltanto una equazione è apparsa. Tuttavia sono presenti anche le derivate parziali spaziali a differenza del caso precedente. 

\section{Simmetrie}
    
    Indaghiamo ora le simmetrie, generalizzando lo stesso concetto espresso per il capitolo precedente. Le simmetrie sono sempre classi speciali di trasformazioni del campo che lasciano l'azione invariata a meno di un termine al bordo. Quindi definiamo una variazione di simmetria dell'azione come una funzione infinitesima $\delta_s \phi$ vincolata a risolvere l'equazione, per ogni arbitraria $\phi(x)$
    \begin{equation} \label{invazionecampi}
        \delta S[\phi, ~\delta_s \phi] = S[\phi + \delta_s \phi] - S[\phi] = \int d^4 x \partial_\mu K^\mu \qquad \forall \phi
    \end{equation}  
    dove $K^\mu$ è un termine al bordo. Anche in questo caso non chiediamo che sia strettamente invariante ma che la variazione sia nulla a meno di una quadridivergenza. Ciò deriva dalla possibilità di aggiungere una quadridivergenza alla densità della lagrangiana senza che le equazioni del moto cambino. Definiamo inoltre una variazione on-shell, ovvero tale che le $\phi$ devono soddisfare le equazioni del moto~\eqref{eullagcampi}, come un'arbitraria variazione infinitesima $\delta \phi$ tale che la variazione dell'azione sia una quadridivergenza
    \begin{equation*}
    \begin{aligned}
        \delta S[\overline \phi, ~\delta \phi] & = \int d^4 x \Big(\pdvd{L}{\phi} \delta \phi + \pdv{L}{\phi,_\mu} \delta \phi,-\mu \Big) \\ & = \int d^4 x \Big ( \delta \phi \Big (\pdv{L}{\phi} - \partial_\mu \pdv{L}{\phi,_\mu} \Big ) Big) + \int d^4 x \partial_\mu \Big ( \pdv{L}{\phi,_\mu}  \delta \phi \Big)
    \end{aligned}
    \end{equation*}
    Applicando le equazioni di Eulero-Lagrange~\eqref{eullagcampi}, il primo integrale si annulla e otteniamo 
    \begin{equation} \label{invonshellcampi}
        \delta S[\phi, ~\delta_s \phi] = \int d^4 x \partial_\mu \Big ( \pdv{L}{\phi,_\mu}  \delta \phi \Big)
    \end{equation}

\subsection{Trasformazioni spaziotemporali come deformazioni dei campi}

    Nel precedente paragrafo, abbiamo imposto che le simmetrie agiscano sui campi e non sulle coordinate, allo stesso modo in cui le simmetrie nel caso discreto interessavano le coordinate e non il tempo. Ciò può sempre essere mostrato notando che le coordinate $x^\mu$ sono variabili di integrazione e quindi mute. Vediamo dunque come trovare la deformazione dei campi in seguito a trasformazioni spaziotemporali. Consideriamo prima il caso di traslazioni spaziotemporali
    \begin{equation*}
        x'^\mu = x^\mu + \epsilon^\mu
    \end{equation*} 
        dove $\epsilon^\mu$ è un quadrivettore dalle componenti costanti. Dato un campo $\phi(\mu)$, costruiamo il campo traslato $\phi'(x^\mu)$, espandendo in serie e tenendo solo i termini lineari al primo ordine in $\epsilon$
    \begin{equation*}
        \phi'(x^\mu) = \phi(x - \epsilon) \simeq \phi(x) - \epsilon^\mu \partial_\mu \phi(x)
    \end{equation*}
        Abbiamo dunque ricavato la deformazione del campo in seguito alla traslazione spaziotemporale 
    \begin{equation*}
        \delta \phi(x) = \phi'(x) - \phi(x) = -\epsilon^\mu \partial_\mu \phi (x)
    \end{equation*}

    Finora abbiamo studiato solamente campi scalari, generalizziamo ora per quelli vettoriali o tensoriali. Solitamente, vengono definiti solitamente in base a come si comportano sotto l'effetto di leggi di trasformazione. Vediamo dunque esempi di come campi scalari, vettoriali, tensoriali di rango $(0,~1)$ si trasformino in seguito all'applicazione di un cambio di coordinate:
    \begin{enumerate}
        \item Per campi scalari
    \begin{equation*}
        \phi'(x') = \phi(x)
    \end{equation*}
        \item Per campi vettoriali
    \begin{equation*}
        V'^\mu(x') = \pdv{x'^\mu}{x^\nu} V^\nu(x)
    \end{equation*}
        \item Per campi tensoriali di rango $(0,~1)$
    \begin{equation*}
        A'_\mu(x') = \pdv{x^\nu}{x'^\mu} A_\nu(x)
    \end{equation*}
    \end{enumerate} 
        Siccome avremo bisogno di trasformazioni infinitesime, rendiamo il cambio di coordinate tale 
    \begin{equation*}
        x'^\mu = x^\mu + \xi^\mu(x)
    \end{equation*}
        Quindi la versione infinitesima dello jacobiano precedentemente utilizzato è 
    \begin{equation*}
        \pdv{x'^\mu}{x^\mu} = \delta^{\mu}_{\phantom \mu \nu} + \partial_\nu \xi^\mu(x)
    \end{equation*}
        mentre per l'inverso è 
    \begin{equation*}
        \pdv{x^\mu}{x'^\mu} = \delta^{\mu}_{\phantom \mu \nu} - \partial_\nu \xi^\mu(x)
    \end{equation*}
        A questo punto applichiamo ai casi precedenti e troviamo le variazioni dei campi
    \begin{enumerate}
        \item Per campi scalari, espandiamo al primo ordine
    \begin{equation*}
        \phi'(x) = \phi(x - \xi) = \phi(x) - \xi^\mu(x) \partial_\mu \phi(x)
    \end{equation*}  
        e otteniamo la variazione
    \begin{equation}\label{variazionescalare}
        \delta \phi(x) = - \xi^\mu(x) \partial_\mu \phi(x)
    \end{equation} 
        \item Per campi vettoriali, espandiamo al primo ordine sia il lato destro che sinistro
    \begin{equation*}
        V'^\mu(x) + \xi^\nu(x) \partial_\mu V^\mu (x) = V^\mu(x) + (\partial_\nu \xi^\mu(x)) V^\nu(x)
    \end{equation*}
        e otteniamo la variazione
    \begin{equation}\label{variazionevett}
        \delta V^\mu(x) = V'^\mu(x) - V^\mu(x) = (\partial_\nu \xi^\mu(x)) V^\nu(x) - \xi^\nu (x)\partial_\mu V^\mu(x)
    \end{equation}
        \item Per campi tensoriali di rango $(0,~1)$, espandiamo entrambi i membtri 
    \begin{equation*}
        A'_\mu(x) + \xi^\nu(x) \partial_\mu A_\mu (x) = A_\mu(x) - A_\nu(x) \partial_\mu \xi^\nu(x)
    \end{equation*}
        e otteniamo la variazione
    \begin{equation}\label{variazione10}
        \delta A_\mu(x) = A'_\mu(x) - A_\mu(x) = - \xi^\nu(x) \partial_\mu A_\mu (x) = A_\mu(x) - A_\nu(x) \partial_\mu \xi^\nu(x)
    \end{equation}
    \end{enumerate}

\section{Enunciato e dimostrazione}
    Il teorema è del tutto analogo alla versione discreta, con l'unica differenza che la legge di conservazione è espressa tramite un'equazione di continuità.

    \begin{theorem}[Primo teorema di Noether in teoria dei campi]
        Sia $\mathcal L(\phi)$ la densità di lagrangiana di un sistema fisico con relativo funzionale di azione
    \begin{equation*}
        S[\phi] = \int d^4 x \mathcal L(\phi,~\phi,_\mu)
    \end{equation*}
        Sia $\delta_s \phi(x)$ una trasformazione del campo che individua una variazione di simmetria del sistema~\eqref{invazionecampi}, ovvero che lascia invariate le equazioni di Eulero-Lagrange~\eqref{eullagcampi}. Allora esiste una quantità $J^\mu$, definita come
    \begin{equation}\label{caricacampi}
        J^\mu = \pdv{\mathcal L}{\phi,_\mu} \delta \phi - K^\mu 
    \end{equation}
        tale che soddisfi l'equazione di continuità
    \begin{equation} \label{thcampi}
        \partial_\mu J^\mu = 0
    \end{equation}
    \end{theorem}

    \begin{proof}
        Inserendo $\phi = \overline \phi$ nella~\eqref{invazionecampi}, ovvero vincolando che $\phi$ soddisfi le equazioni del moto, otteniamo 
    \begin{equation}\label{prova6campi}
        \delta S[\overline \phi, ~\delta_s \phi] = \int d^4 x \partial_\mu K^\mu
    \end{equation}
        Dall'altra parte, inserendo $\delta \phi = \delta_s \phi$ in~\eqref{invonshellcampi}, ovvero vincolando le variazioni on-shell siano anche di simmetria, abbiamo
    \begin{equation}\label{prova7campi}
        \delta S[\overline \phi,~\delta_s \phi] = \int d^4 x \partial_\mu \Big ( \pdv{L}{\phi,_\mu}  \delta \phi \Big)
    \end{equation}
        Notando che i membri sinistri delle~\eqref{prova6campi} e~\eqref{prova7campi} sono identici, abbiamo
    \begin{equation}
        \delta S[\overline \phi, ~\delta_s \phi] = \int d^4 x \partial_\mu K^\mu = \int d^4 x \partial_\mu \Big ( \pdv{L}{\phi,_\mu}  \delta \phi \Big)
    \end{equation}
        A questo punto sottraiamo il secondo con il terzo membro, ottenendo la sottrazione nulla
    \begin{equation}
        \int d^4 x \partial_\mu \Big ( K^\mu - \pdv{L}{\phi,_\mu}  \delta \phi \Big)
    \end{equation}
        Infine avremo che l'integranda si annulla e, riconoscendi la definizione di corrente~\eqref{caricacampi}, otteniamo la tesi
    \begin{equation}
        \partial_\mu \Big ( K^\mu - \pdv{L}{\phi,_\mu}  \delta \phi \Big) = \partial_\mu J^\mu = 0
    \end{equation}
    \end{proof}

    Chiameremo d'ora in poi la quantità $J^\mu$ corrente di Noether associata alla simmetria. 

\subsection{Equazione di continuità}
    Soffermiamoci ora sulla nozione di legge di conservazione. L'equazione di continuità esprime la conservazione di una quantità fisica asserendo che la variazione di questa quantità sarà uguale al suo flusso attraversante una superficia chiusa. Per vedere ciò, scriviamo esplicitamente la~\eqref{thcampi}
    \begin{equation}
        0 = \partial_\mu J^\mu = \partial_0 J^0 + \nabla \cdot \mathbf J
    \end{equation}
    Troviamo che l'equazione di continuità può essere riscritta come 
    \begin{equation*}
        \pdv{J^0}{t} = - \nabla \cdot \mathbf J
    \end{equation*}
    Integrando entrambi i membri su un volume $V$ abbastanza grande da permettere che $J$ vada a zero più velocemente della crescita dell'area di superficie
    \begin{equation*}
        \int_V d^3 x \pdv{J^0}{t} = - \int_V d^3 x \nabla \cdot \mathbf J
    \end{equation*} 
    Utilizziamo il teorema della divergenza e riscriviamo il secondo termine 
    \begin{equation*}
        \int_V d^3 x \pdv{J^0}{t} = - \int_{\partial V} d \mathbf A \cdot \mathbf J
    \end{equation*} 
    Dalle ipotesi fatte su $J$ e su $V$, il secondo membro si annulla e troviamo la carica che si conserva 
    \begin{equation*}
        Q = \int_V d^3 x J^0(x)
    \end{equation*} 
    che soddisfa la legge di conservazione
    \begin{equation*}
        \dv{}{t} Q = 0 
    \end{equation*}

\subsection{Tensore energia-impulso}

    Speciale interesse poniamo sulle traslazioni spaziotemporali, poichè ci conducono alla definizione del tensore energia-impulso, la cui fondamentale applicazione è data dalle equazioni di campo di Einstein in relatività generale. 
    Definiamo il tensore energia-impulso $T^\mu_\nu$ come le correnti di Noether associate alle 4 traslazioni spaziotemporali
    \begin{equation*}
        x'^\mu = x^\mu + \epsilon^\mu
    \end{equation*}
    che possono essere decomposte nelle varie componenti: per $\mu=0$ abbiamo una traslazione temporale $t' = t + \epsilon^0$ mentre per $\mu=i=1,~2,~3$ abbiamo una traslazione spaziale $x'^i = x^i + \epsilon^i$. Dunque insieme ci sono 4 correnti di Noether che si conservano, che possiamo scrivere senza separarle come 
    \begin{equation} \label{tensenimp}
        J^\mu = T^\mu_\nu \epsilon^\nu
    \end{equation}
    una per ogni indice $\mu$. La conservazione di $J^\mu$, ovvero l'equazione $\partial_\mu J^\mu = 0 $ implica anche che si conservi $\partial_\mu T^\mu_\nu = 0$, essendo $\epsilon^\nu$ una costante.

    Il significato fisico delle componenti del tensore energia-impulso è il seguente: 
    \begin{enumerate}
        \item $T^{00}$ è la densità di energia,
        \item $T^{0j}$ è il flusso di energia lungo la j-esima direzione,
        \item $T^{i0}$ è la densità di quantità di moto lungo la i-esima direzione,
        \item $T^{ij}$ è il tensore degli sforzi.
    \end{enumerate}

\section{Campo scalare}
    Consideriamo una densità di lagrangiana per un campo scalare $\phi$, che avrà la forma $\mathcal L(\phi,~\phi,_\mu)$. Dato che la lagrangiana non dipende esplicitamente dalle coordinate $x^\mu$, sarà invariante per la variazione~\eqref{variazionescalare}, dove poniamo $\xi^\mu = \epsilon^\mu$
    \begin{equation}
        \delta \phi(x) = - \epsilon^\mu \partial_\mu \phi(x)
    \end{equation}
    Calcoliamo la variazione dell'azione
    \begin{equation*}
    \begin{aligned}
        \delta S & = \int d^4 x ~ \delta \mathcal L = \int d^4 x ~ \Big ( \pdv{L}{\phi} \delta \phi + \pdv{L}{\phi,_\rho} \delta \phi,_\rho \Big) \\ & = - \int d^4 x ~ \epsilon^\sigma \Big ( \pdv{L}{\phi} \phi,_\sigma + \pdv{L}{\phi,_\rho} \phi,_{\rho\sigma} \Big) = \int d^4 x ~ \partial_\sigma (\epsilon^\sigma \mathcal L)
    \end{aligned}
    \end{equation*}
    Quindi, secondo la~\eqref{invazionecampi}, abbiamo trovato che il termine al bordo è
    \begin{equation*}
        K^\mu = - \epsilon^\mu \mathcal L
    \end{equation*}
    Ora applichiamo il primo teorema di Noether e troviamo la corrente~\eqref{caricacampi} generata dalle traslazione spaziotemporali
    \begin{equation*}
        J^\mu = \pdv{\mathcal L}{\phi,_\mu} \delta \phi - K^\mu = \pdv{\mathcal L}{\phi,_\mu} (- \epsilon^\sigma \phi,_\sigma) + \epsilon^\mu \mathcal L = - \epsilon^\sigma \Big ( \pdv{\mathcal L}{\phi,_\mu} \phi,_\sigma - \delta^\mu_{\phantom \mu \sigma} \mathcal L \Big)
    \end{equation*}
    Quindi il tensore energia-impulso, definito dalla~\eqref{tensenimp}, è
    \begin{equation}
        T^\mu_{\phantom \mu \sigma} = \pdv{\mathcal L}{\phi,_\mu} \phi,_\sigma - \delta^\mu_{\phantom \mu \sigma} \mathcal L
    \end{equation}

    \hfill

    Proponiamo un esempio concreto, in cui prendiamo la lagrangiana più semplice.

    \begin{example}
    Sia la densità di lagrangiana per un campo scalare nella forma
    \begin{equation*}
        \mathcal L = \frac{1}{2} \partial_\mu \phi \partial^\mu \phi
    \end{equation*}
    Il suo tensore energia-impulso è
    \begin{equation*}
        T_{\mu\nu} = \partial_\mu \phi \partial_\nu \phi - \frac{1}{2} g_{\mu\nu} \partial_\sigma \phi \partial^\sigma \phi
    \end{equation*}
    \end{example}

\section{Campo elettromagnetico}
    I fenomeni elettromagnetici sono interamente caratterizzati dalle equazioni di Maxwell, che possono essere derivate dal principio di Hamilton a partire dall'azioned di Maxwell
\begin{equation} \label{azionemaxwell}
    S[A_\mu] = - \frac{1}{4} \int d^4 x ~ F_{\mu\nu} F^{\mu\nu}
\end{equation}
    dove $A_\mu$ è il tensore $(0, 1)$ definito attraverso il potenziale vettore e scalare, e $F^{\mu\nu} = \partial_\mu A_\nu - \partial_\nu A_\mu$ è il tensore di Maxwell. Calcoliamo le equazioni del moto attraverso le equazioni di Eulero-Lagrange~\eqref{eullagcampi}
\begin{equation*}
    0 = \pdv{\mathcal L}{A_\nu} - \partial_\mu \pdv{\mathcal L}{A_{\nu,\mu}} = \partial_\mu (\partial^\nu A^\mu - \partial^\mu A^\nu) = \partial_\mu F^{\mu\nu}
\end{equation*}
    che sono proprio le equazioni di Maxwell in assenza di sorgenti\footnote{in realtà ci sarebbe un'altra equazione $\partial_\mu F*^{\mu\nu} = 0$, dove $F*^{\mu\nu}$ è il tensore duale.}
\begin{equation}\label{eqmax}
    \partial_\mu F^{\mu\nu} = 0
\end{equation}

    Indaghiamo ora le simmetrie che l'elettrodinamica presenta.

\subsection{Simmetria spaziotemporale}
    La prima simmetria che indaghiamo è una traslazione spaziotemporale, che porta come abbiamo precedentemente visto al tensore energia-impulso. Sappiamo che l'elettrodinamica è invariante per trasformazioni di gauge 
\begin{equation*}
    \delta A_\mu = \partial_\mu \lambda(x)
\end{equation*}
    dove $\lambda(x)$ è una generica funzione, chiamata funzione di gauge, dipendente dalle coordinate. Come conseguenza, chiediamo che anche le cariche conservate siano invarianti di gauge come la teoria. Osserviamo tuttavia che prendendo una variazione di $A_\mu$, che abbiamo calcolato in \eqref{variazione10}, 
\begin{equation*}
    \delta A_\mu = - \epsilon^\nu \partial_\nu A_\mu
\end{equation*}
    e che lascia effettivamente invariata l'azione, a meno di un termine al bordo
\begin{equation*}
\begin{aligned}
    \delta F_{\mu\nu} F^{\mu\nu} & = 2 F^{\mu\nu} \delta F_{\mu\nu} = 2 F^{\mu\nu} \delta (\partial_\mu A_\nu - \partial_\nu A_\mu) = 2 F^{\mu\nu} ( \partial_\mu \delta A_\nu - \partial_\nu \delta A_\mu) \\ & = 2 F^{\mu\nu} ( \partial_\mu - \epsilon^\sigma \partial_\sigma A_\nu - \partial_\nu - \epsilon^\sigma \partial_\sigma A_\mu) = 2 F^{\mu\nu} ( \partial_\mu (- \epsilon^\sigma \partial_\sigma A_\nu) - \partial_\nu (- \epsilon^\sigma \partial_\sigma A_\mu)) \\ & = - 2 \epsilon^\sigma F^{\mu\nu} ( \partial_\mu \partial_\sigma A_\nu - \partial_\nu \partial_\sigma A_\mu) = - 2 \epsilon^\sigma F^{\mu\nu} \partial_\sigma F_{\mu\nu} = \partial_\sigma (-\epsilon^\sigma F_{\mu\nu} F^{\mu\nu})
\end{aligned}
\end{equation*}
    Tuttavia la variazione non è gauge invariante e si può già prevedere che non lo sarà neanche la corrente. Per ovviare a ciò, introduciamo una nuova variazione che questa volta rispetta i prerequisiti 
\begin{equation*}
    \delta A_\mu = - \epsilon^\nu \partial_\nu A_\mu + \partial_\mu (\epsilon^\nu A_\nu) = F_{\mu\nu} \epsilon^\nu
\end{equation*}
    che è un invariante di gauge essendo il tensore di Maxwell $F$ tale. Calcoliamo la variazione dell'azione
\begin{equation*}
\begin{aligned}
    \delta F_{\mu\nu} F^{\mu\nu} & = 2 F^{\mu\nu} \delta F_{\mu\nu} = 2 F^{\mu\nu} \delta (\partial_\mu A_\nu - \partial_\nu A_\mu) = 2 F^{\mu\nu} (\partial_\mu \delta A_\nu - \partial_\nu \delta A_\mu) \\ & = - \epsilon 2 F^{\mu\nu} (\partial_\mu F_{\nu\sigma}  - \partial_\nu F_{\mu\sigma}) = 2 F^{\mu\nu \epsilon^\sigma (\partial_\mu F_{\nu\sigma} + \partial_\nu F_{\sigma\mu})} = - 2 F^{\mu\nu} \epsilon^\sigma \partial_\sigma F_{\mu\nu} \\ & = \partial_\sigma (-\epsilon^\sigma F^{\mu\nu} F_{\mu\nu})
\end{aligned}
\end{equation*}
    che essendo un termine al bordo, mostra che è una variazione di simmetria, con 
\begin{equation*}
    K^\sigma = -\epsilon^\sigma \mathcal L
\end{equation*}
    Ora applichiamo il primo teorema di Noether e troviamo la corrente \eqref{caricacampi} 
\begin{equation*}
\begin{aligned}
    J^\mu & = \pdv{L}{A_{\rho,\mu}} \delta A_\rho - K^\mu = - F^{\mu\rho} \epsilon^\sigma F_{\rho\sigma} + \epsilon^\mu L = \epsilon^\sigma (- F^{\mu\rho} F_{\rho \sigma} + \delta^\mu_{\phantom \mu \sigma} L ) \\ & = - \epsilon^\sigma \Big ( F^{\mu\rho} F_{\rho\sigma} + \frac{1}{4} \delta^\mu_{\phantom \mu \sigma} F^{\alpha \beta} F_{\alpha \beta} \Big)
\end{aligned}
\end{equation*}
    ed usando la relazione che lega il tensore energia-impulso, otteniamo 
\begin{equation*}
    T^\mu_{\phantom \mu \sigma} = - F^{\mu\rho} F_{\sigma\rho} + \frac{1}{4} \delta^\mu_{\phantom \mu \sigma} F^{\alpha \beta} F_{\alpha \beta}
\end{equation*}
    Notiamo che $T$ è gauge invariante dato che $F$ lo è. A questo punto mostriamo che è veramente una carica che si conserva
\begin{equation*}
    \partial_\mu T^\mu_{\phantom \mu \sigma} = - \partial_\mu F^{\mu\rho} F_{\sigma\rho} + \frac{1}{4} \partial_\mu \delta^\mu_{\phantom \mu \sigma} F^{\alpha \beta} F_{\alpha \beta} = 0
\end{equation*} 
    dove abbiamo usato le equazioni di Maxwell $\partial_\mu F^{\mu\sigma} = 0$.

\subsection{Simmetria conforme}
    In realtà, la teoria elettromagnetica è invariante rispetto ad un gruppo molto più grande di trasformazioni, chiamato gruppo conforme. Consideriamo una trasformazione generica
\begin{equation*}
    x'^\mu = x^\mu + \xi^\mu(x)
\end{equation*}
    dove $\xi^\mu$ è un generico quadrivettore. La variazione che subisce $A_\mu$ in seguito a questa trasformazione spaziotemporale è data dalla \eqref{variazione10}
\begin{equation*}
    \delta A_\mu = - \xi^\nu \partial_\nu A_\mu - \partial_\mu \xi^\nu A_\nu
\end{equation*}
    Tuttavia, come nel paragrafo precedente, chiediamo che la variazione sia gauge invariante e quindi aggiungiamo un termine alle variazione
\begin{equation*}
    \delta A_\mu = - \xi^\nu \partial_\nu A_\mu - \partial_\mu \xi^\nu A_\nu + \partial_\mu(\xi^\nu A_\nu)
\end{equation*}
    Potrebbe sembrare che l'aggiunta di un termine dovuto ad una trasformazione di gauge, porti ad una corrente di Noether differente e duqnue ad una sbagliata legge di conservazione. Ciò non avviebne perche associate alle trasformazioni di gauge, non sono presenti cariche come nel caso di simmetrie incontrate finora, ma bensì vincoli. Nel prossimo capitolo, tratteremo più in dettaglio questa differenza. 

    Il funzionale di azione \eqref{azionemaxwell}, non è invariante per arbitrarie scelte di $\xi^\mu$, ma soltanto per trasformazioni appartenenti al gruppo conforme, cioè tali che soddisfi l'equazione
\begin{equation}\label{conforme}
    \xi_{\mu,\nu} + \xi_{\nu,\mu} = \frac{1}{2} g_{\mu\nu} \xi^\alpha{,\alpha}
\end{equation} 
    Dunque qualsiasi soluzione di questa equazioni ci fornisce una simmetria a cui possiamo associare una corrente di Noether. È possibile separare le soluzioni in quattro differenti categorie 
\begin{enumerate}
    \item Traslazioni, già incontrate nel paragrafo precedente, in cui
\begin{equation*}
    \xi^\mu = \epsilon^\mu
\end{equation*}
    dove $\epsilon^\mu$ è un quadrivettore dalle componenti costanti.
    \item Trasformazioni di Lorentz, tali che 
\begin{equation*}
    \xi^\mu = \Lambda^\mu_{\phantom \mu \nu} x^\nu \qquad \Lambda_{\mu\nu} = - \Lambda_{\nu\mu}
\end{equation*}
    dove $\Lambda^\mu_{\phantom \mu \nu}$ è la matrice di Lorentz antisimmetrica.
    \item Dilatazioni, tali che
\begin{equation*}
    \xi^\mu = \lambda x^\mu
\end{equation*}
    dove $\lambda$ è un fattore di scala.
    \item Trasformazioni conformi speciali, tali che
\begin{equation*}
    \xi^\mu = 2 x^\nu b_\nu x^\mu - b^\mu x^\nu x_\nu
\end{equation*}
\end{enumerate}

    Mostriamo ora che l'azione è invariante per trasformazioni conformi 
\begin{equation*}
\begin{aligned}
    \delta \mathcal L & = \delta \frac{1}{4} F^{\mu\nu} F_{\mu\nu} = \frac{1}{2} F^{\mu\nu} \delta F_{\mu\nu} = \frac{1}{2} F^{\mu\nu} 2 \partial_\mu (\xi^\rho F_{\nu \rho}) = F^{\mu\nu} F_{\nu\rho} \partial_\mu \xi^\rho + F^{\mu\nu} \xi^\rho \partial_\mu F_{\nu\rho} \\ & = F^{\mu\nu} F_\nu^{\phantom \nu \rho} \partial_\mu \xi^\rho + \frac{1}{2} F^{\mu\nu} \xi^\rho (\partial_\mu F_{\nu\rho} + \partial_\nu F_{\rho\mu}) = - F^{\mu\nu} F^\rho_{\phantom \rho \nu} \partial_\mu \xi^\rho - \frac{1}{2} F^{\mu\nu} \xi^\rho \partial_\rho F_{\mu\nu} \\ & = - F^{\mu\nu} F^\rho_{\phantom \rho \nu} \partial_\mu \xi^\rho - \frac{1}{4} \partial_\rho (F^{\mu\nu} F_{\mu\nu}) \xi^\rho \\ & = - \frac{1}{2} F^{\mu\nu} F^\rho_{\nu} (\partial_\mu \xi_\rho + \partial_\rho \xi_\mu) + \frac{1}{4} F^{\mu\nu} F_{\mu\nu} \partial \cdot \xi - \frac{1}{4} \partial_\rho (\xi^\rho F^2) \\ & = - \frac{1}{2} F^{\mu\nu} F^\rho_{\phantom \rho \nu} \Big ( \partial_\mu \xi_\rho + \partial_\rho \xi_\mu - \frac{1}{2} g_{\mu\rho} \partial \cdot \xi \Big) - \partial_\rho(\xi^\rho \mathcal L) = - \partial_\rho(\xi^\rho \mathcal L)
\end{aligned}
\end{equation*}
    dove abbiamo usato l'identità di Bianchi e il fatto che $\xi$ soddisfi l'equazione \eqref{conforme}. Il termine al bordo è dunque 
\begin{equation*}
    K = - \xi^\rho \mathcal L
\end{equation*}

    Ora applichiamo il primo teorema di Noether e troviamo la corrente \eqref{caricacampi} 
\begin{equation*}
    J^\mu = \pdv{L}{A_{\rho,\mu}} \delta A_\rho - K^\mu = F^{\mu\alpha} \xi^\beta F_{\alpha\beta} +  \xi^\mu \frac{1}{4} F^{\alpha\beta} F_{\alpha\beta} = \xi^\beta \Big ( F^{\mu \alpha} F_{\alpha \beta} + frac{1}{4} \delta^\mu{\phantom \rho \beta} F^{\alpha \alpha} F_{\alpha \beta} \Big)
\end{equation*}
    tale che soddisfi l'equazione di continuità $\partial_\mu J^\mu$. 

\section{Campo di Schroedinger}

    Anche l'equazione di Schoedinger alla base della meccanica quantistica può essere derivata attraverso un funzionale di azione 
\begin{equation*}
    S = \int dt \int d^3 r \Big( i \hbar \psi^* \dot \psi - \frac{\hbar^2}{2m} \nabla \psi^* \cdot \nabla \cdot \psi - V(\mathbf r) \psi^* \psi \Big)
\end{equation*}
    Consideriamo $\psi$ e $\psi^*$ come due campi differenti e applichiamo le equazioni di Eulero-Lagrange \eqref{eullagcampi} per verificare la nostra tesi 
\begin{equation*}
    0 = \pdv{L}{\psi^*} - \dv{}{t} \pdv{L}{\dot \psi^*} = i \hbar \dot \psi + \frac{\hbar^2}{2m} \nabla^2 \psi - V \psi 
\end{equation*}
    che dunque ci porta alla nota equazione di Schoedinger
\begin{equation*}
    i \hbar \dot \psi = - \frac{\hbar^2}{2m} \nabla^2 \psi + V \psi = H \psi
\end{equation*}
    Analogamente può esser visto come con $\psi$ invece che $\psi^*$ porta alla stessa equazione ma complessa coniugata. 

\subsection{Conservazione della probabilità}

    L'azione è invariante rispetto alla trasformazione di fase globale costante $\alpha$
\begin{equation*}
    \psi' = \psi e^{i\alpha}
\end{equation*}
    e dunque è una simmetria $\delta \psi = i \alpha \psi$. Il termine al bordo $K$ è nullo, dato che sono presenti termini $e^{i\alpha}$ e $e^{-i\alpha}$ che si elidono a vicenda. Effettuando la variazione on-shell, otteniamo 
\begin{equation*}
    \delta S = \int dt \int d^3 r i \hbar \dv{}{t} (\psi^* \delta \psi) = - \alpha \hbar \int dt \dv{}{t} \int d^3 r \psi^* \psi
\end{equation*}
    Dunque abbiamo ottenuto che la probabilità si conserva 
\begin{equation*}
    Q = \int d^3 r \psi^* \psi
\end{equation*}
    che normalmente si pone $Q = 1$ per avere la condizione di normalizzazione. 

\subsubsection{Riferimenti bibliografici}
    I riferimenti bibliografici per questo capitolo sono~\cite{landaucampi},~\cite{barone},~\cite{banados} e~\cite{weinberg}.
