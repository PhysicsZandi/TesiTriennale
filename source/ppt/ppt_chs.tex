\begin{document}

\begin{frame}
    \titlepage
\end{frame}

%\begin{frame}
%\tableofcontents[sectionstyle=show,subsectionstyle=show/shaded/hide,subsubsectionstyle=show/shaded/hide]
%\end{frame}

\begin{frame}{Introduzione}
    Contenuti nell'articolo~\say{Invariante Variationsprobleme} del $1918$ della matematica tedesca Emmy Noether. 

    \hfill

      Possiamo riassumerli nel seguente modo:
    \begin{itemize}
        \item simmetria globale $\iff$ legge di conservazione,
        \item simmetria locale $\iff$ identità tra le equazioni del moto.
    \end{itemize}
\end{frame}

\section{Il primo teorema in meccanica classica}

\begin{frame}{Formalismo lagrangiano}
    In meccanica classica\footnote{In questo caso classica ha l'accezione di non relativistica e non quantistica}, un generico sistema fisico è descritto dal funzionale di azione
    \begin{equation*}
         S[q^i(t)] = \integ{t_1}{t_2}{t} L(t, ~q^i, ~\dot q^i)
    \end{equation*} 
      Attraverso il principio di azione stazionaria, si ricavano le equazioni del moto
    \begin{equation*}
        \delta S = 0 \quad \Rightarrow \quad \pdv{L}{q^i} - \dv{}{t} \pdv{L}{\dot q^i} = 0
    \end{equation*}
\end{frame}

\begin{frame}{Che cos'è una simmetria?}
    Una simmetria è una classe speciale di trasformazioni delle coordinate che lascia l'azione invariata a meno di un termine al bordo
    \begin{equation*}
        \delta S[q^i, ~\delta_s q^i] = \integ{t_1}{t_2}{t} \dv{}{t} K
    \end{equation*}
      Una variazione on-shell è una arbitraria trasformazione delle coordinate che usa le equazioni del moto
    \begin{equation*}
        \delta S[\overline q^i, ~\delta q^i] = \integ{t_1}{t_2}{t} \dv{}{t} \Big( \pdv{L}{\dot q^i} \delta q^i \Big )
    \end{equation*}
\end{frame}

\begin{frame}{Il primo teorema di Noether}
    \begin{block}{Teorema}
        Sia $\delta_s q^i$ una simmetria dell'azione. Allora esiste una carica di Noether $Q$ definita come
        \begin{equation*}
            Q = K - \frac{\partial L}{\partial \dot q^i} \delta_s q^i
        \end{equation*}
            tale che sia conservata lungo le equazioni del moto e quindi soddisfi la legge di conservazione
        \begin{equation*} 
            \frac{d}{dt} Q = 0
        \end{equation*}
    \end{block}
\end{frame}

\begin{frame}

    \begin{block}{Dimostrazione}
        Vincolando che le $\overline q^i$ soddisfino le equazioni del moto e che le $~\delta_s q^i$ siano una simmetria dell'azione, troviamo
        \begin{equation*}
            \delta S[\overline q^i, ~\delta_s q^i] = \integ{t_1}{t_2}{t} \dv{}{t} K = \integ{t_1}{t_2}{t} \dv{}{t} \Big( \pdv{L}{\dot q^i} \delta q^i \Big )
        \end{equation*}
          Sottraendo il terzo dal secondo
        \begin{equation*}
            \integ{t_1}{t_2}{t} \dv{}{t} \Big (K - \pdv{L}{\dot q^i} \delta q^i \Big ) = 0
        \end{equation*}
          otteniamo la tesi ricercata
        \begin{equation*}
            \dv{}{t} \Big (K - \pdv{L}{\dot q^i} \delta q^i \Big ) = \dv{}{t} Q = 0
        \end{equation*}
        $\qquad \qquad \qquad \qquad \qquad \qquad \qquad \qquad \qquad \qquad \qquad \qquad \quad q.e.d.$
    \end{block}

\end{frame}

\begin{frame}{La particella libera}
    L'azione di una particella libera è
    \begin{equation*}
        S = \integ{t_1}{t_2}{t} \frac{m}{2} \mathbf{\dot r}^2 = \integ{t_1}{t_2}{t} \frac{m}{2} (\dot x^2 + \dot y^2 + \dot z^2)
    \end{equation*} 
\end{frame}

\begin{frame}  
    La carica di una traslazione temporale è l'energia
    \begin{equation*}
        \delta_s t = \epsilon \quad \Rightarrow \quad \delta S = \integ{t_1}{t_2}{t} \dv{}{t} \Big( -\epsilon \frac{m}{2} \mathbf{\dot r}^2 \Big) \quad \Rightarrow \quad  Q = \epsilon \frac{m}{2} \mathbf{\dot r}^2 
    \end{equation*}
      La carica di una traslazione spaziale è la quantità di moto
    \begin{equation*}
        \delta_s \mathbf r = \mathbf a \quad \Rightarrow \quad \delta S = 0 \quad \Rightarrow \quad  Q = - m \mathbf{\dot r} \cdot \mathbf a
    \end{equation*}
      La carica di una rotazione spaziale è il momento angolare
    \begin{equation*}
        \delta_s \mathbf r = \boldsymbol \omega \times \mathbf r \quad \Rightarrow \quad \delta S = 0 \quad \Rightarrow \quad  Q = \boldsymbol \omega \cdot \mathbf r \times m \mathbf{\dot r} 
    \end{equation*}
      La carica di un boost di Galileo è il moto del centro di massa
    \begin{equation*}
        \delta_s \mathbf r = \mathbf v t \quad \Rightarrow \quad \delta S = \integ{t_1}{t_2}{t} \dv{}{t} \Big( m \mathbf r \cdot \mathbf v \Big)  \quad \Rightarrow \quad  Q = \mathbf v \cdot (m \mathbf r - m \mathbf{\dot r} t)
    \end{equation*}
\end{frame}

\section{Il primo teorema in teoria classica dei campi}

\begin{frame}{Formalismo lagrangiano}
    In teoria classica\footnote{In questo caso classica ha l'accezione di relativistica e non quantistica} dei campi, un generico sistema fisico è descritto dal funzionale di azione
    \begin{equation*}
        S[\phi] = \int d^4 x ~ \mathcal L (x^\mu,~\phi,~\phi_{, \mu})
    \end{equation*} 
    dove la densità di lagrangiana è definita come
    \begin{equation*}
        L = \int d^3 x ~ \mathcal L
    \end{equation*}  
      Attraverso il principio di azione stazionaria, si ricavano le equazioni del moto 
    \begin{equation*}
        \delta S = 0 \quad \Rightarrow \quad \pdv{\mathcal L}{\phi} - \partial_\mu \pdv{\mathcal L}{\phi,_\mu} = 0
    \end{equation*}
\end{frame}

\begin{frame}{Che cos'è una simmetria?}
    Una simmetria è una classe speciale di trasformazioni dei campi che lascia l'azione invariata a meno di un termine al bordo
    \begin{equation*}
        \delta S[\phi, ~\delta_s \phi] = \int d^4 x ~ \partial_\mu K^\mu
    \end{equation*}
      Una variazione on-shell è una arbitraria trasformazione dei campi che usa le equazioni del moto
    \begin{equation*}
        \delta S[\overline \phi, ~\delta \phi] = \int d^4 x ~ \partial_\mu \Big ( \pdv{\mathcal L}{\phi,_\mu}  \delta \phi \Big) 
    \end{equation*} 
\end{frame}

\begin{frame}{Il primo teorema di Noether}
    \begin{block}{Teorema}
        Sia $\delta_s \phi$ una simmetria dell'azione. Allora esiste una corrente di Noether $J^\mu$ definita come
        \begin{equation*}
            J^\mu = \pdv{\mathcal L}{\phi,_\mu} \delta \phi - K^\mu
        \end{equation*}
            tale che sia conservata lungo le equazioni del moto e quindi soddisfi l'equazione di continuità
        \begin{equation*} 
            \partial_\mu J^\mu = 0
        \end{equation*}
    \end{block}
\end{frame}

\begin{frame}{Il tensore energia-impulso}
    Dalle 4 traslazioni spaziotemporali
    \begin{equation*}
        x'^\mu = x^\mu + \epsilon^\mu
    \end{equation*}
    definiamo il tensore energia-impulso
    \begin{equation*} 
        J^\mu = T^\mu_{\phantom \mu \nu} \epsilon^\nu
    \end{equation*}
\end{frame}

\begin{frame}{Esempio: elettrodinamica}
    L'azione dell'elettrodinamica in assenza di sorgenti è
    \begin{equation*}
        S[A_\mu] = - \frac{1}{4} \int d^4 x ~ F_{\mu\nu} F^{\mu\nu}
    \end{equation*}
    dove il tensore elettromagnetico è definito come 
    \begin{equation*}
        F_{\mu\nu} = \partial_\mu A_\nu - \partial_\nu A_\mu
    \end{equation*}
      Le equazioni del moto sono le equazioni di Maxwell
    \begin{equation*}
        \partial_\mu F^{\mu\nu} = 0
    \end{equation*}

\end{frame}

\begin{frame}
    La corrente di una traslazione spaziotemporale, compatibile con l'invarianza di gauge, è 
    \begin{equation*}
    \begin{aligned}
        & \delta A_\mu = F_{\mu\nu} \epsilon^\nu \quad \Rightarrow \quad \delta S = \int d^4 x ~\partial_\sigma (-\epsilon^\sigma F^{\mu\nu} F_{\mu\nu}) \\ & \Rightarrow \quad J^\mu = - \epsilon^\sigma \Big ( F^{\mu\rho} F_{\rho\sigma} + \frac{1}{4} \delta^\mu_{\phantom \mu \sigma} F^{\alpha \beta} F_{\alpha \beta} \Big)
    \end{aligned}
    \end{equation*}
      ed il tensore energia-impulso è
    \begin{equation*}
        T^\mu_{\phantom \mu \sigma} = - F^{\mu\rho} F_{\sigma\rho} + \frac{1}{4} \delta^\mu_{\phantom \mu \sigma} F^{\alpha \beta} F_{\alpha \beta}
    \end{equation*}
\end{frame}

\section{Secondo teorema e teorie di gauge}

\begin{frame}{Formalismo hamiltoniano}
    Definiamo l'hamiltoniana attraverso la trasformazione di Legendre 
    \begin{equation*}
        H(q^i, ~p_j, ~t) = p_i \dot q^i - L
    \end{equation*}
    che, per il principio di azione stazionaria, porta alle equazioni del moto
    \begin{equation*}
        \dot q^i = \pdv{H}{p_i} \qquad \qquad \dot p_j = - \pdv{H}{q^j}
    \end{equation*}
\end{frame}

\begin{frame}{Teoria di gauge}

    Definiamo una simmetria di gauge come una trasformazione contenente un'arbitraria funzione delle coordinate spaziotemporali che lascia invariata l'azione del sistema. 

    \hfill
    
    L'elettrodinamica è una teoria di gauge, infatti prendendo 
    \begin{equation*}
        \delta A_\mu = - \partial_\mu \Lambda
    \end{equation*}
    e mettendolo nelle equazioni di Maxwell, otteniamo
    \begin{equation*}
        \partial_\nu \partial^\nu A_\mu - \partial_\mu (\partial^\nu A_\nu) = 0 \quad \Rightarrow \quad \partial_\nu \partial^\nu A'_\mu - \partial_\mu (\partial^\nu A'_\nu) = 0
    \end{equation*}
\end{frame}

\begin{frame}{Il secondo teorema inverso}
    \begin{block}{Teorema}
        Un'azione hamiltoniana nella forma
    \begin{equation*} 
        S[q^i,~p_i,~\lambda^a] = \int dt ~ (p_i \dot q^i - H_0(q^i, ~p_i) + \lambda^a \phi_a (q^i, ~p_i))
    \end{equation*} 
        dove l'hamiltoniana totale è 
    \begin{equation*}
        H = H_0 - \lambda^a \phi_a
    \end{equation*} 
        tale che presenti vincoli (di seconda classe) è invariante per trasformazioni di gauge.
    \end{block}
\end{frame}

\begin{frame} {Esempio: elettrodinamica}
    Consideriamo l'azione associata all'elettrodinamica di Maxwell
    \begin{equation*}
        S = \int d^4 x ~ \Big (\frac{1}{2} \dot A_i \dot A^i - \dot A_i \partial^i A_0 + \frac{1}{2} \partial_i A_0 \partial^i A_0 - \frac{1}{4} F^{ij} F{ij} \Big )
    \end{equation*}
    che nella descrizione hamiltoniana diventa
    \begin{equation*}
        S[A_i, ~\pi_i, ~A_0] = \int d^4 x ~ \Big ( \pi_i \dot A^i - \Big ( \frac{1}{2} \pi_i \pi^i + \frac{1}{4} F_{ij} F^{ij} \Big) + A_0 \partial_i \pi^i \Big)
    \end{equation*}
    dove il momento coniugato ad $A_i$ è 
    \begin{equation*}
        \pi_i = \pdv{\mathcal L}{\dot A_i} = \dot A_i + \partial_i A_0 = - E_i
    \end{equation*}
\end{frame}

\begin{frame}
    Attraverso il teorema inverso, il vincolo
    \begin{equation*}
        \phi = \partial_i \pi^i = \boldsymbol \nabla \cdot \mathbf E = 0
    \end{equation*}
    genera la simmetria di gauge
    \begin{equation*}
        \delta A_\mu = - \partial_\mu \Lambda(x)
    \end{equation*}
\end{frame}

\begin{frame}
   \begin{center}
        Grazie per l'attenzione.
   \end{center} 
\end{frame}

\end{document}