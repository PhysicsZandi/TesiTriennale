\begin{document}

\begin{frame}
    \titlepage
\end{frame}

%\begin{frame}
%\tableofcontents[sectionstyle=show,subsectionstyle=show/shaded/hide,subsubsectionstyle=show/shaded/hide]
%\end{frame}

\section{Introduzione}

\begin{frame}{I teoremi di Noether}
    Contenuti nell'articolo\say{Invariante Variationsprobleme} del $1918$ della matematica tedesca Emmy Noether $(1882-1935)$. 

    \hfill 

    \pause Possiamo riassumerli nel seguente modo:
\begin{itemize}
    \item simmetria globale $\iff$ legge di conservazione;
    \item simmetria locale $\iff$ identità tra le equazioni del moto.
\end{itemize}

    \hfill 

    \pause Che cosa vuol dire simmetria? Che cosa vuol dire che una quantità fisica si conserva? Che cosa vuol dire che è presente un'identità tra le equazioni del moto? 
\end{frame}

\section{Il primo teorema}

\begin{frame}{Formalismo lagrangiano}
    
    Un generico sistema fisico è descritto dal funzionale di azione

    \begin{columns}
    \begin{column}{0.5\textwidth}
       \begin{equation*}
            S[q^i(t)] = \integ{t_1}{t_2}{t} L(t, ~q^i, ~\dot q^i)
       \end{equation*} 
    \end{column}
    \begin{column}{0.5\textwidth}
       \begin{equation*}
            S[\phi] = \int d^4 x ~ \mathcal L (x^\mu,~\phi,~\phi_{, \mu})
       \end{equation*}
    \end{column}
    \end{columns}

    Attraverso il principio di azione stazionaria,

    \begin{equation*}
        \delta S = 0
   \end{equation*}

    si ricavano le equazioni del moto 

    \begin{columns}
        \begin{column}{0.5\textwidth}
           \begin{equation*}
                \pdv{L}{q^i} - \dv{}{t} \pdv{L}{\dot q^i} = 0
           \end{equation*} 
        \end{column}
        \begin{column}{0.5\textwidth}
           \begin{equation*}
                \pdv{\mathcal L}{\phi} - \partial_\mu \pdv{\mathcal L}{\phi,_\mu} = 0
           \end{equation*}
        \end{column}
        \end{columns}
\end{frame}

\begin{frame}{Simmetrie}
    
    Una simmetria è una classe speciale di trasformazioni che lascia le equazioni del moto invariate, delle coordinate
    \begin{equation*}
         \delta_s q^i \colon \delta S[q^i, ~\delta_s q^i] = \integ{t_1}{t_2}{t} \dv{}{t} K \quad \forall q^i
    \end{equation*} 
    o dei campi
    \begin{equation*}
         \delta_s \phi \colon \delta S[\phi, ~\delta_s \phi] = \int d^4 x ~ \partial_\mu K^\mu \quad \forall \phi
    \end{equation*}

    Una variazione on-shell è una arbitraria trasformazione che usa le equazioni del moto, delle coordinate
    \begin{equation*}
        \delta q^i \colon \delta S[\overline q^i,~\delta q^i] = \integ{t_1}{t_2}{t} \dv{}{t} \Big( \pdv{L}{\dot q^i} \delta q^i \Big )
    \end{equation*} 
    o dei campi
    \begin{equation*}
        \delta \phi \colon \delta S[\overline \phi, ~\delta_s \phi] = \int d^4 x ~ \partial_\mu \Big ( \pdv{\mathcal L}{\phi,_\mu}  \delta \phi \Big) 
    \end{equation*}

\end{frame}

\begin{frame}{Deformazioni}
    
    Non abbiamo tuttavia considerato trasformazioni temporali perché producono deformazioni delle coordinate.
    \begin{equation*}
        t' = t + \epsilon \Rightarrow \delta q(t) = - \epsilon \dot q(t)
    \end{equation*}
    Analogamente per campi 
    \begin{equation*}
        x'^\mu = x^\mu + \xi^\mu(x) \Rightarrow \delta \phi(x) = - \xi^\mu(x) \partial_\mu \phi(x)
    \end{equation*}
    o in particolare per campi tensoriali $(0,~1)$ 
    \begin{equation*}
        \delta A_\mu(x) = - \xi^\nu(x) \partial_\nu A_\mu (x) - A_\nu(x) \partial_\mu \xi^\nu(x)
    \end{equation*}
    dove un campo tensoriale $(0, ~1)$ viene definito dalla sua trasformazione
    \begin{equation*}
        A'_\mu(x') = \pdv{x^\nu}{x'^\mu} A_\nu(x)
    \end{equation*}

\end{frame}

\begin{frame}{Legge di conservazione (ed equazione di continuità)}
    
    Una quantità fisica si conserva se la sua derivata temporale è nulla
    \begin{equation*}
        \dv{}{t} Q = 0
    \end{equation*} 
    che può essere anche espressa tramite l'equazioni di continuità 
    \begin{equation*}
        \partial_\mu J^\mu = \partial_0 J^0 + \partial_i J^i = \partial_0 J^0 + \boldsymbol \nabla \cdot \mathbf J = 0
    \end{equation*}
    in cui a conservarsi è 
    \begin{equation*}
        Q = \int_V d^3 x ~ J^0(x)
    \end{equation*}
    
\end{frame}

\begin{frame}{Il primo teorema}
    
    \begin{block}{Il primo teorema di Noether (in meccanica classica)} 
        Sia $\delta_s q^i$ una simmetria del sistema. Allora esiste una carica di Noether $Q$ definita come
        \begin{equation*}
            Q = K - \frac{\partial L}{\partial \dot q^i} \delta_s q^i
        \end{equation*}
            tale che sia conservata lungo le equazioni del moto e quindi soddisfi la legge di conservazione
        \begin{equation*} 
            \frac{d}{dt} Q = 0
        \end{equation*}
    \end{block}

\end{frame}
\begin{frame}

    \begin{block}{Il primo teorema di Noether (in teoria classica dei campi)} 
        Sia $\delta_s \phi$ una simmetria del sistema. Allora esiste una corrente di Noether $J^\mu$ definita come
        \begin{equation*}
            J^\mu = \pdv{\mathcal L}{\phi,_\mu} \delta \phi - K^\mu
        \end{equation*}
            tale che sia conservata lungo le equazioni del moto e quindi soddisfi l'equazione di continuità
        \begin{equation*} 
            \partial_\mu J^\mu = 0
        \end{equation*}
    \end{block}
    
\end{frame}

\begin{frame}

    \begin{block}{Dimostrazione} 
        \begin{equation*}
            \delta S[\overline q^i, ~\delta_s q^i] = \integ{t_1}{t_2}{t} \dv{}{t} K = \integ{t_1}{t_2}{t} \dv{}{t} \Big( \pdv{L}{\dot q^i} \delta q^i \Big )
        \end{equation*}
        \begin{equation*}
            \integ{t_1}{t_2}{t} \dv{}{t} \Big (K - \pdv{L}{\dot q^i} \delta q^i \Big ) = 0
        \end{equation*}
        \begin{equation*}
            \dv{}{t} \Big (K - \pdv{L}{\dot q^i} \delta q^i \Big ) = \dv{}{t} Q = 0
        \end{equation*}
    \end{block}

\end{frame}

\section{La particella libera}

\section{Elettrodinamica}

\section{Il secondo teorema e teorie di gauge}

\begin{frame}
    \begin{itemize}
        \item C
        \item I
        \item A
        \item O
    \end{itemize}
\end{frame}

\section{Conclusioni}

\begin{frame}
    \begin{center}
        Grazie per l'attenzione
    \end{center}
\end{frame}



\end{document}