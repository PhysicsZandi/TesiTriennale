\begin{document}

\begin{frame}
    \titlepage
\end{frame}

%\begin{frame}
%\tableofcontents[sectionstyle=show,subsectionstyle=show/shaded/hide,subsubsectionstyle=show/shaded/hide]
%\end{frame}

\begin{frame}{Introduzione}
    Contenuti nell'articolo~\say{Invariante Variationsprobleme} del $1918$ della matematica tedesca Emmy Noether. 

    \hfill 

    Possiamo riassumerli nel seguente modo:
\begin{itemize}
    \item simmetria continua globale $\iff$ legge di conservazione,
    \item simmetria continua locale $\iff$ identità tra le equazioni del moto.
\end{itemize}

\end{frame}

\section{Il primo teorema in meccanica classica}

\begin{frame}{Formalismo lagrangiano}
    In meccanica classica\footnote{In questo caso classica ha l'accezione di non relativistica e non quantistica}, un generico sistema fisico è descritto dal funzionale di azione
    \begin{equation*}
         S[q^i(t)] = \integ{t_1}{t_2}{t} L(t, ~q^i, ~\dot q^i)
    \end{equation*} 
    Attraverso il principio di azione stazionaria,
    \begin{equation*}
        \delta S = 0
    \end{equation*}
    si ricavano le equazioni del moto 
    \begin{equation*}
        \pdv{L}{q^i} - \dv{}{t} \pdv{L}{\dot q^i} = 0
    \end{equation*} 
\end{frame}

\begin{frame}{Che cos'è una simmetria?}
    Una simmetria è una classe speciale di trasformazioni delle coordinate che lascia le equazioni del moto invariate
    \begin{equation*}
         \delta_s q^i \colon \delta S[q^i, ~\delta_s q^i] = \integ{t_1}{t_2}{t} \dv{}{t} K
    \end{equation*} 
    Una variazione on-shell è una arbitraria trasformazione delle coordinate che usa le equazioni del moto
    \begin{equation*}
        \delta q^i \colon \delta S[\overline q^i, ~\delta q^i] = \integ{t_1}{t_2}{t} \dv{}{t} \Big( \pdv{L}{\dot q^i} \delta q^i \Big )
    \end{equation*} 
    Non abbiamo tuttavia considerato trasformazioni temporali perché producono deformazioni delle coordinate
    \begin{equation*}
        t' = t + \epsilon \quad \Rightarrow \quad \delta q(t) = - \epsilon \dot q(t)
    \end{equation*}
\end{frame}

\begin{frame}{Che cosa vuol dire che una quantità si conservi?}
    Una quantità fisica si conserva se la sua derivata temporale è nulla
    \begin{equation*}
        \dv{}{t} Q = 0
    \end{equation*}    
\end{frame}

\begin{frame}{Il primo teorema di Noether}
    \begin{block}{Teorema}
        Sia $\delta_s q^i$ una simmetria dell'azione. Allora esiste una carica di Noether $Q$ definita come
        \begin{equation*}
            Q = K - \frac{\partial L}{\partial \dot q^i} \delta_s q^i
        \end{equation*}
            tale che sia conservata lungo le equazioni del moto e quindi soddisfi la legge di conservazione
        \begin{equation*} 
            \frac{d}{dt} Q = 0
        \end{equation*}
    \end{block}
\end{frame}

\begin{frame}

    \begin{block}{Dimostrazione}
            Vincolando che le $\overline q^i$ soddisfino le equazioni del moto e che le $~\delta_s q^i$ siano una simmetria dell'azione, troviamo
        \begin{equation*}
            \delta S[\overline q^i, ~\delta_s q^i] = \integ{t_1}{t_2}{t} \dv{}{t} K = \integ{t_1}{t_2}{t} \dv{}{t} \Big( \pdv{L}{\dot q^i} \delta q^i \Big )
        \end{equation*}
        Sottraendo il terzo dal secondo
        \begin{equation*}
            \integ{t_1}{t_2}{t} \dv{}{t} \Big (K - \pdv{L}{\dot q^i} \delta q^i \Big ) = 0
        \end{equation*}
        otteniamo la tesi ricercata
        \begin{equation*}
            \dv{}{t} \Big (K - \pdv{L}{\dot q^i} \delta q^i \Big ) = \dv{}{t} Q = 0
        \end{equation*}
        $\qquad \qquad \qquad \qquad \qquad \qquad \qquad \qquad \qquad \qquad \qquad \qquad \qquad q.e.d.$
    \end{block}

\end{frame}

\begin{frame}{La particella libera}

    L'azione di una particella libera è
    \begin{equation*}
        S = \integ{t_1}{t_2}{t} \frac{m}{2} \mathbf{\dot r}^2 = \integ{t_1}{t_2}{t} \frac{m}{2} (\dot x^2 + \dot y^2 + \dot z^2)
    \end{equation*} 

\end{frame}

\begin{frame}  

    La prima simmetria è quella di una traslazione temporale
    \begin{equation*}
        t' = t + \epsilon
    \end{equation*}
    che porta ad una variazione dell'azione
    \begin{equation*}
        \delta S = \integ{t_1}{t_2}{t} \dv{}{t} \Big( -\epsilon \frac{m}{2} \mathbf{\dot r}^2 \Big)
    \end{equation*}
    con termine al bordo 
    \begin{equation*}
        K = - \epsilon \frac{m}{2} \mathbf{\dot r}^2
    \end{equation*}
    
    La carica di Noether è quindi l'energia, a meno di un fattore costante
    \begin{equation*}
        Q = K - \frac{\partial L}{\partial \mathbf{\dot r}} \cdot \delta_s \mathbf r = - \epsilon \frac{m}{2} \mathbf{\dot r}^2 
    \end{equation*}

\end{frame}

\begin{frame}  

    La seconda simmetria è quella di una traslazione spaziale 
    \begin{equation*}
        \mathbf r' = \mathbf r + \mathbf a
    \end{equation*}
    che porta ad una variazione dell'azione
    \begin{equation*}
        \delta S = 0
    \end{equation*}
    con termine al bordo 
    \begin{equation*}
        K = 0
    \end{equation*}
    
    La carica di Noether è quindi la quantità di moto, a meno di un fattore costante
    \begin{equation*}
        Q = K - \frac{\partial L}{\partial \mathbf{\dot r}} \cdot \delta_s \mathbf r = - \mathbf a \cdot m \mathbf{\dot r} 
    \end{equation*}

\end{frame}

\begin{frame}  

    La terza simmetria è quella di una rotazione spaziale 
    \begin{equation*}
        \mathbf r' = \mathbf r + \boldsymbol \omega \times \mathbf r
    \end{equation*}
    che porta ad una variazione dell'azione
    \begin{equation*}
        \delta S = 0
    \end{equation*}
    con termine al bordo 
    \begin{equation*}
        K = 0
    \end{equation*}
    
    La carica di Noether è quindi il momento angolare, a meno di un fattore costante
    \begin{equation*}
        Q = K - \frac{\partial L}{\partial \mathbf{\dot r}} \cdot \delta_s \mathbf r = \boldsymbol \omega \cdot \mathbf r \times m \mathbf{\dot r} 
    \end{equation*}

\end{frame}

\begin{frame}  

    La quarta simmetria è quella di un boost di Galileo
    \begin{equation*}
        \mathbf r' = \mathbf r + \mathbf v t
    \end{equation*}
    che porta ad una variazione dell'azione
    \begin{equation*}
        \delta S = \integ{t_1}{t_2}{t} \dv{}{t} \Big( m \mathbf r \cdot \mathbf v \Big) 
    \end{equation*}
    con termine al bordo 
    \begin{equation*}
        K = m \mathbf r \cdot \mathbf v 
    \end{equation*}
    
    La carica di Noether è quindi il moto del centro di massa, a meno di un fattore costante
    \begin{equation*}
        Q = K - \frac{\partial L}{\partial  \mathbf{\dot r}} \cdot \delta_s \mathbf r = \mathbf v \cdot (m \mathbf r - m \mathbf{\dot r} t)
    \end{equation*}

\end{frame}

\section{Il primo teorema in teoria classica dei campi}

\begin{frame}{Formalismo lagrangiano}
    In teoria classica\footnote{In questo caso classica ha l'accezione di relativistica con metrica $g_{\mu\nu} = diag(1,~-1,~-1,~-1)$ e non quantistica} dei campi, un generico sistema fisico è descritto dal funzionale di azione
    \begin{equation*}
        S[\phi] = \int d^4 x ~ \mathcal L (x^\mu,~\phi,~\phi_{, \mu})
    \end{equation*} 
    Attraverso il principio di azione stazionaria,
    \begin{equation*}
        \delta S = 0
    \end{equation*}
    si ricavano le equazioni del moto 
    \begin{equation*}
        \pdv{\mathcal L}{\phi} - \partial_\mu \pdv{\mathcal L}{\phi,_\mu} = 0
    \end{equation*} 
    dove il quadrigradiente è definito come
    \begin{equation*}
        \phi_{, \mu} = \partial_\mu \phi = \dv{\phi}{x^\mu} = (\frac{1}{c} \pdv{\phi}{t}, ~ \nabla \phi)
    \end{equation*} 
\end{frame}

\begin{frame}{Che cos'è una simmetria?}
    Una simmetria è una classe speciale di trasformazioni dei campi che lascia le equazioni del moto invariate
    \begin{equation*}
        \delta_s \phi \colon \delta S[\phi, ~\delta_s \phi] = \int d^4 x ~ \partial_\mu K^\mu \quad \forall \phi
    \end{equation*}
    Una variazione on-shell è una arbitraria trasformazione dei campi che usa le equazioni del moto
    \begin{equation*}
        \delta \phi \colon \delta S[\overline \phi, ~\delta \phi] = \int d^4 x ~ \partial_\mu \Big ( \pdv{\mathcal L}{\phi,_\mu}  \delta \phi \Big) 
    \end{equation*} 
\end{frame}

\begin{frame}
    Non abbiamo tuttavia considerato trasformazioni delle coordinate perché producono deformazioni dei campi
    \begin{equation*}
        x'^\mu = x^\mu + \xi^\mu(x) \Rightarrow \delta \phi(x) = - \xi^\mu(x) \partial_\mu \phi(x)
    \end{equation*}
    o in particolare per campi tensoriali $(0,~1)$ 
    \begin{equation*}
        \delta A_\mu(x) = - \xi^\nu(x) \partial_\nu A_\mu (x) - A_\nu(x) \partial_\mu \xi^\nu(x)
    \end{equation*}
    dove un campo tensoriale $(0, ~1)$ viene definito dalla sua trasformazione
    \begin{equation*}
        A'_\mu(x') = \pdv{x^\nu}{x'^\mu} A_\nu(x)
    \end{equation*}
\end{frame}

\begin{frame}{Che cosa vuol dire che una corrente si conservi?}
    Una corrente si conserva se la sua quadridivergenza è nulla, che può essere espressa tramite l'equazione di continuità 
    \begin{equation*}
        \partial_\mu J^\mu = \partial_0 J^0 + \partial_i J^i = \partial_0 J^0 + \boldsymbol \nabla \cdot \mathbf J = 0
    \end{equation*}
    in cui a conservarsi è 
    \begin{equation*}
        Q = \int_V d^3 x ~ J^0(x)
    \end{equation*}  
\end{frame}

\begin{frame}{Il primo teorema di Noether}
    \begin{block}{Teorema}
        Sia $\delta_s \phi$ una simmetria dell'azione. Allora esiste una corrente di Noether $J^\mu$ definita come
        \begin{equation*}
            J^\mu = \pdv{\mathcal L}{\phi,_\mu} \delta \phi - K^\mu
        \end{equation*}
            tale che sia conservata lungo le equazioni del moto e quindi soddisfi l'equazione di continuità
        \begin{equation*} 
            \partial_\mu J^\mu = 0
        \end{equation*}
    \end{block}

    La dimostrazione è analoga al caso precedente.
\end{frame}

\begin{frame}{Il tensore energia-impulso}
    Dalle 4 traslazioni spaziotemporali
    \begin{equation*}
        x'^\mu = x^\mu + \epsilon^\mu
    \end{equation*}
    definiamo il tensore energia-impulso
    \begin{equation*} 
        J^\mu = T^\mu_{\phantom \mu \nu} \epsilon^\nu
    \end{equation*}
    il cui significato fisico delle componenti è 
    \begin{enumerate}
        \item $T^{00}$ è la densità di energia,
        \item $T^{0j}$ è il flusso di energia lungo la j-esima direzione,
        \item $T^{i0}$ è la densità di quantità di moto lungo la i-esima direzione,
        \item $T^{ij}$ è il tensore degli sforzi.
    \end{enumerate}
\end{frame}

\begin{frame}{Elettrodinamica}
    L'azione dell'elettrodinamica in assenza di sorgenti è
    \begin{equation}
        S[A_\mu] = - \frac{1}{4} \int d^4 x ~ F_{\mu\nu} F^{\mu\nu}
    \end{equation}
    dove il tensore elettromagnetico è definito come 
    \begin{equation*}
        F^{\mu\nu} = \partial_\mu A_\nu - \partial_\nu A_\mu
    \end{equation*}
    che conduce alla formulazione covariante delle equazioni di Maxwell
    \begin{equation*}
        \partial_\mu F^{\mu\nu} = 0
    \end{equation*}

\end{frame}

\begin{frame}
    La variazione compatibile con la simmetria di gauge è 
\begin{equation*}
    \delta A_\mu = F_{\mu\nu} \epsilon^\nu
\end{equation*}
    che conduce ad una variazione dell'azione
\begin{equation*}
    \delta S = \int d^4 x ~\partial_\sigma (-\epsilon^\sigma F^{\mu\nu} F_{\mu\nu})
\end{equation*}
    ed al termine al bordo
\begin{equation*}
    K^\sigma = -\epsilon^\sigma \mathcal L
\end{equation*}
    La corrente di Noether è
\begin{equation*}
    J^\mu = - \epsilon^\sigma \Big ( F^{\mu\rho} F_{\rho\sigma} + \frac{1}{4} \delta^\mu_{\phantom \mu \sigma} F^{\alpha \beta} F_{\alpha \beta} \Big)
\end{equation*}
    ed il tensore energia-impulso è
\begin{equation*}
    T^\mu_{\phantom \mu \sigma} = - F^{\mu\rho} F_{\sigma\rho} + \frac{1}{4} \delta^\mu_{\phantom \mu \sigma} F^{\alpha \beta} F_{\alpha \beta}
\end{equation*}

\end{frame}

\section{Secondo teorema e teorie di gauge}

\begin{frame}{Formalismo hamiltoniano}
    Definiamo l'hamiltoniana attraverso una trasformazione di Legendre 
    \begin{equation*}
        H(q^i, ~p_j, ~t) = p_i \dot q^i - L
    \end{equation*}
    che attraverso il principio di azione stazionaria porta alle equazioni del moto
    \begin{equation*}
        \dot q^i = \pdv{H}{p_i} \qquad \qquad \dot p_j = - \pdv{H}{q^j}
    \end{equation*}
    
    \hfill

    È possibile riformulare con l'uso delle parentesi di Poisson
    \begin{equation*}
        \poi{f}{g} = \pdv{f}{q^i} \pdv{g}{p_i} - \pdv{g}{q^i} \pdv{f}{p_i}
    \end{equation*}
    diventando 
    \begin{equation*}
        \dot q^i = \poi{q^i}{H} \qquad \qquad \dot p_j = - \poi{p_j}{H}
    \end{equation*}
\end{frame}

\begin{frame} {Teoria di gauge}
    Consideriamo dunque un'azione hamiltoniana nella forma
    \begin{equation*} 
        S[q^i,~p_i,~\lambda^a] = \int dt ~ (p_i \dot q^i - H_0(q^i, ~p_i) + \lambda^a \phi_a (q^i, ~p_i))
    \end{equation*}
    dove l'hamiltoniana totale è definita come 
    \begin{equation*}
        H = H_0 - \lambda^a \phi_a
    \end{equation*} 
    Le equazioni del moto sono 
    \begin{equation*}
        \dot q^i = \pdv{H}{p_i} = \pdv{H_0}{p_i} - \lambda^a \pdv{\phi_a}{p_i}
    \end{equation*}
    \begin{equation*}
        \dot p_i = - \pdv{H}{q^i} = - \pdv{H_0}{q^i} + \lambda^a \pdv{\phi_a}{q^i}
    \end{equation*} 
    \begin{equation*} 
        \phi_a(q^i, ~p_i) = 0
    \end{equation*}
\end{frame}

\begin{frame}
    Calcoliamo la derivata temporale dei vincoli 
    \begin{equation*}
        \dv{}{t} \phi_a(q^i, ~p_i) = \poi{\phi_a}{H_0} - \lambda^b \poi{\phi_a}{\phi_b} = \poi{\phi_a}{H_0} - \lambda^b C_{ab}
    \end{equation*}
    e la poniamo debolmente nulla
    \begin{equation*} 
        \poi{\phi_a}{H_0} - \lambda^b C_{ab} \approx 0
    \end{equation*}

    \begin{itemize}
        \item Vincoli di prima classe, i moltiplicatori di Lagrange $\lambda^a$ sono fissati dall'equazione
        \begin{equation*}
            \lambda^b (t) = C^{-1}_{ab} \poi{\phi_a}{H_0}
        \end{equation*}
        dove il loro ruolo è quello di mantenere il vincolo nullo $\phi = 0$ nel tempo.
        \item Vincoli di seconda classe, i moltiplicatori di Lagrange non impongono alcuna condizione e le equazioni del moto rimangono non determinate univocamente 
        \begin{equation*} 
            \poi{\phi_a}{H_0} = C_a^{\phantom a b} \phi_b \approx 0 \qquad \poi{\phi_a}{\phi_b} = C_{ab}^{\phantom{ab} c} \phi_c \approx 0
        \end{equation*}
            Siamo in presenza di teorie di gauge.
        \end{itemize}

\end{frame}

\begin{frame}{Il secondo teorema}
    L'azione è invariante per trasformazioni di gauge
    \begin{equation} \label{gaugee1}
        \delta q^i = \poi{q^i}{\phi_a} \epsilon^a (t)
    \end{equation}
    \begin{equation}\label{gaugee2}
        \delta p_i = \poi{p_i}{\phi_a} \epsilon^a (t)
    \end{equation}
    \begin{equation}\label{gaugee3}
        \delta \lambda^c = \dot \epsilon^c (t) + \epsilon^a(t) C_a^{\phantom a c} - \lambda^a \epsilon^b(t) C_{ab}^{\phantom{ab} c}
    \end{equation}  

\end{frame}

\begin{frame} {Elettrodinamica}
    L'elettrodinamica è una teoria di gauge, infatti prendendo 
    \begin{equation*}
        A'_\mu = A_\mu - \partial_\mu \Lambda
    \end{equation*}
    e mettendolo nelle equazioni di Maxwell in assenza di sorgenti
    \begin{equation*}
        \partial_\nu \partial^\nu A_\mu - \partial_\mu (\partial^\nu A_\nu) = 0
    \end{equation*}
    otteniamo 
    \begin{equation*}
        \partial_\nu \partial^\nu A'_\mu - \partial_\mu (\partial^\nu A'_\nu) = 0
    \end{equation*}

\end{frame}

\begin{frame}
    Consideriamo l'azione associata all'elettrodinamica di Maxwell
    \begin{equation*}
        S = \int d^4 x ~ \Big (\frac{1}{2} \dot A_i \dot A^i - \dot A_i \partial^i A_0 + \frac{1}{2} \partial_i A_0 \partial^i A_0 - \frac{1}{4} F^{ij} F{ij} \Big )
    \end{equation*} 
    che  nella descrizione hamiltoniana per l'elettrodinamica diventa
    \begin{equation*}
        S[A_i, ~\pi_i, ~A_0] = \int d^4 x ~ \Big ( \pi_i \dot A^i - \Big ( \frac{1}{2} \pi_i \pi^i + \frac{1}{4} F_{ij} F^{ij} \Big) + A_0 \partial_i \pi^i \Big)
    \end{equation*}

    Attraverso il teorema inverso, il vincolo
    \begin{equation*}
        \phi = \partial_i \pi^i = \nabla \cdot E = 0
    \end{equation*}
    genera la simmetria di gauge
    \begin{equation*}
        \delta A_\mu = - \partial_\mu \Lambda(x)
    \end{equation*}

\end{frame}

\begin{frame}
   \begin{center}
        Grazie per l'attenzione.
   \end{center} 
\end{frame}

\end{document}